\title{
4 Riskanalys av miljöhälsofaktorer - principer och exempel
}
I denna utredning behandlas många olika typer av hälsonisker som har samband med faktorer i miljön, från välkända men mindre allvarliga risker som vi vet att många människor utsätts för, till hypotetiska risker som kan vara mycket allvarliga på sikt, men där vi inte vet om ens några är eller kommer att bli drabbade. Bullerstörningar är exempel på det förra och t.ex. miljöföroreningar med hormonliknande effekter exempel på det senare. Dessa risker upplevs på skilda sätt av olika människor, och hanteras var och en för sig av våra myndigheter. Även Miljöhäloutsutredningens förslag är olika långt gående på olika områden. Några kommentarer till praxis för riskbedömning, riskvärdering och riskhantering kan därför vara motiverade. Flera av termerna kan förefalla vara synonymous, men markerar olika steg i riskanalysen och skiljer sig åt kvalitativt och principiellt.
\subsection*{4.1 Riskbedömning}
Med riskbedömning i miljöhälsosammanhang menas en beskrivning av de hälsoeffekter som kan uppkommata till följd av exponering för en viss miljöförorening eller blandning av föroreningar i en viss situation. Riskbedömning är en i huvudsak analytisk process, baserad på ett naturvetenskapligt underlag, till skillnad från riskuärdering och riskhantering, som även innefattar samhälleliga värderingar av risk, nytta, kostnader, alternativ, konsekvenser m.m. Vid myndighetsbeslut som baseras på riskbedömningar är det därför viktigt att man klart redovisar den vetenskapliga bedömningen och andra överväganden var för sig.
Riskbedömningens två första steg, riskidentifiering och riskuppskattning, beskrivs i kapitel 1.2 i Bilaga 1, och diskuteras därför endast kortfattat här.

deltog. Dessa dokument innebär ett viktigt steg för att internationellt och nationellt lägga ökad vikt vid och ge ökade resurser för att lösa miljö- och utvecklingsproblem.
I Agenda 21 konstateras att det råder ett nära samband mellan människors hälsa och utveckling och ett särskilt kapitel berör människors hälsa. Ett antal delmål för att minska hälsorisker orsakade av olika miljöfaktorer finns antagna om luftföroreningar i städer och inomhus, vattenföroreningar, bekämpningsmedel, bostadsområden, buller, strålning samt industri- och energiproduktion.
Riodeklarationen, Agenda 21, Hälsa för alla-programmet samt Europeiska stadgan för miljö och hälsa innehåller övergripande principer, som bör vara vägledande för och i samstämmighet med de mål som formuleras på olika nivåer i samhället.
Riksdagen har angett ett stort antal mål för miljöarbetet i Sverige, både av övergripande karaktär och mer konkreta mål inom olika områden. Många av de antagna målen baseras på regeringens proposition "En god livsmiljö" (1990/91:90) där följande övergripande målsättningar anges:
- skydda människors hälsa,
- bevara den biologiska mångfalden,
- hushålla med uttagen av naturresurserna så att de kan nyttjas längsiktigt, samt att
- skydda natur- och kulturlandskap.
Riksdagen har också antagit vissa miljömål med särskild koppling till hälsan och dessa redovisas för respektive exponering i kapitlen 5, 60 h. 8 När det gäller nationella mål för folkhälsan överlämnade regeringen 1991 en folkhälsoproposition till riksdagen. Där betonades att folkhälsaorbetet måste bygga på tre principer: lokalt engagemang, kunskap och samarbete över sektorsgränserna.
Regeringen har vidare i december 1995 fastställt direktiv för en parlamentarisk kommitté som ska utarbeta förslag till nationella mål för hälsoutvecklingen. Målen ska vara vägledande för samhällets insatser för att främja folkhälsan, förebygga ohälsa samt förhindra förtida och undvikbar funktionsnedsättning, sjuklighet och död.
Det kan finnas motsättningar mellan vissa åtgärder som förbättrar en typ av ohälsa men kan riskera att förvårra andra. Så t.ex. kan en förbättrad ventilation i samtliga bostäder kräva

\title{
1 Sammanfattning
}
Sverige har en god miljö och ett gott hälsotillstånd bland befolkningen jämfört med många andra länder. De stora hygieniska riskerna åtgärdades tidigt. Värt välfärdssamhälle och goda utbildningsnivå, lagstiftningen och en omfattande tillsynsorganisation har bidragit till det förhållandevis gynnsamma läget. En del åtgärder har emellertid varit svåra att genomföra eller har inte finansierats i tillräcklig utsträckning. För行业发展, sprids också mellan olika länder, nya produkter och varor introduceras, forskningen utvidgar kunskapen och i vissa fall blir inte hälsoeffekterna uppenbara förrän efter många decennier. Därför är den miljöbetingade ohälsan fortfarande betydande och påtaglig. Miljön måste förbättras och beredskapen inför nya risker måste vara hög.
De miljörelaterade hälsoniskerna finns i många olika sektorer och hanteras av många olika myndigheter och organisationer. Det behövs en samlad överblick av och gemensamma mål för denna del av miljö- och folkhälsoområdena, vilket varit utredningens huvudsyften. Genom att underteckna "Declaration on action for environment and health in Europe", som antogs av miljö- och hälsoministarna i Europa vid ett möte i Helsingfors 1994, har Sverige åtagit sig att senast 1997 ha utarbetat ett nationellt handlingsprogram för att minska hälsoniskerna i miljön.
Utredningens uppgift har såldes varit att presentera ett översiktligt nationellt handlingsprogram, som också är jämförbart med andra handlingsprogram i Europa, för att minska den miljörelaterade ohälsan. I enlighet med direktiven har detta begränsats till kemiska, fysikaliska och vissa biologiska risker samt skador på grund av olycksfall i miljöerna utanför arbetsplatserna. Utredningen har sammanfattat dagens kunskap om riskfaktorer i miljön som kan leda till ohälsa (bilaga 1), och även inventerat de myndigheter, lagar, mål, aktörer och verktyg

SOU 1996:124
Sjukdomar 47

tial radon radiation. Avhandling, Handelshögskolan i Stockholm, 1995.
Victorin K. Gränsvärden - vad de innebär och hur myndigheterna använder dem. Rapport 13/91. Kemikalieinspektionen, 1991.

luftföroreningar, 100-1 000 per år, är dock av samma storleksordning som den från naturlig bakgrundstrålning (ca 400 fall per år). Luftföroreningar ger dessutom en rad andra hälsoeffekter.
När det gäller lindriga effekter i form av subjektiva besvär accepteras risker av en helt annan storleksordning. I åtskilliga undersökningar har det t.ex. påvisats att flera procent av befolkningen i tätorter är mycket besvärade av buller och luftföroreningar från trafiken. Även vid Naturvårdsverkets miljömål på lång sikt för buller utomhus, 55 dBA, beräknas \(10 \%\) vara störda.
Ett exempel från inomhusmiljön är Socialstyrelsens gränsvärden för sanitär olägenhet för ventilation, som dels är uttryckt som att det ska vara minst en halv luftomsättning per timme i bostäder eller \(0,35 \mathrm{l} / \mathrm{s}\) och \(\mathrm{m}^{2}\) golvta, och dels att mängden koldioxid inte ska överstiga 1000 ppm. Det senare är exempel på ett indikatortänkande, som inte så mycket bygger på vetenskap utan mer på beprövad erfarenhet. Koldioxiden som kommer från människors utandning är inte giftig förrän vid koncentrationer upp emot \(2 \%\), men erfarenhetsmässigt vet man att problem med lukt av kroppsdörer, trötthet och huvudvärk undviks om koldioxidhalten hålls under 1000 ppm. Briställig ventilation, som avspeglar sig i så höga eller högre koldioxidhalter är tyvärr vanligt förekommande i skolor och daghem.
\subsection*{4.5 Riskuppfattning (riskperception)}
Sädana risksiffbror som de ovan nämnda kan naturligtvis vara svåra att ta till sig, men det är ändå anmärkningsvärt att radonproblemet har rönt så pass litet intresse. Rent allmänt tycks det vara så att man tenderar att skatta risker lägre som man själv kan påverka och delvis är medansvarig för, som rökning och risker med biläkande. Inom utredningens område kan det samma tänkas gälla överdriven solning, vedröken från den egna pannan och radon i den egna villan (som kan åtgärdas ventilationstekniskt). I motsvarande grad tenderar man att skatta risker som man själv inte kan påverka högre, som t.ex. industriutsläpp, livsmedelstillsatser och strålningsrisker i

ekonomiska, praktiska och andra överväganden. I praktiken kan det vara svårt att urskilja vilka överväganden som har gjorts. Myndigheternas gränsvärden är dock alltid en följd av administrativt fattade beslut, och utgör inte någon gräns mellan farligt och ofarligt. De olika miljömyndigheternas gränvårdessättningar kommenteras kortfattat nedan.
Inom miljöskyddet bestämmer Koncessionsnämmden för Miljöskydd eller länsstyrelserna hur stora utsläppen får vara från enskilda industrier eller andra miljöstörande industrier. Regeringen har fastlagt vissa generella krav, t.ex. för bilvgaser. Tekniska och ekonomiska skäl har i allmänhet varit dominerande för dessa gränsvärden.
Naturvårdsverket har fastställt gränsvärden för luftkvalitet i tätorter, som i stort sett överensstämmer med de miljömedicinskt grundade förslag som Institutet för Miljömedicin tagit fram som underlag. Halterna av exempelvis kvävedioxid överskirder gränsvärdet vid vissa starkt trafikerade gator. Det förslag till riktlinjer för vägtrafikkbuller i tätorter som Naturvårdsverket tog fram för ca tio är sedan har däremot aldrig fastlagts i form av rikt- eller gränsvärden, eftersom det skulle medföra mycket stora samhälleliga kostnader för åtgärder. Detta är ett exempel på tillämpning av den s.k. begränsningsförordningen enligt vilken myndigheterna inte får ge ut föreskrifter eller allmänna råd förärn de kostnadsmässiga konsekvenserna har utretts och man i förekommande fall fått godkännande av regeringen.
Liusmedelsverket sätter gränsvärden för bl.a. livsmedelstillsatser och bekämpningsmedelsrester i livsmedel. Dessa bygger oftast på internationella riskbedömningar enligt ovan nämnda principer, som mynnar ut i rekommenderade högsta acceptabla dagliga intag (ADI). För bekämpningsmedel och andra främmande ämnen brukar man använda begreppet högsta tolerabla dags- eller veckointag (TDI, TWI) i stället för ADI. Det ska dock påpekas att frågan om allergier och överkänslighet för livsmedelstillsatser faller utanför ADIkonceptet, och måste behandlas på annat sätt. Högre mängd av livsmedelstillsater än vad som behövs för livsmedlets hantering medges inte, och för bekämpningsmedelsrester får halterna inte vara högre än den halt som blir följden av en korrekt utförd bekämpning. Detta gör att gränsvärdet i vissa fall kan vara lägre än vad som motiveras av ADI-värdet. Normalt ligger halterna av bekämpningsmedelsrester väl under gränsvärdena.

\title{
3 Sjukdomar som kan vara miljörelaterade
}
\begin{abstract}
I Sverige har den ekonomiska och sociala utvecklingen och fördelningen av välfärden haft avgörande betydelse för befolkningens hälsa. Folkhälsan har förbättrats genom bättre hygien och kost, förbättrade livsvillkor och arbetsmiljöer, allmän skolgång, bättre boendemiljöer samt medicinskt förebyggande arbete. Även den allmänna kunskapsnivån har höjts inom områden av betydelse för att förebygga ohälsa. Hälsoläget i Sverige har i nästan alla avseenden förbättrats för genomsnittssvensken medan däremot sociala skillnader i hälsa kvarstår eller ökar.
Många olika faktorer kan samverka vid uppkomst av sjukdom. I detta kapitel kommer några sjukdomar att diskuteras utifrån förekomst och bidragande faktorer, och exemplifieras med vad man känner till om olika miljöförorenings arbetsydelse.
Till de stora folksjukdomarna brukar räknas hjärt- och kärlsjukdomar, cancer, psykisk ohälsa, skador, rörelesorganens sjukdomar samt allergier och infektionssjukdomar. Några av dessa tas upp nedan. Dessutom berörs luftvägssjukdomar samt påverkan på immunsystemet, nervsystemet och reproduktionen.
\end{abstract}
\subsection*{3.1 Hjärt-kärlsjukdomar}
Ärligen dör ca 50000 personer i hjärt- och kärlsjukdomar, varav omkring 30000 i kranskärlsjukdom och 10000 islaganfall, och de är därmed den vanligaste dödsorsaken för både män och kvinnor. Hjärt-kärlsjukdomarna svarar för mer än hällften av alla dödsfall, ca \(30 \%\) av vårddagar i sluten vård och \(10 \%\) av alla förtidspensioneringar. Män har ca \(30 \%\) högre dödlighet i hjärtkärlsjukdom än kvinnor. Dödligheten har dock minskat med

Ett annat exempel är en beräkning av samhällsekonomiska kostnader till följd av allergiska besvår, som gjorts åt Folkhälsoinstitutet (Persson m.fl., 1994). De totala kostnaderna skattades till 5,6 miljarder kronor per år. Kostnaderna har mer än fördubblats under det sista decenniet, främst på grund av en kraftig ökning av antalet förtidspensioneringar och antalet sjukskrivningsdagar samt en ökad allergiförekomst. Som nämmdes i kapitel 3 är orsaken till den ökade allergiförekomsten oklar, och därmed effekten av allergiförebyggande åtgärder, men i rapporten gjordes antagandet att förbättrad ventilation i skolor skulle leda till mindre allergibesvär. Den årliga investeringskostnaden för ventilationsförbättringar beräknades till mellan 128 och 337 miljoner kronor. För att denna investering ska vara samhällsekonomiskt lönsam måste enligt den ekonomiska modellen sjukdomskostnaderna för barn och ungdom sjunka med 12-33 \% (oräknat värdet av hälsoförluster i sig). Hur stora förbättringar av allergibesvär som ventilationsåtgärderna verklingen skulle medföra kan man naturligtvis inte uttala sig om, och man har heller inte försökt att värdera den eventuella hälsovinsten i sig.
Naturvårdsverket har nyligen presenterat en litteratursammanställning av använda metoder för att värdera miljötillgångar i ekonomiska termer med applikation på värdering av luftföroreningars påverkan på miljö och hälsa (Naturvårdsverket 1996).
En ekonomisk värdering av hälsoeffekterna av luftföroreningar har utförfts för Oslo (Rosendahl 1996). Dorsensponssamband mellan förhöjda partikelhalter (PM10) och olika hälsoeffekter användes för att beräkna antal för tidiga dödsfall, sjukfrånvaro, sjukhusvistelser m.m. Man räknade också på ett samband mellan förhöjda halter av kvävedioxid och ökat intag på sjukhus för astma. De totala samhällsekonomiska kostnaderna beräknades till 160 miljoner kronor per år, varav 4 miljoner hänfördes till kvävedioxid (reducerad livskvalitet till följd av sjukdom oräknat).
En ekonomisk värdering av hälsoeffekter av bilavgaser har också gjorts för Göteborg (Leksell och Lövgren 1995). De ekonomiska beräkningarna bygger på antaganden om betalningsviljan hos allmänheten för att minska luftföroreningshalterna. Denna antas enligt tidigare betalningsviljestudier vara 2500 kr för att halvera halterna. Deras beräkningar utmynnar i en

Arbetarskyddsstyrelsens hygieniska gränsvärden för luften på arbetsplatser är betydligt högre än motsvarande rekommendationer för utomhusluft, och avser högsta tillätna halt för vuxna under 40 av veckans 168 timmar. De fastställs med hänsyn till både toxikologiska, tekniska och ekonomiska faktorer. De utseluter inte att vissa hälsoeffekter kan upppkomma vid och under gränsvärdesnivån. Halterna på vissa arbetsplatser kan också ligga nära gränsvärdena.
Kemikalieinspektionen har hittills arbetat i begränsad omfattning med gränsvärden. Exempel är regler om halten krom i cement, avgivning av formaldehyd från träbaserade byggskivor samt nickel i varor som kommer i kontakt med huden. Regler för konserveringsmedel i kemiska produkter håller på att utarbetas. Kemiska produkter får normalt inte fritt överlåtas till privat bruk om de innehåller mer än \(0,1 \%\) av ämnen som har medelhög eller hög cancerframkallande förmåga.
Strålskyddsinstitutet har angivit dogsränser för allmänheten och för anställda i verksamhet med joinserande strållning, som bygger på internationella rekommendationer och kvantitativa cancernskruppskattningar. All planering utgår dock ifrån att endast berättigad verksamhet får äga rum, och att doserna ska hållas så låga som möjligt. Denna potimering leder normalt till att de faktiska stråldoserna blir väsentligt lägre än dosgränserna.
Socialstyrelsen och Boverket har satt riktvärden för bl.a. radon i existerande bostäder respektive vid nybyggnad. Dessa gränsvärden, 400 resp. \(200 \mathrm{~Bq} / \mathrm{m}^{3}\) är relativt sett höga, men överskrids ändå i många bostäder. Utslaget på hela bostadsbeständet beräknas den genomsnittliga radioaktiviteten (ca \(100 \mathrm{~Bq} / \mathrm{m}^{3}\) ) medföra ca 900 lungcancerfall per år enligt Strålskyddsinstitutet, och ca 400 fall enligt en riksomfattande epidemiologisk studie. Radonet i våra bostäder framstår därmed som ett av de största miljomedicinska problemen i landet. Att gränsvärdet trots detta inte har kunnat sättas lägre beror på de stora kostnaderna som ett lägre gränsvärde skulle medföra i åtgärder.

in terms of acceptable exposure, mainly on the basis of medical data and the existing documentation and conventions.
Estimates have also been made of the cost of measures proposed for certain specific areas. As a rule, the cost estimates only relate to measures designed to improve environmental quality, without reference to the health benefits. These estimates are not very precise, and no estimates have been made of the cost of not implementing preventive measures. Costbenefit analysis of measures designed to improve environmental health is an area that is in urgent need of further development.
Many of the necessary improvements can be made in connection with new investment, reconstruction, increased knowledge and reallocations. Funds allocated under employment programmes can be used for health-related environmental improvement. Some measures can be financed by charges, others will have to be paid for by the individual. Some objectives can be attained speedily, while others may take decades, either because they require concerted international action, because the cost of rapid changes would be too high or because of the slow degradation of certain substances in the food chain.
The draft proposals will be submitted to the Government for further consideration and may subsequently be presented to the Parliament. It is important that the plan should be evaluated with respect to guidelines, objectives and indicative measures; such an evaluation might be carried out after a period of five years. A revised plan might take into account new research findings, an evaluation of the integration of health issues into public and private sector planning and quality systems and the experience gained of coordination between the environmental and health sectors. Furthermore, research on the connection between environmental factors and the living conditions should by that time have reached a point where the findings can be integrated into a revised plan.
The plan of action proposed by the Commission is intended to heighten awareness of the impact on health of various environmental factors, to promote long-term planning measures that will maintain and further improve the high level of health in this country, to achieve improvements in neglected areas and to enhance knowledge both of suspected risks and of the effects of the measures taken. The plan should help to mobilize resou-

De viktigaste faktorerna vid klassningen var hälsoeffekternas allvarlighetsgrad och antalet personer som kan beräknas vara berörda. Antalet drabbade är dock genomgående mycket svårt att uppskatta, utom på vissa områden som t.ex. antalet bullerstörda i tätorter. Antaganden fick därför göras för många föroreningar för att kunna ange en högsta trolig gräns för antalet drabbade.
Människors exponering för olika miljöföroreningar är också generellt sett dåligt känd, men vid riskklassificeringsseminariet pekades viktiga exponeringssituationer ut med angivande av högexponerade grupper. De klassningar som gjordes på seminariet har legat till grund för den prioritering av miljörelaterade hälsonisker som framgår av kapitel 5.
\subsection*{4.7 Risk-kostnad-nyttoanalyser}
Om en hälsoniskbedömning utmynnar i att ett gränsvärde bör sänkas eller att andra riskreducerande åtgärder bör vidtagas av myndigheter eller regeringen, krävs i allmänhet en kostnadsberäkning enligt den tidigare nämnda begränsningsförordningen. Kostnaderna för tekniska eller andra åtgärder kan visserligen vara svåra att beräkna, men det är ändå oftast möjligt att göra det. I kapitel 5 kommer t.ex. beräknade kostnader för att sänka buller- och radonhalterna i bostäder att refereras. Men hur skulle man kunna beräkna de eventuella hälsovinster som åtgärderna kan medföra?
Denna fråga kräver för det första att man vet vilka hälsoeffekter den aktuella miljöfaktorn medför. Som redan framgått av kapitel 3, och som kommer att framgå av kapitel 5, är det svårt att ange hur stor andel av sjukligheten i befolkningen som kan hänföras till olika miljöfaktorer, bl.a. eftersom uppgifter om exponering saknas. För vissa miljöfaktorer kan man dock göra en ungefärlig uppskattning av hur många människor som är drabbade, t.ex. antalet bullerstörda människor i tätorterna och antalet människor som upplever besvär till följd av inomhusmiljöfaktorer. Ungefärliga uppskattningar av cancerrisken med några miljöföroreningar har gjorts, se kapitel 5. Härutöver har kvantitativa beräkningar i denna utredning

method used. In conclusion, the proposed action plan is presented.
These proposals, the structure of which differs somewhat from the presentation of the background material, outline the main analyses and measures. To start with, ten general guidelines are proposed for the task of reducing environmental health risks, after which objectives and indicative measures are proposed for various health-related environmental factors. The Commission's task may be summed up in three crucial questions:
- What are the most significant environment-related health risks?
- What resources and instruments exist to prevent and eliminate these risks?
- What more can be done without recourse to new financial resources?
When all the background material for the report had been prepared, a workshop was held at which the Commission's experts and members of two reference groups discussed the quality of the information and classified the health risks associated with various environmental factors. The gravity, extent and anticipated future significance of their effects on health were considered. On the basis of these discussions I have identified the following five key problem areas:
- Asthma and respiratory disorders, which are on the increase and can lead to lifelong suffering or even death. Although the causes are not fully understood, it is clear that pollution in indoor and outdoor environments plays a large part.
- Lung cancer, which claims many hundreds of victims every year, on account of polluted air, radon and environmental tobacco smoke.
- Malignant melanomas (skin tumours), which are spreading at an alarming rate due, in particular, to excessive exposure to sun.
- Accidents and injuries, which still cause many deaths and disabilities despite the success of preventive measures in the last few decades.
- Accumulation of persistent substances in the body and the food chain, which may affect future generations, although the

\title{
Summary
}
Compared with many other countries, both Sweden's environment and its people are in good health. The classical sanitary risks were eliminated a long time ago. Our welfare system, our high level of education, our legislation and our comprehensive system of administrative supervision have contributed to this comparatively favourable situation. However, some measures have proved difficult to implement or finance in full. Moreover, pollution is spread from one country to another, new products are brought onto the market, scientific progress is constantly being made, and the impact on health of certain substances sometimes only becomes apparent after several decades. Consequently, several major environmental health problems remain to be addressed. The environment must be improved and we must see to it that we are equipped to deal with new risk.
The present-related health risks occur in many different sectors and are dealt with by a variety of authorities and organizations. It is therefore essential to obtain an overall picture of this aspect of environmental and public health and agree on appropriate objectives, and this was the primary task of the Commission on Environmental Health. By signing the "Declaration on Action for Environment and Health in Europe", which was adopted by the European ministers of health and the environment at a meeting in Helsinki in 1994, Sweden has undertaken to present, by 1997, a national action plan designed to reduce environmental health risks.
The Commission's task was thus to present a coherent national action plan, similar to those to be presented by other European countries, with the above objective in view. In accordance with the Commission's terms of reference, the draft plan only covers chemical, physical and, to some extent, biological risks, as well as injuries, other than industrial injuries, caused by accidents. The report summarizes the research on which the report is based, and this is followed up by proposals on environmental and health quality objectives and indicative measures appropriate to various environmental factors and sectors. In addition, there is a brief description of diseases that may be caused by environmental factors and of the risk analysis

experimentella studier med enskilda ämnen på både försöksdjur och människor. Luftföroreningar beräknas årligen försaka några hundra fall av intag på sjukhus för luftvägssjukdomar i landet, och betydligt fler fall med lindrigare effekter (se Bilaga 1, kapitel 2).
De ämnen som oftast ger upphov till allergi är pollen från gräs, lövträd och andra växter, epitel från pälsdjur i hemmet som katt, hund eller marsvin, samt dammklvalster. Mögelsporer innehåller tämligen svaga allergen, och därför drabbads endast starkt allergibenägna individer.
Problem med allergi och annan överkänslighet har ökat under de senaste decennierna. På 1960-talet hade 2 procent av den vuxna befolkningen astma. I dag har mellan 4 och 10 procent av de vuxna astmabesvär. Av läkare diagnosticerad astma förekommer hos \(5-7 \%\) av svenska skolbarn.
Undersökningar på värnpliktiga 17-20-åringar 1971, 1981 och 1992 visar att förekomsten av estma och allergisk rinit (hösnuva) under denna tid tredubblats, se tabellen nedan. Astmafrekvensen var högst i norra Norrland, medan hösnuva var vanligast i mellansverige. Förekomsten av allergiska ögon- och näsbesvär har också ökat hos skolbarn, och är nu ca \(7-10 \%\).
Tabell 3.1. Prevalensen av estma och allergisk rinit (hösnuva) bland värmpliktiga vid mönstringen (Åberg m.fl., 1995)
\begin{tabular}{lccc}
\hline & 1971 & 1981 & 1992 \\
\hline Astma & \(1,9 \%\) & \(2,8 \%\) & \(5,7 \%\) \\
Hösnuva & \(4,4 \%\) & \(8,4 \%\) & \(14,6 \%\) \\
\hline
\end{tabular}
Flera internationella studier har visat att förekomsten av estma och allergi är högre i städer än på landsbygd. En undersökning på barn i Sundsvall visade att förekomsten av allergier, men inte av estma, var högre i tätorten än på landsbygden. Barnen i Sundsvall var mer allergiska än barn i två städer i Estland och Polen. I både Väst- och Östeuropa tenderar dock allergierna att vara vanligare i städer än på landsbygd.
Både engetiska faktorer och miljöfaktorer har betydelse för att utveckla allergi och astma. Mellan 25 och 40 procent av befolkningen löper av genetiska skäl en ökad risk att bli allergiska. Dessa förutsättningar är tämligen konstanta, liksom

ytterligare energi och inget av de viktigare energislagen är helt miljövänliga. Ökad cykling skulle minska motoravgaser men kan å andra sidan öka olycksfallsrisken om inte samtidiga åtgärder vidtas för att minska dessa risker. Ökad avfallsåtervinning kan medföra nya smittorisker genom att matavfall förvaras i närheten av bostaden under längre tid. Det är således viktigt att beakta helheten vid miljöhålsoförbättringar kilsom vid allt annat förbättringsarbete. I många fall kan ytterligare miljömedicinsk forskning behövas för att precisera vilka åtgärder som är viktigast och vilka faktorer som samverkar eller motverkar varandra.
\subsection*{2.3 Handlingsprogrammets och betänkandets utformning samt utredningens genomförande}
Utredningens uppdrag har främst gällt att framställa ett nationellt handlingsprogram för att minska miljörelaterade hälsorisker. Målsättningen har varit att handlingsprogrammet skall vara mål-, problem- och åtgärdsinriktat, uppföljningsbart, relativt kortfattat och internationellt användbart. Förslaget har utformats så att det efter eventuell remissbehandling och revidering ska kunna förelägags riksdagen.
Som ett första steg i utredningen färdigställdes en inventering och beskrivning av miljörelaterade hälsorisker (bilaga 1) och av de myndigheter, lagar, mål, aktörer och verktyg som står till buds för att förebygga dessa risker (bilaga 2). I betänkandet har inte alla frågor som behandlats i bilagorna tagits med, utan enbart de frågor vi ansett bör prioriteras.
Ett riskklingsseminarium som utredningen arrangerade i februari 1996 har varit av stor betydelse för prioriteringen - se vidare kapitel 4.
Det finns många olika slags mål, t.ex. övergripande mål, process-, struktur-, produktions-, resultat-, effekt-, kvalitetsmål m.fl. I ett miljöhålsoprogram är det naturligt att främst använda god hälsa som effektmål och en låg exponeringsnivå som skyddar mot ohälsa som resultatmål. Eftersom de önskvärda hälso- och exponeringsmålen (miljökvalitetsmålen) ofta ligger långt i framtiden och kan kräva internationell samverkan kan det

intellektuell kapacitet till följd av exponering för t.ex. metylkvicksilver, dioxiner eller PCB under fosterlivet eller tidiga barnaår.
För att förbättra möjligheterna till kvantitativ (siffermässig) riskbedömning krävs det forskningsinsatser både vad gäller mätningar och modeller som beskriver exponeringen, samt experimentella och epidemiologiska studier som är upplagda så att det går att få fram dos-responssamband i dosområden som överlappar aktuell exponering. Epidemiologiska data är naturligtvis de mest användbara, och i utredningen har t.ex. nyare epidemiologiska studier utnyttjas för att beräkna hur många personer som tas in på sjukhus för luftvägsproblem till följd av luftföroreningar, samt de viktigaste bidragande orsaksfaktorerna till astma och/eller luftrörskattarh sos små barn.
Den kvantitativa riskbedömningen är betydligt lättare att genomföra för genotoxiska cancerframkallande ämnen och joinserande strållning. Man kan då beräkna hur stor sannolikheten för cancer är vid en viss dos. Även om sådana beräkningar innehåller stora osäkerheter, så kan det lätt räknas om till antal fall per år i landet under förutsättning av viss exponering. I utredningen har sådana beräkningar refererats för t.ex. några luftföroreningar, radon, radioaktivt cesium, arsenik i dricksvatten och stekytematgener i föda.
Att det är så jämförelsevis lätt att beräkna kvantitativa risker för cancerframkallande ämnen, är en av orsakerna till att cancerframkallande miljöföroreningar ofta blir utslagsgivande i riskbedömningssammanhang. Detta gäller särskilt i nordamerikanska riskbedömningar, eftersom man där gör kvantitativa riskuppskattningar på alla cancerframkallande ämnen oavsett på vilket sätt de framkallar cancer. Man gör t.ex. kvantitativa cancerriskuppskattningar för dioxiner och PCB.
\subsection*{4.2 Riskvärdering, riskhantering}
Riskbedömning är huvudsakligen en vetenskaplig process, men när intresset vänds från hur stor risken är till hur den ska värderas och vad man kan göra åt den, så övergår frågan till att bli myndigheternas ansvar och slutligen en politisk fråga. Vid riskvärdering och riskhantering är resultatet av riskbedömningen

tioner, även om kunskapen både om exponering och konsekvenser är ofullständig. Det finns misstankar om påverkan på bl.a. immunologiska mekanismer, hormonsystem, fortplantningen och fosterutvecklingen.
Därtill kommer många hälsobesvår, som inte är livshotande men berör stora delar av befolkningen, främst bullerstörningar och mag-tarmbesvår från födoämnes- och vattenföroreningar.
Omfattande resurser satsas på att begränsa de miljörelaterade hälsoriskerna, t.ex. av näringslivet, kommunerna, statliga myndigheter, länsstyrelser och landsting, universitet och högskolor, forskningsstiftelser och forskningsfonder samt genom frivilligt arbete inom t.ex. patientföreningar och Agenda 21arbetet.
Grundprincipen för arbetet med att förebygga miljörelaterade hälsorisker är att hälsaospekterna tydliggörs och beaktas inom alla sektorer och i de många system och principer som finns för miljöarbetet. Det gäller t.ex. internationellt samarbete, producenternas ansvar, kommunal planering, livscykelanalyser och kvalitetssäkringsarbete, försiktighetsoch substitutionsprinciperna, miljölagstiftningen och tillsynen av denna, miljökonsekvensbeskrivningar, miljöövervakning, forskning och utbildning. Det är viktigt att hälsoaspekterna fortsätter att ha en central roll även när andra miljörfägor lyfts fram. De hälsoproblem som utredningen har prioriterat kräver att många olika verktyg kombineras. Vissa åtgärder kan vara kostsamma medan andra bygger på personligt initiativ.
Utredningen har inte övervägt strukturella förändringar eftersom hälsoarbetet i allmänhet bara är en delaspekt av berörda myndigheters uppgifter. Utredningen föreslår dock en förstärkning av FoU-resurserna inom det miljömedicinska området, en utökad samverkan mellan arbets- och miljömedicinsk forskning och ett fast miljöhålsoprogram, som utgår från pågående verksamhet inom Folkhälsoinstitutet.
För övrigt föreslås följande riktlinjer för det nationella miljöhålsoarbetet:
* Hälsoskyddets höga nivå bör bevaras, trots ekonomiska åtstramningar inom privat och offentlig sektor, för att inte omfattande ohålsa ska bli följden och åtgärderna bli mer kostsamma.

\title{
3.3.2 Hudeksem och födoämnesallergi
}
Cirka 3 procent av småbarn har allergiskt s.k. atopiskt eksem (böjveckseksem), som ökar med åldern upp till puberteten och sedan minskar i förekomst. Mellan 7 och \(18 \%\) av barn i åldern 7-14 år har atopiskt eksem. Förekomsten av handeksem är hos vuxna ca \(11 \%\). Nickelallergi utgör ca \(20 \%\) av kontakteksemen.
Många upplever sig ha besvår vid intag av vissa födoämnen. Bevisad födoämnesallergi är emellertid tämligen ovanligt utom under de första levnadsåren, medan icke-allergisk överkänslighet är vanligare. Cirka 8 procent av barn under 3 års ålder är allergiska/överkänsliga mot olika födoämnen, varav 2 procent mot mjölkprotein och 4 procent mot ägg. Födoämnesallergi föreligger hos 1-2 \% av 7-10-åriga barn, och sannolikt isamma frekvens bland vuxna. Förekomsten av födoämnesöverkänslighet i vidaste bemärkelse uppskattas till \(10-15 \%\).
\subsection*{3.4 Påverkan på immunförsvaret}
Immunsystemet har som uppgift att försvara kroppen mot yttre och inre fiender såsom infektioner och canceromvandlade celler. Det måste dessutom reglera sin egen funktion så att det inte överreagerar mot ämnen i den yttre miljön (allergi) eller angriper kroppens egna vävnader (autoimmunitet). Rubbningar i immunsystemet kan därför resultera i ökad förekomst av infektioner och cancersjukdomar, i allergi och i autoimmun sjukdom, t.ex. reumatism. Immunsystemet har dock normalt en stor reservkapacitet, och en något nedsatt funktion behöver därför inte medföra ökad infektionsrisk. Vid allvlarliga sjukdomar som cancer, HIV/AIDS eller diabetes kan dock immunförsvarets reservkapacitet vara nedsatt.
Kemiska ämnen kan också påverka immunförsvaret. Från djurförsök är det visat att sådan påverkan kan vara annorlunda och allvarligare om påverkan sker under fosterlivet och nyfddhetsperioden då immunsystemet fortfarande är omget, jämfört med påverkan på vuxna individer.
Det finns djurexperimentella data som visar att olika kemiska ämnen kan skada immunsystemets funktion. För människor har immunosuppressiva effekter påvisats i situationer med

avgashalter. Även om kolmonoxid kan förvärra tillståndet hos personer som redan har hjärt-kärlsjudkom, så är det dock inte visat att det medverkar i uppkomsten av sådan sjukdom.
Både rökning och s.k. passiv rökning (miljötobaksrök) medför ökad risk för hjärt-kärlsjukdom. I Socialstyrelsens Folkhälsorapport 1994 anges att flera hundra dödsfall årligen kan ha samband med passiv rökning. Från yrkesepidemiologiska undersökningar är det visat att vissa yrkesgrupper även i Sverige som t.ex. sotare, sopförbränningsarbetare, gasverksarbetare och arbetare på aluminiumsmältverk haft en ökad risk för hjärtkärlsjukdom. Dessa yrkesgrupper exponeras för olika förbränningsprodukter. Bl.a. har det diskuterats om polycykiska aromatiska kolväten (PAH) skulle kunna vara av betydelse.
\subsection*{3.2 Cancer}
Cancer är den näst vanligaste dödsorsaken och svarar för ca \(20 \%\) av alla dödsfall. Ärligen dör ca 20000 människor i någon tumörsjukdom och dubbelt så många insjuknar. De vanligaste cancerformerna är hos män prostatacancer följt av lungcancer, och hos vinknor bröstcancer följt av cancer i tjocktarmen. Risken att insjukna i cancer ökar kraftigt med ökande ålder, varför de flesta cancerfallen inträffar i hög ålder. Detta gör att en stor del av det ökade antalet cancerfall kan tillskrivas befolkningens ökade genomsnittsålder. Den resterande ökningen är ca \(0,7 \%\) per år (av totalt \(1,7 \%\) ).
Den cancerform som ökar snabbast är hudcancer, både malignt melanom och annan hudcancer som ökat med ca \(3-4 \%\) per år vardera under de senaste 20 åren. Denna cancerform anses till största delen kunna hänföras till ökat solande både hemma och på semesterresor söderut; särskilt överdrivet solande som leder till brännskador.
En annan upppärksammad ökning gäller lungcancer bland kivinnor, där ökningstakten för närvarande är \(3,7 \%\) per år. Lungcancer har tidigare ökat kraftigare bland män och är fortfarande drygt dubbelt så vanlig hos män som bland kivinnor. Sedan mitten på 80-talet visar dock lungcancer hos män en sjunkande trend. Den främsta anledningen torde vara att mäns

\title{
2 Bakgrund
}
\begin{abstract}
Sverige har en god miljö och ett gott hälsetillstånd bland befolkningen jämfört med de många länder. Vi har sedan lång tid lagt stor vikt vid åtgärder för att förebygga ohålsa och reducera exponering för hälsorisker i miljön. Miljö- och hälsoskyddsnämnderna inrättades på 1800-talet för att arbeta med åtgärder mot kolera, TBC och andra infektionssjukdomar, som till stor del var ett resultat av brister i den fysiska miljön. Ätgärder för förbättrad vatten- och avloppsförsjörning, bostadshygien, kost och utbildning, samt generellt högre levnadstandard bidrog på ett avgörande sätt till att dessa sjukdomar minskade i omfattning. Under senare decennier har också de stora industriella utsläppen åtgärdats, och det olychsfallsförebyggande arbetet har varit framgångsrikt i flera sektorer.
De aktuella riskfaktorerna i den fysiska miljön är därför delvis av annat slag i dag. Den ökade fordonstrafiken såväl i Europa som i Sverige samt bostadsuppvärminng med fossila bränslen och ved bidrar till ett flertal luftföreningar med både kort-och långsiktiga effekter. Byggnadsmateralet innehåller olika tillsatsämnen och eftersom bostäderna har gjorts tätare innebär detta att vi exponeras för en mängd olika kemiska föreningar inomhus. Den relativa betydelsen av de många, små, "diffusa" utsläppen har ökat. Vissa av dessa föroreningar ansamlas i miljön, växter och djur och hamnar så småningom i den föda vi åter och dricker. Effekterna kan visa sig först efter flera decennier. Mattförgiftningar och andra miljöbetingade infektionssjukdomar är fortfarande ett betydande problem. Nya risker kommer att uppstå på grund av nya kemikalier, ny teknik, förändrade mikroorganismer och nya riskgrupper.
För många av dessa föroreningar är hälsoeffekterna dåligt kända. Ofta är exponeringen för en enstaka förorening inte tillräckligt stor för att kunna förklara eventuella hälsoeffekter
\end{abstract}

<|im_end|>};

nervsystemet hos fostret eller det diande barnet som visar sig i försenad mental och motorisk utveckling. I övrigt är underlaget dåligt, och det är mycket svårt att skatta yttre miljöfaktorers betydelse för fosterpåverkan. Från djurförsök är det dock visat att många olika kemikalier kan ge fosterskador, men doserna i dessa försök är i allmänhet mycket högre än vad människor utsätts för i den allmänna miljön.
En annan typ av reproduktionspåverkan är den som hormonliknande ämnen kan utöva. Östrogenliknande effekter, som diskuteras närmare i avsnitt 4.4 samt kapitel 4.13 i Bilaga 1, misstänks kunna ha ett samband med bl.a. försämrad spermiekvalitet hos män och påverkan på den sexuella utvecklingen.
\subsection*{3.6 Påverkan på nervsystemet}
Orsaksfaktorer till kroniska hjärnsjukdomar som Alzheimers sjukdom, Parkinsons sjukdom eller multipel skleros (MS) är, förutom en genetisk komponent, i stort sett okända. Förekomsten av åldersdemens hos personer över 75 år är ca 10 procent i Stockholm, och är således ett betydande problem.
Beträffande miljöfaktorer så finns det indicier som pekar på ett samband mellan exponering för aluminium och Alzheimers sjukdom, men något samband med intag av aluminium via vatten eller föda är inte säkerställt (se Bilaga 1, kapitel 4.11).
Yrkesmässig exponering för höga halter av mangan i gruvor har visat sig kunna ge upphov till hjärnskador, och att andra tungmetaller som bly, klucksilver och arsenik kan ge upphov till skador på nervsystemet är väl känt från yrkesmässig exponering. Yrkesmässig exponering för organiska lösningsmedel, liksom hög alkoholkconsumtion, kan också ge skador på det centrala och perifera nervsystemet och symtom som minnesförlust, koncentrationssvårigheter och nedsatt reaktionshastighet.
De nämnda ämenna förekommer alla som föroreningar i den allmänna miljön, men i halter som understiger dem där effekter setts i djurförsök eller i yrkeslivet. Det går därför inte att precisera miljöföroreningars betydelse för nerv- eller hjärnskador i befolkningen.

samband med kärnkraft. Det är särskilt svårt att värdera risker med låg sannolikhet men allvarliga konsekvenser, eller risker på lång sikt. I fallet kärnkraft värderar strålningsexperterna riskerna som mycket mindre än vad allmänheten gör, och det samma gäller omhändertagande av kärnkraftsavfallet.
Många miljörelaterade hälsonisker presenteras som "larm" i pressen, utan att läsaren får någon information om problemets storleksordning. Detta bidrar naturligtvis till svårigheten att bedöma och relatera risker storleksmässigt till varandra. I många fall har dock även experterna svårt att göra bedömningar, om underlaget är osäkert och bristfälligt. Detta gäller särskilt potentiella risker på sikt.
\subsection*{4.6 Riskklassificering inom utredningen}
Vid ett riskklassificeringsseminarium med utredningens expertgrupp, vetenskapliga referensgrupp och "aktörsgrupp" (representanter för kommuner, länstyrelser och landsting) som hölls i februari 1996, klassades de olika miljöfaktorerna som tas upp i Bilaga 1 utifrån deras betydelse ur hälsosynpunkt.
Alla de föroreningar eller företeelser som tas upp i Bilaga 1 behandlades, uppdelat på de olika områdena utomhusluft, inomhusmiljö, vatten, föroreningar i föda, buller, joinserande och icke-joniserande strålning, elektromagnetiska fålt samt skador. För några områden gjordes också sammanfattande bedömningar. Som underlag fanns en sammanställning av de viktigaste hälsoeffekterna, känsliga grupper, viktiga exponeringssituationer med i förekommande fall angivande av uppskattat antal exponerade, samt försök still uppskattningar av antalet drabbade. Viktiga föroreningskällor angavs också med utpekande av ansvariga sektorer eller aktiviteter (t.ex. trafik, energi, personliga aktiviteter).
Utifrån det sammanställda materialet, uppgifterna i Bilaga 1 och de deltagande experternas bakgrundskunskaper klassades de olika föroreningarna/företeelserna utifrån deras betydelse ur hälsosynpunkt. Klasserna var "hög", "medel" och "låg". För vissa faktorer ansågs ingen klassning vara möjlig på grund av bristande kunskaper om effekter eller exponeringsförhållanden. Behovet av forskning ansågs vara särskilt stort för vissa utpekade ämnen eller företeelser.

utan den sammanlagda effekten av många olika riskfaktorer är avgörande. Ibland samverkar yttre riskfaktorer med livsstilsfaktorer, t.ex. radongas med tobaksrök. Därför krävs ofta en kombination av olika åtgärder för att minska en viss typ av sjuklighet eller ohälsa.
Några av de mest diskuterade hälsoeffekterna i dag till följd av olika exponeringar i miljön är risken för utveckling av cancer och luftvägsbesvär. För flera faktorer är sådana samband väl belagda. I andra fall finns misstankar om samband med olika grad av sannolikhet som behöver utredas.
Det är inte enbart effekter i form av sjuklighet och annan ohälsa som bör beaktas i dessa sammanhang. Störningar och oliägenheter till följd av buller och lukt är inte lika allvarliga effekter som utveckling av cancer och allergier, men berör många människor och kan ha stor inverkan på välbefinnandet.
Det finns också påtagliga brister i kunskaperna om vad människor faktiskt exponeras för i den allmänna miljön. Även om miljöövervakning pågått i viss utsträckning under lång tid har den främst varit inrikstad på den allmänna miljön och inte särskilt studerat upptaget för människor. Individrelaterade mätningar är nödvändiga för att kunna bedöma sambandet mellan hälsoitllstånd och de faktiska mängder som människor får i sig genom luften och födan eller genom huden.
Även om miljöarbetet alltid haft människors hälsa som främsta mäslättning är inte hälsoperspektivet alltid tydliggjort i det miljöpolitiska arbetet.
Arbetet med att förebygga och minska de miljörelaterade hälsoriskerna - i fortsättningen kallat miljöhälsoarbetet spänner över hela skalan från ekonomisk politik och lagstiftning till hälsoupplysning och kan delas in i följande åtgärdstyper:
- generella hälsopolitiska åtgärder
- processintegrerade aktiviteter, dvs. hälsoaspekterna integreras i produkt- och verksamhetsutveckling eller områdesplanering
- föreskrivande åtgärder, t.ex. lagstiftning och andra regleringar
- institutionalisering av verksamheter, t.ex. miljö- och hälsoskyddsförvaltning, folkhälsosekretariat, Agenda 21enheter
- informationsinsatser och utbildningsinsatser
- aktiviteter av de som berörs av verksamheten, t.ex. aktionsgrupper.
Utredningen har granskat och sökt beakta alla dessa olika typer av åtgärder i analysen och förslagen.
Det finns inget "ramverk" som håller samman alla de aktiviteter som syftar till att förebygga och undanröja hälsorisker i miljön. Varje näringslivssektor (transport-, industri-, energi-,

\title{
4.1.2 Riskuppskattning
}
Nästa steg i riskbedömningsprocessen handlar om att göra kvantitativa förutsågelser om vid vilka nivåer skadliga effekter kan uppträda, i relation till de halter som kan finnas t.ex. som förorening i vatten, luft eller föda eller buller- och strålningsnivär. Detta kallas riskuppskattning eller dosresponsanalys. För detta fördras data från välgjorda epidemiologiska studier eller djurförsök som innehåller olika dosgrupper. I vissa fall finns data från kontrollerade försök med människor. Så t.ex. har effekterna av luftföroreningarna ozon och kvävedioxid undersökts i kammarförsök, där försökspersoner fätt andas in gasen och effekter på lungfunktionen registrerats.
Riskuppskattning av cancer har speciella förutsättningar. Man skiljer mellan cancerframkallande ämnen och andra toxiska substanser när det gäller riskbedömning av kemikalier. För sådana cancerframkallande ämnen som också är mutagena (genotoxiska) tror man att det finns ett samband mellan exponeringsdos och sannolikheten för att drabbas av cancer även vid mycket låga doser. Detta bygger på hypotesen att en kritisk mutation är den händelse som inleder den process som kan leda till cancer, och att i princip en enda molekyl kan ge den DNAskada som framkallar mutationen. På samma sätt ser man på riskerna med joniserande strålning. Cancerrisken vid låga doser kan beräknas med hjälp av matematiska modeller. Osäkerheterna i sådana beräkningar är dock stora. I utredningen har vi refererat sådana cancerriskuppskattningar för luftföroreningarna eten, bensen, butadien och bens(a)pyren (en indikator på cancerframkallande PAH-föreningar). Detsamma gäller för aflatoxin, arsenik, nitrosaminer och stekyte-mutagener som finns som föroreningar i föda, samt för radon i inomhusluft och för annan joinerande strålning.
För andra ämnen tror man att det finns en tröskelnivå under vilken inga effekter uppkommer. Om man vill ange ett medicinstk grundat riktvärde som ska skydda den allmänna befolkningen mot oönskade hälsoeffekter dividerar man därför den högsta dos som inte har givit några effekter (eller den lågsta dos som givit effekter) i djurförsök eller epidemiologiska studier, med en säkerhetsfaktor. Vid framtagande av ADI-värden (Acceptable Daily Intake - Högsta acceptabla dagliga intag) för t.ex. livsmedelstillsatser används oftast säkerhetsfaktorn 100, där

oavsktligt hög exponering för vissa kemiska substanser. När det gäller läggradig exponering under lång tid, som är den mest relevanta situationen för allmänhetens exponering för miljöföroreningar, så har vi mycket liten kunskap om påverkan på immunfunktionen, men det finns data som antyder att PCB och andra klarerade miljöföroreningar kan påverka immunsystemet i befolkningar som åter mycket fisk från förorenade områden.
Djurexperimentella undersökningar har visat att luftföroreningar som ozon och kvävedioxid ger en minskad motståndskraft mot luftvägsinfektioner, som delvis kan hänföras till påverkan på immunförsvaret. Det finns också humandata som pekar på att luftföroreningar kan ha betydelse för förekomsten av luftvägsinfektioner, särskilt hos små barn.
En annan faktor, som indirekt har samband med miljöföroreningar, är den immunosuppressiva effekten av ultraviolett ljus (både UV-A och UV-B). Denna effekt uppträder framför allt hos personer med ljus hudfärg, och kan vara i flera veckor efter exponeringen för soljlus eller i solarium. Den immunosuppressiva effekten av soljlus är sannolikt en bidragande faktor till den ökande hudcancerfrekvensen, som primärt beror på UV-ljusets DNA-skadande effekt. Uttunningen av det stratosfäriska ozonskiktet ger ökad UV-instrålning och beräknas komma att leda till såväl nedsatt immunförsvar som ökad förekomst av hudcancer.
Endast i undantagsfall och i samband med hög exponering kan en immunotoxisk påverkan förväntas leda till nya sjukdomar. Snarare kan immunologiska effekter i en befolkning förväntas leda till en liten ökning av t.ex. infektionssjukdomar eller cancer, en ökning som kan vara svår att registrera.
\subsection*{3.5 Reproduktionspåverkan}
Med reproduktionspåverkan menas en avvikelse från det normala biologiska förloppet i samband med befruktning, graviditet och barnets utveckling före och efter födseln. Sådana avvikelser kan bero på skador som uppkommit i könscellen före befruktningen, eller i fostercellerna under graviditeten. Vid störningar på reproduktionen kan man antingen få effekter på

bilagorna. Erfarenheten visar att spridningen i de beräknade kostnaderna är stor, vilket förklaras av att beräkningen av hälstoutfallet av olika åtgärder innehåller stora osäkerheter och att de ekonomiska beräkningarna kan göras på många olika sätt och grundas på skilda förutsättningar. När de närmare kostnaderna för de föreslagna åtgärderna bedöms är det viktigt att hänsyn tas till olika parters möjlighet att finansiera förslagen.
Analyserna kompletteras sällan med beräkningar av kostnaderna för att inte sätta in hälsoförebyggande åtgärder (se vidare kap. 4.7). Sådana analyser försvåras av att det saknas gemensam grund för värdering av lidande och död i olika åldrar. När hälsoeffekterna ska skattas i monetära former används antingen "statistiskt liv", medborgarnas betalningsvilja för olika förbättringar eller minskat produktionsbortfall, sjukvårdskostnader m.m. Det finns ingen värderingsfri skala för prioriteringarna och att människovärdet beräknas i pengar.
I bedömningen av kostnads- och nyttoanalyser är det viktigt att beakta att investeringar för en förbättrad hälsa ofta sammanfaller med investeringar av andra orsaker. Principen bör vara att kostnaderna i första hand ska belasta den som orsakar den skadliga exponeringen.
Kvalificerade analysmetoder för beräkning av kostnader och nytta med olika hälsoinriktade miljöätgärder är ett angeläget utvecklingsområde även om miljöhälso- och folkhälsaorbetet inte i första hand syftar till att undvika produktionsbortfall eller minska framtida sjukvårdskostnader utan till att minska sjukligheten och öka hälsan och välbefinnandet.
\subsection*{2.4 En hållbar hälsoutveckling}
I rapporten från Världskommissionen för miljö och utveckling (Brundtlandkommissionen 1987) myntades begreppet "en hållbar utveckling". En hållbar utveckling definierades som en utveckling som tillfredsställer dagens behov utan att äventyra kommande generationers möjligheter att tillfredsställa sina behov. På motsvarande sätt talade Folkhälsogruppens nationella strategi för "Hela folkets hälsa" (1991) om en hållbar utveckling i ett hälsopolitiskt perspektiv. Med det avsågs en utveckling som i ett långsiktigt perspektiv befrämjar möjligheterna för

en faktor 10 avser att kompensera för skillnad i känslighet mellan försöksdjur och människor, och ytterligare en faktor 10 avser variation i känslighet mellan individer. Större säkerhetsfaktor kan förekomma om särskilt allvarliga effekter uppträder. Ä andra sidan har en så låg säkerhetsfaktor som 2 använts för gränsvärden för luftföroreningar, där lindriga effekter på människor varit utgångspunkten. I vissa fall har man kunnat relatera effekter med biologiska mått som exempelvis halterna av bly och kolmonoxid i blod, och då har man också använt mycket små säkerhetsmarginaler.
Det finns många kunskapsluckor och osäkerheter i de metoder som används för riskuppskattning, men de kan ändå användas för att ta fram medicinska "lågriskniväer" för miljöföroreningar. Riskuppskattningarna bör göras av experter, men eftersom det kan finnas olika syn både på det vetenskapliga underlaget och värderingen av det så kan olika experter mycket väl tänkas komma fram till olika resultat.
\subsection*{4.1.3 Exponeringsanalys}
När riskbedömningen handlar om att ta ställning till existerande föroreningsniväer i miljön blir processen oerhört mycket svårare. Här tillkommer riskbedömningens tredje del, nämligen exponeringsanalysen, där vi så gott som alltid har mycket ofullståndig kunskap om människors faktiska exponering. Människor kan exponeras för miljöföroreningar och andra kemikalier genom inandning (t.ex. i inomhus- eller utomhusluft), genom intag via vatten eller föda, eller genom direkt huddkontakt. Även i det fall vi känner exponeringen för en grupp människor är det svårt att veta om det föreligger någon risk, och hur många människor som i så fall kan befaras bli drabbade av en viss effekt. Om exponeringen ligger runt en rekommenderad lägrisknivå eller ett ADI-värde, så ska det ju finnas en marginal till den halt där effekter förväntas, men å andra sidan är marginalen satt med kante på spridning i känslighet mellan individer. Om exponeringen närmar sig den halt där effekter har påvisats i djurförsök eller epidemiologiska studier, så ökar naturligtvis risken för att effekter ska uppkomma, men det är ändå svårt att tutala sig om hur många som kan tänkas vara drabbade. Därför går det inte att säga om det t.ex. finns barn, och i så fall hur många, som har nedsatt

positivt på möjligheterna till mobilisering för det förebyggande miljöhälsoarbetet i Sverige.
Uppdraget att ta fram ett handlingsprogram mot hälsorisker i miljön är också ett led i en process där WHO:s Europakontor ursprungligen tog initiatvet. 1984 antog WHO:s europeiska medlemsländer det hälsopolitiska programmet "Hälsa för alla år 2000", där flera av de 38 målen berör hälsorisker i miljön. I Frankfurt 1989 genomfördes WHO:s "första konferens om miljö och hälsa". Där antogs den "Europeiska stadgan för miljö- och hälsa" gemensamt av miljö- och hälsoministrarna i Europaregionens medlemsländer.
I juni 1994 träffades återigen miljö- och hälsoministrarna i Europa och antog "Declaration on Action for Environment and Health in Europe". I deklarationen åtar sig varje land att senast 1997 ha utarbetat ett nationellt handlingsprogram för att minska hälsorisker i miljön. Handlingsprogrammen ska baseras på det enskilda landets behov och tas fram gemensamt av hälso- och miljödepartementen. Handlingsprogrammen ska inte enbart fokuseras på hälsorisker utan också inventera och bedöma den struktur och de verktyg som behövs för åtgärdsarbetet.
WHO:s "drejde konferens om miljö och hälsa" planeras till år 1999 i London och syftar till att följa upp de nationella och internationella aktiviteter som initierats av Helsingforsdeklarationen.
\subsection*{2.1 Utredningens avgränsningar}
Centralt för utredningen är hälsoeffekter av faktorer i den fysiska miljön. Med fysisk miljö menas här fysikaliska, kemiska och vissa biologiska faktorer utomhus och inomhus. Begreppet fysisk miljö är dock mycket omfattande. Det har därför varit nödvändigt att göra vissa avgränsningar, varav flertalet återfanns redan i direktiven (se bilaga).
Avgränsningar har gjorts så att frågor som rör arbetsmiljö samt vissa livsstilsfrågor (alkohol, tobak, kost, narkotika, läkemedel, hygieniska och kosmetiska produkter) inte omfattas av utredningen. Spridning av smitta mellan människor och mellan djur och människor ingår inte i utredningen. Däremot ingår smittspridning via vatten och föda.

antikroppar mot. Dessa allergier, som uppträder i tidiga barnaär, har dock i sig en god prognos och immunologisk tolerans utvecklas i regel inom några år. Sådana tidiga allergier visar emellertid på att individen har årvt en allergibenägenhet och de följs oftast av luftvägsallergi och astma senare under uppväxten. Luftvägarnas slemhinnor verkar inte ha samma förutsättningar att hantera immunogent material, och förmodligen bidrar detta förhållande till att sensibilisering och allergiutveckling lättast sker via luftvägarna. Lokala skador, som t.ex. vid inflammation, gynnar uppkomsten av allergi.
Mycket tyder på att allergiutvecklingen hos en individ till stor del bestäms under de första levnadsmånaderna, när immunsystemet annu inte är moget.
\subsection*{3.3.1 Luftvägssjukdomar}
Luftvägssjukdomar har ett klart samband med rökning och exponering för höga halter av luftföroreningar. Dessa sjukdomar inkluderar kronisk bronkit, atsma, lungemfysem, ökad känslighet i luftvägsslemhinnan för ospecifika stimuli (s.k. hyperreaktivitet), ökad känslighet för akuta luftvägsinfektioner samt ospecifika luftvägssymtom som hosta, slembildning, pip i bröstet och andfäddhet. Även symtom från näsa och hals har relaterats till exponering för luftföroreningar. I epidemiologiska studier har effekten av luftföroreningar studerats med sådana sjukdomsmätt som nedsatt lungfunktion, symtom från luftvägarna, försämrat tillstånd hos astmatiker och personer med andra kroniska luftvägssjukdomar, luftvägsinfektioner och nedre luftvägssjukdomar hos barn, ökat intag på sjukhus i luftvägssjukdomar och till och med ökad dödlighet i samband med episoder med höga luftföroreningshalter.
Förekomsten av atsma diskuteras i avsnittet nedan. Förekomsten av kronisk bronkit varierar med älder och rökvanor, och uppgick till 8,6\% bland vuxna män i en studie från Norrbotten 1991. Studier på förekomsten av bronkiell hyperreaktivitet i den allmänna befolkningen har inte gjorts i Sverige.
Rökning är den viktigaste orsaksfaktorn för kronisk bronkit, men också andra luftföroreningar kan spela en roll. Att allmänna luftföroreningar kan ha betydelse för luftvägssjukdomar och symtom har visats i epidemiologiska undersökningar och

livsmedel kan ske genom t.ex. förorenat foder. Vid slakt av smittat djur kan smittämnen överföras till andra slaktroppar. Campylobacterinfektion är den vanligaste rapporterade livsmedelsburna bakterieinfektionen, med ca 2500 inhemska fall per år. Bakterien är vanlig hos t.ex. höns, kor och svin. Den tillväxer dock inte lika lätt i livsmedel som Salmonella.
Antalet anmälda inhemska salmonellafall är ca 300-800 per år. Anledningen till det låga antalet inhemskt salmonellasmittade personer beror på att mindre än \(1 \%\) av alla livsmedel och djur i Sverige är smittade med Salmonella.
Vissa bakterier som tillväxer i livsmedel kan bilda toxiner som kan ge upphov till mag-tarmsjukdom. Vissa bakterietoxiner är värmestabila och bryts inte ner vid koking eller stekning.
\subsection*{3.8 Slutsatser}
Orsakerna till uppkomst av sjukdom är endast delvis kända, men sannolikt är det alltid fråga om multifaktoriella processer. Detta gör det vanskligt att dra slutsatser om miljöfaktorers betydelse. Kunskaperna är särskilt ofullståndiga när det gäller sådana miljöfaktorer som vi exponeras för i den allmänna miljön i dagens Sverige där rökning och alkoholkonsumtion ofta överskuggar effekten av yttre miljöfaktorer. Den mesta informationen om miljöföroreningars toxiska effekter föreligger från djurförsök med höga doser eller epidemiologiska undersökningar av yrkesexponerade personer, där halterna också ligger betydligt högre än i den allmänna miljön. För vissa cancerframkallande ämnen tror man att det föreligger en risk även vid låga doser, och för några sådana har man kunnat beräkna cancerrisken vid lågdosexponering. Sambandet mellan lungcancer och radon och luftföroreningar har kunnat påvisas i epidemiologiska studier, liksom andra effekter på luftvägarna av luftföroreningar. Skattningar har också kunnat göras av dessa miljöfaktorers bidrag till cancer och luftvägssjukdomar. I övrigt tillåter dock underlaget sällan kvantitativa förutsäggelser om olika miljöfaktorers bidrag till sjuklighet. Detta innebär inte att samband skulle saknas, men belyser de metodologiska svårigheter som det innebär att studera effekten av miljöföroreningar som oftast förekommer i låga halter men i en

rces in the interests of a sound environment for sustainable health development.

\title{
4.4 Riskjämförelser
}
Gränsvärdena för radon och joniserande strålning är intressanta att jämföra ur cancerisksynpunkt. Vid fastställandet av dosgränserna för joniserande strålning har man gjort jämförelser med andra risker i samhället. Dosgränsen 1 mSv per år för allmänheten, som gäller för summan av all strålning från artificiella källor, är ungefär lika stor som den från naturlig bakgrundstrålning. Det är också den dos som inte ska överskridas om kostrekommendationerna följs för konsumtion av fisk, ren, vilt, bär och svamp från områden som förorenades i samband med det radioaktiva nedfallet från Tjernobylolyckan. 1 mSv per år beräknas medföra 5 cancerödöksfall per 100000 personer och år (årlig risk för cancer \(5 \cdot 10^{4}\) ). Man hänvisar till att samhället brukar acceptera dödfallsrisker i storleksordningen 1 förolyckad per 10000 personer och år. T.ex. drunkningsolyckor har ungefär den frekvensen i Sverige, medan risken för dödsolyckor i trafiken är nästan 10 gånger större.
Gränsvärdet för radon, \(400 \mathrm{~Bq} / \mathrm{m}^{3}\), beräknas enligt de modeller som användsion strålskyddet medföra en livstidslrsk för lungcancer på ca \(2 \%\) (årlig risk ungefär 3 fall per 10000 personer, \(3 \cdot 10^{-4}\) ), vilket börjar närma sig den risk som rökare löbner med för ökrigsning, 10000 personer.
Det finns inga andra gränsvärden som bygger på kvantitativa cancerriskuppskattningar, men WHO har t.ex. rekommenderat gränsvärden för cancerframkallande föroreningar i dricksvatten vid den halt som teoretiskt motsvarar en livstids cancerrisk på 1 på 100000 (1.10²), vilket motsvarar en genomsnittlig årlig risk på 1,4 bland 10 miljoner människor (1,4-10²). Samma risknivå har används av Institutet för Miljomedicin (IMM) för att ange "lågrisknivär" för några cancerramkallande luftföroreningar. Denna risk är alltså mer än 100 gånger lägre än den som den naturlig skorra stråslningen och dosgränsen 1 mSv per år beiktans medföra.
Det finns dock inte bara en, utan kanske handfratals cancerframkallande luftföroreningar som kan tänkas verka additivt, medan den befintliga strålningen från artificiella källor er en dos som kraftigt understiger 1 mSv per år. I praktiken kan därför luftföroreningar ge upphov till fler genercall än artificiell strållning. Det på olika sätt (epidemiologiskt och/eller utgående från enskilda substanser) beräknade antalet cancerfall till följd av

Förslaget till nationellt handlingsprogram för att minska miljörelaterade hälsorisker i Sverige presenteras i detta huvudbetänkande som ett eget kapitel, med beaktande av att förslaget ska kunna utgöra grund för ett program som efter remissbehandling kan behandlas av riksdagen och presenteras internationellt.
Stockholm den 1 oktober 1996
\section*{Christer Hogstedt}
Katarina Victorin
Martin Eriksson
Titus Kyrklund

och kvinnors rökvanor utvecklats olika. Både bland män och kvinnor är lungcancer vanligast i storstäderna.
Både bröstcancer och testikelcancer ökar med drygt \(1 \%\) respektive knappt \(3 \%\) per år. Denna ökning har diskuterats i samband med hormonliknande miljöföroreningar.
Olika riskfaktorers relativa betydelse för uppkomst av cancer har skattas av den statliga cancerkommittén 1984. Cancerkommittens bedömningar är visserligen mycket osäkra, men några uppdaterade samlade svenska uppskattningar har inte presenterats. På grund av samverkanseffekter mellan de olika faktorerna kan angivna procenttal inte adderas för att få en rättvisande totalsumma.
Tobak bedöms allmänt vara den största säkert kända enskilda riskfaktorn för uppkomst av cancer. (Cancerkommittén angav ca \(15 \%\) av all cancer.) Många epidemiologiska studier har visat ett klart samband mellan antalet rökta cigaretter per dag och lungcancerrisk. Andra cancerformer som har samband med tobaksbruk är cancer i munhåla, svalg och sannolikt även njuroch annan urinvägscancer. Cancerkommittén bedömde att kostens orsaksandel till cancer är minst lika stor, t.ex. för magsäckscancer, grovtarms- och ändtarmscancer, bröstcancer och prostatacancer, men här är underlaget betydligt sämre. Riskfaktorer i kosten är högt intag av animaliskt fett och lågt intag av fibrer, frukt och grönsaker. Osäkerheten i bedömningen av matvanornas betydelse är dock mycket stor. Det kan t.ex. nämnas att man tidigare har ansett att det finns ett samband mellan intaget av fett och risk för bröstcancer hos kvinnor. Nya stora epidemiologiska undersökningar ger dock inget stöd för detta samband (Adami och Wolk, 1995).
Cancerkomittén uppskattade vidare att t.ex. allmänna luftföroreningar orsakar ca \(1 \%\) av antalet cancerfall, arbetsmiljöfaktorer \(2 \%\), radon \(1 \%\), övrig joinserande strålning \(2 \%\) och UV-ljus \(5 \%\).
Av miljöfaktorers betydelse för canceruppkomst finns det bästa underlaget för lungcancer. Cirka 2700 lungcancerfall inträffar årligen i Sverige. Av dessa uppskattas rökning orsaka 75-80 procent och radon 400-900 fall, dvs. 15-30 procent. De som både röker och bor i radonhus har mängdubbelt högre risk än de som enbart bor i radonhus. Miljötobaksrök beräknas ge upphov till ca 40-80 fall av lungcancer. Som nämnts år lungcancer vanligare i städer än på landsbygden. En genomgång av epidemiologiska studier från olika länder visar sammantaget

komplex blandning med många ämnen varav endast några få kan mätas och kvantifieras. Det betyder också att påverkan på allmänbefolkningen kan pågå under lång tid utan att upptäckas och att det förebyggande arbetet måste utgå från starka misstankar snarare än säkerställda bevis.
I jämförelse med den ottillräckligt kända betydelse som kemiska och fysikaliska miljöfaktorer kan ha för förekomsten av sjukdomar i befolkningen, så är mag-tarmsjukdomar till följd av mikrobiologisk smitta betydligt enklare att dokumentera. Kopplingen till faktorer i miljön är dock inte klar. En stor del av livsmedelsburna smittor uppkommer sannolikt till följd av brister i hanteringen och förvaringen av livsmedel, bristande hygien samt okunskap.
\subsection*{3.9 Litteratur}
Adami H-O, Wolk A. Samband fett-bröstcancer? Omhuldad hypotes överges? Läkartidningen, 98, 557-558, 1996.
Ahlbom A, Alfredsson L, Andersson I, Axelsson I, Foucard T, Pershagen G, Skering S, Stiernberg N. Miljö och Folkhälsa. Underlag till ett epidemiologiskt forskningsprogram. Naturvårdsverket Rapport 3988, 1992.
Björkstén B. Allergi och annan överkänslighet ur ett miljöperspektiv. En kunskapsöversikt. Naturvårdsverket Rapport 4523, 1996.
Böttinger M. Mikroorganismer och miljö, samt Bakteriologisk livsmedelshygien. Kapitel 4 och 5 i Miljömedicin (Rylander R, Friberg L, Skering S red.). Almqvist \& Wiksell, 1991.
Bråbäck L. Respiratory symptoms and atopic sensitization among school children in different settings around the Baltic Sea. Avhandling No 442, Linköpings Universitet, 1995.
Bråbäck L, Björkstén B. Trångboddhet skyddar mot allergi? Hypoteser om Östeuropas låga allergifrekvens. Läkartidningen, 92, 3908-3912, 1995.

sällan kompletterats med beräkningar av kostnaderna för att inte sätta in hälsoskyddande åtgärder. Kostnads- och nyttoanalyser av åtgärder för att förbättra den miljörolaterade hälsan är ett viktigt utvecklingsområde.
I stor utsträckning kan nödvändiga förbättringar genomföras inom ramen för nyinvesteringar, ombyggnader, kompetensutveckling, ökad medvetenhet och omprioriteringar. Sysselsättningsmedel kan användas för hälsorelaterade miljöförbättringar. I vissa fall kan verksamhet avgiftsbeläggas och andra åtgärder får hushällen lov att bekosta. Vissa mål kan nås omgående medan andra kommer att ta flera decennier att uppnå, t.ex. därför att internationella åtgärder krävs, kostnaderna för snabba ändringar skulle vara orimligt höga eller på grund av långsam nedbrytning i näringskedjor. När konkreta uppdrag formuleras måste noggranna ekonomiska konsekvensanalyser göras.
Detta programförslag överlämnas till regeringen för vidare behandling och eventuell presentation för riksdagen. Det är angeläget att programmet utvärderas med avseende på riktlinjer, mål och åtgärdsinriktningar, t.ex. efter ca fem år. Ett reviderat program kan inkludera nya kunskaper, utvärdera hur integreringen av hälsospekterna lyckats i de offentliga och privata planerings- och kvalitetssystemen samt dra nytta av erfarenheter som vunnits av samverkan mellan miljö- och folkhälsoarbetet. Dessutom bör forskning kring samverkan mellan miljöfaktorer och övriga livsvillkor ha utvecklats så att dessa aspekter kan integreras i ett reviderat program.
Utredningens förslag till handlingsprogram avser att stimulera till ökad medvetenhet om hälsokonsekvenserna av olika miljöfaktorer, långsiktiga planeringsinsatser för att upprätthålla och förbättra den höga hälsonivån, förbättringar av eftersatta områden och ökad kunskap om såväl misstänkta risker som åtgärdernas effektivitet. Programmets uppgift ska vara att bidra till mobilisering för en god miljö för en hållbar hälsoutveckling.

\title{
Innehåll
}

slutsumma på 720 miljoner kr per år i Göteborg och 8 miljarder kr i hela landet, som värdet av bilavgasernas hälsoeffekter.
Ett annat exempel gäller radon i bostäder. Betalningsviljan för att minska halterna undersöktes som del i en ekonomisk kostnad-nyttoanalys där "nyttan" alltså kan anges som betalningsvilja. Denna beräknades utifrån en enkästudie till i medeltal för ett hushåll \(64000 \mathrm{kr}\) för en sänkning av radondotterhalten från 500 till \(70 \mathrm{~Bq} / \mathrm{m}^{3}\), och \(19000 \mathrm{kr}\) för en minskning från 500 till \(200 \mathrm{~Bq} / \mathrm{m}^{3}\) (Söderquist 1995). Det är upperbanligen en diskrepans mellan den teoretiska betalningsviljan och antalet hushåll som faktiskt bekostar radonsanering (jämför kapitel 5.2).
\subsection*{4.8 Litteratur}
Haag Grönlund M. An introduction to health risk assessment of chemicals. Rapport 6/95. Kemikalieinspektionen 1995.
Kemikalieinspektionen. Riskbedömning och riskhantering inom kemikaliekontrollen. Rapport 11/95. 1995.
Leksell I, Lövgren L. Värdering av lokala luftföroreningeseffekter. KFB-rapport 1995:5, Kommunikationsforskningsberedningen 1995.
Naturvårdsverket. Luftföroreningar. Vad kostar de samhället? En kunskapsöversikt. Naturvårdsverket Rapport 4592, 1996.
Persson U, Svarvar P, Ödegaard K. Samhällsekonomiska kostnader avseende allergiska besvär för barn/vuxna i Sverige 1983-1993. Folkhålsoinstitutet 1994.
Rosendahl, K E. Helseeffekter av luftforurensninger og virkninger på økonomisk aktivitet. Statistisk sentralbyrå 96/8, Norge 1996.
Söderquist T. Benefit estimation in the case of nonmarket goods. Four essays on reductions of health risks due to residen-

som står till buds för att förebygga och undanröja dessa risker (bilaga 2). Olika samhällssektorers inverkan på och ansvar för de miljörelaterade hälsoniskerna har beskrivits kortfattat.
I huvudbetänkandet sammanfattas det bakomliggande faktamaterialet och kompletteras med förslag till hälso- och miljökvalitetsmål samt åtgärdsinriktningar för olika miljöfaktorer och sektorer. Därutöver beskrivs kortfattat de sjukdomar som kan vara miljörelaterade och principerna bakom riskanalyser. Huvudbetänkandet avslutas med ett förslag till handlingsprogram.
Förslaget till handlingsprogram sammanfattar de viktigaste analyserna och förslagen i en delvis annorlunda struktur än bakgrundsmaterialet. Det inleds med 10 övergripande riktlinjer för arbetet med att minska de miljörelaterade hälsoniskerna (miljöhäsorbetet), därefter föreslås mål och åtgärdsinriktningar för olika hälsonelaterade miljöfaktorer.
Utredningens uppdrag kan kondenseras till tre huvudfrågor.
- Vilka är de viktigaste miljörelaterade hälsoproblemen?
- Vilka resurser och verktyg finns för att förebygga och åtgärda dessa hälsoproblem?
- Vad kan göras ytterligare utan nya ekonomiska resurser?
När faktamaterialet till utredningen hade sammanställts genomfördes ett seminarium där utredningens experter och medlemmarna i två referensgrupper diskuterade kunskapens kvalitet och klassificerade olika miljöfaktorers hälsonisker. Hänsyn togs till sjuklighetens allvarlighet, omfattning och befarade framtida betydelse. Med stöd av detta underlag har jag funnit att fem problemområden är särskilt viktiga:
- Astma och luftvägsbesvär, som ökar i samhället och kan medföra livslångt lidande eller vara livshotande. Även om inte orsakerna är helt klarlagda är det uppenbart att föroreningar i utom- och inomhusluft spelar stor roll.
- Lungcancer, som i onödan drabbar många hundratals människor årligen på grund av luftföroreningar, radon och miljötobaksrök.
- Maligna melanom (elakartade hudtumörer), som ökar i en mycket oroande utsträckning och där ogynnsamma solvanor har stor betydelse.
- Olycksfall och skador, som fortfarande skördar många dödsfall och invalidiserande skador trots att det förbyggande arbetet varit framgångsrikt under senare decennier.
- Uplagrings av svärmedbrytbara ämnen i människokroppen och i näringskedjor och som befaras påverka kommande genera-

jordbruks-, handelssektorn m.fl.) har ett ansvar för att verksamheterna och produkterna inte orsakar ohälsa och olägenheter. Flera olika departement, centrala myndigheter och länsstyrelserna har ansvar att förebygga och undanröja hälsorisker i miljön. Lokalt har kommunerna ansvar för hälso- och miljöskyddet. Hälso- och sjukvården har också en viktig roll, bl.a. genom sin miljömedicinska verksamhet. Dessutom har varje enskild individ ansvar för det hon kan påverka.
De befintliga resurserna bör kunna nyttjas effektivare om de inblandade aktörerna kan enas om gemensamma mål, åtgärder och former för samverkan. Ett viktigt syfte med detta handlingsprogram är att sammanfatta den aktuella kunskapen om miljörelaterad ohälsa och de verktyg som finns för att förebygga sådan ohälsa för att samlat kunna visa upp problembilden och därmed underlätta gemensamma åtgärder. En nationell samling kring en handlingsplan mot våra viktigaste hälsorisker i miljön bör också stimulera det lokala och regionala åtgärdsarbetet.
Som ett gott exempel på framgångsrikt miljöhälsaorbete kan nämnas den ökade upppärksamheten kring allergi och överkänslighetsreaktioner som redovisas i bilaga 2 till utredningen. Där beskrivs hur ökningen av atsma och lutfvägsbesvär började uppmärksammas i mitten av 80-talet för att i dag vara en risk som de flesta är medvetna om. Som framgår av beskrivningen var det samspelet mellan bl.a. kvalificerat utredningsarbete i nära samarbete med berörda forskare och myndigheter, god massmedial bevakning med information, lokal och regional foränkring, stöd för nätverksbygge, statliga åtgärdsmedel, förstärkta forskningsresurser och enstaka, massiva kampanjer som effektivl lyft upp "allergifägan" på dagordningen. Byggherrar, fastighetsägare och hyresgäster har vidtagit åtgärder mot bristfalliga byggmaterial, produktionsfel och låg ventilation. Nya forskningsresurser har tillkommit för att belysa samspelet mellan olika faktorer och mekanismer bakom allergioch överkänslighetsreaktionerna.
Agenda 21-arbetet och det framgångsrika arbetet med att minska olcyksfallen inom olika sektorer är andra exempel på framgångsrikt miljöhälsoarbete som använt andra kombinationer av verktyg men som också karaktäriserats av samspel mellan de internationella, nationella, regionala och lokala niväerna, privat och offentlig sektor, forskning och åtgärder, information och frivilligarbete. Exemplen visar att det finns all anledning att se

<|im_end|>};

endast kunnat göras beträffande det antal fall av sjukhusinläggningar som kan tänkas bero på luftföroreningar.
Nästa svårighet är att kunna förutse hur stor positiv effekt på hälsan som den planerade åtgärden kan medföra. För att kunna göra detta fordras kunskap både om åtgärdens betydelse för människors exponering för det aktuella ämnet, och om dosresponssamband. Med sådana data skulle kostnadsberäkningar kunna göras av olika åtgärdsprogram och effekten på hälsan jämföras, t.ex. hur många som inte längre skulle vara besvärade efter bullerbekämpning jämfört med färre lungcancerfall efter en intensiv informationskampanj om behovet av radonreducering i vissa smähus.
Om man förutsätter att antalet sjukdoms- eller dödsfall till följd av en viss föroreningssituation kan anges, så uppkommer frågan hur detta ska värderas i ekonomiska termer. Under senare år har ekonomer börjat intressera sig för detta i syfte att kunna väga kostnader mot miljö- och hälsoeffekter uttryckt i samma "sort", nämligen pengar. En metod för att nå ett ekonomiskt värde på hälsoeffekter är att beräkna alla direkta kostnader som kan kopplas till sjukdomen, t.ex. sjukhuskostnader, läkarbesök, läkemedel och förlorad arbetsinkomst. En annan metod som kallas humankapitalmetoden går ut på att värdera produktionsbortfall med antagande om att en människas värde står i relation till vad hon producerar och att lönen ger information om produktivitet. Det är uppenbart att denna metod måste användas med stor försiktighet. För att komma runt problemen med dessa typer av kostnadsberäkningar har ekonomer i stället använt enkåtmetoder för att uppskatta människors betalningsvilja för en viss miljö- och hälsoförbättrande åtgärd.
Betalningsviljan har av Vägverket används för att beräkna värdet av ett "statistiskt liv", som används i olycksriskkalklyler. I Sverige skattas detta s.k. humanvärde till ca 14 miljoner kronor.
Hur svårt kostnad-nyttoberäkningar är kan illustreras med olkycksfallskador (se kapitel 12 i Bilaga 1). Trots att man har en god uppfattning om omfattningen av och orsakerna till skador, så har man inte kunnat kostnad-nyttoberäkna effekten av skadepreventiva åtgärder. Man har beräknat den sammanlagda samhällskostnaden för personskador till så mycket som 62 miljarder kronor per år, men man har ingen klar bild av vilka åtgärder som leder till vilken effekt eftersom det finns så många samverkande och motverkande faktorer.

\subsection*{4.1.1 Riskidentifiering}
Det första steget i riskbedömningsprocessen brukar kallas för riskidentifiering eller faroanalys. För att kunna bedöma ett ämnes inneboende toxiska (skadliga, giftiga) egenskaper erfordras kunskaper från djurförsök, epidemiologiska undersökningar (studier av exponerade befolkningsgrupper) eller kontrollerade förskö med frivilliga föröskspersoner. Utifrån sådana undersökningar kan man få kännedom om vilka skador som primärt uppkommer (kritiska effekter) och hur ämnet tas upp och sprids i kroppen, omvandlas och utsöndras.
Det ska då märkas att toxikologiska undersökningar endast krävs för godkännande av läkemedel, bekämpningsmedel och livsmedelstillsatser, och före utsättande på marknaden av nya ämnen i kemiska produkter. EU:s nya riskbegränsningsprogram för ämnen i kemiska produkter som redan finns på marknaden kan först på lång sikt medföra att dessa ämnen blir väl undersökta. Detta betyder att många miljöföroreningar är mycket dåligt undersökta, och det kan vara ganska slumpartat om toxikologiska data finns tillgängliga i litteraturen eller ej. Vissa ämnen har varit forskningsmässigt intressanta att undersöka, medan andra kanske inte undersökts alls. Detta gäller särskilt sådana ämnen som inte tillverkas, utan bildas i samband med t.ex. förbränning. Toxikologiska undersökningar omfattar inte heller alla tänkbara toxiska effekter. Så t.ex. är cancerstudier mycket tidskrävande och kostsamma, och utförs bara på utvalda substanser. För exempelvis effekter på immunförsvaret eller nervsystemet under fosterutvecklingen finns inga etablerade testmetoder.
De flesta miljöföroreningar som tas upp \(i\) detta huvudbetänkande, är relativt väl undersökta. Där finns exempel på ämnen som ger likartade effekter hos försöksdjur och människor, t.ex. skador på luftvägarna av luftföroreningar, men det finns också exempel på ämnen där man sett mycket allvarliga effekter i djurförsök men där man saknar motsvarande information från epidemiologiska studier, t.ex. dioxiner. Det finns även exempel där man sett effekter i epidemiologiska undersökningar där den biologiska verkningsmekanismen inte hittills kunnat förklaras utifrån erfarenheterna i djurförsök och andra modellsystem, t.ex. när det gäller samband mellan cancer och elektromagnetiska fält.

en relativ risk på upp till 1,5 efter justering för rökning. Detta skulle innebära att upp till en tredjedel av lungcancerfallen är hänförbara till tätortsboende, vilket i så fall motsvarar 300 fall i våra tre storstäder. Hur stor andel som beror på luftföroreningar är svårt att uppskatta, men Cancerkommittén angav ca 100 fall per år. Förhöjd lungcancerrisk har också påvisats för boende runt smältverk och vissa andra industrier.
Det finns också en ökad risk för cancer totalt i tätorter. Av denna ökade cancerrisk uppskattade Cancerkommittén att i storleksordningen 100-1 000 fall per år skulle kunna hänföras till allmänna luftföroreningar.
Ett stort antal cancerframkallande ämnen finns som föroreningar i föda, luft eller vatten; både sådana som har en genotoxisk verkningsmekanism utan s.k. tröskelnivå och sådana som verkar genom indirekta mekanismer och sannolikt har en tröskelnivå under vilken inga effekter uppkommer. Endast en del av dessa är identifierade kemiskt och än färre år undersökta toxikologiskt. Det är därför inte möjligt att göra någon uppskattning av hur stor betydelse de kan ha för cancerförekomsten. För ett fåtal ämnen gjordes försök till kvantitativ cancerriskuppskattning i Bilaga 1. Detta visade att t.ex. eten, butadien och bensen i utomhusluft kan ge upphov till 10-100 cancerrall per år i Sverige.
\subsection*{3.3 Allergiska besvår och luftvägssjukdomar}
Med allergi menas en immunologiskt grundad överkänslighet. Den atopiska allergin avser en ärtligt betingad benägenhet att bilda s.k. IgE-antikroppar. Ju mer uttalad denna benägenhet är, desto mindre doser allergen behövs för att sensibilsera individen, dvs. att initiera produktionen av allergenspecifika IgEantikroppar. De symtom som ofta uppträder vid atopisk sjukdom är astma, snuva, ögonkatarr och eksem. Dessa symtom kan även förekomma vid annan överkänslighet, dvs. utan att någon allergi föreligger.
Atopisk allergi utlöses nästan enbart via slemhinnorna i luftvägarna eller mag-tarmkanalen. Via mag-tarmkanalen utsätts vi redan tidigt för stora mängder av olika födoämnesproteiner, som det atopiskt benägna barnet snabbt utvecklar IgE-

hållbar utveckling. Därför har vi kallat utredningen och förslaget till handlingsprogram "Miljö för en hållbar hällsoutveckling".

förekomsten av vissa typer av allergiframkallande biologiskt material som pollen. Man tror därför att andra faktorer varit avgörande för ökningen av allergier. Dit hör dåligt inomhusklimat där problem med fukt, dålig ventilation och kvalsterförekomst har ökat under det senaste decenniet. Rökning och virusinfektioner i luftvägarna ökar risken för sensibilisering vid samtidig exponering för allergen, sannolikt på grund av inflammation och skador på luftvägsslemhinnan. Barn som vistas på daghem får ofta luftvägsinfektioner, och kan därför i detta avseende tänkas utgöra en riskgrupp. Ä andra sidan tycks trängboddhet skydda mot sensibilisering enligt en jämförande studie i Sverige, Estland och Polen, och en hypotes har formulerats som går ut på att vissa tidiga infektioner kan verka skyddande mot allergier (Bråbäck och Björkstén, 1995).
Exponering för tobaksrök och pålsdjur samt boende i hus med bristfällig ventilation utgör viktiga riskfaktorer för atsma och luftvägsallergi hos små barn, medan amning mer än 3-4 månader har en skyddande effekt. En förstärkt effekt av kombinationen tobaksrök och fuktiga/dåligt ventilerade byggnader har iakttagits. Den andel av atsma och/eller nedre luftvägskatarr hos små barn som kan förklaras av de nämnda riskfaktorema har i olika undersökningar skattats till 20-33 \% för föräldrarnas rökning; 11-21 \% för kortvarig amning; 6-26 \% för pålsdjur i hemmet och 12-17 \% för bristfälligt ventilerade bostäder. (Wickman och Pershagen, 1994) I en ny svensk enkätbaserad studie på 7-9-ärningar befanns övre luftvägsinfektioner vara den viktigaste riskfaktorn för atsma och allergisk rinit, förutom ärfliga faktorer. Fuktproblem i bostaden ökade risken för såväl atsma, allergisk rinit och eksem som för övre luftvägsinfektioner. (Åberg m.fl., 1996).
Luftföreningar utomhus som kvävedioxid och ozon kan påverka lungfunktionen och förorsaka en förhöjd känslighet i luftvägarna och därmed utlösa atmatsika besvär hos personer som redan har atsma. Att dessa ämnen kan försämra tillståndet hos astmatiker och öka känsligheten för inandning av t.ex. kall luft eller allergen är således visat, men däremot saknas bevis för att de kan ge upphov till astmadebut. De kan dock framkalla inflammation i lungan och ge skador på luftvägsslemhinnan, vilket kan tänkas underlätta upptag av inandat allergen och därmed också öka risken för sensibilisering. I djurförsök är det visat att höga halter av ozon och svaveldioxid kan underlätta sensibilisering mot allergen via luftvägarna.

processes involved are not fully understood. There is thought to be a risk of potential impact, inter alia, on immunological mechanisms, hormone systems, reproduction and foetal development.
In addition to these there are many health problems which, while not life-threatening, affect large numbers of people, in particular, annoyance reactions caused by noise and gastrointestinal problems caused by contaminated food and water.
Substantial resources are allocated for the purpose of reducing environmental health risks, e.g. for measures implemented by the sectors concerned, industry, local and central authorities, county administrative boards and county councils, universities and research foundations, as well as for voluntary work done by patients' associations and Agenda 21 groups.
The key to preventing environmental health risks is to define and address health concerns in all the relevant sectors and in implementation of the various systems and principles on which environmental activities are based, i.e. international cooperation, producer responsibility, community planning, life cycle analyses, quality assurance, the precautionary principle and the substitution principle, environmental legislation and its enforcement, environmental impact assessments, environmental monitoring, research, and education and training. Health concerns must be accorded a central role even where other major environmental issues emerge. Addressing the priority health problems identified by the Commission will require a combination of many different instruments. Some measures will be costly, while others will rely for their success on personal initiatives.
The Commission has not considered the need for structural changes, since public health is generally only one of the responsibilities of the authorities concerned. However, it does propose that better use be made of the research and development resources in the field of environmental medicine, a closer cooperation between research on industrial medicine and on environmental medicine and the implementation of a permanent environmental health programme by the National Institute of Public Health based on its existing operations.
In addition, the following guidelines are proposed in the context of national environmental health activities:

Äke Larsson, Miljönämnden, Malmö stad, Kia Regnér, Miljöoch hälsoskyddsnämnden, Österäkers kommun, Rolf Wickström, Miljö- och byggnämnden, Kramfors kommun, CarLennart Ästedt, Miljö- och hälsoskyddsnämnden, Uppsala kommun, Karin Andersson, Miljö- och stadsbyggnadsnämnden, Nynåshamns kommun, Bertil Fermstad, Miljö- och hälsoskyddsnämnden, Marks kommun, Göran Jansson, Miljövårdsenheten, Länsstyrelsen, Kristianstads län, Jan Johansson, Miljövårdsenheten, Länsstyrelsen, Östergötlands län, Christian Blücher, Miljövårdsenheten, Länsstyrelsen, Kalmar län, Lars-Göran Jonsson, Miljövårdsenheten, Länsstyrelsen, Jönköpings län, Gunnar Önnevall, Miljöskydd, Länsstyrelsen, Västerbottens län, Lars Hagmar, Yrkes- och miljömedicinska kliniken, Universitetsjukhuset i Lund och Kjell Andersson, Yrkes- och miljömedicinska kliniken, Regionsjukhuset i Örebro.
Utredningen har presenterats för Nationalkommittén om Agenda 21, Miljöbalksutredningen, Kommunikationskommittén och Generaldirektörsgruppen för folkhälsofrågor vid Folkhälsoinstitutet. Utredningens förslag har också presenterats för styrelsen för Folkhälsoinstitutet, generadirektören för Boverket Gösta Blücher, och föreståndaren för IMM professor Sten Orrenius med medarbetare. Utredningen var medarrangör till en nationell miljöhalsokonferens i Sundsvall 13-14 maj 1996. Utredningens sekretariat har också sammanträtt med sekretariaten för olika utredningar (Alternativbränsleutredningen, Kretsloppsdelegationen, Utredningen om Konsumenterna och miljön). Utredaren har samrätt med överdirektör Olof Edhag, styrelseordförande för Institutet för Miljömedicin, generaldirektör Anders L. Johansson, Arbetslivsinstitutet och professor Hans Wigzell, rektor för Karolinska Institutet.

* Internationell samverkan är nödvändig för att minska de gränsöverskridande föroreningarna, åtgärder mot uttunning av ozonskiktet, antagandet av förordningar och handelsregler som är gynnsamma ur hälsosynpunkt m.m. och kräver kvalificerade insatser.
* Producenterna och ägarna ska vidta de försiktighetsmätt som behövs för att förhindra skada på hälsan under produktionens, varornas och byggnadernas hela livscykel.
* Hälskonsekvensanalyser, som är en redan lagfåst del av miljökonsekvensbeskrivningar (MKB), bör även genomföras inför strategiska, politiska beslut.
* Forskningen och riskvärderingen behöver förstärkas, särskilt om samverkan mellan miljöfaktorer, levnadsvanor, arbetsoch livsvillkor samt de svårnedbrytbara ämnenas långsiktiga hälsopäverkan.
* Det lokala och regionala miljö- och folkhälsoarbetet behöver tillgång till central kunskaps- och informationsförmedling samt stöd för nätverk i folkhälsoarbetet.
* Tillsyns- och kunskapsorganisationen bör vara pluralistisk, eftersom åtgärderna mot de miljörelaterade hälsoeffekterna är integrerade delar av andra verksamheter, men också samordnad och effektiv.
* Inom hälso- och sjukvärden, där kunskapen om de miljörelaterade hälsoriskerna utvecklas inom miljömedicinen, bör resurserna användas bättre genom regional samverkan.
* Den hälsorelerade miljöövervakningen behöver förbättras och användas för fortlöpande analyser, bl.a. för att följa upp detta handlingsprogram.
Utöver dessa riktlinjer föreslås också en rad mål och åtgärdsinriktningar för att minska ohälsan från luftföroreningar, inomhusmiljön, föroreningar i dricksvatten och föda, buller, strållning och skador genom olycksfall.
De övergripande hälsomålen har formulerats så att all förebyggbar ohälsa på sikt bör elimineras. Målen för miljökvalitet anges som riktvärden för exponeringen och har huvudsakligen baserats på medicinska data samt existerande dokumentation och överenskommelser.
Kostnader för vissa delområden har belysts. De kostnadsberäknade åtgärderna tar i allmänhet bara hänsyn till en förbättrad miljökvalitet men den medicinska nyttan har sällan beräknats. Skattningarnas intervall är stora och analyserna har

Mona Blomdin (till november 1995), Maria Delvin (november 1995-juni 1996) och Eva Sandberg (från juni 1996), Miljödepartementet, Leif Busk, Livsmedelsverket, Göran Enander, Länsstyrelsen i Älvsborgs län, Ann-Sofie Eriksson (till januari 1996), Mona Åkerström (från januari 1996), Svenska Kommunförbundet, Erik Jannerfeldt/Margareta Mårtensson, SAfIndustriförbundet, Anders Jepppson, Arbetsmarknadsdepartementet, Enn Kivisäkk, Strålskyddsinstitutet, Michael Kramers, Socialdepartementet (t.o.m. december 1995), SvenÅke Larsson, Landstingsförbundet, Lars Lindau, Naturvårdsverket, Nils Gunnar Lindquist, Kemikalieinspektionen, Laila Linnergren-Fleck, Socialstyrelsen, Göran Pershagen, Institutet för miljomedicin, Pia Stork, Kommunikationsdepartementet, Inger Sävenstrand-Rådö, Folkhälsoinstitutet och Kerstin Wennerstrand, Näringsdepartementet/ Inrikesdepartementet.
Därutöver har utredningen haft stöd av en vetenskaplig referensgrupp och en referensgrupp med representanter för kommuner, länsstyrelser och landsting, för att diskutera uppläggningen, analyserna och förslagen. I den vetenskapliga referensgruppen ingick:
Prof. Anders Ahlbom, Institutet för Miljomedicin, prof. Olav Axelson, Yrkes- och miljomedicinska kliniken, Universitetssjukhuset i Linköping, doc. Jørgen Bäckström, Kemikontoret, prof. Erik Dybing, Statens Institutt for Folkehelse, Norge, doc. Carl-Gustav Elinder, Njurkliniken, Huddinge sjukhus, prof. Bo Jansson, Institutet för Tillämpad Miljöforskning, Stockholms Universitet, prof. Max Kjellman, Barnkliniken, Universitetssjukhuset i Linköping, prof. Tord Kjellström, Office of Global and Integrated Environmental Health, WHO, Genève, prof. Roland Möllby, Mikrobiologiskt och Tumörbiologiskt Centrum, Karolinska Institutet, prof. Gunnar Nordberg, Institutionen för Miljomedicin, Umeå universitet, prof. Staffan Skerfving, Yrkes- och Miljomedicinska institutionen, Lunds universitet, dr Jan Sundell, Folkhälsoinstitutet och prof. Leif Svanström, Institutionen för Folhälsovetenskap, Karolinska Institutet.
I referensgruppen med representanter för kommuner, länsstyrelser och landsting ingick:

kräver kvalificerade insatser
9.2.5 Syselsättningsmedel bör användas till
hälsorelaterade miljöförbättringar
9.2.6 Hälsokonsekvensbedömningar av större politiska beslut bör göras
9.2.7 Miljöhäsarbetet bör utvecklas tillsammans med miljö- och folkhälsoarbetet
9.2.8 Tillsyns- och kunskapsorganisationen bör vara pluralistisk
9.2.9 Forsningen och utbildningen bör förstärkas
9.2.10 Handlingsprogrammet bör följas upp och den hälsorelaterade miljöövervakningen bör förstärkas
9.3 Miljörelaterade hälsonisker - mål och åtgärdsinnriktningar
9.3.1 Utomhusluft
9.3.2 Inomhusluft
9.3.3 Föroreningar i dricksvatten
9.3.4 Smitta och föroreningar i föda
9.3.5 Buller
9.3.6 Strålning
9.3.7 Skador genom olycksfall
Bilaga: Kommittédirektiv

förmågan att alstra barn, uttryckt som förmågan att bli gravid och fullfölja graviditet, eller så kan man få effekter som kan iaktas på det framfödda barnet. Förmågan att alstra barn skulle kunna vara en funktion som är känslig för miljöpåverkan, och en viktig funktion härvidlag är spermiernas förmåga att befrukta ägg, samt det befruktade äggets förmåga till implantation och växt i livmodern.
Andelen graviditeter som slutar i missfall är så hög som 1015 procent, och har inte ökat med tiden. Orsakerna till missfall kan vara många. I omkring hälften av alla missfall som inträffar i början på graviditeten kan grava kromosomrubbningar påvisas, och i en fjärdedel föreligger missbildningar.
Kemikaliers eventuella påverkan på fostrets naturliga utveckling har varit av stort intresse alltsedan neurosedynkatastrofen i början på 60-talet. Frekvensen tidigt upptäckta missbildningar kan skattas till 2-3 procent, och totalt sett får 4-5 procent av alla barn någon missbildningsdiagnos. Någon ökning med tiden har inte iakttagits. Lindrigare påverkan kan ge upphov till minskad födelsevikt och försenad utveckling.
För att en kemisk substans ska kunna vara fosterskadande måste det passera placentabarriären, dvs. nå fostret via moderns blodcirkulation. Risken för strukturella missbildningar är störst under andra till och med tolfte graviditetsveckan då de olika organen anläggs. Under den sista delen av graviditeten växer nervsystemet mycket snabbt, vilket gör att nervsystemet under detta stadium är mycket känsligt för toxisk påverkan. Motsvarande hög känslighet föreligger under barnets första levnadstid, eftersom nervsystemets snabba tillväxt fortsätter efter födelsen. Vissa miljöföroreningar som ansamlats i kroppen utsöndras med modernsjölken och kan även på det sättet påverka det lilla barnet.
Orsakerna till fosterskador är till stor del okända. Alkoholmissbruk är dock en känd riskfaktor, liksom vissa infektioner och joinserande strållning. Rökning ökar risken för missfall och låg födelsevikt och har även satts i samband med nedsatt förmåga att alstra barn. Ett samband mellan låg födelsevikt och boende i förorenade områden har påvisats i några studier men inte i andra.
Efter epidemiologiska undersökningar på exponerade grupper av människor är det känt att t.ex. bly, metylkvicksilver och PCB kan påverka den normala utvecklingen av festret. Den känsligaste effekten av dessa ämnen tycks vara påverkan på

<smiles>[I] at (-8.1, 6.7)

\title{
3.7 Mag-tarmsjukdomar till följd av smittspridning
}
\begin{abstract}
Bakterier, virus, protozoer etc. är den del av miljön som alltid utövat det största inflytandet på människans hälsa. Som exempel kan nämnas tyfoidfeber- och koleraepidemierna på 1800-talet. Till följd av förbättrad hygien har antalet fall av epidemisk gulset (hepatit A) sjunkit från 12000 fall år 1948 till omkring 300 fall per år.
\end{abstract}
Av de tarmburna bakteriella smittorna har tyfoid- och paratyfoidfeber minskat markant. Däremot är salmonellaneriterima vanliga och har ökat igen i slutet av 1980talet. Till stor del sker smittan utomlands. Campylobakterinfektion som orsak till diarré har också visat sig vara vanligt. Den vanligaste orsaken till barndiarré är rotavirus.
Tidigare inträffade bakteriella tarminfektioner främst på sommaren och hösten. I och med att matvaror nu förvaras i kylskåp, och att resor till sydliga länder ökat, har det skett en viss utjämning över året. Indirekt kontaktsmitta är särskilt vanlig i miljöer där hygienen är dålig, t.ex. vid gemensamt bruk av handduk eller briställig diskning av bestick och porslin. Tarmsmittor kan överföras genom bristande renhållning i toaletter och badrum.
Smittspridning via vatten och föda har beskrivits i Bilaga 1, kapitel 4.2, 4.4 och 5.3. Smittspridning via badvatten är dåligt dokumenterat. Antalet utbrott av mag-tarmsjukdom till följd av förorenat dricksvatten är ca 2-10 varje år, med totalt ca 3000 insjuknade personer. Antalet rapporterade livsmedelsburna utbrott var 1995 134 st med sammanlagt ca 3000 sjuka. Mörkertalet är dock stort, och antalet kan vara så högt som 500000, vilket innebär att varje svensk drabbas vart femtonde till tjugonde år av magsjuka.
Vattenburna smittor kan förekomma om dricksvattnet förorenas med avloppsvatten. Tidigare förekom vattenburna tyfoidfeber- och dysenteriepidemier. I dag förekommer insjuknande till följd av framför allt Campylobacterios. Virus kan också spridas via avloppsvatten, liksom giardiasis och amöbadysenteri.
Risen för smittspridning via livsmedel är särskilt stor, eftersom bakterier (men inte virus) kan tillväxa i livsmedlet om temperaturen överstiger ca \(+10^{\circ} \mathrm{C}\). Förorening av animaliska

\(23 \%\) bland män och \(31 \%\) bland kvinnor under perioden 19751991.
Från att tidigare ha varit en sjukdom som var vanligare bland högutbildade män har hjärt-kärlsjukdomar nu blivit vanligast bland arbetare inom industri- och transportnäringarna. Bland kvinnor har risken alltid varit högre i lägre socialgrupper. Att högutbildade män i ökad utsträckning slutat röka antas spela en stor roll. I alla äldrar har såväl risken att insjukna som att avlida i hjärtinfarkt minskat. Även dödligheten i slaganfall har minskat.
De socioekonomiska skillnaderna i dödlighet i hjärtkärlsjukdom har upppärksammats under senare år. Härutöver finns regionala skillnader i landet. Under de senaste decennierna har dödligheten i hjärt-kärlsjukdom genomgående varit högre i norra Sverige. Ur Socialstyrelsens Folkhälsaropport 1994 kan dock utläsas att dödligheten i kranskärssjukdom är hög även i t.ex. Värmland och Södermanland och lägst i Halland för både män och kvinnor.
Förutom rökning är höga blodfetter (kolesterol) och högt blodtryck de viktigaste kända riskfaktorerna. Andra riskfaktorer är faktorer i arbetsmiljön, låg fysisk aktivitet, övervikt, passiv rökning, diabetes, arbetslöshet, social utsatthet och brister i socialt stöd och nätverk. Ärflliga faktorer spelar också in. Dessa riskfaktorer kan naturligtvis samverka med varandra. Detta innebär att enskilda riskfaktorers bidrag inte på ett enkelt sätt kan adderas.
Orsakerna till de regionala skillnaderna i Sverige har diskuterats. Kostvanorna har säkerligen bidragit till den höga dödligheten i Norrland, liksom andra riskfaktorer som arbetslöshet, övervikt och diabetes. Värmlands höga dödlighet har dock inte kunnat förklaras med vare sig rökning, högt blodtryck eller höga kolesterolvärden.
Beträffande miljöfaktorers betydelse för uppkomst av hjärtkärtsjukdom, så är detta relativt lite utforskat. Man har dock påvisat samband mellan hjärt-kärlsjukdom och "mjukt" dricksvatten, dvs. vatten med lågt innehåll av kalcium och magnesium, i epidemiologiska undersökningar i såväl Sverige som i andra länder. Några säka slutsatser om orsakssamband har dock inte kunna dras. Dessa studier diskuteras närmare i Bilaga 1, kap. 4.13.
Förbräningen av organiskt material ger upphov till kolmonoxid och olika organiska förbränningsprodukter, i högre halter ju sämre förbräningen är. Kolmonoxid binder till blodets hemoglobin och bildar s.k. karnoxyhemoglobin och ger försärmrad syretransport till hjärta och hjärna. Det är väl känt att personer med kärkramp kan få förvärrade besvär om halten av karnboxyhemoglobin i blodet överstiger ca \(2,5 \%\). Detta kan inträffa vid vistelse i rökiga lokaler eller platser med höga

människan att leva ett hälsosamt liv. Även i vårt land måste ojämlikheten i hälsa mellan olika sociala grupper minska för att skapa en hållbar utveckling.
Utredningen har inte analyserat den miljörelaterade ohälsans utbredning i olika sociala grupper eller skillnader mellan män och kvinnor. Det beror på att uppgifter inte funnits tillgängliga och att hela utredningen baserats på känd kunskap eftersom resurser saknats för egna forskningsinsatser. Sverige har inte längre några slumområden och därför är de miljöbetingade hälsoskillnaderna mellan olika sociala grupper mindre än i många andra länder. Det är emellertid troligt att de som har låg utbildning och mindre personliga resurser har sämre möjligheter att tillgodogöra sig nödvändig information för att minska många miljörelaterade hälsoproblem. Tobaksrökning är mycket utbredd hos lågutbildade kvinnor och deras barn kommer därför att drabbas av miljötobaksrök i högre utsträckning än andra barn. Olycksfallsrisken för barn till resurssvaga föräldrar är större än för andra barn. Bristfälliga bostadsområden med dålig ljudisolering och dålig fritidsmiljö bebos av de utan valmöjlighet. Ekonomisk resurrsbrist ökar risken för sämre kosthåll. Det är således troligt att det räder en viss social snedfördelning av den miljörelaterade ohälsan likaväl som av yrkesmässiga och sociala riskfaktorer.
Hur samhället beskyddar och börjer för sina barn, mäter dess grad av civilisation och utvecklingsförmåga. Barnen ska vara de första som gynnas av mänskighetens framgångar och de sista som drabbas av dess misslyckanden. Ostimulerande utemiljö, nedslitna bostadsområden och dåliga skollokaler välkomnar inte de unga och ger inte optimala förutsättningar för en utvecklande uppväxtimiljö. Barns större känslighet än vuxnas motiverar att grånsvärdesättning för att tilltåa ämnen i miljön och olika produkter utgår från barn i större utsträckning än vad som i dag är fallet.
Den tilltagande upplagringen av giftiga ämnen som inte bryts ner och befaras påverka fortplantningen och fosterutvecklingen, liksom ökningen av luftvägsbesvär och elakartade hudtumörer är exempel på hot mot en hållbar halsoutveckling. Det behövs sett samlat program för en hälsosam utveckling, som skapar förväntningar om större hänsyn till hälsoeffekterna i planerings- och kvalitetssystem, medborgarnas engagemang, ökad kunskap, långsiktig planering för att underhålla fungerande system, samordnade insatser, goda säkerhetsmarginaler, bättre miljö och en

grundläggande, men vid val av åtgärder måste andra hänsyn vägas in, exempelvis tekniska och ekonomiska möjligheter att minska exponeringen samt tillgången på alternativ. Hur risken uppfattas av samhället inverkar också på bedömningen av olika åtgärders genomförbarhet och på val av styrmedel. Riskvärdering och -hantering innefattar alltså överväganden av såväl naturvetenskaplig som politisk, social, kulturell, ekonomisk och teknisk natur.
De olika lagar och verktyg som står till buds för att minska utsläpp och människors exponering för miljöförroreningar, buller, strålning etc., har beskrivits i bilaga 2.
De riskniväer vid vilka den allmänna opinionen och myndigheterna anser sig föranledda att ingripa kan variera avsevärt från fall till fall. Vi får t.ex. röka och dricka alkohol fritt fastän dessa vanor förorsakar tijotusentals förtida dödsfall och allvarliga sjukdomsfall ärligen. À andra sidan kan farliga leksaker dras in som följd av ett enda olyckstillbud med ett barn. Ett aktuellt exempel på tidigt ingripande gäller den s.k. galna kosijukan i England, där hundratusentals kor ska slaktas ut på grund av en icke-bevisad risk för att en allvarlig men mycket sällsyt hjärnsjukdom hos människan kan ha koppling till galna kosijukan och spridas via köttet. Att andra faktorer än de rent vetenskapliga har vägts in i riskvärderingen som ledde till EUländernas beslut är uppenbart. Den massmediaal uppmärksamheten torde ha spelat en icke oväsentlig roll.
\subsection*{4.3 Gränsvärden}
Ett av styrmedlen för att minimera miljörelaterade hälsorisker är att ange gränsvärden för t.ex. högsta tillätna utsläpp till luft och vatten eller halter av föroreningar i vatten, luft och föda. Egentligen avses med gränsvärden bindande sådana som är fastställda av regering eller myndighet, och där överträdanden medför någon form av sanktioner. Myndigheterna ger dock också ut riktvärden och rekommendationer som i praktiken ofta fungerar som gränsvärden.
Idealt bör gränsvärden sättas efter att en vetenskaplig riskbedömning först har gjorts enligt principerna ovan, och en riskvärdering sedan gjorts där riskerna vägts mot tekniska,

Bylin G, Boström C-E. Luften vi andas utomhus. Utomhusluftens betydelse för allergi och annan överkänslighet. Folkhälsoinstitutet, 1994.
Cancerkommittén. Cancer. Orsaker, förebyggande m.m. SOU 1984:87, 1984.
Formgren H. Omfattningen av allergi och annan överkänslighet. Vetenskaplig kunskapssammanställning. Folkhälsoinstitutet, 1994.
Institutet för Miljömedicin. Hälsorelaterad miljöövervakning ett programförslag. IMM-rapport 7/92, 1992.
Lövik M. Miljöets virkning på immunsystemet. Manuskript som underlag till Miljöhålsoutredningen. Statens Institutt for Folkehelsa, Oslo, 1996.
Socialstyrelsen. Folkhälsorapport 1994.
Sundell J, Kjellman M. Luften vi andas inomhus. Inomhusmiljöns betydelse för allergi och annan överkänslighet. Folkhälsointistutet, 1995.
Wickman M, Pershagen G. Utvärdering av allergiprevention: Mätning av hälsorelaterade mål i Folkhälsoinstitutets allergiförebyggande program. MME-rapport nr 25, Miljömedicinska enheten, Stockholms läns landsting, 1994.
Åberg N, Lundbäck B, Möller C, Åberg B. Tredubbling av astma och allergisk rinit hos värnpliktiga mellan 1971 och 1992. Abstrakt, Svenska Läkaresällskapets Riksstämma, 1995.
Åberg N, Sundell J, Eriksson B, Hesselmar B, Åberg B. Prevalence of allergic diseases in school children in relation to family history, upper respiratory infections, and residential characteristics. Allergy, 1996, under tryckning.

* The high level of health protection should be maintained, despite cutbacks in the private and public sectors, to avoid a deterioration of public health which would require even more costly measures.
* International cooperation is essential in order to reduce transboundary pollution, take action to combat depletion of the ozone layer and adopt rules, including trade rules, that are conducive to health protection, and such cooperation requires expert knowledge of high quality.
* Producers should take the necessary measures to prevent health risks in connection with the production process and throughout the life cycles of their plants and the goods they produce.
* Health impact assessments should be carried out prior to strategic political decisions.
* Research and risk analysis should be given higher priority, in particular as regards interaction between environmental factors, lifestyles, and work and living conditions and the long-term impact on health of persistent substances.
* Local and regional environmental and health workers need access to national sources of information and support for networks.
* Since environmental health risks are the responsibility of several sectors, supervision, research and educational activities should be pluralistic, while at the same time coordinated, efficient and effective.
* Better use should be made, through closer regional cooperation, of the resources allocated to environmental medicine.
* Environmental health monitoring should be improved and used for the purpose of continuous analyses, in particular, of implementation of the proposed action plan.
In addition to these guidelines, a number of objectives and indicative measures are proposed with a view to reducing health problems caused by air pollution, indoor environments, contaminated drinking water and food, noise, radiation and injuries due to accidents.
The general health objectives are formulated with a view to eventual elimination of all health problems that can be prevented. The environmental quality objectives are expressed

Avgränsningen mot arbetsmiljön motiveras med att denna är välutredd samt att tillsyn, forskning och utveckling inom arbetsmiljöområdet är väl samlat inom Arbetsmarknadsdepartementet. I de fall en arbetsmiljö sammanfaller med offentlig miljö, t.ex. i skolor och daghem, omfattas dock dessa av utredningen.
Avgränsningen mot livsstilsfrågor motiveras av att dessa faktorer vanligtvis inte räknas till den fysiska miljön. Bruk av alkohol, tobak och narkotika bör främst ses som sociala frågor. När riskerna i den fysiska miljön samverkar med livsstilsfrågor beaktas dock dessa kopplingar i utredningen, t.ex. mellan radon och tobaksrökning. Även kosmetiska och hygieniska produkter har betraktats som livsstilsfaktorer och har därför inte behandlats.
Frågor som rör läkemedel och medicinska produkter har inte medtagits på grund av att dessa är välkontrollerade och vanligtvis inte räknas som fysiska miljöfaktorer.
Sociala och psykologiska frågor har inte behandlats eftersom utredningen annars blivit en folkhälsoutredning, vilket inte varit avsikten. Psykosociala effekter av störningar i den fysiska miljön ingår i utredningen, men inte frågor som rör t.ex. arbetslöshet, sociala nätverk eller mobbing.
Avsikten med detta handlingsprogram har således varit att ge en sammanhållen bild av mål och åtgärdsinriktningar för att förbättra den miljörelaterade hälsan men inte att föreskriva detaljplaner för att nå varje mål. Sådana detaljplaner finns för vissa områden och för andra kan de utarbetas av berörda sektorer och myndigheter men det har inte varit möjligt inom utredningens ram, som koncentrerats på helheten.
\subsection*{2.2 Hålsorelaterade miljömål och åtgärdstyper}
"I strävan mot en hållbar utveckling står människan i centrum. Hon har rätt till ett hälsosamt och rikt liv i samklang med naturen."
(Riodeklarationens första paragraf)
Riodeklarationen och Agenda 21 antogs vid FN:s konferens om miljö och utveckling i Rio de Janeiro 1992 där 150 statschefer

ibland vara lämpligt med delmål som bör uppnås inom en överskådlig tid för att kunna utvärdera igångsättandet av förbättringen. Därmed skiljer sig målformuleringarna i ett handlingsprogram från t.ex. organisationers och myndigheters målformuleringar, som i allmänhet är mer verksamhetsinriktade.
Alla mål och åtgärdsinriktningar har diskuterats tämligen ingående i kapitlen 5-8, som i sin tur bygger på faktamaterial från bilagorna. Därutöver finns flera förslag som berörs i huvudbetänkandet men som inte passar i formen för ett framtidsinriktat, nationellt handlingsprogram med internationell användbarhet. Det kan t.ex. gälla åtgärder som redan är beslutade och genomförs, förslag som berör specifika myndigheter m.m. Därför finns fler förslag i kapitlen 5-8 än vad som återfinns i kapitel 9. Skillnaden betingas således av handlingsprogrammets karaktär och inte av utredningens värdering av mer eller mindre betydelsefulla förslag och åtgärder.
I betänkandet har en del åtgärder behandlats på flera olika ställen. Det beror på att vi har tagit upp frågor såväl i samband med olika miljöfaktorer, som i samband med beskrivning av de olika sektorernas betydelse för den miljörelaterade ohälsan. Handlingsprogrammet har, i enlighet med direktiven, baserats på en genomgång och sammanställning av olika miljöfaktorers hälsoeffekter. Det har därför varit naturligt att relatera mål och åtgärdsinriktningar till dessa miljöfaktorer. Ä andra sidan kommer huvuddelen av åtgärderna att genomföras inom olika samhälls- och näringslivssektorer, varför ett avsnitt om angelagna åtgärder inom olika sektorer har inkluderats (kapitel 5.10).
I kapitlen 5-8 föreslås åtgärdsinriktningar och uppdrag som i vissa fall riktas till namngvina myndigheter, men i sektorsavsnittet är förslagen i allmänhet inte riktade till enskilda verksamheter. Många samarbeten försiggår inom branscher av betydelse för de miljörelaterade hälsoriskerna. I det nationella handlingsprogrammet har skrivningarna hållits mer principiellt och angivande av enskilda myndigheter undvikits i största möjliga utsträckning.
I direktiven ingick önskemål om att utföra kostnads- och nyttoanalyser av de miljöåtgärder som behövs för att rädda liv, sänka skadliga exponeringar eller minska antalet drabbade. Vissa åtgärder har kostnadsberäknats av olika myndigheter eller andra institutioner och dessa beräkningar återges i betänkandet och

\title{
Till statsrådet och chefen för Socialdepartementet
}
\begin{abstract}
Regeringen beslutade i april 1995 att tillsätta en särskild utredare med uppdrag att utarbeta ett handlingsprogram för att minska miljörelaterade hälisoriker i Sverige. Utredningens övergripande syfte är att identifiera de miljöproblem som är en risk för hälsan samt att ta fram ett nationellt handlingsprogram för att minska hälisorikerna i miljön. Utredarens uppdrag preciseras i kommittédirektivet, Dir 1995:68 (se bilaga).
Utredare har varit professor Christer Hogstedt, överdirektör vid Arbetslivsinstitutet och överläkare vid Stockholms läns landsting. Huvudsekreterare har varit docent Katarina Victorin, Institutet för miljömedicin. Övriga sekreterare på halvtid har varit sektionschef Martin Eriksson, Socialstyrelsen, och dr med. vet. Titus Kyrklund, Naturvårdsverket. Utredningens sekretariat har under några månader biträtts av fil. lic. Ann Mari Skorpen för framtagande av underlagsmaterial, fil. kand. Sara Victorin för språkbehandling och redigering och Barbro Nilsson som assistent. Översättningen av sammanfattningen har gjorts av Robert Crofts.
\end{abstract}
Utredningen består av detta huvudbetänkande (SOU 1996:124) och två bilagor: Miljörelaterade hälorsiker och Aktörer och verktyg i miljöhälsoarbetet. Huvudbetänkandets faktamaterial bygger på underlagsmaterialet i bilagorna där många utöver utredaren och sekretariatet bidragit till texten vilket framgår av förord i dessa bilagor. En förkortad version av huvudbetänkandet kommer att publiceras på engelska.
Utredningen har biträtts av en expertgrupp med följande ledamöter: