\title{
Information om handlingen
}
\author{
Handlingtyp: Rutin \\ Gäller för: Västra Götalandsregionen \\ Innehållsansvar: Ingemar Qvarfordt, (ingqv), Universitetssjukhusö \\ Granskad av: Ingemar Qvarfordt, (ingqv), Universitetssjukhus ö \\ Godkänd av: Ingemar Qvarfordt, (ingqv), Universitetssjukhusö \\ Dokument-ID: SSN11800-2140136717-247 \\ Version: 1.0
}
Giltig från: 2023-03-22
Giltig till: 2025-03-22

\title{
Åtgärder vid fall av legionella
}
Utredning av smittkälla vid fall av legionella sker enligt flera olika lagstiftningar och flera olika aktörer behöver involveras. Samverkan mellan dessa är nödvändig, varför en utredningsgrupp rekommenderas. Exempel på aktörer som bör ingå; verksamhetsföreträdare, representant för Hälso- och sjukvården i kommunen, Medicinskt ansvarig sjuksköterska, Smittskydd Västra Götaland, VVS-ansvarig för fastigheten, Miljö- och hälsoskyddsinspektör från aktuell kommun och Vårdhygien m.fl.
\section*{Förebyggande åtgärder - verksamhetens ansvar}
- Vattentemperaturen på utgående varmvatten från varmvattenberedaren ska vara minst \(60^{\circ} \mathrm{C}\) och vattnet vid tappstället ska vara minst \(50^{\circ} \mathrm{C}\) efter 1 minuts spolning.
- Tappställen som inte används ska spolas igenom en gång per vecka. Spola igenom alla kranar och duschar enligt följande:
1. Spola kallvatten i 1 min, känn med handen att det blir kallt.
2. Spola varmvatten i 1 min, känn med handen att det blir varmt.
Vid spolning i dusch låt dörren stå öppen. Ätgärden bör dokumenteras och signeras i checklista.
- Vid misstanke om felaktig kall- eller varmvattentemperatur, eller om vattentrycket upplevs för lågt bör tappningsstationen felanmälas till fastighetsansvarig och inte användas i väntan på åtgärd.
- Kranar och andra tappställen som inte används ska tas bort.
- Duschslang ska hängas på översta duschhållaren för att tömma slangen på vatten, samt inte vara längre än 1,5 m. Duschslangen bör förses med snabbkoppling (nitokoppling). Koppla bort duschslangen från blandaren efter varje användning, låt den hänga rakt ner utan att nudda golvet.
- Duschslang ska vara ljustät och strilmunstycke ha stora hål.
\section*{Andningshjälpmedel}
Rengöring av innerdelar till trachealkanyler, nebuliseringskoppar och liknande ska avslutas med sköljning i sterilt vatten alternativt natriumklorid. Se Rengöring av andningshjälpmedel i hemmet under rubrik "Åtgärder för att förhindra tillväxt av bakterien legionella i flergångsprodukter".
\section*{Övrigt}
Luftfuktare, vattenautomater, bubbelbad och ismaskiner rekommenderas inte.

\title{
Legionella
}
Regional rekommendation för kommunal vård och omsorg som komplement till Vårdhandboken. Framtagen av Vårdhygiens nätverksgrupp för kommunal vård och omsorg i Västra Götaland.
\section*{Förändringar sedan föregående version}
Inga förändringar.
\section*{Syfte}
Förhindra spridning av legionella inom kommunal vård och omsorg samt tydliggöra förebyggande åtgärder för att förhindra smittspridning. Det är ett komplement till Vårdhandbokens mer omfattande avsnitt om Legionella.
\section*{Bakgrund}
Legionella pneumophila är en allmänt förekommande bakterie i jord och vatten.
Legionella kan orsaka allvarlig lunginflammation (s.k. legionärssjuka).
Inkubationstiden är 2-10 dygn, vanligast 5-6 dygn. Legionella är anmälnings- och smittspärningspliktig enlig smittskyddslagen.
Legionella förökar sig lättast i stillastående vatten vid temperaturer mellan 20-45 \({ }^{\circ} \mathrm{C}\). Tillväxt sker framför allt i den biofilm som bildas i vattenledningsrör och duschslangar. För att undvika legionellatillväxt är det viktigt med bra flöde och rätt temperatur i vattensystemets alla delar. Se förebyggande åtgärder nedan.
\section*{Riskgrupper}
Personer med nedsatt immunförsvar, t.ex. pga hög ålder, rökning, immunnedsättande medicinering, kroniska sjukdomar såsom diabetes, KOL, har ökad risk att insjukna.
\section*{Åtgärder}
Vårdtagarnära och vårdrelaterat arbete
- Patienter med legionellainfektion kan vårdas tillsammans med andra.
- Arbeta efter Grundläggande vårdhygieniska rutiner

\title{
Kunskapsunderlag
}
Legionella i miljön - en kunskapssammanställning om hantering av smittrisker Folkhälsomyndigheten
BOV - Byggenskap och Vårdhygien 3.dje utgåvan. sid. 26, Svensk Förening för Vårdhygien- SFVH
\section*{Uppföljning, utvärdering och revision}
Vårdhygiens nätverksgrupp i Västra Götaland ansvarar för uppföljning, utvärdering och revision av denna rutin.