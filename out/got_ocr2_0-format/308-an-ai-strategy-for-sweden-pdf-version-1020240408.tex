\title{
Skills \& Talent | The demand for Al skills is growing
}
The demand for Al skills \& talent is growing, and a highly skilled workforce is a critical long-term need of any Al ecosystem. Frontrunner nations, regions, and companies have for many years made, and continue to make, significant efforts to attract, develop and train talent to build and use Al.
The issue of skills \& talent in an ecosystem is enormous, and includes education of future workforce on many levels, training of existing workforce (re-/upskilling), attracting and retaining talent to the field of Al, and attracting international talent to Sweden. The responsibilities for the various aspects are highly distributed which needs to be considered when setting a strategy for how to address the needs.
It's equally important to not only focus on technical skills and talent to drive Al adoption and initiatives. Al is a cross-disciplinary field requiring skills as well in change management, interaction design, legal, business models, communication, innovation management, and many more context-dependent skills.
\section*{Collaboration | Al has unique properties requiring collaboration in ecosystems}
Unlike many technologies, the deployment and adoption of Al technology and solutions come with value creation and synergies across all sectors of society. Moreover, very few organizations/companies, or even nations, can build enough capability and competence themselves. Extensive collaboration is therefore essential and Sweden should recognize the need to invest in the building of Al ecosystems by facilitating and incentivizing collaboration.
For Sweden especially, with the high decentralization of the public sector (e.g. 21 regions, 290 municipalities, 350+ government agencies), collaboration becomes even more important as very few organizations individually are capable of building and managing necessary infrastructure, data, and solutions. To really scale the benefits of Al across both organizations and the country, significant activities and investments compensation for the decentralization must be put in place.
"With new opportunities, risks, and threats to prosperity and security at stake, the promise and peril associated with this foundational technology are too vast for any single actor to manage alone. As a result, cooperation is inherently needed to equally mitigate international security risks, as well as to capitalize on the technology's potential to transform enterprise functions, mission support, and operations."
From: An Artificial Intelligence Strategy for NATO
"The research community, industry, and Government play important, interconnected roles to enable the development and deployment of Al solutions. It is critical for Singapore to enhance partnerships across these stakeholders to strengthen our collective capabilities and drive our overall national Al effort." From: Singapore's Al Strategy.

\title{
Availability | General AI technology is widely available, while data is often proprietary
}
Mature and powerful AI technologies are already today freely available or accessible as readyto-be-used services and tools. Although generative AI is used to substantially improve the quality and productivity across organizations and processes, the potential value of available technology is far from fully realized.
While AI technology is typically openly available, data held by companies and organizations is often proprietary and constitutes untapped value and opportunity for strengthened competitiveness and societal solutions. Hence, a strategy aiming at accelerating the use of AI broadly should prioritize investments driving adoption and value-creation based on such proprietary data.
In academia and research institutes, there is also an untapped potential for adopting existing AI technologies and accelerating and enhancing research in various fields (e.g. medicine, biology, materials, etc).
As the adoption of AI happens broadly, research related to the consequences on society becomes increasingly important. Such research could include research on the economic potential, societal value, consequences on the labor market, and the safety of AI systems.
\section*{Leadership | Broad Al adoption requires bold leadership}
Determined and knowledgeable leadership understanding the potential value and risks of AI, setting direction, and making concrete investments is critical for any nation, organization, or company that wants to broadly benefit from the opportunities of AI. Lack of leadership leads to missed opportunities and increased risks of AI being used in ways not aligned with responsible AI principles, not beneficial for society, or misaligned with other objectives.
Further, the competence of an organization will determine how well and sustainably AI solutions will be built and deployed. Leaders understanding the potential of AI therefore immediately and forcefully strive to increase knowledge level and competence broadly among its leaders and workforce in a purposeful way on a strategic, tactical, and operational level.
Lastly, investing in capabilities and competence is essential but not enough. Without a shift in mindset, those investments will not render expected returns, what took us here is not what will lead us into the future. Leaders need to have an ambidextrous approach, both working strategically and visionary to take the nation and organization on a long-term journey while simultaneously creating value and solving problems using currently available AI solutions. This needs to be communicated and embodied by executive leaders and decision-makers.

\title{
More reading and background
}
\section*{GLOBAL RANKING}
According to the Global Al Index Sweden ranks number 17 in total, and as follows in the seven categories:
- Talent - 15
- Infrastructure - 21
- Operating Environment - 2
- Research - 13
- Development - 17
- Government Strategy - 44
- Commercial - 16
What stands out is the lack of Government Strategy.
Tortoise Media's Global Al Index is one of the most established global benchmarks for countries and encompasses several important metrics and data.
Sweden is currently ranked 17 out of 62 countries, most notable is that we rank 44th in the category "Governmental Strategy". Looking at the metrics for Government Strategy, the following conclusions can be drawn:
- Developing and adopting a national strategy will significantly increase our position
- The strategy should have dedicated funding, include elements of reskilling and up-skilling
- Involvement by senior government members or ministers gives increasingly high points
This comparative analysis by Brookings into national Al strategies and capabilities also complements the Global Al Index and is worth reading.

\title{
Invest in strategic knowledge, technology, and solutions shared by many
}
The development of strategic and important knowledge, technology, and solutions for sovereignty, independence, and scaled adoption should be incentivized and built-in broad collaborations, and made available as widely as possible. Strategic knowledge, technology, and solutions (and other relevant resources) must evolve and should include national computational infrastructure, foundational models, privacy-preserving methods, access to data, data-sharing mechanisms and structures, and AI adoption tools for leaders.
AI Sweden (among others) has demonstrated that the willingness among international and Swedish companies and organizations to jointly invest in knowledge, technology, and solutions enabling the adoption of AI is significant. When incentivized by government funding and structured on a truly collaborative basis with strong sharing mechanisms, the return on investment can be significant.
Furthermore, to fully benefit from AI, strategic investments addressing specific sectors' underlying prerequisites are necessary. This specifically applies to the public sector, where investments in digitalization, data platforms, and data governance are still required.
\section*{Ensure responsible use and good governance}
By building on the Swedish values centered around democracy, inclusion, transparency, and diversity, Sweden has the opportunity to scale and govern the use of AI in a responsible way. AI systems should be designed, operated, and governed in a manner that respects laws, human rights, values, and cultural diversity. This includes mitigating risks such as bias and discrimination, as well as risks related to security and safety.
As a principle, responsible use of AI should be approached and ensured by active experimentation and exploration while developing AI solutions in practice, and by fully recognizing the need for diversity and inclusion of stakeholders, domain experts, end users, and AI developers in all parts of the process.
Moreover, leaders in any organization should ensure that employees have access to, and proficiency in, available modern tools independent of for example their age, native language, disabilities, educational background, or where they live in Sweden. Related to AI, this extends to providing prerequisites for everyone to use AI responsibly and minimize the risks that groups of people are left behind.
Responsible use of AI also means that we have a responsibility to use AI where appropriate and to address our time's most pressing global challenges.

\title{
How can Al Sweden support?
}
Launching this first version of An Al Strategy for Sweden, Al Sweden is committed to taking responsibility and contributing to the journey to a defined and adopted national strategy with clear priorities, investments, and allocated responsibilities, learn more here!
For questions about the strategy as such, reach out to Martin Svensson or Mikael Ljungblom.
Sign up to receive news about the strategy.
\section*{About Al Sweden}
Al Sweden is the national center for applied Al and brings together I20+ partners across the public and private sectors as well as academia. Al Sweden is funded by Vinnova, regions and municipalities, and its partners. Together, we invest in generating tools and resources to accelerate the use of Al for the benefit of our society, our competitiveness, and everyone living in Sweden.

\title{
Vision - a desired future state
}
A future Swedish society making the most of the potential of Al should draw on existing strengths centered around democracy, social stability, a highly developed welfare state, high levels of equality, and a business environment that fosters innovation and entrepreneurship.
Based on who we are as a country, our values, and our strengths, the future Sweden seizes the opportunity of Al to solve key societal and business challenges for the benefit of a prosperous society characterized by democratic values and high quality of life. This is powered by a robust, highly attractive, and ambitious Swedish Al ecosystem that has a positive global impact on people and the planet.

The Vision can also be expressed as Sweden striving to become a worldwide benchmark for how \(\mathrm{Al}\) is embedded nationally and generates sustainable value throughout businesses and society with a positive impact globally.
As an overall indicator of immediate progress, Sweden should aim to be ranked among the top 10 countries in the Global Al Index (current ranking is 17) by 2025, at the latest.
A broader framework for benchmarking and measurement based on the Global Al Index and this Strategy is suggested further down.

\title{
Roadmap ahead | Open invitation to act
}
An Al Strategy for Sweden is Al Sweden's suggested common ground for continued discussion and action. So what does the journey to a defined and adopted national strategy with clear priorities, investments, and allocated responsibilities look like?
As it addresses a vast number of leaders throughout business and society, this journey is both highly important and challenging.
Therefore, let's agree that it is our actual ability to take action that determines the progress, not the choice of words. It is our firm belief that bold actions guided by this strategy will propel us towards the suggested Vision.
Passivity is coupled with great risks in terms of diminished competitiveness, less tax revenue, highly qualified jobs being created elsewhere, an overburdened public sector, and an overall significantly less prosperous society.
On the other hand, if we succeed in taking action throughout society, we are looking at vast opportunities in attracting talent and foreign investments, increasing tax revenues, creating highly skilled jobs, providing excellent service and welfare for our inhabitants, and creating a key foundation for a prosperous future.
We think that a few key questions will further aid us in making priorities, taking concrete action, and distributing responsibilities.
- As a country or as an organization, what are our priorities and what is the potential value we could generate?
- What investments are necessary at what level to address those priorities and realize the value?
- What leadership, structures for collaboration, and incentives would enable the public sector to collectively invest in solutions for the future?
- What needs to happen to allow Swedish healthcare to build life-saving solutions for patients and solutions to increase the healthcare system's productivity?
- What policies do we need to make it easier to attract international talent and investments?
- How does Sweden identify, manage, and future-proof its top priorities and distribute the benefits? Does the government need a dedicated long-term national task force?
Again, it's worth emphasizing that Sweden's success requires bold leadership, broad collaboration, and significant investments. Individual organizations or companies might prevail on their own, but for a prosperous Sweden, no entity, or the government for that matter can carry the responsibility alone.
The above questions are formulated for the national level, the government being a key enabler, and similar questions will be equally important for a hospital, municipality, or company management to answer and act upon.

\title{
The heart of the strategy \& Key principles
}
As set out in the Executive summary, An Al Strategy for Sweden encourages Sweden to recognize the necessity of Al and apply a fully adoption-centric strategy that ensures significant and broad value across sectors, contributes to a prosperous democratic society, strengthens our national security, and enables Sweden to have a positive global impact while navigating potential risks.
The heart of this adoption-centric strategy should be bold leadership focusing on value-creation, cross-sector collaboration, and collective investments, fueled by international relations and strong frontrunners in both the private and public sectors boldly leading the way.
To guide politicians, business leaders, decisionmakers, civil servants, and changemakers, the strategy includes a mindset that should permeate, guide, and align actions, investments, and initiatives associated with this strategy.
The mindset is set out in the form of the following Key Principles.
Relevant actions, investments, and objectives should be based on and mapped against these principles for Sweden to quickly see positive effects and over time maximize the long-term impact throughout the ecosystem.
The Key Principles should apply generally but might need adjustment to fit a specific context or leadership role.
\section*{Enable the best to perform and incentivize them to share generously}
Ambitious frontrunners, i.e. the early adopters and leading companies and organizations investing and demonstrating real value by \(\mathrm{Al}\), already exist in both private and public sectors.
Such frontrunners should be recognized, enabled, and incentivized to move even faster and generate more value given that they share generously with others in the ecosystem. This would result in Sweden quickly reinforcing a pay-it-forward culture encouraging responsibility, investments, talent attraction, and collaboration, as well as contributing to making Sweden a global example of how to create substantial value through Al adoption.
This strategy's focus on frontrunners is based on the trend that breakthroughs typically appear in companies and sectors where Al is applied to real problems, often in service and product development.

\title{
Executive Summary
}
Sweden should act now and apply a bold strategy aligned with democratic values, incentivizing substantial investment by companies and organizations and aiming at generating and spreading the benefits across sectors and society.
An Al Strategy for Sweden, therefore, encourages Sweden to recognize the necessity of Al and apply a fully adoption-centric strategy that ensures significant and broad value across sectors, contributes to a prosperous democratic society, strengthens our national security, and enables Sweden to have a positive global impact while navigating potential risks.
The heart of such a strategy should be bold leadership focusing on value, cross-sector collaboration, and collective investments, fueled by international relations and strong frontrunners in both the private and public sectors boldly leading the way.
An Al Strategy for Sweden has been written to guide politicians, business leaders, decisionmakers, civil servants, or changemakers with ambition and responsibility to lead and develop Sweden, the Swedish ecosystem, or individual organizations and companies towards a positive future through one of the most transformative times in history.
This first version of An Al Strategy for Sweden is intended to catalyze actions by leaders in national and regional governments, government agencies, corporations, and public organizations.
It can not be emphasized enough that Sweden's success in adopting Al requires broad and collective actions from a wide range of companies and organizations. It is not down to the government alone, although the government plays an important role as a key enabler.
This Strategy has been developed by Al Sweden, the national center for applied Al with the mission to Accelerate the use of Al for the benefit of our society, our competitiveness, and everyone living in Sweden, with feedback and input from selected experts in Sweden, Canada, USA and Singapore. The Strategy is also inspired by the strategies of EU, Canada, the USA, the UK, Germany, and Singapore.

\title{
Benchmark and measurement
}
An Al strategy for Sweden points to Sweden's opportunity to use Al to solve key societal and business challenges for the benefit of a prosperous society characterized by democratic values and high quality of life.
As stated in the Vision, Sweden should strive towards becoming a worldwide benchmark for how Al is embedded nationally and generates sustainable value throughout businesses and society with a positive impact globally. What would constitute such a worldwide benchmark that can inspire others to follow?
Ultimately the benchmark must demonstrate that real value is generated broadly throughout society. Measuring only enabling factors, the size of investments, or the progress in some sectors would not be enough.
Existing frameworks (for example the Global Al index ) are mainly focused on enablers for Al such as talent, infrastructure, and innovation (research and development) and to some degree investments. These could serve as indicators of use and adoption, but will not provide a holistic view.
To follow up on Sweden's progress, a framework combining existing well-established indicators and additional indicators to further measure the actual use and value creation is therefore relevant. There are few established metrics for use and value creation and such indicators need to be further developed.
More work is required to develop a pragmatic and detailed framework of indicators, but the following high-level structure of three main pillars is suggested
- Enablers - the status of necessary prerequisites for broad adoption
- Investments - the level of investments of various types
- Use and value creation - quantification of solutions deployed and the estimated value
\section*{ENABLERS}
Fundamental prerequisites for broad adoption in a country, sector, or organization include operating environment, access to talent and competence, infrastructure and other relevant technology, and solutions.
Examples of indicators:
Many of the necessary indicators are covered by the Global Al Index, such as number of graduates, data scientists, diversity of Al workforce, computing capabilities, and more.
In addition, indicators covering the level of cross-sector collaboration, level of international collaboration, access to crucial knowledge, infrastructure and supercomputing capabilities, data sharing and open data sources, and foundational models for development and deployment should be added to the framework.

\title{
Fundamental standpoints \& perspectives
}
This Strategy is based on a strong sense of urgency and an analysis summarized in a set of fundamental standpoints and perspectives that must be recognized by any politician, business leader, decision-maker, civil servant, or changemaker with ambition and responsibility to lead and develop Sweden, the Swedish ecosystem, or individual organizations and companies.
These standpoints and perspectives also underpin the heart of the strategy and the Key principles.
\section*{Competitiveness | The international private sector constantly expands its lead}
The last years have clearly shown that the use of Al at scale and profound breakthroughs happen in the private sector, not least in the USA. Companies with access to relevant data, resources, leadership, and talent constantly increase their lead through a positive reinforcement loop of attracting more talent, generating more value, increasing investments, attracting more talent, and so forth.
The trend is clear, organizations and companies that understand Al the best, also increase their investments the most. As a consequence, the gap between the best and laggards continues to widen. As the value and potency of Al typically scale exponentially, the importance of not falling behind cannot be emphasized enough.
To stay relevant, Sweden as a nation, and Swedish companies and organizations, should engage and learn from the best on all levels.
Reference: HAI Al Index Report 2023
\section*{Speed | The extreme pace of Al development must be recognized and acted upon}
The speed at which Al is being developed is unprecedented, recently demonstrated by generative Al (on language, images, and video). New services have reached hundreds of millions of users and started to change knowledge work in an exceptionally short period of time.
This pace of change can be expected to continue and force companies and organizations to adapt. The pace of development in the private sector, where there is access to real-world data and significant resources, also challenges academic research on Al methodologies and technology to stay relevant.
To stay adaptive and strengthen Sweden's competitiveness, the Swedish Al ecosystem needs to lift to a higher level. This means that new ways of working, new incentives, new funding processes, and new ways of collaborating should be identified and deployed.
As noted in the introduction, even countries that have invested in Al research and talent through research-focused strategies early on(e.g. Canada) now face a crucial choice to invest beyond research and experimentation to enable adoption or lag behind.

\title{
Invest in strategic knowledge, technology, and solutions shared by many
}
To maximize the benefits of AI, Sweden should scale initiatives that catalyze collaboration across all sectors and regions. This should include a strong ambition to compensate for the high decentralization of the public sector through investments in collaboration and sharing between similar public organizations.
Facilitation of collaboration should include incentives for consortiums to build joint advanced solutions, structures, and ways to enable access to real-world data and in general build on mechanisms that allow fast pace and adaptivity.
Sweden should also consider investing in one or two "moonshots", e.g. large-scale initiatives where many actors come together to solve a grand challenge.
REFERENCES TO NATIONAL, EU, NATO AND OECD STRATEGIES
OECD overview of national Al strategies (from 2021)
USA National AI R\&D Strategy.
USA National Security Strategy on AI
USA NATIONAL ARTIFICIAL INTELLIGENCE RESEARCH AND DEVELOPMENT STRATEGIC PLAN, 2023 UPDATE Strengthening and Democratization the U.S. AI Innovation System
Pan-Canadian Al Strategy.
UK National AI Strategy.
Singapore National AI Strategy.
Japan Revised AI Strategy.
Norway National Strategy Al
Denmark National Strategy for Al
Australia Strategy, for Generative Al
NATO Al Strategy.
EU thoughts on AI Strategy.
REFERENCES TO STRATEGIES AND GUIDELINES RELATED TO RESPONSIBLE AI
EU ethics guidelines for trustworthy. Al
UNESCO: recommendation on the ethics of artificial intelligence
OECD: OECD AI Principles overview
EU AI Act Proposal
NATO AI Strategy.

\title{
Engage with and learn from the best
}
Given the pace of development internationally and that many of the objectives are best managed on an international level, Sweden as a nation as well as organizations and companies should actively seek partnerships with like-minded countries, leading companies, academia and research institutes, and non-profit organizations, where there is a ground for mutual learning and development.
Sweden and Swedish companies should especially build relations with leading global companies and ensure access to knowledge, infrastructure, foundational models, services, and products critical for the development and deployment of Al solutions.
International collaboration is often challenging and generating actual, mutual, and sustainable value takes time. Sweden's investments in international collaborations should therefore preferably be pragmatic, focused and strive towards a high density of interactions with selected ecosystems.
Moreover, it should be mentioned that international collaboration must be developed considering the current geopolitical, trade, security, and competitive situation. This can be difficult to manage on an organizational level, and government guidance and facilitation should be provided.
\section*{Substantially invest in skills and talent everywhere in the ecosystem}
Leaders throughout society should invest in upskilling their workforce and attracting new talent to their respective organizations, whether that is within government, private companies, public organizations, ministries, government agencies, or academia.
For Sweden as a nation, dedicated efforts are necessary on many fronts, not least
- Education of future workforce on many levels and in all fields,
- Training of existing workforce (re-/upskilling) including all professional roles relevant to Al adoption (e.g. leaders, changemakers, legal experts, Al practitioners, and domain experts).
- Attracting talent to the field of Al
- Attracting international talent to Sweden
From a governmental perspective, this represents a significant challenge. Clear direction and coordination of the entire educational system should be prioritized as well as investments to scale initiatives quickly increasing the overall competence in the ecosystem.
In parallel, Sweden must make it easier for international talent to discover, arrive, study, work, live, and remain in Sweden, as well as incentivize and enable the private and public sectors to recruit talent internationally.

\title{
More reading and background
}
RELEVANT STRENGTHS AND WEAKNESSES OF SWEDEN
Sweden has some comparative advantages and disadvantages worth considering.
As mentioned in the Vision above, the Swedish values and strengths should be embraced and considered when developing an Al strategy. As a nation, Sweden is known for its social stability, highly developed welfare state, high levels of equality, a skilled workforce, and a business environment that fosters innovation and entrepreneurship. Furthermore, Sweden offers a competitive, open economy with access to new products, technologies, and innovations, a welleducated workforce, and a stable political environment.
Relative to its size, Sweden has many diverse exporting enterprises as well as a rich startup community, which should mean that many companies are exposed to competitive markets and need to be early adopters.
Sweden has also traditionally ranked high on innovation, is internationally recognized in quality management, and has a history of successfully managing change both within organizations and for society at large. The culture of early delegation with a lack of management involvement can on the other side lead to slow Al adoption and poor prioritization.
The public sector is well organized and holds exceptionally valuable data, but is also highly decentralized with many small organizations facing difficulties in establishing relevant leadership, attracting relevant competence, and investing according to the opportunities. Regulatory frameworks like GDPR and public procurement also lead to slower progress. Several other sectors are also lagging for similar reasons, not least the vast number of SMEs.
The education system is divided between humanities and science. The transformation initiated by Al implies the need for more workers skilled in several fields. Sweden has few cross-disciplinary programs and there are few indications that this will change during the coming years.
As an EU member state, Sweden has in practice focused on regulation rather than enabling innovation. The dynamics in the two other relevant continents, North America and Asia are clearly different which has resulted in faster development.
This comparative analysis by Brookings into national Al strategies and capabilities points towards Sweden having weakness in data management and algorithmic management and lacking focus on adoption in both industry and public services.

\title{
More reading and background
}
\section*{AI AND CURRENT INVESTMENTS IN SWEDEN}
Early 2024 the Swedish Al ecosystem is well prepared and Sweden stands a good chance to benefit from Al. The Swedish private sector is highly diverse, has a strong track record of being innovative, and includes several significant frontrunners. The interest in Al and the collaborative spirit in the public sector are quickly growing and good examples of valuable solutions are being deployed. The combination of WASP's investment in academic research and talent, and Al Sweden's dedicated effort to driving adoption across sectors, is from what we can see a unique strength that few countries can match.
However, larger investments in Sweden are necessary. Research and development (R\&D) is a key driver of digital innovation, and according to OECD Going Digital, R\&D expenditure in information industries in Sweden is a lagging indicator.
While the Swedish Innovation Agency, Vinnova, is making important resources available, it is notable that the Swedish Government has not made any new dedicated investments in Al during the last six years, in stark contrast to other leading countries such as Canada, Singapore, the UK, France, and Germany.
For reference, one of the most relevant countries for Sweden to collaborate with, is Canada where between 2022 and 2023, \$2.57 billion was invested in Al research in the country, as total Al investment reached \(\$ 8.64\) billion. Read more here.
Read further on the top 10 Countries Leading in Al Research \& Technology, in 2023

\title{
INVESTMENTS
}
Indicators related to investments are very relevant as they both give a perspective on the belief in the future value to be realized, but also serve as a proxy for the current leadership commitment regardless of level. Metrics should be on government investments as well as investments on a sector level.
Given the importance of collaboration, the framework of indicators should also measure investments in crossover collaboration, and international collaboration as well as investments in knowledge, technology, data, and solutions related to sovereignty, independence, and scaled adoption.
Examples of indicators:
Many but not all of the requirements are covered by the Global Al Index. The level of the government's actual Al spending and targets, the scale, funding, and volume of business investments are partly covered. Indicators to be added are for instance investments by public organizations, international companies' engagement and investments in Sweden, and Swedish organizations' investments in competence transfer and international relations.
\section*{USE AND VALUE CREATION}
An adoption-focused strategy needs indicators of actual use and value creation. Further development is needed to measure the ecosystem at large concerning use and value creation specifically by frontrunners per sector and in total. As value and impact created by Al can be direct or indirect, proxy indicators such as the number of Al models or solutions deployed might be relevant.
Examples of indicators:
Few indicators of use are covered by Global Al index beyond investments and personnel capacity however, one indicator is companies implementing Al as a key part of their business process.
Indicators to be added are for example, public organizations implementing Al as a key part of their operations, the number of deployed Al models among selected organizations (frontrunners), organizations investments in Al deployment, indicators of value creation, and number of employees with access to Al tools.
Going forward, a framework and its indicators need to be further defined, specifically indicators on Use and value creation. Preferably indicators should be developed to match already publicly available data. A first step would be to measure the current state of the Swedish Al ecosystem with regard to the developed framework and set up tracking mechanisms to follow the progress.
As an overall and immediate objective, Sweden should aim to be ranked among the top 10 countries by 2025 at the latest.

\title{
An Al Strategy for Sweden
}
\section*{Using Al is necessary for solving key societal challenges, staying competitive, and in general preserving and developing a democratic society with a high quality of life. The objective of An Al Strategy for Sweden is therefore to develop Sweden's capabilities to strongly benefit from the potential of \(\mathrm{Al}\).}
\begin{abstract}
"Based on who we are, our values, and our strengths, Sweden has a tremendous opportunity to solve key societal, democratic, and business challenges with the use of Al. To get there we should focus on generating value, leading with boldness, scaling collaboration, and investing significantly across sectors."
\end{abstract}
MARTIN SVENSSON
MANAGING DIRECTOR, Al SWEDEN
Like many other countries, Sweden faces urgent grand challenges such as climate change, increasing energy demand, an aging population, and threats to our democracy and security. The consequences of the demographics alone turn the usage of Al and modern technology into a necessity, as the welfare sector otherwise would need a significant share of the newly graduated workforce at the same time as there are limits to tax raises. Today's solutions are simply not the solutions that will build the future of society.
In the private sector, many companies are seeing rapidly increasing international competition and signs of a coming dramatic disruption, often driven by new possibilities enabled by Al. To preserve and strengthen competitiveness, Sweden should adopt policies that spur its private sector and, not least, large exporting corporations to accelerate their investments in Al.
Countries that early on have invested heavily in Al research and talent through research-focused strategies (e.g. Canada) now face a crucial choice to invest beyond research and experimentation to enable adoption or lag behind. While others have to change their strategies, Sweden on the other hand has an opportunity to acknowledge that the benefits from Al come when the technology is put to use in companies and organizations, and implement an adoptionfocused strategy from the beginning.

\title{
Team and contributors
}
An Al Strategy for Sweden has been written by the leadership team and other team members at Al Sweden led by Martin Svensson, Managing Director, and Mikael Ljungblom, Director of Public Policy and International Relations.
While Al Sweden is the sole responsible organization for the strategy, the following contributors have given input and feedback in their individual capacity:
- Elissa Strome, Executive Director, Pan-Canadian Al Strategy, at CIFAR
- José-Marie Griffiths, President at Dakota State University, and one of the commissioners in the National Security Commission on Artificial Intelligence
- Laurence Liew, Director, Al Innovation at Al Singapore
- Birgitta Bergvall-Kåreborn, Vice-Chancellor (Rector) at Luleå University of Technology
- Christian Hedelin, Chief Strategy Officer, SAAB
- Darja Isaksson, Director General of Vinnova, the Swedish Innovation Agency
- Markus Lingman, Chief Strategy Officer, board of Halland hospital group
- Mikael Sjöberg, Secretary General, the Swedish Confederation of Professional Employees
- Peder Blomgren, VP and Head of Data Office, R\&D at AstraZeneca and Chairperson in AI Sweden
Many others have contributed with inspiration and perspectives.