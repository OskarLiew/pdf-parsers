\title{
Covid-19
}
\section*{Ta prov på dig själv - För medarbetare inom kommun, region och samhällskritisk verksamhet}
\author{
Läs igenom hela instruktionen innan du börjar \\ Kontrollera att du har vad du behöver innan du börjar \\ - En större förvaringspåse med zip-lock och QR-kod som innehåller: \\ o Ett provrör med skruvlock, etikett och provtagningspinne \\ - En tallrik eller mugg \\ - En plastpåse att förvara skråp i och en bit hushållspapper \\ Förberedelser innan provtagning \\ 1. Öppna den större förvaringspåsen och ta fram innehållet. Du kan lägga det på bordet framför dig. \\ 2. Ta ut provröret och lägg det på hushållspappret.
 \\ Registrera ditt prov \\ 1. Gå till direkttest.se på din mobil eller dator, välj ”Registrera prov”.
}
\section*{ELLER}
Skanna QR-koden på påsen med din mobilkamera genom att rikta kameran mot QR-koden.
2. Skriv ditt mobilnummer, epost (valfritt) och arbetsplatskod i de avsedda rutorna.
3. Välj vårdcentralen du är listad på i rullisten, hittar du inte din vårdcentral, välj "Annan".
4. Efter att du legitimerat dig med BankID är registreringen klar.
5. Om du fyllt i epostfältet under registreringen skickas ett mail med referensnumret till epostadressen du angett.

\title{
Vanliga frågor och svar
}
F: Jag har inte Bank-ID.
S: Om du saknar Bank-ID hänvisas du till vårdcentral för provtagning.
F: Vad gör jag om svaret är positivt?
S: Läs igenom smittskyddsbladet som finns länkat i ditt positiva provsvar. Därefter kontaktar du vårdcentralen du är listad på, och som du angav vid registreringen, samt din arbetsgivare.
F: Jag är osäker på om jag registrerade provet korrekt.
S: Provet är registrerat när du kommer till sidan ”Ditt prov är registrerat”. Om du av någon anledning är osäker på om det blev rätt kan du logga in under ”Provsvar” på direkttest.se. "Testresultat är ej tillgängligt" betyder att provet är registrerat men att analysen inte är klar. "Det finns inget prov registrerat" betyder att du inte registrerat provet korrekt och det finns därför ingen koppling mellan dig och ditt provrör i systemet.
F: Hur läng tid tar det innan jag får svar?
S: Vanligtvis får du ditt provsvar inom 48 timmar från då du lämnar ifrån dig provet med reservation för sent inlämnade prover som inte hinner med transporten. Då kan det ta upp till 72 timmar.
F: När kan jag börja arbeta igen om provet är negativt?
S: När du är så frisk att du kan arbeta. Kontakta din arbetsgivare när svaret kommer
F: Om jag blir sjukare vad gör jag då?
S: Om du upplever att du behöver stöd av vården för att du är sjuk gör du precis som vanligt. Du kontaktar 1177 eller din ordinarie läkare till exempel på den vårdcentral där du är listad.
F: Vad är biobankning och varför är det bra?
S: Biobankning betyder att provet du skickar in kan frysas ner och sparas för att kunna användas igen. Svenska forskare kan sedan begära tillstånd hos etikmyndigheter för att använda prover för forskning. Utan material donerat av de som provtas kommer forskare i Sverige inte ha möjlighet att utföra viktig fortsatt forskning om covid-19. Av det skälet ber vi att du klickar i rutan som tillåter biobankning när du registrerar ditt prov med QR-kod eller på direkttest.se. Det är frivilligt.

1. Provtagning mot
2. Provtagning i nāsan
3. Spotta och doppa
4. Skruva pā
svalgvägg
Ta prov pā dig sjālv
1. Skruva av locket pā provōret, provtagningspinnen sitter fast i skruvlocket.
2. Fōr den mjuka delen av pinnen in genom munnen och sā lāngt bak i din hals som du kan komma, sā att du nāstan kräks.
3. Gnid runt pinnen mot svalgväggen (lāngst in i halsen) sā länge du kan. Det kommer att kānnas obehagligt, men 5 sekunder räcker. Se skiss.
4. Stoppa sedan fõrsiktigt in samma ände av pinnen rakt in i nāsborren, sā lāngt in som du kan. Vinkla inte pinnen uppāt. Gnid runt pinnen i 10-20 sekunder. Se skiss.
5. Samla mycket spott i munnen och spotta pā talirien eller i muggen, undvik att fā med slem eller snor.
6. Doppa den mjuka delen av pinnen i spottet sā att den blir ordentligt fuktig.
Överför provet till provröret och fõrpacka provröret pā ett säkert sätt
1. Fòr fõrsiktigt tillbaka pinnen i provröret.
2. Skruva pā locket pā provröret ordentligt.
3. Lägg tillbaka röret i fõrvaringspäsen med QR-kod.
4. Stäng pāsen ordentligt.
5. Fõrvara pāsen med provröret i ett kylskäp, tills du kan lämna in det eller fār det hämtat.
Hantera avfallet pā ett säkert sätt
1. Lägg hushāllspappret i en vanlig plastpāse.
2. Knyt ihop plastpāsen ordentligt och släng som vanliga hushāllssopor.
3. Tallrik och mugg kan diskas pā vanligt vis.
Du fār ett meddelande till din mobil när provsvaret är färdigt
När provsvaret är klart fār du ett sms. Det sker vanligtvis inom 48 timmar från dā du lämnar in röret men kan dröja nāgot i samband med helgdagar. Logga in med BankID pā webbplatsen direkttest.se. för att läsa svaret.
Se separat papper för vanliga frågor och svar.