<|im_end|>};

När en offentlig aktör ska använda sig av en molntjänst kan det medföra att uppgifter som omfattas av sekretess behöver lämnas ut till leverantören av molntjänsten. Foto: Lantmäteriet, Nataja Kamenjäsrevic.
Förtydliga möjligheterna för offentliga aktörer att använda molntjänster
En särskild fråga när det gäller användningen av AI rörligången till molntjänster. För många företag och myndigheter är tillgången till denna tjänst en förutsättning för att lyckas med sin digitalisering och användning av AI. Tillgång till molntjänster gör det möjligt att användA al-Verktyg, vilka inte är möjliga att använda utan att data försver till motinel. Det finns även fler fördelar med att använda molnabsaserade lösningar så somat ett kediterna vid framtagande av mya Al- tjänster kan förkortas men även att kontinuiteten och beredskapken kan förbättras.
För offentliga aktörer finns det dock ned ran olika rättsliga växrigheter, och libland hinder, mot användningen av denna typ av tjänster. Eftersom molntjänstererna ofta tillhandahålls av en aktör antfurför Sverige, och även antfutar EJ, kan GDPR innebära ett hinder mot att använda tjänsten. Mot den bakgrundner han det nylinje förelagsit is det tidigare nämnda Draghi-rapporten att EJtör under lätretta för europeiska aktörter offentå vandås sig av molntjänster i exempelvis USA, det länd som dominerar marknaden. \({ }^{[59]}\)
När en offentlig aktör ska använda sig av en molntjänst ken dat medfödra att uppgifter som omfattas av sekretess behöver lämnas att vill leverantören av molntjänsten. Om en uppgiftom attpattas av sekretess år det inte tillätet att lämna ut den om inte en sekretessbrytande bestämmelse är tillämplig. I OSL (10 kap. 2 a §) finns det sedan 2023 en sekretessbrytande bestämmelse för enbart teknisk berbetning eller lagring, om det med hänsynt till omständigheterna inte är ölmpligt att uppgiften lämnas ut. Det är emellertid osset omhren dena bestämmelse gör det möjligt för offentliga aktörer att föra över sekretessbelada uppgifter till en molntjänst vid behov. Detta eftersom en molntjänst kan innefatta för member ann ektnisk berbetning eller lagring, såsom olika typer av analyser med hjälp av AI.
För offentliga aktörer är emellertid användandet av molntjänster inte bara en fråga om juridik, utan också om lämplighet. bör svenska myndigheter lämna kontrollen över uppgifter i den verksamket som vi benämner som samhällsbärande till privata företag eller andra länder? Till detta kommer olika säkehretsrelaterade aspekter. Som exempel kan vi nämna en ökad allmän sårbarhet, ökade riserker att obtenhöriga fält länggång till data samt svårlagteret säkerhetrsproduvå personal och uppträtå rättvändsare risk- och sårbarhetsanalyser. Al-kommissionen ansern att denna osäkerhet är olycklig. \({ }^{}\)
- Vi föreslår att tillämnningen av bestämmelsen i 10 kap. 2 a § OSLB för beddass samt att det bör förtydligas under vilka omständigheter offentliga aktörer kan och bör använda sig av molntjänster i sin verksamhet. \({ }^{}\)
\footnotetext{
[57] The future of European competitiveness. Part B in-depth analysis and recommendations (september 2024), s. 77 ff.
[58] See cokal en disikussion i kaplett Al och samhälletå säkerhet.
[59] I kaplett Al för en offentligt saktor i framkatt lämnas vidare förslag på en gemensam digital kärminfrastrutur (Al-verkstadt) vilkan även föresläs innefatta molntörmigar.
}

\title{
Al för
}

Omslagsbild: Kentaroo Tryman/Johnér

\title{
Figur 3: Antal ansökta patent per miljon invånare
}
IT-metoder för ledning

Telekommunikation

Digital kommunikation

\title{
Bilaga B KPI:er för uppföljning
}
\author{
Allt policyarbete syftar till någon sorts måluppfli \\ leise. Målen kan se olika ut. men ofta har de det \\ gemensamma draget att de inte är så lätta att mätta, \\ till exempel välfärd, som i någon mening är målet för \\ nästan all politik. Al-kommisionsen förslag är inga \\ undantag. Enligt direktiven ska de bidra till att" ... \\ stärka utveckling och användning av artificiali intel- \\ ligens (Al) i Sverige på ett hållbart och säkert sätt" \({ }^{\text {II. }}\) \\ Fört att följa upp utvecklingen, och se i vilken mån vi \\ rör oss mot målet, föreslår vi därför användningen av \\ nyckelindikatorer (Key Performance Indicator, KPIer) \\ framtagna av en oberoende extern part man har kom- \\ petens att bedöma olika ländersjör på Al-om- \\ rådet. Den mest användbara externa bedömningen \\ det avseendet görs i The Global Al Index (GAILI). Där \\ bedöms länders relativa styrka inom Al utifrån en rad \\ olika inductarore lindele i sju områden: Polifisk styr \\ ning, utveckling, Infrastruktur, Forskning, Kommersi- \\ alvisering, Talang och Operativ miljö. GAI är sedan en \\ sammanvängig av samtliga inductaroren inom dessa \\ sju områden. Länderna rankas även separat inom \\ varje område. \\ GAI omfattar för närvarande 83 länder. Dara vanligt \\ förekommande index är Stanford Al Index, som dock \\ främst fokuserar på Al-utveckling i allmänhet utan att \\ prioritera landjämförelser, och The Government Al \\ Readiness index, som har en snärvare inlittning mot \\ offentlig sektor. inget index är perfekt i alla avseenden, \\ men det är Al-kommisionsens bedömning att GAI1 i \\ nuläget är bäst lämpat för uppföljning av vår Färdplan, \\ inte minst eftersom det möjliggör jämförelser mellan \\ länder. Det kan dock inte utselutas att något annat \\ index med tiden kommer att visa sig mer användbart. \\ Det är dessutom inget självändamål att klättra på \\ ranklingistor. Syftet med våra åtgärdsfjärslag är att \\ förbättra de verligha jutsättningarna för utveckling \\ och användning av Al i samhället. \\ De indikatorer som används i GAI skiljer sig är på två \\ druglighagande sätt. Vissa inslåtatorer visra mätt på \\ ett lands absoluta kapacitet, till exempel antalet Al-fö- \\ retag, eller totala Al-investeringar mätt i dollar. Darna \\ inlikatorer är justerade för ett land storlek, exem- \\ pelvis antalet Al-företag per capita eller jämfört med \\ BPN. När GAI presents varger man inhop alla dessa \\ indikatorer och ger varje land en pånag. Poängen är \\ en blandning av ländermas absoluta kapacitet inom \\ Al (som i någrad speglar hurstort landet och dess \\ Al-sektar på och deras storleksjusterade kapacitet på \\ Al-området.
}

<|im_end|>};

\begin{abstract}
Aven I förhållande till GDPR finns det åkål att göra en översyn. Förordningen började tillämpas 2018 vilket innebär att över sex år har gått sedan dess. Foto: Shutterstock
\end{abstract}
dataskyddsregleringen i grunden. Vi anser att myndigethers behandling av personuppgifter bör regleras i en lag i stället för separata registerförfattningar, vilket har medför en fragmentiserad lagstiftning och liche enhetlig tillämpning. Med en samlad reglering skulle tillämpningen kunna blom er enhetlig.
\section*{Förslag}
- Al-kommissionen föreslår att arbetet med att moderniseria befintliga registerfattningar skara fortsätta. Samtidigt anser kommissionen att regeringen bör utreda hur en ramlag för personuppgirtshbehandling skulle kunna utformas i syfte att ersätta de separata registerförfattningarna.
Aven i förhållande till GDPR finns det skål att göra en översyn. Förordningen började tillämpas 1208 vilket innebär att över sex år har gått sedan dess. Efforsom GDPR är ett EU-rättsligt regelverk bör översyngom atta swenska författningar som kompletterar GDPR i vilset av såväl svensk som europeisk rättspraxis, samt nationella författningar i andra medlemsstater. Att GDPR tillämpas olika i EU:s 27 medlemsstater har i den tidigare nämnda Draghi-rapporten identifierats som ett hinder mot Al-utvecklingen. I rapporten rekommenderar Draghi bland annat att reglerna bör förenklas samt att resurser ska läggas på att se till så att de tillämpas på ett harmoniserat vis i hela unionen. \({ }^{[50]}\) Ett budskap som mått Al-kommissionen vid ett stort antal möten med olika aktörer är att den svenska tolkningen av GDPR skulle vara mer restriktiv än andra EU-länders.
Mot den bakgrunden anser Al-kommissionen att en översyn bör innefattan anas y av praxis från swenska domstolar och förvaltningen ogdigheter i syfte att ante för att för att för att för att för att för att heter tolkar förordningen på ett mer restriktiv sätt än vad andra medlemsstater gör. Översynen bör även utreda om det finns möjlighet att genom ändring av svenska nationella föreskrifter kunna göra det mindre komplicerat att dela och använda data som innehåller personuppgifter. Samtidigt är det viktigt att si väk av om ett fortsatt robust skydd för den personliga integristeten. En rimlig balans mellan behovet att tillgängliggöra data och skyddet för den personliga integristeten är därför nedvändig.

\title{
Telekom
}
\section*{ChatGPT sammanfattar:}
\begin{abstract}
Telekomsektorn har en central roll i samhällets digitalisering, och dess betydelse ökar i takt med Al:s framväxt.
\end{abstract}
Idetta kapitel utforskar vi de möjligheter och utmaningar som uppstår när telekom och Al samverkar, särskilt kring realtidsdata och låg latens. Hur snabbt och stabilt vi kan överföra information blir avgörande för framtidens digitala lösningar, från sjukvård till industri. Trots detta har Sverige halkat efter när det gäller investeringar i mobilnät och 5G, vilket riskerar att påverka landets konkurrenskraft och innovation negativt. Kaplitét ger en överblick av telekoms nuvarande roli och dess växande betydelse i en Al-driven värld.
\section*{Vad är telekom?}
Telekommunikation, som även kallas telekom, är det traditionella uttrycket för överföring av information över avstånd med hjälp av tekniska hjälpmedel. Det inkluderar alla typer av överföringar av röst, data och video. Det innefattar olika tekniker och tjänster som stortefoni, förbeportk, sattelter, radio, tv och internet. I lagstiftning och reglering används framför allt bespreppet elektronisk kommunikation. Ofta, därbland \(\mathrm{PA}\) en \(\mathrm{U}-\mathrm{NiN}\), används uttrycket konnektivitet \({ }^{[12]}\) för att beskriva det överprigande policymördet.
Telekoms roll i digitaliseringen och utvecklingen av Al
Med hjälp av telekom är det möjligt att transporera stora mängder information och data mellan människor, maskiner, företag och myndigheter. Ottawa format det om information som ligger till grund för de digitala tjänster vi använder dagligen. Tack verearelekom kan vi till exempel deklarera online, strömma film och musik, handla, upmåg och ta del av nyheter och samhällsinformation på våra datorer och mobiler.
1 takt med att digitaliseringen ökar, har det också skett en explosionsartåd ökning av mängden dataflöden som på olika sätt ska delas mellan oss, oftast retillid för att möta denna utveckling är en snabb och tällortilig uppkoppling nödvändig. Kort sagt är telekom en grundförutsättning för det digitala samhället, och dess betydelse ökar på grund av Al.
För samhället innebär det stora vinster att snabb kunna genomföra realtidsoverförningar av data. Dett gäller inte minst utvecklingen av nya smarta system som ofta förlitar sig på Al-teknik. Itt sådant exempel är färrövervankning av patienter. Med hjälp av så kallad kropensära teknologi, såsom smarttkolcor och medicinska sensorer, är det möjligt att måta vitala tecken som puls och blodtryck, för att sedan dela denna måtdata direkt med en läkare via mobilnätet. Ytterligare ett exempel är att ambulanspersonal med hjälp av mobilnätet kan överföra patientdata under transporten, vilket gör det möjligt för vårdpersonal att förbereda sig innan patientan ändler till sjukhuset.
I framtidan karmen förler digitala lösningar och smarta system var på fandre de av datafiöden i realtid, vilket ställer var på fördröpingin (så kallad /ateny). Detta för att för att för att information att nå slutdestinationsen en ett digitalt varket, till exempel fördröjningen som kan uppstår när stö ska må nottagaren under ett telefonsamt. Refen i<|im_start|> på finet i dag pline det exempel på digitala lösningar som behöver väldigt låg fördröj- ning, såsom fjärsförning av maskiner och robotar, finansiella transaktioner och onlinespel. Takt med att Al-system börjar analysera och reagera på informa function från om mögningen i realtid, kommer behovet av lag förder av förmetet en att öka Yyterligare.
Tytterlare en koppling mellan Al och telekom är att Al-modellerna ofta är för stora för att sprata på datorn. De kräver i stället särskild hårdvara som i sin tur finns på särskilda datacenter. Al- tjänster kräver därför telekommunikation, i synnherhet snabbt och stabilt internet, för att koppla samman dessa datacenter med våra telefoner eller datorer.
Sammantaget kan vi konstatera att användandet av Al kommer att påskynda samhällets digitaliseringer ochsra nabb och stabil uppkoppling avgörande för att hantera vardagvislig och arbetsuppgifter. För företag kommer tillgängt ih hökglavitativ uppkoppling bli en all stättore konkurrensförelch od avgöra var de lokaliser sig Al-tekkinen kommer på så väs att öka internetuppkopplingens betydelse för regional utveckling. Al-lösningar kommer sannolikt också

För det svenska samhället är det potentiella mervärdet av den här typen av samarbete betydande. Dels kan vår tillgång på högkvalitativa data locka hit utländsk Al-kompetens. Dels kan det hjälpa svenska teknologföretag att växa och utvecklas. På lite sikt är det troligt att företagen på egen hand hittar marknadslönsningar, som i exemplemt med Astrazeneca. På kort sikt kan det dock finnas anledning att överväga behovet av ett begränsat offentligt stöd för att stärka den här typen av samarbete mellan företag.
Al har emelertid också potitalen att bidra till att lösa bredare, bransch- och sektoröverskridande samhältsummangar- utmaningar som ingen enskild aktär har möjlighet att lösa på egen hand, och där värdet av lönsingen kommer fler till godo än berörda parter. I dessa fall är det inte troligt att marknadslöns. Sålv löser detta. Det kan handa om att hitta innovasom till att äkletta, att äkletta, att äkletta, att äkletta, att företag, heltet, underlätta klitamtomställningen eller åtgärda bristen på läkare och sujksöterskor. I det arbetet bör aktörer från både privat och offentlig sektor ingå. Det är angelåget att hitta lönsningar som bidrar till att även den har formen av samarbeten kommer till ståndt. Ett visionart exempel på vad som skulle kunna åstadkommas med den här typen av samverkan presenteras i ruta, En vision om en svensk hälsomodell. Då den här typen av projekt sker relativt såilan och då kan kräva relativt mycket pengar är det inte lämpligt att täcka kostnaden med årliga budgetenslag, Det är därför viktigt att hitta alternativa sätt att finansiera denna typ av projekt.
Offentlig sektor har tillsammans med privata aktörer en mycket viktig roll i innovationssamarbetet. Innovas tionfellig sektor kan också gynnas avsevärt genom samarbete inom offentlig sektor. I kapitlet Al för en offentlig sektor i framkant redogörs bland annat för en Al-verkstad i offentlig sektor. I den skulle mycket innovation kunna äga rum. Det finns också goda exempel på framgångsrikt samarbete mellan stora och små aktörer i offentlig sektor. Där kan de mindre dra nytta av de större aktörrenas Al-kompetens och utveckling. Samtidigt kan de större aktörerna använda de mindre som testbäddar för sina enga Al-innovationer. Här finns potential som kan leda till ökad innovations- och omställningsförmåga och främja en jämlik utveckling hos alla involverade parter, oavsett storlek eller geografisk placering. Den föreslagna Al-verkstaden är också käntt att fungera som en plattform där privata företag kan bidra med lönsningar till offentliga aktörers utmaningar.
\section*{Förslag}
- Vinnova bör ges i uppdrag att tillsammans med näringslivet och aktörer från offentlig sektor så snart som möjligt utreda vad som kräva för att få tillstånd större sektörsøverskridande projekt, där värdet av en innovation kommer fler till del än deinblandade parterna. Utredningen bör innehålla en analys av hur staten finansiellt kan stödja dessa projekt, inklusive möjligheten till offentlig-privata partnerskap, givet att projekten inte kan förväntas ske på regelbunden basis, och ofta torde kräva relativt stora belopp. Uppdraget bör också innebära att skapa forum, med representanter för privat och offentlig sektor, för att identifiera möjliga sektörsöverskridande projekt.
- Vinnova bör få uppdraget att tillsammans med näringsliv och offentlig sektor utreda hur möjligheterna kan förbättras för att data ska delas för att främja innovation.

\title{
Foto: Shuterstock
}
\section*{Ena - Sveriges digitala infrastruktur}
Id agfVrolatar Myndigheten för digital förvaltning (Digg) en förvaltningsgemensam digital infrastruktur. Ena - Sveriges digitala infrastruktur, är ett samlingsnamn för olika system, komform för att för att för att för att för att för att för denffertliga förärlingen att dela digital information till medborgare och med andra myndigheter i Sverige och inom EU. It datta sammanhangar åt det dock viktigt at nottera att Ena inte är en gemensam infrastruktur för utveckling och avländer av att för att infentlighat intänter infentlighat infentlighat infentlighat infentlighat infentligha av infentlighat infentlighat infentlighat infentlighats medikning. I infentlighat infentlighat infentlighat infentlighå av infentlighat infentlighat infentlighat infentlighet av infentlighat infentlighat infentlighat infentlichts medikning. I infentlighat infentlighat infentlighats medikning. II infentlighat infentlighat infentlighat infentlighatt
\section*{Al-verkstaden}
För att tillgodose de behov som finns föreslår Al-kommissionen följkaltignat ett and itätrens en gemensam kärnrinfrastruktur för utveckling och leverans av Al-drvina tjänster in omfentlig sektor - en "Al-verkstad" . En sådan verkstad blir en central komponent i ett större ekosystem för hantering av Al. I den skulle offentliga aktörer var för sig chitilSAMmans, och i samspel med näringslivet, kunna utforska och utveckla nya Al-tjänster och funktioner och dela och använda kvalitetssäkrade data, modeller och komponenter. Via kopplingar till andra molninfrastruturer skulle verkstaden få tillgång till nödvändig beräkningskraft utöver den egna. Al-verkstaden skulle kunna föra katalog över de Al-Iösningar och Al-modeller som redan finns framtagna för att främja en återanvändning av andra offentliga aktörers arbete. En viktig uppighat för vekstaden olirt att säkerställa att allt detta sker på ett säkeret hag lenagligt sätt. Att bygga upp och använda en Al-verkstad kommer även att bidra till att bygga viktig kompetens brett i den offentliga sektorn, något som också kommer bidra till att stärka vår civila med för att för att för att för att.
De aktörer som har kapacitet och förmåga kommer att behöva bidra genom att ta fram centrala tjänster och funktioner som också kan användas av andra, och genom att dela med sig av sin kompetens. Även mindre aktörer ska kunna bidra, trots att dessa har begränsade tekniska resurser och kompetens. I kraft av sin lägre organisatoriska komplexitet och mer snabbrörliga styrning kan de spela en viktig roll som plattformar för innovation och testtäbdär för nya Iösningar - något som kan förväntas öka innovations- och omställingsförmågman i offentlig sektor som helhet. På så vis kan ett ömsesidigt utbyte av förmågor åstadokommas till nyta för alla inblandade, och en jämlik utveckling främjas hos alla involverade parter, oavsett storlek eller geografisk plats.
Det finns bara ett fåtal offentliga organisationer som har gjort stora investeringar i att bygga en IT- och Al-finfrastruktur som klarar moderna krav på tillgänglighet och säkerhet, och som dessutom har vana av att tillhandahålla IT-tjänster åt andra aktörer. Föräskingskassan och skatterveket intert här en sårställning. \({ }^{154}\) En uppskaling av dessa myndigheters förmågor skulle leda till en kostnadseffektiv och ändamålsenligösning på de ovan beskrivna infrastrukturutmanningarna samt påtagligt accelerera offentlig verksamhets förmåga att utveckla och använda Al.

\title{
1 Inledning och sammanfattning
}
Utvecklingen av Al, artificiell intelligens, kommer att påverka hela vårt samhälle. Hur det sker beror i hög grad på hur vi själva agerar. Genom att satsa proaktivt och involvera alla grupper i samhället kan vi dra stor nytta av en ökad Al-användning, samtidigt som vi hanterar risker och utmaningar.

\title{
Innehåll
}
1 Inledning och sammanfattning ..... 7
2 En stabil grund att bygga på ..... 21
Energi ..... 22
Telekom ..... 27
Beräkningskraft ..... 30
Data som en förutsättning för Al-utvecklingen ..... 36
Al och samhällets säkerhet ..... 46
Spetsforskning i samverkan ..... 53
Tillgång till internationella Al-resurser ..... 57
3 Al för alla ..... 61
Kompetenslyft för alla ..... 62
Innovation, entreprenörskap och riskkapital ..... 72
Al för en offentlig sektor i framkant ..... 92
4 Ledarskap och styrning ..... 107
Internationella positioner ..... 108
Ledarskap och styrning för att genomföra Färdplanen ..... 116
Bilaga A Förslagslista ..... 120
Bilaga B KPl:er för uppföljning ..... 123

\title{
Al för en offentlig sektor i framkant
}
\section*{ChatGPT sammanfattar:}
Al har potentialen att förvandla Sveriges offentliga sektor och skapa en verksamhet som inte bara är effektivare utan även mer träffsäker, robust och anpassningsbar för att möta framtidens för att förvärsäder.
I detta kapitel uttorskas hur Al kan bidra till att stärka väffärdssystemet, förbättra offentliga tjänster och höja Sveriges internationella konkurenskräft, Här ställs frånan hur den offentliga sektorn genom Al-teknikens hjälp - kan möta de växande kraven från en åldrande befolkning och allt mer individanpassade samhällsbehov. Kapitel kommer att belysa konkreta exempel på Alts bidrag, såsom snabbare handläggningstider och ökad rättsäkerhet, och beskriva den tekniska och strukturella omvandling som kravs för att framgängskrift integrera Al. Vi diskuterar också de juridiska och etiska överväganden som föller med Al-änvändning i myndigheter och offentligt förvaltning. Genom att förvärsäder och ökad rättsäker och ökad rättsäker och ökad rättsäker och ökad var att<|im_start|>amtjälnär. Damerna, digitala och högkvaltätta offentliga samtliga invånare - avosett var de bor - får tillgång till moderna, digitala och högkvaltätta offentliga tjänster. Kapitelt belyser vägen framåt för offentlig sektor och pekar på de centrala investeringar och förändringar som kravs för att Al ska bli ett kraftfullt verktyg för samhällsnytta.
\section*{Det offentliga åtagandet}
Offentlig verksamhet handlar i hög grad om förvaltning - det gäller oavsett om ansavret ligger hos stat, region eller kommun. Men i det offentliga uppdraget ligger också att långsiktigt utveckla samhället och rusta Sverige för framtiden. Givet teknikens utveckling och samtidens samhälleiga utmaningar ställer detta höga krav på såväl omställningsförmåga som utvecklingskapacitet.
\section*{Den offentliga sektorn}
Den offentliga sektorn är en stor och viktig del av den svenska ekonomin. De ansvariga för verksamheten sat raten genom cirka 367 myndigheter under regeringen, 21 regioner och 290 kommuner. \({ }^{[141]}\) Spännivdden mellan aktörerna är stor, både i fråga om storleck och när det gäller bredd och mångfald av ansvarsområden. Verksamheten handlar om till från förlossningsvård, förskola, pensioner och äldreomsorg till rättsväsende, butbildningvässende, infrastruktur och kultur - för att nämna några exempel.
\section*{I Sverige är det offentliga åtagandet mycket omfattande. Det har skapat en grundtrygshet för många människor, inte minst under tider av förändring och perioder då påfrestningarna på vårt samhälle varit stora. Vårt välfärdssystem, i kombination med en anda av samförstånd och samarbete, har haft en mycket stor betydelse för vårt lands utveckling och ekonomiska välstånd. Så vill också de allra flesta människor som bor i Sverige att det ska fortsätta vara.}
Tillsammans utgår de offentliga utgifterna ungefär hätigen av Sveriges BNP, 2841. miljärder kronor, vilket är mycket vid en internationell jämförelse. Av detta går nästan 700 miljärder kronor till upphandlade varor och tjänster. Det innebär att dessa resurser stärker den privata sektorn. Därutöver utförs viss offentlig verksamhet av privata aktörer, till exempel inom skola och äldreomsorg. Vidare arbetarean för stor andale av befolkningen inom offentlig sektor. 2023 var det 1,5 miljoner människor, vilket motsvarar cirka 30 procent är att förvärsäder och äldreomsorg. I ändet: På de allra flesta orter är offentliga arbetsgivande, dessutom den ensister största arbetsgivaren.

\title{
Det är Al-kommissionsens bedömning att universitet och högskolor skyndamt bör införa relevant AI-innehåll alla utbildningar. Foto: Gorodenkoff/Shutterstock
}
\section*{Mer Al i den högre utbildningen}
Behovet av kvalificerad personal med teknisk abkgrund inom Al är stort, såväl vid universitet och högskolor, som i privat och offentlig sektor. I Sverige rader emellertid inte bara stor brist på teknisk Al-kommetsen, utan det behövs också en bredare förståelse för Al bland studenter och lärare vid universitet och högskolor inom alla områden. Det finns få utbildningar i dag som tillgodoser behovet av kompetens med koppling till Al inom till exempel kemi, bologj, juridik, ekonomi, medicin, samhällsvetenskap och humaniora. För att möta denna brist och stärka Svefiges konkurrenskraft krävs därför en ökad integration av Al i all högre utbildning.
Det ar Al-kommissionsens bedömning att universitet och högskolor skyndamt bör införa relevant Al-innehåll alla utbildning och ökommets der sgrundläggande förståelse för vad Al är och ökommetsverkar den specifika utbildningens område, det förståetslede förståelse för hur Al kan användas för att förbättragt det specifika området. Till exempel bör jurister förstås vilka juridiska frågor Al kan resa. Men juristen behöver också förstå hur dne men kan använda Al-tykørg för att bil mer effektivt inom mätnensområdet.
Al-kommissionen väklomnar att regeringen under 2023 gav Universitetskanslersämbetet (UKA) i uppdrag att analysera hur Al kan påverka högskolans utbildningsutbud i relation till arbetsmärknadens framtida kompetensbehov. UKA likfter i sin redovorning av uppdraget bland annat att det finns ett behov av ökat samarbete mellan lärosätena för att utveckla utbudet inom Al (U2023/02126). Här kan man med fördel bygga vidare på det som gjorts inom WASPPED. \({ }^{}\) Där har man till exempel tagit fram en ny kursplan för det nya bredare Al-ämnet, bedrivit forskning och utveckling för att kunna införa Al på alla högskoleutbildningar, samt skalat upp teknisk högskoleutbildning inom Al.
\section*{Förslag}
- Al-kommissionen anser att 750 miljoner kronor totalt under åren 2025-2027 bör avsättas till lärosätena. Syftet är dels att lärarna ska fär möjlighet till kompetensutveckling inom Al, del inte annellet i utbildningarna framtidssåkras. Universitets- och högskolerådet bör få i uppdrag att ansvarra för satsningen och tilförlas 5 miljoner kronor arläglen för det utökade uppdraget.
Kompetensutveckling för anställda och omställning för arbetsökande kontra tekniskkiffen har historiskett sutt medfört att arbetsuppgifter, men även att vissyka ren, i sin helhet försvinner. Typsikt sett har människors muskelkraft ersattas av maskiner och robutor. Den här gången handlar det mer om mänsklig beräknings- eller analysförmåga och hur maskiner, eller per ssectifkt datorprogram, kommer att sköta dessa arbetsuppigtfer snabbare, billigare och med högre valletet.

Vision: En illustration av hur den framtida Al-verkstaden kan användas
I sayfte att göra behovet av en Al-verkstad tydligt
liustrerar u här hur en sådan kan användas.
En Al-verkstad gör det möjligt att snabbt skapa
anpassningar - såväl säkerhets-, funktions- som
verktygsmässigt - utifrån behoven hos de som
ska använda den. Det gäller oavsett om det är för
utveckling av nya Al-tjänster eller för att använda
färdiga sådana.
För de som har viss erfarenhet av att utveckla Al-
drivna tjänster och använda förtrånade modeller
kommer det att gögans kanse helt, eftersom det i Al-
verkstaden finns standardverktyg som användas av de
allra flesta som bedriver någon form av Al-utveckling.
\section*{Exempelfall i (aktör med begränsad egen}
Al-förmåga)
För en litten kommun eller myndighet som vill börja
använda Al-drivna tjänster från Al-verkstaden
kan processen inledas med ett platsbesök av
Al-insatstykran. Dessa erbjuder hjäll med att
nägon, en med att<|im_start|>ande, en med att<|im_start|>ande, en med att
Al-verkstadens andisuntingsstöd bistår sedan med
ansökan om behörighet till de Al-tjänster som önskas
och ger vägledning om hur nasutling kan speåt ett
säkert sätt. Antnulsingsvat skler elektroniskt, och
samband med det ges information om kostnaderna
såkert skleret, en med ett står, en med att
Tillgångligheten underlättas av att det finns en.
katalog med generella Al-tjänster som aktören kan
välja från, Tänkbara tjänster kan vara särka tolknings,
översättnings, transkriberings, masknings- och
digitalia assistenttjänster. Aktören kan efter anslutning
nu säkert använda och drnyta av effekterna av
jänsterna i sin verksamhet.
Exempelfall 2 (aktör med innovativt ides som kräver
välja)
I en region har man tidigare haft Al-insatstykran på
besök och fått hjälp med att identifiera Al-önsingar
för sina behov. Regionen är ansluten till Al-verkstadten
och har börjat använda några av Al-tjänsterna. Nu
behöver man utveckla en tjänst som inte finns.
Medarbetarna tittar på Al-verkstadens webblpats för att teda på om någon annan redan håller på att lösa
samma eller liknande problem, men det gör det inte.
Regionen kontaktar Al-insatstykran för att få hjälp,
insatstyrkan hjälpert till med att öka förståelsen för
vad som krävs för att få till en lönsing - till exempel
genom att rådgöra med IMY och kanske genomföra
en regulatorisk sandlåda för att undersöka om
revelergen sätter några hinder. På vägen störter
medarbetarna på några andra regioner och
myndigheter med liknande problem.
I arbetet dyker det också upp ett antal rättäksut
frågor som man få hjälp att lösa med stöd av
vägledningsteam fram av härhet av tärkut oss
verkstånande om att få fram data från lädre
system. Här erbjuds stöd från Al-verkstaden för att
gemensamt ställa krav på leverantörenna om hur data
ska kunna földa från befintliga system till den nya Al-
tjänsterna.
Meden för utvecklingen av Al-tjänster än
lagd beställs ett verkstadstrus som är virtuellt.
Anslutningsstödet ser till att det finns ett anpassat
rum med tillgångt till grundmodeller och andra verktyg
som är lämpliga för uppgiffen. Nu kan utvecklingen
och realiseringen av den nya Al-drivna tjänsten ske,
antingen i ensam regå eller tillsammasn med andra.
Exempeln illustrerar hur Al-verkstaden kan bli
en katalysator för kunskap om Al och utveckling
av lönsingar som påtagligt kan bidra till Sveriges
verkstånande om härhet av tärkut oss
verkstaden ska aktörer också kunna ta del av
randas lönsningar som kan användas i den dagliga
verksamheten. Al-verkstaden, tillsammans med
Al-insatstykran och vägledningsstöd, kan också
bidra till att på sikt lösa problemet med att en del av
systemen är garnla och att det är svårt att få ut data
från dem.
Al-verkstaden - en väg in
Den finns sen stor efterfrågan på samlad information
om frågor som är användningen av Al och hur
det kommer att påverka samhället. Det kan gälla
tillämpning av GDPR, Al-förordningen eller något
annat relevant regelverk. Var kan man träna och
utveckla Al-modeller i en säker miljö? Hur kan Al
påverka utbildnings- och yrkesval? vilka Al-verktyg
finns och hur kan man lära sig att använda dem? Hur
kan man som kommersiell leverantör bidra till att utveckla Al-önsingar för offentlig sektor? Var finns
lämpliga data som kan användas för att lösa ett
specifikt problem? Det här år frågor som alla har rätt
att enkelt för svår på - privatpersoner, företagare och
aktörer från akademi och offentlig sektor.
Det är därför angeläget att ettablera en tydlig portal för
olika Al-relaterade frågor - en plats dit man kan vända
sig för att hittta svar på sina frågor, eller vägledning om
hur man tarar sig vidare. \({ }^{[161]}\) Det handlar sålunda inte tim

förmår. I syfte att centralisera och effektivisera hanteringen av fråfag gällande Al bör en särskild task force tillsättas vid Regeringskansliets Statsrädsberedning. \({ }^{}\) En sådan task force ska fungera som en brygga mellan politiken och de medarbetare på Regeringskansliet som implementerar och följer upp regeringens Landsstrategi. Al-kommissionen föreslår i övrigt inga förändringår i Regeringskansliets organisation. Vi att det att det att det att det att det att ett att ett att ett att ett att ett att ett att ett att ett att ett att ett atttattning på det arbete Al-kommissionen bedrivit. Genom att ta vid där kommissionens arbete slutar kan det momentum som byggts upp under 2024 med ett mycket stort antal sammhållskontakter tas till vara. Al-kommissionen anser att 35 miljoner kronor årligen för avsätta för denna task force.
Gruppen bör ledas av en tastsekreterare med efterarenhet från arbete i Regeringskansliet. Det är nödvändigt att det inom gruppen finns både generalist- och specialistkompetenser. Specialister rör exempelvis Al-teknologi (motsvarande en Chief Technology Officer), dataförvaltning (motsvarande Data stewardkompetens \({ }^{}\) ), personer med kompetens kring der ujidiska regelverk och de säkerhetsfrågor som är av betydelse för utvecklingen och användningen av Al, samt internationell förhandlingserfarenhet. Efter fem år bör det utvärderas om denna task force ska fortsätt silt arbete, eller om man kan återgå till ett meter normalt läge avseende hanteringen av Al. relaterade frågor inom förvaltningen.
Till denna task force bör det kntyas en tastsekreterargrupp och en interdepartemental arbetsgrupp. Den bör även ges ansvar för att bevaka viktiga frågor i vår omvård, inklusive Inom EU, som rör Al ysite thet åt hålla ansvariga politiker Informerade. Den bör ha regelbundna samråmed mændingsliv, offentlig sektor, akademi och arbetsmarknadens parter för att hålla sig Informerad om den tekniska utvecklingen. Alkommissionens medlemmar kan med fördel användas som ett vettsenskapligt råd i det fortsatta arbetet.
\section*{Regeringen bör årligen följa upp de åtgärder som vittdas i syfte att uppfylla målen för Sveriges Al-strategi. Detta bör odrevisas i anrlig rapport, vilken den task force Al-kommissionen föreslår ska ansvarfar för. För att underlätta uppföljning av implementeringen av åtgärderna bör
uppföljningsindikatorer (Key Performance Indicators) tas fram. I bilaga B ger vi förslag på sådana. I syfte att tydliggöra de offentliga investeringar som görs för att skärka Sveriges Al-förmåga bör dessa Al-utgifter redovisas särskilt under samtliga utgiftsområden i budgetpropositionen.
- Regeringen bör uppmuntra myndigheter att använda Al i sin verksamhet. Detta kan ske på flera olika sätt:
- Ett återrapporteringskrav bör riktas till samtliga myndigheter om hur de arbetar för ansvarfull implementering av Al i sina respektive myndigheter. måför med tidigare uppdrag som gält Agenda 2030, jämställdhet eller kring hemarbete under pandemin. Flera liknande uppdrag har lags tredan av olika departement, men kommissionen vill understryka den signal som ges när samtliga myndigheter får samma uppdrag.
- Regeringen har även möjlighet att triva på enskilda myndigheter som den uppfattar ligger efter i sin Al-utveckling. Det sker genom att ge mer specifika uppdrag om att öka Al-användningen, utreda vilka av myndighetens verksamheter som skulle kunna utföras med stöd av Al och ta fram en Al-strategi eller liknande uppdrag.
- Frågor om Al bör även vara en del av de regelbundna uppföljningssamtal som hålls mellan Regeringskansliet och myndigheter.
- Regioner och kommuner bör, likt det vi föreslår avseende regeringen, uppmuntra och uppmana regionala och kommunala förvaltningar att öka användningen av Al.
Asvlutningsvis, vilten av personligt ledarskap vil store takniskkiften kan inte nog betonas. Al-kommissionen anser därför att regeringen, Regeringskansliet, generaldirektör, defer heim kommuner och regioner, tillsammans med verkställande direktörer och styrelser i privata företag för föregå med gott exempel och skaffa sig nödvändig Al-kompetens under 2025. Det rå av central betydelse att våra besultsfattare förstår vad Al är, samt vilken potential och risker den nya tekniken för med sig. Vi anser också att det inom Regeringskansliet bör sätts upp ett mål om att en viss andel av anställda på varje departement ska Ha-1 Kompetens.
\footnotetext{
111 kommahrantiget kan det notera att Produktörkotskommmissionen nyligen förenslägt att en amordningfunktion (Stattdärdsbendringen ska skapas för priorerande støkostvergipande frågor. Se SOU 2024-29 Goda möjligheter till det var-Livåndt. s. 492.
112 Med Data Steward means här en person med mycket stor kompetens kring dataforvaltning. Kommissionen föreslår även i kapiktet Data som en förutsättning för Al att en Balsa Steward-Kutkinö ska rättas vilket är ett spezutt förslag i forländande till det som namis här.
}

bra. Det här är särskilt viktigt eftersom det räder brist på tempeters inom både IT- och cybersäkerhetsområdet. Genom att öka snabbheten och precisionen vid upptäckt och hantering av hot kan Al också bidra till att minska konsekvenserna av cyberangrepp. \({ }^{}\)
Samtidligt väcker dessa möjligheter att bevira brottslighet av olika slag också befogade frågor om personlig integritet och rätten till privatliv. Oktrisk och bred Al-avändning av det slag som beskrivs ovan bör inte vara tillåten i en rättsstat och liberal demokrati av det slag som vi vill att Sverige ska vara. Den kommer inte heller att vara det, givet Al-förordningens förbud mot användning av Al-system med oaceptabel risk, samt härda reglering av sådana med hög risk. Det här är viktigt för människors långsiktiga förtorende för såväl Al-tekniken i sig, som de offentliga verksamheter som använder den. \({ }^{}\)
\section*{Skrätt forskning om Al och cybersäkerhet}
Mot bakgrund av vad som sagset ovan finns det skäl att satsa på forskning om Al och cybersäkerhet. Liksom ofta är fallet med Al-utvecklingen är det nödvändigt att få till samarbetem mellan det privata, det offentliga och lärosäten. En bra plats för sådan forskning är cybercampus Sverige som invigdes i fberauri 2024 och finns på Kungliga Tekniska högskolan (KTH). Cybercampus är ett sverst konsnolent initiality av unvergåttet för att hänkölskor samt privat och offentliga samverkraft. En för att hänkölskor samt privat och offentlig som inom cybersäkerhet och cyberförsvar. På KTH finns också Centrum för cyberförsvar och informationsätset herket (CDIS) som fokusera för forskning i syfte att skära Sveriges försvarsförmåga. Vid CDIS vollitsad även Sveriges cybersolderat, CDIS grundades 2020 som ett samarbete mellan KTH och Försvarsmatten. I dag deltar även Totalförsvarets forskningsinstitution (FOI), Myndigheten för samhällskysdd och beredskap (MSB), Försvarets radioanstalt (FRA) och Forsvarshökskolan i dess verksamhet. Det finns också planer på att integrera CDIS i Cybercampus Sverige.
När det gäller specifika områden för forskning anserppl Al-kommissionen att mer forskning om hur Al kan användas för att stärka cybersäkerheten är av stort interesse. Det kan exempelvis haldla om kodgrankning och andra medeter för att upptäcka särbacher i programvara eller automatiserad penetrations-
testing och etisk hackning. Al kan även hjälpa oss det kraftbäta säkerheten i större system på ett bättet sätt än vad som tidigare varitoljiget. Dessutom kan kuskapen om automatiserad incidenthantering stäkas, det vill såga hur vi kan träna Al att försvvara datmonärtverk mot angrepp. Stora språkbmodeller kan
till exempel användas för att utveckla angreppskod och för att utnyttja sårbarheter hos system. Det är därför motiverat att vi lär oss mer om hur en potentiell angripare kan använda den senaste tekniken för att på så vis kunna anpassa cybersäkerheten ringk viktiga system. I detta arbete är det naturligt att etablera och bygga vidare på befintliga samarbeten inom såväl Norden, EU som Nato.
\section*{Förslag}
- Al-kommissionen anser att det finns skäl att ytterligare stärka forskningen kring Al och cybersäkerhet under en huvudman, KTH genom cybercampus Sverige. I syfte att stärka forskningen inom Al och säkerhet föreslår Alkommissionen att 50 miljoner kronor årligen ska skjutas till för att stärka och utveckla den forskning som id dag bedrivs. Genom att ansål dessa medel skapas förutsättningar att ansöka om ytterligare finansiering från EU samt från det privata selstom och andra forskningsfinansärier. Vi bedöder ett sådant anslag vara tillåtkligt för att bilda en särskild forskningsavledning för Al och säkerhet bestäende av såväl seniora som och av.
\section*{Icke-antagonistiska hot- olyckkor kommer att niträffa}
De hot som vi tillstils har diskuterat är alla antagonistiska. Det vill såga sådana hot som bygger på att en allinsnad aktör avskiltligt riktar ett angrepp mot der stavenska samhället eller svenska intressen. Det finns även icke-antagonistiska hot, ett vilå såga hot inne uppstår avskiltigt. Detta inkluderar bland annat pandemier, olyckor och klimatförändringar, vilka samtliga medför allvarliga utmanfang. Et jortsdarked kan rasera en viktig motorväg och att detacanter kan brinna ner. Det innefattarck osåk handhavandefel, bristande kompetens och bristande rutiner. Sju av tio IT-incidenten som rapporterates till MSB under 2023 berodde på andra anledningar än angrepp somism taslaget eller systemfel. \({ }^{}\) Vi behöver därfor antarbe aktivt för att förebygga de händelser som kraftbejggsav. Vi måste påck såkserställa en beredskap för de allvarliga hot som idlagr kan elimineras helt, och förmåga att hantera de konsekvenser de kan medföra.
\(\Delta\) Ven på detta område kan Al användas för att stärka samhället på många sätt. Tekniken kan hjälpa oss både att planera för allvarliga händelser i fremdstid och att hantera dem när de val iiträfar.
\footnotetext{
(82) See exemplivis rapporten Foreign Cybersecurity(Threats for 2020 - Udøtøs av Enfisa, EUs: cybersäkerhetsbyrå. Tilljänglig på https://www.enfisa.europa.eu/ publications/forenig-the-penocracy-threats-for-2030-updata-2040-exteded-report@edbnovia4/fullreport.
(83) Bevövet av stark tillå när det gäller Al dalsutens (vere) 1 Kapført. Al för en offentlig saktor i framant.
(84) MSB, Arrasport i-Incidentrapportering 2023. s. 21.
}

\title{
Figur 8. Operativ miljö - ranking och poäng
}
\author{
Italien - plats 1 \\ USA - plats 2 \\ 96 \\ 90 \\ 88 \\ 88 Poäng för \\ plats 5 \\ 84 \\ 81 \\ 88 Poäng för \\ plats 58 \\ 50 \\ 47 \\ 47 \\ 100 \\ 80 \\ 100 \\ 100 \\ 100 \\ 100 \\ 110 \\ 110 \\ 110 \\ 110 \\ 100 \\ 110 \\ 110 \\ 100 \\ 100 \\ 100 \\ 120 \\ 120 \\ 120 \\ 120 \\ 100 \\ 100 \\ 100 \\ 130 \\ 130 \\ 130 \\ 130 \\ 140 \\ 140 \\ 140 \\ 140 \\ 130 \\ 130 \\ 130 \\ 100 \\ 100 \\ 100 \\ 140 \\ 140 \\ 140 \\ 100 \\ 100 \\ 100 \\ 180 \\ 180 \\ 180 \\ 180 \\ 100 \\ 100 \\ 100 \\ 0 \\ 0 \\ 0 \\ 0 \\ 0 \\ 0 \\ 0
}
Notera: Placeringen för varje land inom Operativ miljö visas efter landets namn. Den horisontella axeln visar poängen för varje land, beräknad utfrån indikatorre relaterade till området. Den högsta poäng som ett land kan få år 100. Den grå stapeln representerar den poign som krävs för att plæra sig på plats 5 i Operativ miljö 12024 års upplaga.
Källa: The Global AI Index, 2024 års upplaga.
Indikatorerna på området skildrar ett stort spann av olika samhällsaspekter, till exempel allmänhetens distillt till AI och I-Företag, nivån på dataskyddslagstiftningen (GDPR framhälls här som ett föredöme), cybersäkerhet, könsfördelningen bland nytuxaminerade ingenjörer samt kostnader för arbetsvism.
För att Sverige ska upprätthålla en välfungerande operativ miljö obehövs åtgärder inom flera områden. I kapitel Data som en förutsättning för Al-utveckling och förörsäld. Vårläder har GDPR tillämpas Vervigel, för att föräld för att föräld för att föräld det delandeling. På samma sätt föreslås en föräld för att onfettlighets- och sekretesslagen. En annan katgeori av relevanta åtgärder är de som kan förstärkta tilltion och förtroendet för tekniken. I kapitel Kornpetensfört förla alln ägrförslag på stödt till fokbildande aktörer för att skapa legitimitet och acceptans för AI. I kapitlet Al och samhällets säkerhet föreslås bland annat att forskningen inom Al och cybersäkerhet samt att ett institut för Al-säkerhet skapas.
Vad kan då vara en ambitiös och realistisk målsättningen för området Operativ miljö? Sveriges stykror i form av tillit, tillgång till unika datakållor, en stark tradition av fokbildning, understödda av förslagen i denna Fårdplan, talar för att väen förtsättningsvis ska tillhöra de högst rankade länderma på detta område, det vill säga bland de 5 främåt länderna år 2030.
- Sverige ska fortsatt rankas topp 5 i området Operativ miljö
\section*{Global AI Index (GAI)}
\(V\)uen om Sverige harhallkat ner i det sammanvägda GAI under senare år finns det ingenting som säger att den utvecklingen ska behöva förtsätta - tvårtom. Sverige har många styrkor att bygga på, och med ett beslutsamt genomförande av de breda åtgärder som föreslås i denna Fårdplan för Sverige borde situationen snabbt kunna förbättras.
Figur 9 visar Sveriges placerering i GAI, vilket bygger på samtliga indikatorer från de sju olika områdena. Sverige återfinns på plats 25 med en poäng in livå med många av våra jämförelseländer.

\title{
2 En stabil grund att bygga på
}
I den här delen av rapporten går vi igenom de områden som är viktiga möjliggörare för utveckling och användning av Al i Sverige. Dessa kan ses som den nödvåndiga grunden som övriga åtgärder bygger vidare på, vilka utvecklas i del 3, Al för alla och del 4 Ledarskap och styrning.
Grundläggningen börjar med mer traditionell infrastruktur, i form av energi och telekom, utan vilka det inte blir någon digital verksamhet alls. Andra viktiga grundstenar är tillgång till beräkningskraft, i form av datorer, och data. Dessa två kan sägas utgöra motorn och bränslet för Al. För att Al ska kunna utvecklas i en hård global konkurrens krävs också en forskningsmiljö i världsklass. Därefter följer en diskussion om säkerhets- och etikfrågorna, som är också centrala för en harmonisk och balanserad implementering av Al i samhället. Slutligen diskuteras behovet av fortsatt tillgång till Al-resurser från övriga världen. Sverige är ett litet land och vi kommer alltid att vara beroende av omvärlden, inte minst på Al-området.
Denna del innehåller:
\begin{tabular}{lr} 
Energi & 22 \\
Telekom & 27 \\
Beräkningskraft & 30 \\
Data som en förutsättning för Al-utvecklingen & 36 \\
Al och samhällets säkerhet & 46 \\
Spetsforskning i samverkan & 53 \\
Tillgång till internationella Al-resurser & 57
\end{tabular}

Fössilfri el som en konkurrensfördel utveckling och användning av Al förutsätter datacenter för att träna och driftsätta algoritmer. I den digitala sektorn spelar geografisk placering mindre roll, eftersom aktörer kan leverera tjänster globalt via internet. Det gör att datacenter kan koncentreras till platters med gynnsamma förutståttningar varifrån de kan försörja en global kundkrets. Elförbrukningen i ett visst land påverkas därmed inte bara av Al utvecklingen, utan även av datacentrens placering.
Sverige och våra närmsta grannar har utmärkta förutsättningar för att husera datacenter. Det borer främst på att vi har relativ politisk stabilitet, ett kallt klimat, god tillgång till vatten och framför allt billig, stabilit och fossilfri el. Just fossifri el blir alltmer intressant i takt med att röster höjs för att göra Al klimatmässigt hållbart. Dettta har gjort Norden attraktivt för betalering av datacenter.
Figur 1: Elproduktionen under 2023
Elproduktion, andel per energislag

\section*{TWh per en miljon invånare 2023}

Note: Fossila bränljen inneftart naturgas, kol och olla.
Källa: Ember (2024), Energy Institute - Statistical Review of World Energy via https://ourworldindata.org/grapher/shareelectricity-nuclear.

Figur 1: FoU-utgifter som andel av intäkteri tre olika kategorier av branscher

\title{
Draghirapporten
}
Nylgen publicerade Mario Draghi rapporten Den europeiska konkurrenskraftens framtid. I rapporten pekar Draghi på flera aspekter som är av stor vikt för utvecklingen inom Al, framför allt i relation till Europas konkurrenskraft.
Här menar Draghi att den europeiska produktiviteten har halkat efter USA:s. Detta med följden att den reala disponiba linkomsten \({ }^{1129}\) per capita, det vill säga inkomsten efter skatt och justerat för inflationen, för att förbättra produktivitetsutvecklingen Europa sedan 2000. Enligt Draghi bero skillnaden i produktivitet till största delen på utvecklingen av tech-sektorn. Medan USA och Kina snabbt utökar sitt teknologiska ledarskap, visar Draghi att Europa släpar efter inom nyckelområden som Al-användning och investeringar i avancerad teknologi. Endast fyrav av världens 50 största technolog är europeiska, annäfflig första plattformarna i Europa sägs av merikanska, av skorning och av skorning.
Draghi belyser även att det i dagsläget är svårt för europeiska startups inom Al att växa och konkurrera globalt. Det beror främst på att de europeiska kapitalmarknaderna är både fragmenterade och för att för att för att för att för att för att älesdes många europeiska företag att söka finansiering utanför unionen, vilket försvagar EU:s teknologiska suveränitet. Detta problem är särskilt allvarligt inom Al-sektorn. Av de mest framgångsrika Al-startupbolag som finns i världen, går endast 6 procent av investeringarna till företag inom EU, lämfört med 61. procent till företag i USA och 17 18. 19. 19. 19. 19. 19. 18. 19. 19. 19. 19.
Al-sektorn präglas av stordriftsfördelar och nätverseffekter, vilket gör Europas fragmenterade kapitalmarknader och regleringar särskilt problematiskt för mindre länder, även om Sverige skljer ut sig halet och rägning. 19. 19. 19. 19. 17. 19. 19. 19. 19. 14. 19. 19. 19. 19. 13. 19. 19. 19. 19. 15. 19. 19. 19. 19. 16. 19. 19. 19. 19. 10. 19. 19. 19. 19. 11. 19. 19. 19. 19. 12. 19. 19. 19. 19. 1. 19. 19. 19. 19. 20. 19. 19. 19. 19. 21. 19. 19. 19. 19. 3. 19. 19. 19. 19. 22. 19. 19. 19. 19. 23. 19. 19. 19. 19. 33. 19. 19. 19. 19. 4. 19. 19. 19. 19. 24. 19. 19. 19. 19. 34. 19. 19. 19. 19. 45. 19. 19. 19. 19. 25. 19. 19. 19. 19. 36. 19. 19. 19. 19. 26. 19. 19. 19. 27. 19. 19. 28. 19. 19. 29. 19. 30. 19. 29. 19. 31. 19. 29. 19. 32. 19. 33. 19. 33. 19. 33. 18. 19. 33. 18. 33. 18. 33. 18. 34. 19. 34. 19. 34. 19. 35. 19. 35. 19. 35. 18. 35. 18. 35. 18. 36. 18. 36. 18. 36. 19. 36. 19. 37. 19. 37. 19. 38. 19. 38. 19. 38. 18. 38. 19. 39. 19. 39. 19. 39. 18. 39. 19. 40. 19. 41. 19. 41. 19. 41. 18. 40. 40. 40. 40. 40. 41. 41. 41. 41. 41. 42. 42. 42. 42. 42. 43. 43. 43. 43. 43. 44. 44. 44. 44. 44. 45. 45. 45. 45. 45. 46. 46. 46. 46. 46. 47. 47. 47. 47. 47. 48. 48. 48. 48. 48. 49. 49. 49. 50. 50. 50. 50. 50. 51. 51. 51. 51. 51. 52. 52. 52. 52. 52. 53. 53. 53. 53. 53. 54. 54. 54. 54. 54. 55. 55. 55. 55. 55. 56. 56. 56. 56. 56. 57. 57. 57. 57. 57. 58. 58. 58. 58. 58. 59. 59. 59. 60. 60. 60. 61. 61. 61. 61. 61. 62. 62. 62. 62. 63. 63. 63. 63. 63. 64. 64. 64. 64. 64. 65. 65. 65. 65. 65. 66. 66. 66. 66. 66. 67. 67. 67. 67. 67. 68. 68. 68. 68. 68. 69. 69. 69. 70. 70. 70. 70. 71. 71. 71. 71. 71. 72. 72. 72. 72. 72. 73. 73. 73. 73. 73. 74. 74. 74. 74. 74. 75. 75. 75. 75. 75. 76. 76. 76. 76. 76. 77. 77. 77. 77. 77. 78. 78. 78. 78. 78. 79. 79. 79. 80. 80. 80. 80. 80. 81. 81. 81. 81. 81. 82. 82. 82. 82. 82. 83. 83. 83. 83. 83. 84. 84. 84. 84. 84. 85. 85. 85. 85. 85. 86. 86. 86. 86. 86. 87. 87. 87. 87. 87. 88. 88. 88. 88. 88. 89. 89. 89. 90. 90. 90. 90. 90. 91. 91. 91. 91. 91. 92. 92. 92. 92. 92. 93. 93. 93. 93. 93. 94. 94. 94. 94. 95. 95. 95. 95. 95. 96. 96. 96. 96. 96. 97. 97. 97. 97. 98. 98. 98. 98. 99. 100. 100. 100. 100. 101. 101. 101. 101. 102. 102. 102. 102. 103. 103. 103. 103. 104. 104. 104. 104. 105. 105. 105. 105. 106. 106. 106. 106. 107. 107. 107. 107. 108. 108. 108. 108. 109. 109. 109. 110. 111. 111. 111. 111. 112. 113. 113. 113. 113. 114. 114. 114. 115. 115. 115. 115. 116. 116. 116. 116. 117. 117. 117. 117. 118. 118. 118. 119. 119. 120. 120. 120. 120. 121. 121. 122. 122. 122. 122. 123. 123. 123. 123. 124. 124. 124. 124. 125. 125. 125. 125. 126. 126. 126. 126. 127. 127. 127. 127. 128. 128. 128. 128. 129. 129. 130. 130. 130. 130. 131. 131. 131. 131. 132. 132. 132. 132. 133. 133. 133. 133. 134. 134. 134. 134. 135. 135. 135. 135. 136. 136. 136. 136. 137. 137. 137. 137. 138. 138. 138. 138. 139. 140. 140. 140. 141. 141. 141. 141. 142. 142. 142. 143. 143. 143. 143. 144. 144. 144. 145. 145. 145. 145. 146. 146. 146. 147. 147. 147. 147. 148. 148. 148. 148. 149. 149. 150. 150. 150. 151. 151. 151. 151. 152. 152. 152. 153. 153. 153. 153. 154. 154. 155. 155. 155. 155. 156. 156. 156. 156. 157. 157. 157. 157. 158. 158. 158. 158. 159. 159. 160. 160. 161. 161. 161. 161. 162. 163. 163. 163. 163. 164. 164. 164. 165. 165. 165. 166. 166. 166. 167. 167. 167. 167. 168. 168. 168. 168. 169. 169. 169. 169. 161. 162. 163. 164. 165. 166. 166. 167. 168. 168. 169. 161. 161. 162. 164. 165. 166. 167. 168. 169. 169. 161. 161. 161. 163. 164. 165. 165. 166. 167. 167. 168. 169. 161. 162. 164. 164. 165. 166. 165. 166. 166. 168. 169. 169. 160. 161. 161. 162. 162. 163. 164. 164. 165. 167. 167. 167. 169. 169. 161. 163. 164. 166. 167. 168. 167. 168. 169. 160. 161. 162. 163. 165. 166. 166. 169. 1

standarder. Även patientorganisationer är viktiga för att bålla patientperspektivet i fokus.
Stiftelsen ska aktivt söka finansiering från flera källor, såsom investeringar från näringslivet i form av kapital och in-kind-resurser, forskningsanslag från nationella och internationella finanslärer, samt filantropiska donationer från både svenska och internationella givare. Samtidligt kommer statlig grundfinansiering vara nödvändig för att garantera projektets långsiktighet.
För att sætsterålla pransparens och ansvar kommer stiftelsen årligen att redovsia sin verksamhet och ekonomi för riksdagen och allmänheten. Det kommer även att finnas en etisk kommitt som löpande granskar att etiska kritlinjer föjis. Stiftelsens internationella samansamtänstkingar ebar unik plattform för samverkan mellan sektorer och länder. Det gör globalt utbyte av expertis och resurser möjlig, samtidigt som projektet är förkranträt i venenska värderingar in hälso-och sjukvården.
\section*{Regulatorisk kommltätte}
En särskild kommltätte, Kommittén för Hålsatoda och Etik (KHE), bör tillståts med mandat att hantera regulatoriska frågor ring clastyd, delt och patientsäkerhet. KHE ska utforma ett innovativt rättsligt varmerför projektets genomförande, med högsta standard för integritet och etik. Kommittén ska bestå av jurister med expertis in omhås-och datablattfostning, läåker och hälsoexpert mer edfarenhet av klinisk forskning och AI datvattare med kunskap om storskalig answersante, etik med inriktning på medicinsk AI teknologisk let, patientfördetärare samt internationella experter in modastavdd. KHE ska föreslå ny lagstiftning som gør en säker och etisk användning av hälsodata möjlig, inklusive en potentiel: Tåg som av hälsodata mägling, hälsole om tätte för att grejektet nödvändig ära rättsliga raderet.
\section*{H\&K} ska också läda en offentlig dialog om Atts enstålla implikationer för att skapa förtroende och stöd hos allmänheten. Desustom kommer kommittén att ha potential att fungera som rådgivande organ till regeringen och andra relevanta aktörer. KHE ska rapportera direkt till regeringen och stiftelsens styrelse, och dess rekommendationer kommer att vägleda projektets genomförande. Med ett starkt mandat och en bred mansamtsättning kan KHE utveckla ett rättsligt och etiskt ramverk som kan bli en global förebild för att balansera innovation, integritet och samhålsnytta inom Al-driven hälso- och sjukvår.
\section*{Forskningscenter}
ett tvåveretskapligt forskningscenter av värdsklass, Centrum för Al och Hålsa (CAIH), ska etableras som ett internationellt centrum för excellens. CAIH kommer att vara naves för utvecklingen av den svenska hälsomodellen och fungera som en global knutpunkt för banbrytande forskning inom Al och hälsa. Huvudcampus kan placeras vid Karolinska institutet med noder vid andra ledande svenska universitet och internationella partnerskap med toppuniversitet som MIT, Stanford, Oxford och Tsinghua.
Forskningen inom CAIH bör fokusera på att utveckla den svenska hälsomodellen och bedriva grundforskning inom Al och maskinilärning, samt kliniska studier för att vallidera Al-modeller i vårdilmijer. CAIH kan också utföra etisk Al-forskning med fokus på transpasrens och rättvisa, samt kombinera Al med genomik och andra "omics" fåft.
Centret skulle attrahera världsledande forskare, bland annat genom stäforskparprogram för internationella experter. Desusutom utgör initiativet en utmarkt möjlighet att etablera ett doktorandprogram i samarbete med globala partners för att utbilda framtida ledare inom Al och hälsa. För att främja innovation kan CAIH inrätta "Svenska Al-hälsporist", ett årligt pris för banbrytande ieder. Centret kan även fungera rågdivanet till regioner och myndigheter om innovationsupphandeling, utveckla standarder för Al-implementation季度 andorna utbildningar för bestålla med med med med med med med med med med med med med med med med med med med med måttet av för att stimulera Al-utvecklingen i Sverige. Till exempel kan centret arrangera en årlig internationel konfferens om Al och hälsa i Stockholm, där forskare, innovatörer och branschiedare samlas för att dela idéer och insikter. Genom att publicera en open access-tidskift om Al-innovatoner kan CAIH göra banbrytande forskning tillgänglig för en bredare publik. Desustom kan ett felkowship-program för kliniker infätts. Det ger möjlighet attarbeta direkt vid CAIH och därmed överbygga klyftan mellan forskning och praktik.
\section*{Vad skulle det kunna kosta?}
Den totala kostnaden för en femärsperiod skulle kunna vara s miljarder kronor.
Denna kostnad motsvarar cirka fyra procent av Sveriges årliga hälso- och sjukvårdsbudget på cirka 110,3 miljarder kronor (2023/24). Investeringen motiveras av potentialen för betydande effektiviseringar och förbättrad vårdkvalitet genom Al-stödd beslutfastttning och forskning, och i slutdänan genom att svenska folket kan bli friskare, leva längre med hög livskvalitet och få effektivare behandling vid sjukdom.
\section*{Slutsats}
Genom att bygga vidare på Sveriges unika position inom häsodata och Al-forskning har Svensk Hälsomodel (SHM) potentialen att revolutionera hälso-och sjukvården, stärka Sveriges position som ledande forskningsnation och skapa betydande samhälsnytta.

\title{
Ledarskap och styrning för att genomföra Färdplanen
}
\section*{ChatGPT sammanfattar:}
Al-tekniken står inför att omvandla vårt samhälle på djupet, och för att maximera dess fördelar krävs tydliga styrningsmodeller och ett engagerat ledarskap.
Detta kapitel utforskar hur \(\mathrm{Al}\) kan adressera några av våra mest pressande utmaningar, men också varför dagens spilttrade förvaltning riskerar att bromsa utvecklingen. Hur kan vi säkra att Al integreras effektivt och ansvarsfullt i våra system? Genom konkreta förslag på samordning och centraliserade initiativ, ges har en plan för hur Sverige kan stärka sitt ledarskap och ta en global position inn om Al. Kapitlet belyser de viktiga gest som krävs för att säkerställa framgång i detta omfattande teknikskifte.
\section*{Politiskt ledarskap saknas}
Under de många möten kommissionen har haft med aktörer i det svenska samhället har det framgått, med stor lydighet, att den nationella styrning som i dag sker av frågor som rÖl Al, i många avseenden är otyldig och fragmentariskt. Den kan handla om offentliga och privata aktörer som avvaktar med att vidta nödvändiga åtgårder på grund av avsaknad av direktiv, vågledning eller nödvändigas resurser. Når det gäller den offentliga förvaltningen noterar regeringen själv bjudgetpropositionen för 2025 att det finns en stor föräskiftighet och osäkerhet, att det saknas gemensamma Al-satsningar och att förvätningen som konsekvens riskerar att missa potentialen med Al. \({ }^{[17,}\)
Jåmfört med the tidigare stora teknikskiftena år behovet av nationellt politikst ledarskap dessutom möjligen än större denna gång, givet avsaknaden av stora, tekniklande sende vanska företag på Al-området. Fårdplan och större, en större, en större, en större, en större, den frågor finns dock inte I-Sverige i dag. Det kan has in grund den i densska förvaltningsmodellen, en modell som 1 stört jänat oss väl under åhrundraden. Myndighedeterna har i regel ett tydligt ansvar och mandat in somt sitt pefikta ansvarsområde men ett betydligt vasgare sådant når det gäller frågor som spänner över flera sektor. Når det gäller hanteringen av Al-frågor men Overgeringskänlet van \(\mathrm{k}\) konstatera att dessa idag är spilttrade på många departement. Mot bakgrund av att förslagen i denna rapport täcker i stort alla statens utgiftsområden, och därmed samtliga departement, åtta etn unmaking.
Den svenska förvaltningsmodellen
Sveriges decentraliserade förvaltningsmodell har många fördelar. Utvörter att lägga många beslut närade medborgarna, I lokala församlingar, erbjuder den ett mått av demokratisk robusthet. I kriser, eller vid snabb förändring, höjs ofta röster för mer kraftfull nationell styrning, Frågan behandläs i hela sin komplexitet i Kommittén om beredskap enligt regeringforsmens betänkande Stärkt konstitutionell beredskap (SOU 2023:75). Snabba systemövergripande teknikskiften, som Al-teknikens genombrott, utgör inte en kris, men kan ånd ulvatera en tydlig central styrning.
\section*{Ett antal förösk till strukturer för en mer samlad} statlig styrning i tvärskertoerillia frågar har prövats i den svenska förvaltningen. Erfarenheterna från många av dessa initiativ tycks vara att tvärsektoriellt arbeite i Regeringskansliet är svårt. Nödvändiga förut sättningar år både politiskt stöd, tydlig styrning och tillräckliga resurser. \({ }^{[17,}\)
Erfarenhet av tvärsektoriell styrning i Sverige Nationella strategier och handlingsplaner används ofta för tvärskertoerillia frågor, som kräver insatser på flera förvaltningsnivår och inom olika verksamstensområden för att förverkliga regeringens mål. \({ }^{[17,}\) Dessa dokument är dock till sin natur inte bindande. 2018 antog regeringen exempelvis en Nationell inriktning för artifciell intelligens.
En lösning som ofta används för att hantera frågor som påverkar flera offentliga (och privata) aktörer är att skapa olika former av föra för samråd och

\section*{Steg 5: Användning av Al-plattformar och Al-verktyg av aktörer i olika branscher och samhällssektorer}
1. 1. 1. 1. 1. 1. 1. 1 1. 1. 1. 1. 1. 1. 2. 2. 2. 2. 2. 2. 2. \(^{2}\) 3. 3. 3. 3. 3. 3. 3. 4. 4. 4. 4. 4. 4. 4. \(^{2}\) 4. 4. 4. 4. 4. 4. 5. 5. 5. 5. 5. 5. 5. \(^{2}\) 5. 5. 5. 5. 5. 5. 6. 6. 6. 6. 6. 6. 6. \(^{2}\) 5. 5. 5. 5. 6. 6. 7. 7. 7. 7. 7. 7. 7. \(^{2}\) 5. 5. 5. 5. 7. 7. 7. 7. 7. \(^{2} 3\) 3. 3. 3. 3. 3. 3. 5. 5. 5. 5. 5. 5.3. 5. 5. 5. 5. 5. 5 5. 5. 5. 5. 5. 5. 7. 7. 8. 8. 8. 8. 8. 8. 8. \(^{2}\) 5. 5. 5. 5. 3. 3. 3. 3. 3. 3. \(^{2}\) 5. 5. 5. 5. 4. 4. 4. 4. 4. 4.3. 4. 4. 4. 4. 4. 4 4. 4. 4. 4. 4. 4. 3. 3. 3. 3. 3. 3. \({ }^{2}\) 5. 5. 5. 5. 5. \(^{2}\)
5. 5. 5. 5. 5. 5. 5 3. 3. 3. 3. 3. 3. 2. 3. 3. 3. 3. 3. 3.3. 3. 3. 3. 3. 3. 3 3. 3. 3. 3. 3. 3. 1. 3. 3. 3. 3. 3. 3. . 3. 3. 3. 3. 3. 3. 7. 3. 3. 3. 3. 3. 3.4. 3. 3. 3. 3. 3. 3 4. 3. 3. 3. 3. 3. 4. \(^{2}\) 5. 5. 5. 5. 8. 8. 8. 8. 8. 8.3. 3. 3. 3. 3. 3. 4 3. 3. 3. 3. 3. 3. 8. 3. 3. 3. 3. 3. 3.2. 3. 3. 3. 3. 3. 3 5. 5. 5. 5. 5. 5. 3. 3. 4. 3. 3. 3. 3. 3. \(^{2} 3\) 3. 3. 3. 3 3. 3. 3 3. 3. 3. 3 3. 3. 3 . 3. 3. 3. 3. 3. 3. \(3 . 3 . 3 . 3 . 3 . 3 . 3 . 3. 3 . 3 . 3 . 3 . 3 . 3 . 4 . 3. 3. 3. 3. 3. 3.3 . 3. 3. 3. 3. 3. 3 . 3. 3. 3. 4. 3. 3. 3. \(^{2}\) 5. 3. 3. 3. 3. 3. 4.3. 3. 3. 3. 3. 3. 5 . 3. 3. 3. 3. 3. 3. . \(^{2} 3\) 3. 3. 3. 3. \(^{2}\) 3. 3. 3. 3. 4. 3. 3. 4. 3. 3. 3.3. 3. 3. 4. 3. 3. 3 3. 3. 3. 4. 3. 3. 5. 3. 3. 3. 3. 3. \(^{23}\) 3. 3. 3. 3. 3. 3 3 3. 3. 3. 3. 3. 3 3 . 3. 3. 3. 3. 3. 4. 3. 4. 3. 3. 3. 3. \(^{23}\) \(^{2}\) 3. 3. 3. 3. 3 3. 3 3. 3. 3. 3. 3 3. 3 . 3. 3. 3. 3. 4. 3. 5. 3. 3. 3. 3. \(^{2}\) 33 3. 3. 3. 3. 3. 3. \(^{\prime} 3\) 3. 3. 3. 3. 3. \(^{2}\)
5. 3. 3. 3. 3. 3 3. 4. 3. 3. 3. 3. 4. 3. \(^{2}\) 3. 3. 3. 4. 3. 3. 2. 3. 3. 3. \(^{2}\) 3. 4. 3. 3. 3. 3. 5. 3. 3. 3. \(^{2}\) 3. 5. 3. 3. 3. 3. 4. 3.3. 3. 3. 3. 3. 4. 3 3. 3. 3. 3. 3. 4. 5. 3. 3. 3. 3. 3.3. 4. 3. 3. 3. 3. 3 3. 5. 3. 3. 3. 3. 5. 3. \(^{2}\) 3. 3. 3. 5. 3. 3. 4. 3. 3. \(^{2}\) 3. 3. 4. 3. 3. 3. 4. 3. 3. \(^{2} 3\) 3. 3 . 3. 3. 3. 3. 5. 3. 4. 3. 3. 3. 3.3. 3. 4. 3. 3. 3. 3 3. 3. 4. 3. 3. 3. 5. 3. 3. \(^{2} 3\) 3. 3 3. 3. 3. 3. 3 . 3. 4. 3. 3. 3. 3. 1. 3. \(^{2}\) 3. 3. 3. 2. 3. 3. 4. 3. 3. 3. . 3. 3. 3 3. 3. 3. 3 . 3. 3. 4. 3. 3. 3. 1. 3. 3. \(^{2} 3\) 3. 33 3. 3. 3. 3. 3. 4. \(^{\prime} 3\) 3. 3. 3. 4. 3. 3. 1. 3. 3. 3. \(^{2} 3\) 33 3. 3. 3. 3. 3. 5. 3.3. 3. 3. 3. 3. 5. 3 3. 3. 3. 3. 3. 5. \(^{2} 3\) 3. 3. 3. 3 . 3. 3. 5. 3. 3. 3. 5. 3. 3.3. 3. 3. 3. 5. 3. 3 3. 3. 3. 3. 5. 3. 5. 3. 3. 3. 3.3. 3. 5. 3. 3. 3. 3 3. 3. 5. 3. 3. 3. 4. 3. 3.3. 3. 3. 3. 4. 3. 3 3. 3. 3. 3. 4. 3. 1. 3. 3. 3. 3. \(^{2} 3\)
5. 3. 3. 3. 3. 3. 3 1. 3. 3. 3. 3. 3. 4. . 3. 3. 3. 3. 3. 3 33 3. 3. 3. 3. 3. 1. 1. 3. 3. 3. 3. 3. \(^{20}\)
6. 3. 3. 3. 3. 3. 3.1. 3. 3. 3. 3. 3. 3 2. 3. 3. 3. 3. 3. 4. , 3. 3. 3. 3. 3. 3. 6. 3. 3. 3. 3. 3. 3 6. 3. 3. 3. 3. 3. 4. \({ }^{2} 3\) 3. 3. 3. 3. 4. 3. 2. 3. 3. 3. 3. \(^{2} 3 . 3 . 3 . 3 . 3 . 3 . 2 . 3 . 3 . 3 . 3 . 3 . 3 .3 . 3 . 3 . 3 . 3 . 3 . 33 . 3. 3. 3. 3. 3. 3 4 . 3. 3. 3. 3. 3. 4. \(3 . 3 . 3 . 3 . 3 . 3 . . 3 . 3 . 3 . 3 . 3 . 3 . . 4 . 3. 3. 3. 3. 3. \(^{2} \times 3\) 3. 3. 3. 3. 3. 4. 2. 3. 3. 3. 3. 3. \(^{21}\)
5. 3. 3. 3. 3. 3. 4 . 3. 3. 3. 3. 3. 5. 4. 3. 3. 3. 3. 3.3. 5. 3. 3. 3. 3. 3 3. \(^{2} 3\) 3. 3. 3. \(^{2} 3 . 3 . .3 . 3 . 3 . 3 . 3 . 3 . 4. 3. 3. 3. 3. 3. 5. \(^{\prime} 3\) 3. 3. 3. 5. 3. 3. 5. 3. 3. 3.3. 3. 3. 5. 3. 3. 3 3. 3. 3. 5. 3. 3. 6. 3. 3. 3. \(^{2} 3\) 3 . 3. 3. 3. 3. 3. 5. \(3 . 3 . 3 . 3 . 3 . 3 .\) 3. 3. 3. 3. 3. 3. \(_{\text {3 }}\) 3. 3. 3. 3. 3. 3.3 3. 3. 3. 3. 3. 3.33333333333333333333 3. 3. 3. 3. 3. 3 4 3. 3. 3. 3. 3. 3 3, 3. 3. 3. 3. 3. 3. 9. 3. 3. 3. 3. 3. 3. , 3. 3. 3. 3. 3. 3333333333333333333
5. 3. 3. 3. 3. 3. 3 . 4 . 3. 3. 3. 3. 4 . 3. 4 . 3. 3. 3. 3. 4 . 4 . 3. 3. 3. 3. 3.3333 3. 3. 3. 3. 3. \(^{2}^{3}\) 3. 3. 3. 3. 3. 4 . 4 . 4 . 4 . 4 . 4 . 4 . 3. 3. 3. 3. 4 . \(^{2} 3\) 3. 3. 3. 3 . \(^{2} 3 . 3 . 3 . 3 . 3 . . 3 . . 3 . 3 . 3 . 3 . 3. 3 .3 . 3 . 3 . 3 . 3 . 3. 3. 3. 3. 3. 1. 4 . 4 . 4 . 4 . 4 . 4 . \(^{2} 3\) 3. 3. 3. \(^{2}\) 3. 3 . 3. 3. 3. 3. 1. 3. 4 . 4 . 4 . 4 . 4 . \(^{2} \times 3\) 3. 3. 3. 4 . 4 . 4 . 3 . 4 . 4 . 4 . 4 . 4 . 4 . \(^{\prime} 3\) 3. 3. 3. 3 . 3. \(^{2} 3 . 3 . 3 . 3 . . 3 . 3 . . 3 . 3 . 3 . 3. 3 . 3 .3 . 3 . 3 . 3 . 3. 3 . 3. 3. 3. 3. 6. 3. 4 . 3. 3. 3. 3. \(^{2} 3 .3 . 3 . 3 . 3 . 3 . 3 \times 3 . 3 . 3 . 3 . 3 . 3 . 5 . 3 . 3 . 3 . 3 . 3 . 3 . \(^{2} \times 3\) 3. 3. 3. \(^{2} 3 .3 . 4 . 3 . 3 . 3 . 3 . 3 . 3 .\) 4 . 3. 3. 3. 3. 3. 1. \(^{2} 3 . 3 . 3 . 3 . 3 \times 3 . . 3 . 3 . 3 . 3 . 3 . . 3. 3 . 3 . 3 . 3 . 3 .3 . 3. 3 . 3 . 3 . 3 . 3. 3 . 4 . 3. 3. 3. 4 . 3. 3. 4 . 3. 3. 3. 4 . 3. 4 . 4 . 4 . 4 . 4 . 3 . 3. 3. 3. 3. 3. \(^{2}\)

\(33333333

Al-kommissionen anser det viktigt att bygga ett välderingssystem av hög kvalitet dit alla ykersverksamma och arbetsöskande kan vända sig för att få sin Al-kompents certifierad. Konkreta förslag kan hämtas från Valideringsdelegationens rapport SOU 2017:18.
Myndigheten för yrkeshögskolan har också fått ett nationellt ansvar för yrkesvalidering. Detta är bra.
Kommissionen menar dock att det i dagslaget saknas ett specifikt uppdrag till universitet och högskolor att validera kunskap som behov för vidarestudierin om akademin. Detta kan röra tillgodorökanande av tigidare utbildning men även bedömning av reell kompetens. En sådan väldering kan sedan användas för att vara berhögårt att tats<|im_start|> till en viss kurs eller utbildning.
\section*{Förslag}
- Al-kommissionen föreslår att universitet och högskolor ska att större ansvar än nå gör välldering av Al-kunskaper för fortsatta studierin om akademin. \({ }^{[1,}\) uppdraget bör det även ingå att utveckla en Al-tjänst som kan underlätta arbetet, bland annat vad gäller sammanställning av dokumentation från en mängd olika källor. Det kommer sannolikt tår krävas ett större offentligt åtagande än då ogf är topptemenskartlåggning och ett välderingssystem ska fungera i praktiken och för att upprätthålla hög kvalitet. Resurser till detta bör omfördelas inom ramen för vuzenutbildningen.
Effektiv omställning för de arbetslösa även om erfarenheten från tidigare teknikkiften indikerar att nettoeffekten på sysessältningen är oförrändrad på sikt, kan vissa grupper prabbas av arbetslöshet i närtid, då användningen av Al i vissa fall kommer att ersätta befintliga arbetsuppgifter. Samhällets formåga att underlätta för arbetslösa människor att hitta nya jobb kommer därför att vara viktig för omställingen. Den statliga arbetsmarknadspolitiken har successivt utvecklats sedan 1950taliet och utvöas i dag primärt av Arbetsförmedlingen. Detta görs i form av platsförmedling, olika former av rustande insatser såsom arbetsmarknadsutbildning, samt olika typer av lönesubventioner för att de med lägre produktivitet ska ha en chans på arbetsmarknaden.
Nurmer utförs även stora delar av de arbetsmarknadspolitiska insatserna av fristående aktörer, upphandlade av Arbetsförmedlingen. Den statliga arbetsmarknadspolitiken har sedermera kompletterats med partsgemensamma trygghetsråd. Dessa har tilldelats liknade uppgifter som Arbetsförmedlingen, men med en något annorlunda målgrupp. I stora drag kan man säga att trygghetsråden hjälper de arbetsökande som står närmast arbetsmarknaden medan Arbetsförmedlingen, i egen verksamhet eller via en upphandlad aktör, hjälper de som är långtidsarbetslösa eller idlargh att furtaktligt jobb.
Det är Al-kommissionens bedömning att både den statliga arbetsmarknadspolitiken och de partsgemensystem och för att äktnikning. Detta är för att äktnikning, spåla för omställningen på arbetsmarknaden i detta teknikkifte. Med det sagt är det viktigt att insatser för att hjälpa de arbetslösa både är ändamålsenliga och effektiva. Den statliga arbetsmarknadspolitiken har till exempel under senare åk tritiserats för att i alltför liten omfattning vara inriktad på arbetsmarknadsutbildning (ser the ixtempel Finanspolitiska rådets rapport Svensk Finanspolitik från 2024). Al-kommissionen delar den uppfattningen och menar att online-utbildningar i Al-kunskap och hur vanliga Al-verktyg kan användas borde vara en såjklvarkler för alla sinkrivna vid Arbetsförmedlingen och trygghetsoranisationerna. Detta då kostnaden av att vara borta från arbetsmarknaden under snabba teknikkiften är extra stor. Arbetsmarknadspolitiken har således en såskiltivikt劲ól artt kompensera de arbetslösa för bortfallet av kompetensutveckling i Al-färgor som sker på arbetsplatsen. Om arbetsmarknadspolitiken inte tar den rollen, riskerar gapeit i samhället att öta k<|im_start|> mellan de som har och inte har ett arbete. Det är Al-kommissionen sfforhoppning att förslaget om en Al-hubno (se såda 70) kan bilen hjälp för del arbetslösa att skraftsAA relevant Al-kompetens.
\section*{Förslag}
- Al-kommissionen anser att alla arbetslösa ska delta i kurser, inom Arbetsförmedlingsens och omställningsonorganisationernas regl, som syftar till att höjä deras Al-kunskap. Resurser för detta bör tas inom ram för arbetsmarknadspolitiken.

En avgörande faktor för att möta framtidens utmæningar är ett väl fungerande utbildningssystem för yrkesverksamma. Sverige har sedan långe också haft en ambitiös utbildningspolitik för vuxna, och utbildningssystemet och finansieringen har byggets ut på alla nivär. Det saknas inte infrastruktur, eller resurser, för att möta behovet av livslångt lärande framförer i all männhet. I detta sammanhang utgörd dock lärosätena ett undantag, där incitamenten och resurseren, i har ett äkten är att äkten är att äkten är att äkten avvar. Öler. Al-kommissionen välkomnar därför den utredning som har uppdragett at utreda en särskild ersättningsmodelt för omställning och vidareutbildning för yrkesverksamma. \({ }^{[11,13]}\) Utredningen ska dels analysera behovet av en särskild ersättningsmodell, dels lämna förslag om hur en modell för delar av resurstilldelningssystemet kan utvecklas för att stimulera kurser och program på grundlågande eller avancerad nervå nas som stärker indvidins ställning på arbetsmarknaden.
Den historiska satsningen på vuxnutbildning i Sverige är en styrka ett låtge, en styrka som få andra länder kan matcha. Men den är också något av ett problem. Floran av vuxnutbildningar är numera vividvuxen och antalet huvudmän är många med ibland otdylig ansvarsfördelning sinsemlen. \({ }^{[11,14]}\) De satsade resurserna dokumenteras inte alltid och följs inte heller upp på ett systematsistakt sätt. Det är Al-komnisionens opffutpattning att framtida satsningar måste vara evidensbaserade och säkerställa att nuvarande system och resurser används andämälsenligt och effektivt. Det måste också finnas mer systematsik vädelgling för arbetsgivare och arbetstagare om hur arbetsmarknaden kan tänkas utvecklas med tanke på det pågående teknikkfiktet. Under följdande rubriker detta avsnitt diskaturer vi behoven av kompetensutveckling för de som är instalda, egentföretagare, arbetsökande eller arbetsösa.
Kompetensen förveckling för annäter inte att kunna avlängs. Al-komnisionen är inte att kunna täckas enbart av nyutexaminerade personer från gymnasiet eller universitet och högskolor. En bidragande orsak till detta är att det tar tid att genomförat sådana utbildningar, tid som vi inte har i nuläget. Det kommer därför att kråvas en ökad satsning på kompetenutveckling inom Al bland nu yrkesverksamma. Möjligheten till livslångt lärande, utöver det arbetsgivaren normalt erbjuder, kommer därmed att spåela en mycket stor roll för Sveriges konkurrenskraft framöver. Många aktörer, exempelvis Universitetskanslersambbetet (UKÄ) och Myndigheten för yrkeshögskolan (MYH), pekar på behovet av att slutla klyftan mellan Al-kompetens och domän- eller branschokpentens.
Vem ska finansiera kompetensutvecklingen? Det är traditionellt sett en uppgift för företagen och andra arbetsgivare att se till att de anställda har den kompetens de behöver för att utföra sina arbetsuppgifter. Stora teknikkifsten som detta karaktäriseras dock av att den nya teknologi i mängöt och mycket är användbar i alla yrken och verksamheter. Då är det inte självklart att en arbetsgivare kommer att tillhandahlla utbildning till sina anställda i den omfattning som är optimal för sambälle! stort. Det beror delvis på att kompetensutvecklingen av de anställda samtidigt gör dem mer attraktiva för andra arbetsgivare.
Risken med detta är att samhället investerar för lite ny teknik och kompetens. För att undivika det finns det därför en anledning för staten att erbjuda subventionerat individning och studiefinansiering även för anställda. Samtidigt får vi inte övervistera i utbildning på grund av alarmistiska signaler om att vissa kompetenserfsvörniner till följd av ny teknik. Inlehler är det önsväkrat att staten tar över kostaderna för utbildning som företag eller anställda egentligen ska stå för såläva. Detta är ingen hala optimeringsproblem och därför är det viktigt att nograttan identifiera för att för att för att för att för att förderdala enmallståld, företag och stat i framtiden.
I dagsläget är det dock Al-komnisionsen bedömning att problemet om staten skulle investera för lite in kunskap är bydligt värre än om staten investerar för mycket i densamma. Al-kommissionen anser således att tillgången till somtällningsstudietset för arbetstagare är av stor betydelse för omställningen på hela arbetsmarknaden framöver. Likaså anser på ist den Al-hubb, som diskuteras på sida 70, kan vara till stor hjälp för arbetstagare och arbetsgivare att exempelvis identifiera vilka kompetenseren behövs framöver, identifiera vilka kurser som kan gledesa färdigheter, samt att gratis kunna prova vissa Al-jänster.
\section*{Förslag}
- Al-kommissionen uppmutrnar regeringen att bjuda in arbetsmarknadens parter, både från privat och offentlig sektor, till en gemensam diskussion. Fokus bör vara att genom samsyns och en bred och långsiktig samverkan utarbeta lönsningar på de problem som uppstår på arbetsmarknaden till följd av Al.

\section*{Förslag}
- Al-kommissionen ser att det finns behov av ytterligare insatser på området. Bland annat menar vi att Arbetsförmedlingen, med hjälp av myndigheter som MYH och SC9, \({ }^{[110]}\) bör få i uppdrag att halvårsis ta fram prognoser över Al:s inverkan på arbetsmarknaden och vilka utbildningsbehov som finns.
- Kommissionen anser också att Institutet för arbetsmarknads- och utbildningspolitisk utvärdering (IFAU) årligen bör sammanställa forskningsläget om utvecklingen på arbetsmarknaden i relation till utvecklingen inom Al. Detta i syfte att arbetsmarknads- och utbildningspolitiken i framtiden ska bli mer evidensbaserad. Dessa underlag kan med fördel göras tillgängliga inom Al-hubben (för mer information om den föreslagna Al-hubben, se nedan). Kommissionen upskpattart att IFAU bör tillföras 3 miljoner kronor årligen för detta uppdrag.
Som ett ytterligare led i detta anser Al-kommissionen att universitet och högskolor, folkhöksolor och andra utbildningsaktörer skyndsamt behöver utforma ett relevant utbud av kurser och utbildningar anpassade till de rykesverksammas behov av Al-kompetens. Utbildningarna kan med fördel erbjudas digitalt, vara totta samt kostnadsría, eftersom de föret och främst bör betraktas som påbyggnad av nuvarande kompetens. Förf att undvåkla dubelbarete rekom- menderar kommissionen att framtagandet av dessa utbildningar sker i samverkan mellan lärosäten, myndigheter, kommuner, folkhögskolor och andra utbildningsaktörer. Privat och offentlig sektor samt arbetsmarknadsen parter också involveras och ges inflytande i processen.
\section*{Förslag}
- Al-kommissionen anser att utbildningsväsendet bör tillföras 250 miljoner kronor totalt under åren 2025-2027. Syftet är att frigöra tid för lärare att kompetensutveckla sig samt för att genomfara nödvändiga kompletteringar utbildningsutbudet av kurser för rykesverksamma.
- Uppgiften att koordinera och sammanställa den nationella satsningen på livslängt lärande bör ges till Universitets- och högskolerådet (UHR) samt Myndigheten för rykeshögskolan (MYH). Myndigheterna bör samorda sina uppdrag och
de bör vardera tillföras 5 miljoner kronor årligen i två år för uppdraget, därefter 2 miljoner kronor vardera per år. Kursutbudet bör finnas på Al- hubben, som presenteras mer ingående nedan.
För att minska bristen på överskådlig information som grund för viktiga besut på arbetsmarknaden meran Al-kommissionen att en Al-hubb bör etableras. Med hjälp av en Al-hubb är det möjligt att samila information om tillgängliga utbildningar och kurser inom Al, samt information om utvecklingen på arbetsmarknaden. Hubben ger också möjligthet att erbjuda och smala gratis Al-verktyg på en gemensam och lättillgänglig platform. \({ }^{[117]}\) Förutom att främja Al-användningen bland svenskar skulle hubben således fungera som en effektiv lanseringsplattform för gratis, kvalitetssäkrade, Al-tjänster.
Det är Al-kommissionens bedömning att en sådan Al-hubb bör bygga på det arbete som gjorts inom ramen för Regeringskansliets samverkansprogram kompetensförsörinning och livslängt lärande, och det värfunktionella arbete som i dag pågår som resultat av samverkansprogrammed. \({ }^{[1,}\) [18] Framtför all har Trygengstehsfonden TSL komlit långt i silt arbete med att föra samman utbud av offentligt finanserade Al-utbildningar med prognoser på arbetsmarknaden.
\section*{Förslag}
- Al-kommissionen anser att analysen och förslagen som togs fram inom Regeringskansliets Samverkansprogram om kompetensförsörinking och Livslångt lärande ska ligga till grund för en Alhubb. Myndigheten för yrkeshögskolan (MYH) kan vara en lämplig huvudman för uppdraget, men detta behöver snabbutadas av den task force som föresläs i kapitlet Styning för att genomföra Färdplanen. Regeringen bör dock redan nu avsätta 10 miljoner kronor årligen för uppbyggnad och drift av Al-hubben.
\section*{Validering för att bygga på de kunskaper som finns}
För att kompetensutvecklingen för yrkesverksamma ska vara så effektiv som möjligt är det viktigt att bygga vidare på den kompetens som de redan har. Under lång tid har därför kompetenskartläggning och validering av yrkeserafenhet och utländska utbildningar diskuterats och delvis genomförts i Sverige. Med en kartläggning och validering som utgångspunkt kan sedan relevanta Al-utbildningar identifleras.
\footnotetext{
[116] Vårt att notera i detta sammanhang är att SCB har i uppfigt att göra långsiktiga prognoser om belfokning, utbildning och arbetsmarknad.
[117] Se förslaget runt Folbiddning tidigare i deltar kappitel.
[118] Detta är en satsning som regertigen lärmelande under 2019; se Gemensam og förmetsförsöringg och livslängt lärande, Vinnova (022).
}

en utveckling som skulle kunna bidra till att underlättä tillgången till beräkningskraft, inte minst för SMF. Ehteleringarna sker genom att företagen hy prlats i svenska datacenter eller genom att de bygger egna. Detta speglar Sveriges attraktionskraft för sådana åteleringar - det finns fillig, fossilfi el, svalt klimat, gott om kylvatten och relativt stabila spelregler. Dessa etableringar kan ejårsäten, privata företag och offentlig verksamhet ytterligare möjligheter att få tillgång till beräkningskraft. För att dessa etableringar ska ge konkurrensfördelar för Sverige, är det dock viktigt att ställka kriv vid etableringarna. \({ }^{}\)
Superdatorn Arrhenius (som diskuterats merjåande iänslutning till avsnittet om universitet och
högskolor) erbjuder också en unik möjlighet att stärka den företagsanknutna Al-utvecklingen i Sverige. Detta genom att den öppnar möjligheten för Sverige att vara vård för en så kallad Al Factory, ett initiativ från EU-kommissionen som bland annat syftar till att stödja SMF i deras satsningar på Al-utveckling. Det sker främst genom att försе dem med beräkningskraft till konkurrenskraftiga priser och såkra testmiljöler för nya Al-tjänster. Flera intressenter, däribland Vetenkysprådet, NAISS \({ }^{\text {S99 }}\), RISE \({ }^{\text {S100 }}\) och Solifelab, åtar bak som satsningen. Ett godkännande skulle innebära betydande ytterligare EU-finansierling till Sverige för en beräkningskraft som då bland annat känäna SMF.
\section*{Al Factory}
Al Factory är ett av EU-kommissionens centrala initiativ för att stärka unionens konkurrenskraft och ut ötköa investeringarna inom digitalisering och Al. Initiativet finansieras delvis av organisationen
\section*{Målet med initiativet är att skapa en bred kompetens inom Al, Både inom näringsliv och forskning, För att uppnå detta stötter AI Factory olika Al-startups, störe företag, myndigheter och forskare med både infrastruktur och resurser för dem att kunna utveckla Al-modeller och applikatoren. Datta skler bandd annat genom att erbjuda tilläggl instructions till avancerad tränng och vidareutbildning, samt tilläggl till resurser som å öndvåndiga för beräkningsar och lagring av data.
Syftet med Al Factory är inte bara att främja utvecklingen av Al-tjänster. Det är också att se till<|im_start|> av och av och av och av och av och av och av och av och av Factory ska även ta strategiskt ansvar inom områden där medlemsstaten är ledande. För Sveriges del kan detta till exempel omfatta fordons och verkstadsindustri, skogsindustrin, förnybar energi den oikat, cybersåkerhet, livsvetenskap/iäkemedel och av och av och av och av och av och av och av och av.
De organisatoren som kan söka medel för att tillhandahålla en Al Factory är de som redan i dag har vårdskap för något av EuroHPC:s olika system. En sådan organisation är den svenska nationella datorinfrastrukturen NAISS, som innehar vårdskap för superdatorn Arrhenius. I juni 2024 skickade NAISS in en preliminär svensk intresseanmälan för att tillhandahålla en Al Factory. Baserat på intresseanmälan har EuroHPC meddelat att man reservat medel för att Sverige kan komma att anikka om inom<|im_start|>ar forvar för det cry med upt till 20 minjoner euro bikkget över tte år.
EuroHPC har också meddelat att det finns möjlighet att söka medel både för beräkningskraft och för kompetenshöande insatser, eller enbart för det sistmännda. Ledningen för NAISS har i samrård med Vetenskapsrådet gjort bedömningen att det är strategiskt lokt för Sverige att ansöka om båda. Äffått att för att för att för såleds att ansöka om 10 miljoner euro vardera. Äffått att för att det inte av hortivet kompetenshöande aktiviteter för forskning och näringsliv. Ansökan innebär också ett trav om Medfinansiering: 20 miljoner euro för hårdvara och 10 miljoner euro för kompetenshöande aktiviteter. Sverige måste med andra och såkerställa en medfinansiering på 30 miljoner euro.
Den del som finansieras nationellt bestämmer för att för att till<|im_start|> av de för att till den för en finansieras av EuroHPC tillägglig för användare i hela unionen. Det skapar en situation där länder som ligger \(\mathrm{I}\) al-utvecklingens framkant komort är attrahera användare från resten av unionen. If förlängningen leder det till att dessa länder stärker sina möjligheter att attrahera viktig kompetens och utöka sin startuput. Äffått att till<|im_start|> av det strategiskt viktigt för Sverige att höra till de som tar till<|im_start|> av det för att.
Berkängkraft för Al är en mycket central del i infrastrukturinvesteringar för Sverige, särskilt vad gäller ambitionen att öka vår konkurrenskraft. I ljuset av detta anser Al-kommissionen att det är mycket viktigt att ta vara på möjligheten att EuroHPC kan medfinansiera satsningarna.

\section*{Förslag}
- Al-kommissionen anser att regeringen ska tillsätta en utredning med uppdrag att genomföra en översyn av implementeringen och tillämpningen av GDPR i Sverige.
Satsa på forskning om tekniker för att stärka den personliga integriteten
Al-kommissionen anser att Sverige bör borde ta en ledande roll inom så kallade privacy enhancing technologies (PET). PET är avgörande för att förena innovation och integritet. Dessa teknologier minskar risken för att personuppgifter exponeras. Ett exempel är differential privacy, som avidentifierar enskilda personers uppgifter i stora dataset. Ett annat insentrast forskningsområde är syntetiska data, där artificialellt skapade data används i stället för verkliga personuppgifter. Detta minimarer risken för integritetsinträng. Dessutom är teknik som federaread maskininärning, där maskiner trånså på lokala enheter utan att rådata behöver överförlas, av stort ressence. Genom att stödja forskning inom dessa områden kan Sverige inte bara stärka sin konkurrenskraft inom integritetsvängil A1, utan också brätta till en säkeråd digital miljo inom EU och på ett gibölt plan.
\section*{Förslag}
- I kapitel At och samhällets säkerhet föreslår Al-
kommissionen ett höjtt anslag till Cybercampus Sverige för forskning om AI och cybersäkerhet. Vår bedömning är att forskning om PET-tekniker börlig å i detta förslag.
\section*{Lagstiftning ska bli mer digitaliservensvängil} Som vi tidigare berört har en betydande del an vi gällande langstiftning utformats långt innan de blev känt hur pass bett I kan användas i samhället.
Det innebät att gällande regelverk ofta inte utfarmsa på ett andämälsenlighet sätt, om man beaktar hur modern teknik gjort det möjligt att använda data och därmed skapa samhälsynytta på annat sätt änvad som tidigare var möjligt. Mot den bakgrunden året viktigt att utforma lagar och andra regler på ett digitaliservensvängil tys. I harår och OECD bland flera, pekat på behovet avt bålda in digitalisering tidigt i arbetet med att utforma nya regler. Detta arbetssätt kallasl digjölt by desigen. \({ }^{}\)
Myndigheten för digital förvaltning (Digg) har samlat rekommendationer kring hur detta kan ske. Exempelvis bör den som utformar en föreskrift redan från början beakta hur den ska utformas för att det ska vara möjligt att dela data på området. \({ }^{}\)
- Al-kommissionen föreslår att en myndighet, kommitté eller särskild utredare som föreslår ny reglering ska göra en bedömning av om förslaget är utformat på ett digitaliseringsvänglöstätt.
Kravet ska framgå av förordningen (2024:183) om konsekvensutredningar.
Bättre styrning och tillgänglighet för offentliga data (Data Governance och Data Steward)
vi har tidigare konstaterat att data behöver vara av godkvalitet för att kunna användas. I det ligger bland annat att data behöver vara strukturerade på ett enhetligt sätt enlighet med etablerade standarder på respektive område. Sedan lång tid tillbåka finns krav på att den offentliga förvaltningen ska ha en god informationsförvaltning. \({ }^{}\) Kraven har dock byggt på en dokumenterrentare av spn på informationsförvaltning, vilket lett till att data ofta blir instå ist dokument eller svatten. Under processen att digitalisierda den offentliga förvaltningen har lagstifterna inten heller gjort på att för att information lädt ska kunna utbytas digital för.
I syfte att vidareutveckla och förtydliga vad som menas med god informationsförvalting i offentlig sektor, så kallad Data Governance, anser Al-kommissionen att ett tydligt krav bör införas för samtliga aktörer i offentig sektor. Kravet bör vara att upprättbålla en modern digital informationsförvaltning möjlgör interoperabel datadelning på ett säkert och integritetskyddande sätt. Det innebät är aktörer in onform entligt sektor ska se till att de data de ansvarar för och som finns inom respektive organisation är väl strukturerade. Detta innebär också att data ska vara strukturerade på ett sätt som gör det möjligt att behandla dem oberoende av den tekniska infrastrukten UR och detta dela dem på ett säkert, tilltisbaserat och interoperabelsätt i enlighet med tillämpliga standerdar och normer. \({ }^{}\) För statliga myndigheter bör ett sändat krav framgå av myndighetsförordningen (2007:515) och föregoner och kommuner av kommunallagen (2017:725). Ett alternativ till detta förslag kan vara att införa en dataförvaltningslag som omfatter verksamhet på statlig, regional och kommunal nivå.
\footnotetext{
[51] The OECD Digital Government Policy Framework: Six dimensions of a Digital Government", OECD Public Governance Policy Papers, No. 2. 2020.
[52] Rekommendationerna finns tilgängliga på https://www.digg.se/kunskap-och-stod/ufforma-regrelverk-digitallesjenvagning/ldgigs-kremmandelonater-for-att-ufforma-eftragen.
[53] Exempelvis är offentliga organ skyldiga att för ett register över inkomna och uprättade allämna handlingar (5 kap. 1 ØSL).
[54] I sammerhänget för när嚴ms det ramverk för nationella grunddata som tagiga fram av Digg.
}

\title{
Precisionsmedicin kan rädda liv \({ }^{\text {[60] }}\)
}
I dag är det möjligt att analysera en patients gener och på så vis ställa diagnos och skräddarsy behandling utifrån den enskilda patientens genetiska förutsättningar. Denna typ av diagnostik och behandling brukar benämmas precisionsmedicin. Vilka data som är och kommer vara aktuella för precisionsmedicin i framtiden är svårt att säga. Det är idgav genligt att analysera patientens gener med hjälp av bland annat \(\mathrm{Al}\), något som kräver att \(\mathrm{Al}\) får tränas på data främande regelverk, såsom patientatdalagen (2008:355), förkordat PDL, som styr hälso-och sjukvården har tillkommit under en tid när precisionsmedicin inte fanns och är därför inte anpassad efter dagens förutsättningar. Enligt PDL är det i dag inte uttryckgolen tilltätet att ta del av en för att<|im_start|> och att<|im_start|> att<|im_start|> att<|im_start|> att<|im_start|> att<|im_start|> att patient. Det rättsliga stödet för att värdpersonal i ett enskilt fall ska få behandla personuppgifter fall för värddanmål utgår från att personalen deltar i värden av den patientar vamps uppgifter behandlas. En implementering av precisionsmedicin i Sverige skulle underlättas av att värdsporantal för möjlighet att ta del av andra patienters personuppgifter an de som personalen deltar i värden av.
Om personuppgifter ska delas mellan värdg/virare, till exempel mellan två regioner, finns också sekretessgränser att ta hänsyn till. Id lag finns således inte de rättliga förutsättningarna för att fullskaligt kunna implementera precisionsmedicin i Sverige. Precisionsmedicin används dock i viss utsträckning riden i dag. Det är till exempel möjligt att identifiera gener och andra biologiska markörer som den enskilda patienten har. En sådan process kan skapa grundläggande förståelse för patientens medicin medicin i för att värdbet att styr. Vi att en in måmlivnitkad behandling. Förekomsten av den här typen av detaljrika kartläggningar av olika sjukdomar, och mer skräddarsydda behandlingar som kartläggningara öppnor för, ökar. Därdemot sanknas rättsliga förutsättningar för värderspersonalen medicin i för att värdbet att styr. Vi att en det att det ett detta exempel illustrerat tillgigt behovet av att i vissa fall finna en ny avgåvning mellan skyddsintresset för den enskilda personliga integritet och nyttan av att inom offentig verksamhet ha möjlighet att dela data. Det är en nytta med potential att gagna både den enskilda individen och samhället. När det gäller hälso- och sjukvård är det i många fall tykligt att ett ökat tilläggligorande av data kan rädda (160).

\section*{Interoperabilitec}
Ett centralt bepregp nir det handlar om att använda data för att möjligöra Al är interoperabilitet.
Begrepet refererar till förmågan hos olika system, ofta datorsystem, att arbeta tillsammans och utbyta information med varandra. \({ }^{93}\) Interoperabilitet kan beskrivas utifrån fya tolla lager; rättslig, organisatorisk, semantisk och teknisk. Det rättstliga lagrøttar sikte på de juridiska förutsättningarna för att kunma varta i det till att det till att det till att det till att det till att det varha och organisatoriska förutsättningarna - såsomstyrning och mål - är utformade. Med semantisk interoperabilitet menses att data är strukturerade på ett sådant sätt att de kan tillgängliggöras, exempelvis att det finns enheltiga standarder. Slutligen menses med teknisk interoperabilitat ett de tekniska systemen är utformade så att data faktiskt kan delas. Det är alltså förla olika förutsättningar som behöver vara uppfyllda för att data ska kunna delas på ett effektivt sätt.
\section*{Sverige är välförsett med data}
I en internationell jämförelse förforgav i Sverige över alvinigt god offentliga data, data som dessutom är gerölingt välgorganiserade. Det finns flera anledningar till detta. Sverige har varit relativt förskonat från krig, vilket innebär att det arkiv som byggets upp är relativt intakta. Därutöver har såväl staten som kyrkan underåhundranden dokumenterat olika uppgifter och verk. Här har bland annat utbygga välfärdstjänster och beskattning medför en högambitionsivnår äät delgätter allt samia in och kategorisera dessa olika uppgifter och verk. I de register som byggets upp av offentliga aktörer i Sverige finns också olika unika identifierare, till exempel personnummer, vilket ger goda länkningsmöjligheter.
\section*{Vära förutsättningar på datasidan är således goda. Kan vi utnyttja den tillgång som våra offentliga data utgör kommer det att dela della stimt sorhämlysntta och stikkart konkurrenstrkraft. Evän verit påvast kortens finen stor potential till Värskadgame genom ökad datadelning. \({ }^{}\)
\section*{Initiativt för ökad tillgång till data} EU-kommunisen har utforcent 2020-talet till det medigåta årltondet, med amloimenten att göta EU till det
\section*{mest attraktiva, säka, dynamiska och snabrörliga datadrivna ekonomin i världen. Detta har bland annat resulterat i dataförvaltningsförordningen \({ }^{}\) och dataförordningen \({ }^{}\). Sedan tidigare gäller även EU:s öppna data-direktiv \({ }^{}\) med tillhörande genomförandebestämmelser. Med dessa initiativ förbättras villkoner för datadelning på EU:s inre marknad samtidigt som mer data görs tillgänglig för användning.
En annan viktig beståndsdel av EU:s datastrategi är skapandet av gemensamma europeiska datamarkråden. Tanken är att det ska bildas ett fettar olika tidigitala infrastrukturer, inom vilka det ska vara möjligt att dela data på ett enkelt sätt samtidigt som krav på säkerhet och integritet upprätthålls. I skrivande stund pågår utvecklingen av sådana datamadråden inom 14 olika sektorer. \({ }^{}\) Langtsh är arbetet kommit med ett europeisk haldasotaamråde (EHDS). \({ }^{}\)
\section*{I Sverige antog regeringen i oktober 2021 en nationel datastrategi med målet att främja olika former av öppen och kontrollerad datadelning. Syttet är att ökta tillgången till data för bland annat Al. Strategin bygger i sin tur på EU:s datastrategi och OECD:ss rekomindemation om ökad tillgång och delning av data som Sverige har skrivit underer. \({ }^{}\)
\section*{Sverige har skrivet och ett antal stalliga utredningar} På senare tid har även ett antal stalliga utredningar presenterats om innehåler förslag som syftar till att förbättra interoperabiliteten. I december 2023 presenterade utredningen om interoperabilitet vid datadelning sitt slutbietänkande. \({ }^{}\) Utdreningen föreslår bland annat ett nytt politiskit mål att offentlig förvaltnings mest aelgåna datadelning ska vara full interoperabel senast 24/30. Detta ska bland annat uppnås genom en ny lag om den offentliga för detailningen interoperabilitet.
\section*{I maj 2024 lämnade utredningen om infrastruktur \\ förhålsada som nationellt inresses sitts deltetskän \\ kande. \({ }^{}\) Utdrening lämnar ett antal förslag vilka \\ gemensamt syftar till att öka interoperabiliteten vad \\ gäller hästaoch og sikjuvårdsdata. Bland förslagen \\ märks en skyldighet förvärdigare att göra vissa uppsfåger om men patient tillgängliga för andra vårdävlede \\ Detta åtadskationg besm om sammanhållen vård-och}
\footnotetext{
\({ }^{}\) See COVID 2023/598 en form for dettag, ss 38 till 57. If, see VSOI 2007-47. Den version og Infusivta røntruden - on förbättradsardom av ingvall fortifit. I och med medigare, \(^{9}\) \\ \({ }^{1}\) Seld barnat arseumtforeach från AstraZeneca. Kältpetelut vinnom, erentpränskär och riskapät. \\ \({ }^{2}\) Europapartements och rädets fordetning (EU 2022/688 av den 30 maj 2022 om personmarktionning och en användning av forordning (EU 2018/1744. \\ \({ }^{3}\) Umpenning och medigare, 2021/2017/2394 och direktiv (EU 2020/1928 (data/odfordinatning) \\ 4) Umpenning och medigare, 2021/2018/2394 (data/odfordinatning) \\ 5) Umpenning och medigare, 2021/2019/2394 (data/odfordinatning) \\ 6) Umpenning och medigare, 2021/2016/2394 (data/odfordinatning) \\ 7) Umpenning och medigare, 2021/2014/2394 (data/odfordinatning) \\ 8) VÖR 2004-381 Den form for datdelning.
}

\title{
Amerikanska Al-regelverk
}
I kraft av sin dominans inom tech-sektorn och sin ekonomiska styrka år de amerikanska regelverken av särskild betydelse och blir en faktor i den internationella konkurrensen. Det amerikanska regellandskapet är komplext eftersom det omfattar den federala nivån, delstatsnivån samt olika myndigheters riktlinjer och domstolarna.
Antalet amerikanska Al-regelverk har ökat under de senaste åren. 2023 fanns det 23 Al-regelverk, att jämföra med ett enda 2016. Bara under 2023 växte antalet regelverk med 50 procent. Kalifornien är den delstat med flest Al-regelverk (7) följt av Virginia (5).
På federal nivå finns det framför allt så kallade Executive Orders. Den mest välkända är Executive Order 14110 On the Safe, Secure, and Trustworthy Development and Use of Artificial Intelligence. Den innehåller atta generella principer för att styrta utvecklingen och användningen av AI, särskilt av federala myndigheter och utvecklandet av foundation models. Den ger dessutom vissa federala myndigheter i uppdrag att utveckla ytterligare AI-specificka riktlinjer och regleringar. USA har också antaigt en Bluegrint för an AI Bil/ of RighTS som är ett tike-bindande ramverk med principer för design, användning omförvar är att för att för att för att för att för att för att för att för att för att för att hjälp olika
göngalisationer bennäta Al-risk
På delstatsnivå har flera Al-lagar införts de senaste åren, många med mer specificika krav. Colorado var den förstaa amerikanska delstaten att införa en omfattande Al-lag år 2024. I Kalifornien måste till exempel politiska annonser ange om de använt Al för att skapa bilder. Nyligen infördes också en lag om transparens vid utveckling av generativ AI. I Michigan och Washington finns liknande krav på all Al-generard marknadsföring, oavsett om de var avsedda att misleda eller inte. I New York är användningen av automatiserade verktyf för anställningsbeslut förbjuden under särskilda omständigheter.
Evan-samarbetet och arcså antagit and praa gemen samma regelverk med betydelse för utvecklingen av AI, till exempel regler för detaskydd och användning av data. GDPR-förordningen (General Data Protection Regulation) från 2018 är kanske den mest välkända och av resaded att skydda enskildas personuppgifter. Öppna data-direktiver hat gept upthov till lagen om den offentliga sekorts tillgängliggörande av data (2022:818). Syftet är att olika aktörer i samhället ska kunna använda offentlig information för att skapa nya produkter och tjänster. Dessutom finns data Governance Act (DGA) som trädde i kräft i september 2023 och är avsedd att reglera frivillig datadelning. I januari 2024 kompletterredes DGA med Data Act, som klargör vem som har rätt att skapa värde av olika data och under vilka förhållanden. För att att ten kortret exempel - vem åger rätten till den data som din upppkolpade tvåtmäxsnel iler kylskåp generarer?
I diskussionen om balansen mellan att fokusera på säker Al och främnåde av innovation jämförs ofta amerikanska och europeiska regelverk, se faktaruta Amerikanska Al-regelverk för en beskriving av ame rikansk Al-reglering. Hårfins påtagliga skillnader. EU har valt att sätta stratta regler för Al-system med hög risk och förbjörder dessutom visas Al-tillämpningar. USA har däremot valt en mer decentraliserad och case-by-case ansats med större fokus på bästa praxis och frivilliga industrastardar. En annan viktig skillnad är att brott mot EUs Al-förordning kan leda till avsevärda bötesummor vilket tidigare inte varit fallet i USA. I till exempel Kalifornien finns dock nu lagförslag som också innehåller stora bötesbopløp.
Men EU-samarbetet handilar inte bara om regler støytet r också att gynna europeisk forskning och innovation. Som EU-medlem har Sverige till exempel möjlighet att delta i och få finansiering av olika Al-forskningsprogram och upplygbgnaden av beräkningskraft i form av spuroderator. Hur Sverige utnyttjär den här möjligheten diskuteras längre ned i det här kapitlet och i kapitlet Beräkningskraft.
EU-kommissionens ordförande Ursula von der Leyen underströk i sina politiska riktlinjer som hon dale fram i juli 2024 vikten av att EU blir en global ledare in Al-innovation. För att uppnå detta ska supderatordre göras tillgängliga för Al-startups och förettag genom ett AI Factories initiatve. \({ }^{1917}\) Kommissionen avser också lågga fram en ny Apply Al Strategy. Syftet är att främja nya industriella användningsomsråden för Al och förbättra tjänster som tillhandahålls av den offentliga sektorn, till exempel inom sjukvården. Vidare är avsikten att sätta upp ett European Al-

förf att säkerställa svensk spetsforskning inom Al och avancerad tillämpning. En sådan satsning ska ses som ett komplement till Knut och Alice Wallenbergs Stiftlees satsningar på superdattorn Berzelius, men bör givetvis samordnas för att säkerställa en långsiktning relationelt uvelcking av konkurrenskraftiga resurser och tjänster.
\section*{Förslag}
- Al-kommissionen bedömer att staten behöver tillföra Vetenskapsrådet 300 miljoner kronor som en engångsinvestering för utveckling och tränning av Al-modeller. Därefter behövs 25 miljoner kronor årligen för att hålla systemet konkurrenskraftigt.
Även behovet att i nästa steg kunna tillämpka trånade Al-modeller väver. Resultatet av att träna en Al-modell är att de förses med en uppsättning så kallade paratmetrar, som för stora modeller kan vara hundratsal miljarder. Den svenska språmkodellen GPT-SW3, som tränats på superdattorn Berzelius, har 40 miljarder parametrar. De senaste kommersiella modellerna har dock väsentligt fler parametrar, och storleken ökar för varje ny version.
Vid träningen av Al-modeller måste användarna ofta ställa sig i ki för att få tillågg til beråkningskraft. I dagslåget kan det på Berzelius ta flera dagar av köande för att få göra beråknaringar, eftersom datt normalt sett är fullt belastad dygt renutet.
Vid användningen av trånade Al-modeller används parametrarna för att bearbeta information som matas in. Som exempel på detta har GPT-SW3 tränats på att för förutsågå nästa ord i en sekvens, och på så sätt skapa nya texter. För detta krävs beråkningskraft som har möjlighet att mycket snabbst våra på de frågor som användaren ställer. För radgavlig interaktiv antvändninggäller det för dattom artkta var inom bråtkeden in en sekund och till många användare samtidigt.
För tillämpning av Al-tjänster krävs det således en dattor som har tillräcklig kraft att thantera den omedelbara interaktionen mellan dator och användare. Interaktionen med till exempel Al-verktyget ChatGPT Skullie avsevärt minskia i värde om det tog flera minuter för användaren att få ett svar. I framtiden är det sannolikt att vi kommer vara i större behov av beråkningskraft för användning av Al-modeller än för träning av dem.
I Sverige saknar vi dock i dag en samlad beräkningskraft viggd för användningen av Al-tjänster, så kallad inferens. Det är Al-kommissionens bedömning att storskalig inferens-anpassad beräkningskraft nu bör byggas upp för att skapa möjligheten att tillhandåhla svenska Al-baserade användartjänster, Sådana insatser bör givetvis vägas mot molbinserade kommersiella tjänster som erbjuds och fokuseras på de användningar som av olika anledningar inte är aktuellt för molntjänster. Denna beräkningskraft ska vara tillgånglig både för näringsliv och offentlig sektor, som har stora behov av att driftsätta system för att erbjuda företagslönsingar och samhällsservice.
\section*{Förslag}
- Al-kommissionen bedömer att staten behöver tillföra Vetenskapsrådet 200 miljoner kronor en engångsinvestering för att upppradera och utöka befintlig beräkningskraft för användning av Al-tjänster. Därefter behövs 25 miljoner kronor årligen för uppradering och utveckling, En sådan satsning ska även bidra till att täcka behoven av kapacitet för tillämpkningsberäkningar i privat och offentlig sektor. I detta sammanhang är det viktig att noga följa utvecklingen av molntjänster, så att mizenv at bygga egen beräkningskraft och att köpa molntjänster hela tiden är optimal.
\section*{Närginsliv}
Det privatne näringslivet tillgodoser till stor del själva sina behov av beräkningskraft. Svenska storföretagjor och räg slåglaget stora investeringar i egen beräkningskraft, medan små och medelstora företag (SMF) tillgostors sina behov främst via molntjänster och lokala servrar. Detta fungerar tillfredsställande för enklaere och mindre resurskravande beräknangier. Däremot kan vissa SMF, vars verksamhet är berende av beräkningstuga allportimer, behöva beräkningskraften hos en superdorter. Ur dessa företagts perspektivé rore det opmittalt att samordna en eskilda investeringarna i en gemensam beräkningskraft eller upphandling av polnittjänst. Erfarenheten visar dock att detta är svårt att genomföra i praktiken, då antalet företag som behövs för att upppnå praktisk massa är alltför många för aktordniane investeringen. Beräkningskostnader kan därmed skapa en barriåför den el SMF att utnytfåj A. Denna situation har telt till att vissa länder tvat att+kolsera på SMF i sina offentliga satsningar på beräkningskraft för Al: \({ }^{}\)
\section*{Dorseren samte tid har flera företag betalerat datacenter i Sverige, nu senast i Falun och Borligen. \({ }^{}\) Datta

\title{
Al-kommissionens
}
\section*{Fårdplan}
for Sverige

I vårt framtida samhälle är Al ett verktyg i medborgarnas tjänst. Foto: Gorodenkoff/Shutterstock
samarbeten mellan akademi och närlingsliv bidrar det till att svensk forskning och innovation blomstrar och drar till sig internationell kompetens. Svenska aktörer är krafter att räkn med i det internationella samarbetet för att lösa viktiga samhällsproblem.
Människors arbetsuppgifter kommer förvisso att förändras, men oftast till det bättre, med mer intressant och givande innehål och färre repetitiv och monotena inslag. I vissa fall kan uppgifterna försvinna helt: I båda fallen finns det stöd att få inom ramen för ett välfungerande system för omställning, med fokus på skydd och vidareutbildning av individen. Det bidrar till ett varbea och stärka tillitien i samhället.
I vårt framtida samhälle är Al ett verktyg i medborgarnos tjänst.
Sverige har goda förutsättningar . . .
Den här positiva samhällsbilden är en vision. Men det är inte en utopi. Sverige har en lång historia av att gå stärkt genom tekniskkiften - vi har investerat inför framtiden, ofta på osäkert faktaunderlag, men med just en vision om att investeringarna ska bära frukt. Genom stora offentliga investeringar järnvägen kunde vi i mitten av artonhundraatlet påbörja transformationen från att vara ett av Europas fattigaste länder till dagens västånd. Med öppenhet för föråndring och förmåga att dra nyta av de fördelar som my teknikn erbjuder har vi stärkt vår konkurrenskraft och höjtvårt välstand. Vi har gjort det i samförstånd och genom samarbete mellan alla delar av samhället.
... men vi halkar efter
Den positiva utvecklingen vi just har beskrivit är emeltertid inte något vi kan ta för givet - tvårtom. Vårvision kräver stark politiskt ledarskap och insikt om att vi står inför ett vägskål där vårt framtida välstand i höggrad kommer att avgöras av hur väl vi lyckas dra nyttta av Als möjligheter och hantera dess problem.
Dessvärre verkar den här insikten inte ha slägit rot i samhället. Al-kommissionen kan i stället konstatera att Sverige halkar efter i utvecklingen. Det är en

<|im_end|>};

\title{
Al och samhällets säkerhet
}
\section*{ChatGPT sammanfattar:}
Als framsteg öppnar enorma möjligheter för att stärka samhällets säkerhet, men de medför också nya frisker.
Detta kapitel undersöker hur Al kan användas både som ett kraftfullt verktyg för att skydda vårt samhälle och som en potentiell riskfaktor som kan utnyttjas av illvilliga aktörer. Med rätt tillämpning kan Al bidra till att förebygga brott, bekämpa cyberangrepp och stärka vårt försvar. Samtidigt ökar det vår särbarhet för cyberattacker, desinformation och autonoma vapen, vilket kräver proaktiva åtgärder och robusta säkerhetssystem. Här presenteras hur Sverige kan hantera dessa utmaningar genom att utveckla Al-teknik på ett sätt som stärker samhällets motsståndskraft och säkerhet.
\section*{Yttre hot - det säkerhetspolitiska läget}
Sverige och Europa befinner sig i det allvarligasaste säkerhetspolitiska läget på många årtionden - fördelmligen sedan andra vårdiskrigets slut. Rysslands invasion av Ukraina 2022 och den kraftigt försämdrute situationen i Mellanösten under det senaste dryga året r tvast kärt bidragande faktorer. Dagens allvarliga säkerhetsläge förvåntas bestå, eller förvärras, under överskåldig till. \({ }^{}\) Försvångar, för att skärta samhället, förvärt med med med med med med med med med med med med med med med med med med med med vidavårt samhället gjöver-ycberområdet till en arena för hot och angrepp, exempelvis riktade mot förettag och finansiella system som har samhällskritiska funktioner. \({ }^{}\) Samhällets var sbøver vare att våbde möta dessa noch att och själva använda Al för att göra samhället säkare. För de myndigheter som har uppdraget att tvåma Sveriges säkerhet är Al rend att viktigt verkty.
Ny teknik skapar nya bereorden- om försjörningsbederskap och digital suveränitet Det förämsrade säkerhetspolitiska lägt är en faktor som har viktiga implikationer för de nya bereorden som växer fram i utvecklingen al \(\mathrm{V}\) Al. För att kunna att villaar de må myjligheter som den senaste tek utnikvecklingen för med sig krävs - som näms i kapitel Tjängåt till intentionarella Al-surresir och Beräkingskraft - tillgängt till beräkningskraft i form av särskilda datorer. Beräkingskraften i värden är di jag huvudsak delgåen i USA och Kina samt i ett fåtåal europeiska länder. Tilkringen av dessa datorer är i sin tur boreande av tillgängt till hvalledre och sällysnta jordartsmetaller som benernt finns på vissa platser. Vi ser också hur för ståsta hand USA, och i andra hand
Kina, dominerar den globala - och i stor utsträckning också den europeiska - techmarknaden. Exempelvis ägda si te åtörsta onlineplattformarna i Europa av amerikanska (sex) eller kinesiska (fryra) bolag. Dessutom kontrollerar tre amerikanska aktörer i dagslåget tvåt vredjelader al ven de europeiska molntjänstmarknad, medan europeiska bolag endast står för två procent. \({ }^{}\)
I takt med att Sveriges Al-förmåga blir alltmer betydeself till för vår säkerhet växer alltså nya bereoden fram. Tillgången till beräkningskraft samt teknik, kunskap och råvaror som är nödvändiga för att bygga sådan infrastruktur är central. En situation där betydande delar av den globala beräkningskraften befinner sig under kontroll av stater som är fientligt inställda till Sverige skulle vara kart negativ för vår säkerhet. I vilken mån Sverige, ensamt eller tillsmännas med andra, har tillgängt till eller möjlighet att bygga sådana förmågor är därför av betydelse för Sveriges säkerhetspolitiska situation.
Dessa frågor behöver integreras i arbetet med att betalera en god försörjningsbederskap viser. Eigaretta skerställa att vår Al-förmåga kan upprätthållas även oligale tid är Ikä viktig som att industrin fortsätter föres med insatsvaror eller att lete- och datakommunikationsnäten fortsätter fungera. Tyvärr visar erfarenheter från coronapandemin att i ett rättgåt läge riskeran solidaruserten mellan allerade länder och inom EU att krackelera. Då handlade det om sådant som vaccin jord och personlig skyddsutrustning, men i framtiden kan det lika kärna handla om beräkningskraft eller

\title{
Figur 5. Forskning - ranking och poäng
}
\author{
USA - plats 1 \\ Kina - plats 2 \\ 54 \\ Singapore - plats 3 \\ Schweiz - plats 5 \\ Israel - plats 7 \\ Nederländerna - plats 15 \\ Finland - plats 18 \\ Sverige - plats 19 \\ Norge - plats 23 \\ Danmark - plats 22 \\ 0 \\ Poång \(\bullet\) KPI
}
\section*{100}
54
\section*{Poång för plats 10}
\section*{15}
\section*{20}
40
60
80
100
Notera: Placeringen för varje land inom Forskning visas efter landets namn. Den horisontella axeln visar poången för varje land, beräknad utifrån indikatorer relaterade till området. Den högsta poång som ett land kan få år 100. Den grå stæpel representerar den poång som kriks för att placera sig på plats 101 Forskning i 2004 årsøpp ulpaga.
Källa: The Global AI Index, 2024 års upplaga.
Området Forskning bygger på ett salt antal underliggande indikatorer, bland annat uppgifter på antalet STEM-forskare, utveckling av tongivande Al-system, publiceringar om Al i akademiska diktskirfer, deltagande i akademiska Al-konferenser, samt kranteringar av universitet och forskare inom datavetenskap. Dessutom ingår aggregerade mått på hur mycket länderna generellt presenterar på forskning och utveckling.
Kaplitet Spetsforskning i samverkan innehåller förslag som kan ståkra Syreniges position inom Al-forskning Centrala åtgärder är att etablera exceltenescentring. Al i samverkan mellan akademi, privat och offertig sektor, gåstprofresurer samt forskarskolor. Sveriges förhållandelvis goda tillågng till data kan också vara en avgörande faktor för att behållå och attrahera forskare från andra länderen, I kapitet Data som förutsättning för Al-utvecklinge diskuterar vi hurs dessa kan bli mimer flängliga med behållen respekt för personlig integritet och upphovvärt.
Vad kan då vara en ambitiös och realistisk målsättningen för området Forskning? Med tanke på de omfattande förslagen vi lägger I denna Färdplan så borde Sverige kunna rankas bland de 5 främsta länderna år 2030.
- Sverige ska senast år 2030 ha gått från plats 1 till att tillhöra topp 5 i området Forskning
\section*{Kommersialisering}
Kommersialisering handlar om att omvandla innovationert till produkter och tjänster som skapar värde på marknaden. En stark position på detta område är avgörande för Sveriges framtda konkurrenskraft etår inom detta område innovation förvandlas till produkt eller tjänst, nya företag föds och investeringar driver på tillväxt och jobbskapande. Inom området Kommersialisering ligger Sverige på plats 18.
Figur 6 visar Sveriges ranking och poång i relation till jämförelseländerna. Figuren visar att Sverige ligger på en liknande nivå som Finland, Schweiz och Norge. Utveroptpländerna USA och Kina ligger även Israel och Singapore langför Sverige och övriga jämförelseländer.

Nedan visas de upskattade kostnaderna för våra förslag, uppdelat efter if vilket kapitel i Färdplanen förslagen återfinns. De största resurstillskotten föreslås till olika satsningar på spetsforskning. Andra betydande satsningar föreslås för offentlig sektor form av Al-verkstaden, ett kunskapslyft och bera:ningskraft. Det handlar om förslag som bör kunskapets, för att det av den betslutas snabbt. Härutöver lägger vi också förslag om ett antal utredningar avseende frågor som ännnar inte råd reö for betsl. För några av dessa utredningar har vi inkluderat en inklåtav kostnad it bællen, dock inte för alla. Det är därför vittigt att regeringen har beredskap för törtlerage kostnader när utredningsförslagen läggs. Innov sissa omåden, som exempelvis energiet och telekom föreslår vi inga satsningar inulägset. Det är dock viktigt att regeringen följer utvecklingen noga på dessa områden och är beredd att død vita dåtgärder vid behov.
Det är också viktigt att förstå att de uppskattade kostnaderna för de föreslagna åtgärderma är bruttokostnader. Al-kommislionsen är övertgad om att förslagen sammantaget kommer att medföra betydande besparingar och inntäksöknigan, genom högre tillväxt och produktivitet i samhället. Vi presenter har inga aggregerade uppskattningar över dessa effekter, eftersom vil bedömer att det inte finns några tillräckligt envyliga och tillförltliga sådana att tillgå i dagsläget. Vår uppfattning grundar sig i erfarenheten av tidigare tekniksliften, samt de exempel på besparingar och effektivitetsöhnjningar som vi redogör för på olika ställen i Färdplänen.
\section*{Tabell 1: Kostnaderna för våra förslag per område (mnkr)}
\begin{tabular}{lccccccccc}
\hline Område & \(\mathrm{Ar} 1\) & \(\mathrm{Ar} 2\) & \(\mathrm{Ar} 3\) & \(\mathrm{Ar} 4\) & \(\mathrm{Ar} 5\) & \(\mathrm{Ar} 6-10\) & Totaltår 1-10 & \\
\hline Beräkningskraft & 845 & 165 & 165 & 165 & 165 & 250 & 1505 & 1755 \\
\hline Data & 4 & 4 & 4 & 4 & 4 & 20 & 20 & 40 \\
\hline Säkerhet & 130 & 80 & 80 & 80 & 80 & 400 & 450 & 850 \\
\hline Spetsforskning & 655 & 655 & 655 & 655 & 656 & 3275 & 3275 & 6550 \\
\hline Kompetens & 508 & 563 & 607 & 217 & 217 & 85 & 2112 & 2197 \\
\hline Innovation & 108 & 108 & 108 & 108 & 109 & 40 & 540 & 580 \\
\hline Öffentlig sektor & 157 & 512 & 512 & 512 & 512 & 60 & 2205 & 2265 \\
\hline Internationella positioner & 257 & 209 & 211 & 213 & 215 & 1075 & 1105 & 2180 \\
\hline Ledarskap och styrning & 35 & 35 & 35 & 35 & 35 & 0 & 175 & 175 \\
\hline Schablon för utredningar & 60 & 60 & 0 & 0 & 0 & 0 & 120 & 120 \\
\hline TOTALT: & 2759 & 2391 & 2377 & 1989 & 1991 & 5205 & 11507 & 16712 \\
\hline
\end{tabular}

Ytterligare en satsning för att främja innovationskratten bland SMF är EU-kommissionens nya koncept som kallas för Al Factory. Foto: Gorodenkoff/Shutterstock

Det har dock redan skett vissa framsteg inom detta område. Band annat införskaffade Knut och Alice Wallenbergss Tiftelse superdatorn Berzelius, som används till satsningar på grundforskning. Det görs bland annat inom forskningsprogrammet Wallenberg Al, Autonomous Systems and Software Program (WASP). För mer information om WASP, se sida 54.
Berzelius är placerad vid National Academic Infrastructure for Supercomputing in Sweden (NAISS) vid Linköpings universitet. \({ }^{}\) Fokus för verksamheten ligger på grundforskning inom akademin, även om företag har en viss tillgång till beräkningskraft via forskningssamarbeten.
\section*{Sommaren 2023 stod det också klart att Sverige} kommer att vara huudman för en av superdatorema inom ramen för EU-organisationen EuroHPC ju. Ut
Målet för EuroHPC ju. Vå rå bland annat att stäkra EU:s tillgång till beräkningskraft genom att koordinera och samla Europas superdatorer. Budgeten är på totalt 2,1 miljarder euro (knappt 24 miljarder kronor). EuroHPC Ju bygger på samfinansierung, vilket innebår att EUmatchar den finansierung som medlemsländerna sälva bidrar med ed. \({ }^{100}\)
Arrhenius, som superdatorn kallas, kommer att vara belägen vid Linköpings universitet och tas i drift 2025.
EuroHPC ju täcker i dagslägt 35 procent av drift-och finansieringskostnaderna. Resterande kostnader finansieras i dagsläget av Vetenskapsrådet och andra svenska aktörer som tillsammans bidrar med 510 miljoner kronor. \({ }^{}\)
Det är dock viktigt att notera att Arrhenius kommer att ersätta de nuvarande dotarer som används för tradiationella tekniskt vetenskapliga beräknangir i Sverige, Även om den kommer att innehålla en mindre andel GPU-er är den inte optimerad för storskalig Al-trängen. Den inledande fasen av Arrhenius kommer i stället att behövas för att tillgodose traditionella behov, till exempel inom fysik, kemi, klimatvetenskap, biologi och medicin.
Men Arrhenius är oavsett en viktig förutsättning för utvecklingen av AI service. Detta eftersom den, ikratt av att vara knuten till EuroHPC ju, Öppnar möjligheten för Sverige att ansöka om en så kallad Al Factory (see fakturuta på sida 34 för mer information om AI Factory). Om en svensk anskön blir framgång�rik kommer det att medföra investeringar i beräkningskrakt och kompetens som är ämnad för Al.
Framåt ser Al-kommissionen att det är av yttersta vikt med en fortsatt investering i resurser för utveckling och tråning av Al-modeller för svenska forskare. Detta
\footnotetext{
\({ }^{}\) NAISS är en organisation för superdatoren och beräkningskraft i dag är verksam vid elva av Sveriges universitet och högskolor, och man handhar en rad av Moraistenas suerdatoer.
\({ }^{}\) För en mälarare redogörelse för ELUs satningar på Al, se kapitel Internationala positioner.
\({ }^{}\) Vetenskapardädet har avsett 200 miljoner kronor för en satsning på Arrhenius under fem år (2025-2029).
}

Superdatorn Berzelius är placerad vid National Academic Infrastructure for Supercomputing in Sweden (NAISS) vid Linköpings universitet. Foto: Thor Balkhed/Linköpings universitet

\title{
Spetsforskning i samverkan
}
\section*{ChatGPT sammanfattar:}
Al-forskningens utveckling går i rasande fart och kopplar allt tätare samman grundläggande vetenskap och praktisk tillämpning. Sverige har en stark tradition inom både forskning och innovation, men den globala konkurrensen kräver nu nationella satsningar för att bibehålla och stärka vårt ledarskap inom Al.
Detta kapitel utforskar hur spetsforskning, näringsliv och utbildning kan samverka för att påskynda Al-utvecklingen i Sverige. Genom att främja excellenscenter, internationella samarbeten och stöd till unga forskare ges konkreta förslag på hur vi kan bygga den kompetens som behövs. Det är en plan för att säkerställa att Sverige står starkt i framtidens Al-kapplöpingn.
Utgångsläget är relativt bra, men det finns utmaningar
Under de senaste åren har utvecklingen av Al gått oerhört snabbt och nya rön har lett till Al-baserade tillämpningar både inom förväntade och helt nya områden. Ett fenomen som genomsrar yarl-U-tuveklingen är det korta avståndet mellan grundläggande forskning, tillämpning, innovation och produkt. I innovationssammanhang talas det ofta om TechnoJogy Readiness Levels (TRL) för att skrivha hur lång på vägen mot produkt en upptäckt eller innovation och avlängt. Vi att skrivta av att skrivta av att skrivta skalan kan nu inom Al-området gå på bara några månader. Detta leer till att företag som vill liggia framkant inom Al måste vara oerhört forskningsnära genom att antingen bedriva egen grundläggande forskning eller koppla sig till ledande forskning vid universitet och högskolor. Det står i ljuset av detta helt klart att spetsforskning inom Al är avgörande för att Sverige ska kunna behålla och stärka sin konkurrenskraft inom en rad för landet viktigta tillämpningsområden.
I en internationell jämförelse har Sverige länge håvdat sig väl, både på forsknings-och innovationsmrådet. För närvarande pågår också ett kompetenslyft inom forskning och utbildning inom Al med flera viktiga aktörer, varav Wallenberg Al, Autonomous Systems and Software Programme (WASP) är det mest betydande.
Utgångsläget för en satsning på spetsforskning inom Al i Sverige är därmed relativt gott. Samtidigt bedrivs forskningen i en internationellt accelererad konkurrenskontext, och för att inte halka efter ochsatt kunna delta i utvecklingen av Al måste nationella
fokuserade storskaliga satsningar göras både på kort och lång sikt.
En utmaning i en satsning på spetsforskning är att kompetensbasen inom Al i Sverige är begränsad på kort sikt. Vi måste därför rekrytera ledande Al-forskare från andra länder, samtidigt som vi tar vara på yngre talanger. På sikt kan nya doktor, genom satsningar såsom WASP, bli ett viktigt tillskott till den samlande spetskompetensen runt Al i landet. Enligt Al-kommisensoneen ska målet vara att ha excellent forskning på värdsfrönten inom Al i sig, samtidigt som vi satsar på ärmnesspecifik Al-kompetens inom breda vetenståkspområden, som natur-och teknikvetenskap, medicin och hålsa, samt humaniora och samhållsvever tenskap. Detta kapitel beskrivner en rad åtgärder för att säkerställa att viånå detta mål.
Jen-Hsun "Jensen" Hwang och Marcus Wallenberg mitt i superdatorn Berzelius på Linköpings universitet. Foto: Thor Balkhed/Linköpings universitet

\title{
Färdplan för Sverige | AI FÖR ALLA
}
\section*{Dessutom måste den samhällsviktiga offentliga verksamheten alltid fungera, is tora och små kommuner, regioner och myndigheter - ofta dygnet runt och under årets alla dagar. Upprätthålls inte lag och ordning eller försavret av våra territoriella gränser, får inte människor vård när de behöver, betalas inte pensoner ut eller om socialförsäkringens försäkringar för inkomstbortfall fallerar, ja då ärmnar själva grunden för det svenska samhällskontraket. Samtidigt måste har det av skor till att för att för att skor. \\ klara sitt åtagande under krävande förhållanden. \\ befinner oss eit tmycket allvarligt säkerhetspolitik \\ läge och precis har återhämtat oss från en panditor \\ illustrerat utmaningen. Den fortsatta digitala trænformationen av offentlig verksamhet måste därför ske \\ med tydliga krav på uthållighet och motståndskraft. \\ De här utmaningarna år extra problematiska eftersom \\ förutsättningarna skiljer sig avsevät mellan olika \\ aktörer. De statliga myndigheterna skiljer sig i både \\ storlek och uppdrag, medan regioner respektive \\ kommuner i grunden har samma uppdrag, men \\ väldigt varierande storlek. Det gäller i synnerhet \\ kommunerna. Trots det ska medborgarna få likvårdig \\ service avseottm de bor i Dorotea eller Stockholm. \\ För att lyckas med detta kommer det att behövsar \\ användningav al i Verksamheten. Vårt att notera år \\ att i dagslåget visar forskningen på betydande och \\ talltgande ojämlikket ih tillgängligheten till offentliga \\ tjänster i digital form. \({ }^{}\) \\ Samhantfattningsvis måste offentlig verksamhet \\ få förutsättningar att bli effektivare, snabbare, träff \\ sårreakre och mer robust - i stort som smått. Att svarva \\ upp mot den här förvåntningen kommer att vara \\ avgörande för medborgarnas tilltit till det offentligas \\ förmåga att hantera trygghet under omställning. \\ Samhälnsyttor om AI används \\ Vi kan i dag inte överblicka allt som kommer att vara \\ möjligt framöver. Det räder dock inget tivel om att \(\mathrm{AI}\) \\ kommer att spela en avgörande roll i den transforma \\ tion som offentlig verksamhet måste genomgå för att \\ möta utmaningarna. Al leder inte automatiskt till ökad \\ välfär, men rätt använd kommer AI att kunna lösa \\ problem, accelerera digitalisering och på så vis ge \\ stor samhälnsytta. Redan i dag används AI i offentlig \\ verksamhet för det andämål. \({ }^{}\) avä uppfattning år \\ att detta år något som måste göras i mycket större \\ omfattning framöver. Frågan är vilken tatt utveck \\ lingen ska ske. Al-kommissionsen uppsfattning år att \\ tempot behöver vara synverligen högt. \\ Exempel på angelågna områden där ökad Al-an- \\ vändning skapar stor samhälnsytta \\ Al-kommissionen vill ilyta fram några områden där \\ vi anser det vara särskitt angelåget att \(\mathrm{Al}^{2}\) er än ökad \\ användning och blir ett verktyg för att generera stora \\ samhälnsyttor. \\ Brettsbekämpning \\ Al måste i högre grad användas för att förbeygga, för \\ hindra och upptäcka brott. \({ }^{}\) Att exempel på använd- \\ ningnsomärde är dataanalser, för att identifiera \\ mönster in omut betalningar från välfärssystemen \\ (redin vid ansökningar) eller för att avslöja komplex \\ brottnsåkver. Ett annat exempel är användningen av \\ avancerade prediktiva algoritmer för att åf underlag \\ om vilka individer som kommunernas socialtjänster \\ bör prioritera för att förebygga kriminalitet. De fram- \\ tida möjligheterna bör kunna leda till en märkbar \\ ökning av tryggheten i samhället. Minskad brottslighet \\ skulle också ha tyvilliga ekonomiska effekter, utöver de \\ positiva följderna för individer.} \\ \begin{tabular}{l} 
\\
\\
\\
\\
\\
\\
\\
\\
\\
\\
\\ \\
\\
\\
\\
\\
\\
\\
\\
\\
\\
\\ & \\
\end{tabular}

\title{
Figur 6. Kommersialisering - ranking och poäng
}
\begin{tabular}{|c|c|c|c|c|c|c|c|}
\hline USA - plats 1 & 100 & & & & & & \\
\hline Kina - plats 2 & 48 & & & & & & \\
\hline Israel - plats 3 & 29 & & & & & & \\
\hline Singapore - plats 4 & 27 & & & & & & \\
\hline Finland - plats 15 & 13 & & & & & & \\
\hline Sverige - plats 18 & 12 & 26 & Poäng för plats 5 & & & & \\
\hline Schweiz - plats 20 & 12 & & & & & & \\
\hline Norge - plats 22 & 11 & & & & & & \\
\hline Nederländerna - plats & 10 & & & & & & \\
\hline 23 & & & & & & & \\
\hline Darnark - plats 25 & 9 & & & & & & \\
\hline & 0 & 20 & 40 & 60 & 80 & 100 & \\
\hline
\end{tabular}
\section*{Poäng KPI}
Notera: Placeringen för varje land inom Kommersialisering visas efter landets namn. Den horisontella axeln visar poängen för varje land, beräknad utifrån indikatorer relaterade till området. Den högsta poäng som ett land kan få år 100. Den grå stapeln representerar den poäng som krävs för att placera sig på plats 51 Kommersialisering 12024 års uppliga.
Källa: The Global A1 Index, 2024 års uppliga.
Området Kommersialisering utgår från indikatorer som antal Al-företag och Al-startups, tillgång till finanstellt kapital för dessa företag, antalet börsnoterater Al-förateg, samt förekomsten av så kallade "enhörningar" inom landets Al-sektor.
\section*{Aterigen handlar det både om absoluta och relativa indikatorer, där störst vikt läggs vid de absoluta indikatorerna. Som framgår av Figur 6 hindrar detta inte relativt små länder, som Israel och Singapore, att placera sig i den absoluta toppen.}
Kommersialisering, liksom innovation, är starkt beroande av ett välfangerande Al-skeystem. Sverige förutsättningar borde därmed stärkas genom de förslag som finns i Årfdplanen. I exempelvis kapitlett Innovation, entreprenroskap och riskkapital föreslås ett utköt stöd för livskraftiga Al-startups. I kapitlett, Beräkningskraft, föreslås också etableringen av en så kallad Al-Factory I sverige, ett EU-initiativ som ska ge små och medelstora företag tillgång till avancerad infrastruktur och kompetents till ett kraftigt subventionerat pris. Sverige bör dessutom aktivt verka i EU-förhandlinglar för att säkerställa att Al-reglering inte hämmar konkurrenskraften eller skapar onödig regleringsbörda, vilket beskrivs närmare i kapitlet Internationella positioner.
Vad kan då vara en ambitiös och realistisk målsättningen för området Kommersialisering? Ser man bortom Al och tittar på samhället stort, så har Sverige redan välutvecklader riskkapitalmarknader och ett relativt starket ekosystem för startups. Färdplanen lägger också en rad förslag att förstärka och kompletera det som redan finns. Detts kabzar en god grund att närmä sig toppnationerna. Sverige borde därför kunna rankas bland de 5 främsta länderna år 2030.
- Sverige ska senast år 2030 ha gått från plats 18 till att tillhöra topp 5 i området Kommersialisering
\section*{Talang}
Talang handlar om tillgång till kvalificerad arbetskraft inom Al, vilket är en avgörande faktor för ett lands konkurrenskraft. En bred bas av krykesverksamma med Al-kompetens är nödvändig för att integrera tekniken hållbart och effektivt i samhället. Inom området Talang ligger Sverige på plats 15.
Figur 7 visar att Sverige har en liknande ranking som grannländerna Finland och Danmark. Vi ligger långt efter de tre optrankade länderna, i synnerhet USA. Betydande förbättringar krävs också för att nå upp till Singapore och Schweiz som har nästan dubbelt så höga poång som Sverige.

När en gemensam kärninfrastuktur ska byggas kommer det också vara viktigt att så snart som möjligt utveckla en finansieringsmodell där en lämplig andel av kostenderna för drift, förvaltning och vidareutveckling finansieras genom avgifter. En sådan avgiftsmodell behöver vara utformad så att al/A aktörer i offentlig förvaltning - på såväl stalt som regional och kommunal nivå - har råd att vara med och utveckla samt använda de Al-drivna tänster som görs tillgångliga genom Al-verkstadden. Det är dårför i rimligt att difterientera en sådan avgift utifrånd etlagarens storlek samt i vilken utsträckning deltagaren använder sig av Al-verkstadens olika funktioner.
Vid skapande av en gerännfrasturser, med att skapande av en ställäga frågatur för med att skapande av en ställäga frågatur, ingnar, Här märks till exempel konkurrensfrågatan, staten, i form av Försäkingskassa och stäkante av skåter. Vi kala slevera en tjänst till kommuner och stäkante. Det perspektivet är det viktigt att notera att Al-verkstaden de kommer att fylla ett tylligt behov som inte täcktande dagslåget. En Al-verkstad behöver dessutom uppfylla högt ställa krav på informationssäkerhet, vilka upp det vikdångia inmod den offentliga förvaltningen. Då Al-verkstaden dessutom bidrar till den svenska avtänretisten och vårt civila försvår õlar Al-verkstaden inte betraktas som konkurrenspebgränande på ett sätt som är oförenligt med gällande lagstiftning.
\section*{Al-insatstyrka och anslutningsstöd}
Att rekrytera Al-kompetens till verksamheten är en utmaning, särskilt för mindre myndigheter, kommuner och regioner. Det riskerar att göra det svårt för dessa att dra well yntta av potentalen med Al - inklusive att utnyttja Al-verkstadden. Det bör dårför skapas en Al-insatstyrka - särskilda team av experter och generaristel mer med updrag att stätto offentliga aktörer med Al-kompetens på plats i verksamheten. \({ }^{}\) Sådana team kan sätts samman av personer som till vardags arbetar in omolika delar av den offentliga sektorn, och aktiveras vid behov. Al-kommissionen anser att det är mest ändamålsenligt att de myndigheter som ges ansvar för att betarla Al-verkstaden också får i uppdrag att inrätta och administrera Al-insatstyrkan, även om fler aktörer kan och bör bidra med medemarmar till den.
Det kan också behövassärskil hjälp när en ny aktör vill ansluta sig till och börja arbeta i Al-verkstaden. Ett sammanhållet och anpassat anslutningsstöd behöver dårför vara en integrerad del av verkstadens tjänsteerbjudande. Det kan dels handla om att lösa tekniska frågor, dels om att stötta med att kartlägga behov och möjligheter och sedan hjälpa verksamheten att komma I gång med förändringarsbetet.
Stödet bör även omfatta utbildning av medarbetare hos den nyligen anslutna aktören, vilka i sin tur ska kunna utbilda vidare inom sin egen organisation. Al-kommissionen anser att det är mest ändamålsenligt att demyndigheter som ges ansvar för att etabler och förvalta Al-verkstaden även får i uppdrag att tillhandahålla anslutningsstöd.
\section*{Förslag}
- Al-kommissionen föreslår att Försäkingskassan och Skatteverket ges i uppdrag att tillsammas etablera och förvalta en Alverkstad för offentlig sektor och fungera som leverantörsmydigheter för denna. I uppdraget ingår också att tillhandahålla ett anslutningsstöd och tillsammans med andra lämpliga aktörer inrättå och administrera en Al-insatstyrka. Myndigheterna ska utredade det rättsliga frågeställningarna och vid behov lämna författningsförslag för etableringen av ALverkstaden. Det finns fördelar med att i författning tydligögra leverantörsmydigheternas uppdrag.
- Al-kommissionen anser att leverantörsmydigheterna ska tra fram en stuktur för styrning av arbetet med att utveckla och förvatta Al-verkstaden. I det arbetet år det viktigt at det heterogena förutsättningarna i myndigheter, kommuner och regioner beaktas.
- Al-verkstaden föreslåts etableras grestgis under en femärsperiod, 2025-2029. Under de två första åren kan grundfrörmågör sättas och infrastrukturen etableras. Redan under de första åren kommer visas vårdeskapande Al- tjänster kunnta tras fram och börja användas. Den fulla uppskaliningen sker under år tre till fem.
- Kostnaden beräknaps uppåtil 1145 miljoner kronor det förstaår året och därefter 500 miljoner kronor per år, det vill säga totalt 2145 miljoner kronor. Det är möjligt att fördela denna förstaår året och därefter 500 miljoner förstaår sig till kostnad.
- All-kommissionen föreslår dock att betaleringskostnaden för finansieras helt genom anslag, i kombination med lånefinansiering där så är lämpligt, för att särksetälla en hög och förutsbetar katt uppybggnaden samt låga trösklar för stora som små aktör att natsla sig till samarbetet. Som nämns avon ska Al-verkstaden efter att den är full teltablerad huvudsakligen finansieras med avgifter från de deltagande aktörerna.

\title{
Figur 9. Global AI Index (GAlI) - ranking och poäng
}
\begin{tabular}{|c|c|c|c|c|c|c|}
\hline USA - plats 1 & \multicolumn{6}{|l|}{100} \\
\hline Kina - plats 2 & \multicolumn{6}{|l|}{54} \\
\hline Singapore - plats 3 & \multicolumn{6}{|l|}{32} \\
\hline Israel - plats 9 & \multicolumn{6}{|l|}{26} \\
\hline Schweiz - plats 12 & \multicolumn{6}{|l|}{20} \\
\hline Nederländerma - plats 13 & \multicolumn{6}{|l|}{20} \\
\hline Finland - plats 15 & \multicolumn{6}{|l|}{19} \\
\hline Danmark - plats 22 & \multicolumn{6}{|l|}{16} \\
\hline Sverige - plats 25 & \multicolumn{6}{|l|}{16} \\
\hline \multicolumn{6}{|c|}{16} \\
\hline \multicolumn{6}{|c|}{} \\
\hline
\end{tabular}
Notea: Placeringen för varje land inom GAlI visas efter landets namn. Den horisontella axeln visar poängen för varje land, beräknad utifrån innaktorer relaterade till området. Den högsta poäng som ett land kan få år 100. Den grå stäpeil renespenterar den poång som krävs för att placera sig på plats 101 (GAlI 2024 åræ upplaga.
Källa: The Global Al Index, 2024 åræ upplaga.
En övergripande KPI för Sveriges utveckling inom Al-området bör baseras på förvärtningarna som finns inom de sju olika områden som mäts i GAIL. Sveriges ranking inom dessa kategorier varierar, med starkar placeringar inom Operativ miljö och till viis med i Talang, men betydande förbättringspotential inom Politisk styrning och Utveckling. Genom att prioritera de föreslana gårdärena i denna Färdplan bör Sverige kunna förbättra sin ranking ganska markant inom 5 år. En ambitiös och rimlig nivå år därför att Sverige bör kunna rankas bland de 10 främsta länderna år 2030. Detta skulle spega en strategisk satsning som stärker landets konkurrenskraft och kapacitet inom Al på global nivå.
- Sverige ska senast år 2030 ha gått från plats 25 till att tillhöra de 10 högst rankade länderna i GAlI
Avslutningsvis kommer länder i vår omvärld också att genomföra strategiska satsningar på Al-området under kommande år. För att behålla och stärka vår konkurrenskraft är det därför viktigt att vi bevakar utvecklingen noga och är bedreda att vidta ytterligare åtgärder när så behövs. I detta sammanhang är det också viktigt att betona att alla indikatorer har sina för-och nackdelar. De kan därför inte förväntas ge en perfekt bild av den underliggande utvecklingen. De föreslagna inmidatorerna ska därmed inte ses som någon absolut sanning, utan som en hjälp att ien komplexvärld se till att hålla rätt kurs och hastighet i arbetet med att stärka utvecklingen och användningen av Al i samhället.

I samband med en utvecking av den gemensma Al-infrastrukturen är det centralt att så tidigt som möjligt beakta aktuella och kommande regelverk, särskilt inom civil beredskap, säkerhetskydd och cybersäkerhet.
\section*{Vägledningar för att överkomma rättslig osäkerhet}
Vid sidan av den tekniska million finns det ett behov av att användarna av Al-verkstaden också får ta del av insatser för att överkomma den osäkerhet som räder bland aktörer i offentlig sektor kring tillämpningen av olika rättsliga normer i förhållande till användningen av Al. Men även kring frågor om digitaliseringen i stort. I kapitleå Data som en förutsättning för Al-utvecklingen behandlade vi de rättsliga normer som på olika sätt kan försvåra att data görs tillgänglig mellan olika aktörer. Vår samlade bild är att det främstår de regelverk som är satta att skydda personuppifter, såsom GDPR, som offentliga aktörer upplever som rättsliga hinder. \({ }^{}\)
Ett isst mätt av osäkerhet kring tolkning av rättsliga normer kommer mallitt att finnas. Ansvariga medarerbare intoen omfittig verksamhet kommer även fortsatt ett behöva hantera komplicerade juridiska frågeställningar. Den rättstliga osäkerheten på området har dock mobilivt så pass omfattandte att satsningar på ynt teknik utbelir. Det leder i sin tur till en förlut både för den sällda verksamheten och samhället i stort. Ett sätt främ stinks den osäkerheter åt att yne mundersigheter med särskilt kompetens inom ett vist rättsområde uppdragat att vägleda andra offentliga aktörer när det gäller tolkningen av gällande rätt. 1 dessa uppradgår åt den certalt att alla offentliga aktörer omfattas och får ta del av den minskade osäkerheten som uppradgaren syfttar till. Aven om sansarvet för den rättstiga tolkungen in sjultändan åligger den som ansvarfar för den rättstiga henskampen, kan vägledningar leda till att ökäldande henskampen.
\section*{Stärk IMY:s regulatoriska sandlåda och samla rådgivningskompetens avseende Al hos Digg integritskingskindsmyndigheten (IMY) har sedan hösten 2022 en regulatorisk sandlåda för innovationsaktörer. \({ }^{[157]}\) På motsvarande sätt som föreslås för privat sektor i kapitlet Innovation, entreprenörskap och riskspital ånser Al-kommissionen att det finns starka skäl att skala upp och förstärka denna genom att skapa ett särskilt spår niriktat mot offentliga aktörer. På så sätt kan IMY ge vägledning till
offentlig verksamhet, samtidigt som myndigheten sprider kunskaperna och lärdomarna som erhållits inom ramen för ett utvecklings- eller innovationsprojekt så att de när ut till fler. Det bör till exempel vara möjligt för en offentlig aktör som är osäker på om användningen av ett vist Al-system strider mot GDPR, eller annan lagstiftning som skyddar den personliga integriteten, att fråga IMY och i rimlig tid få ett tydligt besked i frågan. \({ }^{[158]}\)
\section*{Förslag}
- Al-kommissionen anser att regeringen bör eg IMY uppradg att etablera ett särskilt spår inlirkatt mot offentliga aktörer i sin regulatoriska sandlåda. Detta särskilda spår bör integreras i verksamheten i Al-verkstaden. IMY bör vidare ges i uppdrag att, inom ramen för Al-verkstaden, ge offentliga aktörer vägledning när det gäller Al och tolkningen av datavksdydssregelverket. Det berkånde behovet av anslag för dessa uppradg är möjlloner kronor årligen.
När det gäller vägledning kring digitaliseringfrågator finns kompetens och erfnerenhet hos Digg. Myndigheten har i dagsläget i uppradg art ge vägledning till den offentliga förvaltningen i juridiska frågor inom ramen för den förvaltningsgemensamma digitaliseringinsfrastrukturen (Ena), om också inkluderan frågor om Al. \({ }^{[169]}\) Flera vägledningar har tagits fram som stöd för digitalisering med hjälp av Al, till exempel Förtroendemodellen, en separat webpf年龄段 med stöd för användning av Al i socialitjänsten, samt vägledningar kring hur regelverk och handlänggår kan anpassas för digitalisering och automulation. \({ }^{[169]}\) Al-kommissionen anser ådrför att det är ändamålisenligt att Digg fortsätter arbetet med att vägledning att offentlig sektor när del gäller Al ål deras respektive verksamhet.
\section*{Förslag}
- Al-kommissionen anser att regeringen bör gigg i uppradg art inom ramen för Al-verkstaden erbjuda vägledning kring användningen av Al officentig sektor. Det berkånde behovet av anslag för detta uppradg är cirka 4 miljoner kronor årligen.

\title{
Figur 3: Antal ansökta patent per miljon invånare
}
Datorteknologi

Bioteknologi

Källa: EPO. Norbäck och Persson, 2024, Den Al-drivna strukturomvandlingen av det svenska näringalivet, mimeo, IFN, Stockholm
Det relativt höga antalet sökta patent förefaller dock i begränsad omfattning vara Al-relaterade. Enligt Artificial Intelligence Index Report 2024 ligger vi inte bland de 15 främsta länderna i antal beviljade Al-patent per 10000 invånare.
Ytterligare ett sätt att studera resultatet av investeringarna i Al är att studera antalet nystartade Al-företag, se Figur 4. Satt i relation till vår folkmängd ligger Sverige på sjunde plats. Vi ligger alltså relativt väl framme, men inte bland de allra bästa.

\section*{Research Council fôr att samla europeiska vetenskapliga resurser i likhet med CERN. \({ }^{}\) \\ Sverige bör vara aktivt och dirva prioriterade Trägøre}
De många Al-initiativ som tas inom EU understrykivitken av att Sverige är aktivt inom området. För att fåt Al-regelverk inom EU som lämpar sig våj fôr svenska förhållanden är det viktig att vit utövar så stort inflytande vi kan i EU:s beslutsprocesser, särskilt iste dseck som Al befinner sig i nu. I våra kontakter med representanter för olika grupper i samhället har Al-kommissionen fåft en förhållandevis samstämning bild av att Sveriges relativa inflytande är för litet och att vi inte prioriterar påverkansarbetet tillträckligt. Att hitta belägg för den bilden är dock inte helt enkelt. Det finns studier, till exempel från Göteborgs universitet, som i stället pekar på att andra länder gärna samarbetat med svenska representanter i förhandlingarna om olika lagförlasg som tagits fram av EU-kommissionen. Detta pekar mot att Sverige skulle kunna få stor utvålzing om man prioriterar påverkansarbetet.
En annan viktig aspekt är i vilken mån vi lyckas påverka lagförlsgas redan innan de har prestersat av EU-kommissionen. Här det av stor vikt att Regeringskansliet försöker ta fram tidliga ståndpunkter att använda i påverkansarbetet getmenot EU-kommissionen, i stället för att vånta tills ett formulelt förslag av försäld och försäld och försäld och försäld och försäld, svenskar på plats i kommissionen. Det skulle derrätta kontakter mellan svenska företrädare, både från myndighetsfären och privat sektor, och bidra med att investskrmp bespektiv i kommissionens interna arbete. Här är det otveytidgå stå att Sverige är underepressenterat. Om vi var reverenterade i linje med vår relativa folkmängd skulle 2,7 procent av kommissionens tjänstemnär vara svenskar, men der velkliga iffrån är bara 0,200. Att rätta till den här obalansten bör vara en prioriterad uppgift för svenska myndigheter. Det är därför välkommet att den svenska regeringen och EU-kommissionen har enats om en handlingsplan för detta.
Behovet av att snabbft åsvenskar på viktiga positioner i EU:s institutioner gäller inte minst Al-området. Eiv bygger för tillfället upp sin organisation för implelementeringen av Al-förordningen. Det innebär bland annat att en särskild Al-byrå skapas inom EU-kommissionen. Att fåk valificerade personer från Sverige att söka sig dít bör också vara en prioriterad uppgift för regeringen. Kostnaden för att tillfälligt bistå med personal till en EU-institution bör inte ensidigt belasta en enhet inom Regeringskansliet eftersom det motverkar sådana placerlanger.
\section*{Förslag}
- Åtgärder bör vidtas för att råda bot på Sveriges underrepressentation inom EUs institutoner. Specifikt kommer det att behövas fler svenska tjänstemän inom geraldirektoratter för kommunikation (DG Connect) och den Al-byrå som nu bildas i Bryssel. För att uppnå detta bör bland annat placeringar av tjänstemän från myndigheter och departement finansieras centralt. Den enhet som lånar ut personal ska inte belastas med kostnaden. Kostnaden kopplat till Al-färgor bedöms gradvis öka och uppgå till 10 miljoner kronor per år inom fyra år.
\section*{De positioner Sverige bör dirva i harfundligarsbetet varier. Menen viktig genera på det aktuella lagstiftningstorf, sageringet. Menen viktig generell position att dirva, som tygut beskråftas i Mario Draghis rapport, år att nyvigt inte får hämma europeisk konkurenskrakt, om det inte finns väldigt starka skål. Dessutom bör regelverket vara tillgict och implementeras på ett harmoniserast ätt i EU:s 27 medlemsländer. Det måste vara "lätt att görar rätt”. Med för tung och komplicerad europeisk regelbörda riskerar användningen av Al inom EU bli ännu mer färgad av sprlåck och värderingar från andra delar av världen. \({ }^{}\)
\section*{Förslag}
- Regeringskansliet och svenska myndigheter bör säkerställa met mer aktivt deltagande i Al-frågorn inom EU-samarbetet. Tillräckliga resurser måste därför ges till de enheter inom Regeringskansliet och till de myndigheter som deltar i EUförhandlinglar på området.
- If förhandlingarsbetet bör svenska förträdare bland annat verka för att:
- Regerlingen på EU-nivå inte otillbörligt hämmar europeiska företags globala konkurrenskrakt, till exempel genom att europeiska företag inte får tillgång till den senaste teknologi från trædje land.
- Regerlingsbördan inte ska vara onödigt tunger företagen. Målet måste vara att det ska vara lätt att göra rätt inom ramen för den samlad regleringen.
- Implementeringen av reglering blir så harmoniserad som möjligt i EU:s medlemsländer, inte minst GDPR.

\title{
Dessa är fritt tillgängliga för allmänheten med öppna licenser.
}
En annan aktör är Ai Sweden, som tillsammans med det statliga forskningsinstitutet RISEl \({ }^{[37]}\) och W\&PP WARA Media \& Language, \({ }^{}\) har utvecklat GPT-SW3, Det är den första riktigt stora språkmodellen för de nordiska språken, främst svenska. Modellen är baserad på samma tekniska principer som OpenAI GPT-4. GPT-SW3 är tränade på klinöpgins universitets supertador, Berzelius. Även denna modell är fritt tillgänglig för allmänheten med öppen licenses.
Upphovsrätten och vägen framåt När det gäller den fortsatta utvecklingen av språkmölnä \(^{[38]}\) för att änder av förgårande att beaktad med förörspräketsliga skyddet för att för att stådande för att stråmåt. Stråpmodeller. Kras uppettning är att myndigheten har möjlighet att träna och göra modeller tillgängliga med stöd av en undantagsbestämmelse inte upphovssträtslagen. \({ }^{[38,}\) Plera rättighetshavare delar dock inte den uppfattningen. De har även invändningar mot metoden som användes för att ta fram GPT-SW3. Det är därför nödvändigt att utveckla och implementera en allmänt accepterad ersättningsmodell, sannolikt byggd på ett system med avtalslicenser. Genom att lösa den frågan kan vi undivka utdragna rättsliga processer samt öka den allmänna acceptansen för de modeller som utvecklas. Om modellerna ska användas i forskningsyfte bör det vara möjligt i princip utan kostnad.
Förslag
- AI-kommissionen aser att nenaletionell samordnare ska tillslätsat inom ramen för kommitvässendet för att samsordna det fortsatta arbetet med att ta fram stora språkmodeller för svenska språket. Givet statens centrala roll, behovet av att involvera flera olika aktörer samt nödvändigheten av att lösa de upphovsrättsliga utmaningarna på ett tillfredsställande sätt, ansar Al-kommissionen att det är nutarligt för staten att samordna och driva på detta arbete. Vid sidan av KB och Al Sweden bör representanter från lösåten, offentlig sektor \({ }^{[14,}\) och det privata näringslivet delta i utvecklingsarbetet.
- Den nationella samordnaren bör ges i uppdrag att samordna och driva på arbetet med att utveckla stora multimodala modeller för det svenska språket tillsammans med berörda aktörer samt eventuella internationella sambaretspartners. Samordnaren bör även driva på arbetet med att ta fram en ersättningsmodell som på ett rimligt sätt ersätter rättighetshavare vars kerv används för att träna stora modeller. Inom ramen för uppdraget bör samordnaren ha ett nära samarbete med rättighetshavare, de som tränar modeller samt organisationer för kollektiv rättighetsförvaltning. Samordnaren bör därutöver ta ställning till behovet av att etablera samarbeten med andra nordiska länder och inom EU. Slutligen bör samordnaren även ta ställning till behovet av att utveckla stora språkmodeller för de nationella minoritetsspråken.

\title{
Figur 3. Utrechtking - ranking och poäng
}
\author{
USA - plats 1 \\ Kina - plats 2 \\ Sydkorea - plats 3 \\ Singapore - plats 5 \\ Israel - plats 6 \\ Finland - plats 12 \\ Schweiz - plats 19 \\ Nederländerma - plats 17 \\ Sverige - plats 30 \\ Danmark - plats 28 \\ Norge - plats 42 \\ \hline
}
\section*{Poäng KPI}
Notera: Placeringen för varje land inom Utveckling visas efter landets namn. Den horisontella axeln visar poängen för varje land, beräknad utfärn indikatorer relevatera till området. Den högeta poäng som ett land kan få år 100. Den grå stapeln representeran den poäng som krävs för att placera sig på plats 101 Utveckling 12024 års upplaga.
Källa: The Global AI Index, 2024 års upplaga.
Området Utveckling baseras ih uwudsak på två klasser av indikatorer. Den första utgörs av indikatorer som fokuserar på patent, som är ett väl etablerat mått på innovation. Den andra klassen, som ska geen bredre bild av Al-innovation, omfattar indikatorer som visar i vilken grad aktörer I landet bidrar till utvecklingen av öppen källkod. \({ }^{40}\) Detta årsåskilt relevant för AI-utveckling, eftersom många framseteg inte patenteras utan delas öppet.
Sveriges innovationskraft inom Al beror på vår förmåga att bygga ett fungerande eko-system, så som beskrivits i till exempel Fårdplanens inledande kapitel, åråde akam<|im_start|>, järningsliv och offentigt sektor samnverkar för att driva utvecklingen framåt. Färdplanen innehåller ett antal förslag för att stärka Sveriges ininnovatskrift på Al-området. Till exempel är det viktigt för att för att känningen mellan akademion och näringslivet, vilket diskutning, för att för att för att för att för att medverkan, bland annat genom att inrätta excellenscenter inom Al. I kapittel Innovation, entreprenöskap och riskpåtal föräsler är utåkat ekonomiskt stöd och vägledning för innovativa företag. Tillsammans med förslaget, i kapittel Beräkningskraft, om att finansiera en Al Factory, skulle detta bidra starkt till att fördbättra förutsättningarna för innovation i Sverige.
Vad kan då vara en ambitiös och realistisk målsättningen för området Utveckling? Med tanke på att
Sverige ofta rankså högt som innovationsland i allmänhet, och med tanke på förslagen i denna Färdplan så borde vi kunna kättra i rankning. Indikatorerna i sin nuvarande form gymnar dock större länder, men detta har inte hindrat länder som Singapore och Israel från att nåhöga placeringar. Det gör att Sverige borde kunna rankas bland de 10 främsta länderna, senast är 2030.
- Sverige ska senast är 2030 ha gått från plats 30 till att tillhöra topp 10 i området Utveckling
\section*{Infrastruktur}
Får att svenska aktörer ska kunna utveckla konkernurskraftiga Al-jänster skravs tillgång till en viss infrastruktur, så som beräkningskraft ämnad för trätning och användning av Al-modeller, telekom och elektricitet. Inom området Infrastruktur ligger Sverige på plats 21.
Figur 4 visar att poängskillnaden till närliggande länder är liten, vilket tyder på att många länder förifen sig på en liknande nivå som Sverige. Mindre förbättringar i infrastruturen kan därför stärka Sveriges position och konkurrenskratt inom området. Singapore och Nederländerna visar att mindre länder kan nå topp-tio även inom detta område.

\[
\mathcal{L} = \left\{ \begin{array}{ll}
\displaystyle \frac{1}{2} \sum_{i=1}^{n} \frac{1}{2} \sum_{j=1}^{n} \frac{1}{2} \frac{1}{2} \frac{1}{2} \frac{1}{{\rm Tr} \left( \frac{1}{2} \frac{1}{2} \frac{1}{4} \right)} \frac{1}{2} \frac{1}{2} \frac{1}{n} \frac{1}{2} \frac{1}{2} \frac{2}{2} \frac{1}{2} \frac{1}{2} \frac{n}{2} \frac{1}{2} \frac{1}{2} \frac{\partial}{\partial n} \nonumber \\
\displaystyle \frac{1}{2} \sum_{i=1}^{n}{\rm Tr} \left( \frac{1}{2} \frac{1}{n} \frac{1}{n} \frac{1}{2} \frac{1}{{\rm Tr} \, \left( \frac{1}{2} \frac{1}{2} \frac{\partial}{n} \frac{1}{2} \frac{1}{2} \frac{\partial}{{\rm Tr} \, \left( \frac{1}{2} \left( \frac{1}{2} \frac{1}{2} \right) \frac{1}{2} \frac{1}{2} \frac{1}{3} \frac{1}{2} \frac{1}{2} \frac{12}{{\rm Tr} \, \left( \frac{1}{2} \right) \frac{1}{2} \frac{\partial}{\partial n}} \right)}{2} \frac{1}{2} \frac{1}{2} \frac{2}{{\rm Tr} \, \left( \frac{1}{2} {\rm Tr} \, \left( \frac{1}{2} \frac{1}{{\rm Tr} \, {\rm Tr} \, \left( \frac{1}{2} \frac{2}{2} \frac{1}{4} \right) \frac{1}{2} \frac{1}{2} \left( \frac{1}{2} \frac{\partial}{\partial n} \frac{1}{2} \frac{1}{2} \frac{11}{{\rm Tr} \, \left( \frac{1}{2} {\bf \partial} \right) \frac{1}{2} \frac{1}{2} \right)} \right)} \nonumber
\end{array} \right\}
\]

\title{
Figur 4: Antal nystartade Al företag under perioden 2013-2023 per miljon invånare
}
\author{
Antal företag per miljon
}

Vilken kunkap àr de yrkesverksamma i behov av?
Bâde yrkesverksamma och arbetsgivare, särskilt de små och medelstora företagen och offentliga verksamheter som saknar strategisk HR-kompetens, behöver kunskap om kila rådigheter som àr nödvändiga framöver. Detta för att kunna ta sitt ansvar att lära sig nya saker i syfte att vara fortsatt anställningsbar, samt för att ha formåga att stödja de anställdas konintuenliga lärande. I dagslägét är det dock ingen enkel uppgåft att på egen hand förstå vilka dessa färdigheter år, varken för yrkesverksamma eller arbetsgivare.
Anledningen till detta är inte att det råder någon allmän brist på information. Hindret utgörs i stället av att den information som finns att tillgå inte årt tillärckligt overskådlig. Vis saknar i dag en sammanhållen, relevant och användarvänlig sammanställning av information om vilken kompetens som kommer efterfrågas framöver och vilka utbildningar som finns för att fylla dessa behov. För att avhjälpa detta behöver vi regelbunden analys av At's effekter. Analyserma bör särskilt fokusera på heterogena effekter där vissa grupper (geografiskt, bransch-, yrkes- eller lönenmåssigt) påverkas negativt. På så vis kan vi vara snabboftade och agera i tid för att bistå dessa identifierade grupper i omställningen. Det här är nödvändigt, givet den snabba utvecklingen. Forskningen ger vissa indkatorer på att sådana heterogena effekter kan skönjas redan nu. \({ }^{}\)
Mycket görs redan på området. För att nämna några exempel tar Myndigheten för yrkeshögskolan (MYH) fram områdesanalyser över olika branschers behov av kompetens, medan Universitetskanslersåmbett ansvarar för att ta fram planeringsunderlag för lärosåtenas dimensionering av utbildning. Inom skolan sijojärts nu även en stor reform som syftar till att arbetsmarknadens behov i större utsträckning ska stra snarare än elevernas önskemål. Mot bakgrund av detta kommer Skolverket att ta fram regionala planeringsvunderlag som beskriver arbetsmarknadens behov, vilka kommunerna kommer ta hänsyn till i utbildningsutbundet.
\footnotetext{
Bâde yrkesverksamma och arbetsgivare behöver kunskap om vilka färdigheter som är nödvändiga framöver. Foto:Scaninav bildbyrå
}

\title{
Inledning
}
och sammanfattning
Det har sagts av många och kan låta som en lykscha, men likarfult är det samt - användningen av antificiell intelligens, eller AI, kommer att förändra och utveckla vårt samhälle. Precis på samma sätt som ångmaskinen, elektriciteten, telefonin och IT tidigare gjort. Processen är redan i gång. Vi använder AI dagligen, ofta utan att tänka på det. Den handlar om allt från enklare tjänster som underlättart vår tillvanor som styrning av en robotdammgusare, till funktioner som bokstavilgen kan raddla liv, som AI-assistered cancerdiagnostik. De här tjänsterna är resultatet av att datorer kan lära sig direkt från data i stället för avt förlja föropprogrammerde ger. De senaste årens dramatiska utveckling av AI har emellertid flytatt gränserna ytterligare för vad som kan åstadkommas. med hjälp av AI. Nu kan vi strya maskiner med vårt naturaliga språk och få hjälp att skapa nyt mittar eller form av text, bilder, programkod, ljud och mycket mer. Hark biomineras vår mänskliga intelligens med den artificiella. Det medfört av tik van bjoba snabbare och med hög hje kvalitet, sättmidg stip vol blir mer treativa och innovativa. AI har mänder vilbit ett verktyg för oss alla.
Årets Nobelpris i fysick och kemi utgör en talande illustration av de framsteg som har gjorts inom AI, fremen framfört all hur AI-utvecklingen leder till genombrott inom olika. Det att ett för att för att skapa hoch Hpfeldt belöns för sitt åførbe sigå kallade artificiella neurala vårtek. Dessa ligger till grund för mykencitet av den utveckling som har gjort dem tidjligt för kempi/stragamka Baker, Hassabis och Jumper att utveckla AI-modeller, som har revolunte nerat förståelsen av protimeter och deras strutturet. Det har i sin tur stor bydelyelse för möjligheterna att förutes sjukdomar och hitta effektiva behandlighet.
Utvecklingen myrner fantastiska möjligheter, men den leder också till oro hos vissa - en oro som röll artifrån vad som ska hånda med det regna jobbet, till om AI skulle kunna utveckla förmågor och en egen vilja som i slutändan kan hota människans existens. Man undrar helt enkeltk vilket samhälle den här utvecklingen komen kommer att leda till.
Det är inte ödesbestämt
Vårt svar på den frågan är en todrfåga. Vad vil/i att förändringen ska leda till? Utvecklingen är nämligen inte ödesbestämd utan ligger i våra eagna händer. Det finns inte en given framtid, utman gÅloika. Vart Just vår våg leder avgörs av hur vi ågarer och förhåller oss till användningen av AI. Ett passivt och reaktivt förhållningssått är enligt AI-kommissionen det såmta och mest riskyllda alternativet. Det skulle innebära att tv iågger vår framtid i andras händer.
Vårl bid av framtiden är i stället ett samhälle om speglar en medveten stråvan efter att dra maximal nytta av de möjligheter användningen av AI erbjuder, samtidigt som vi hanterar de problem som uppstår. Det här innebär inte att AI används till all. Ät aret väljdet kraftfull vetryk, men det betyder inte att varje problem löses bäst med hjälp av AI.
Vårv vision är ett samhälle där människors vardag förenlaks med hjälp av AI-tjänster, och där det finns en grundläggande förståelse för vad AI är, och inte är, ett samhälle där det finns en levande diskussion om möjligheter och risker, på vetenskaplig grund. Kunskap är nämligen den första och kanske viktigaste försvarslivenj i kampetom i mot villivillig användning och utveckling av AI. Samtidigt använder vi just AI för att försvåra för organiserad brottslighet och bygga mot ständskraft mot cyberhot.
Det är också ett samhälle där vår unika tillgång till data utnytjas betydligt bättre än i dag, men med bibehållen respekt för den personaliga integriteten och upphovsrätten. Genom ökade möjligheter till delning av data, och ett fördjupat samarbete mellan privat och offentlig sektor, möjliggör vi inte bara innovation och utveckling in onffettlig sektor. Vi förbättrat rocks index möjligheter att leva upp till sitt växande åtagande - ett åtagande som annars riskerat all bli över mäktigt till följad av den demografiska utvecklingen.
Med god tillgång till data, beräkningskapacitet i form av datforkapacipitet samt en bred AI-kompetens, kan svenska företag bli mer innovativa och framgångsrika på vårdsmarknaderna, inte minst genom att tillämpa AI på nya områden, i kombination med fördjupade

\title{
Al-kommmissionen anser att regeringen behöver ta initiativ för att säkerställa en god beredskapsplanering på de områden där bortfall av digital formåga annars riskerar att så ut viktiga samhällsfunktioner. Foto: Försvarsmäkten
}
\section*{Med hjälp av Al kan vi till exempel}
- Optimera underhåll och reparationer av kritisk infrastruktur inom exempelvis transport-, el- och kommunikationssystemen. Detta genom att analysera data från sensorer som upptäcker tecken på försämring eller felfunktion innan det går så långt som ett haveri eller en kollaps.
- Studera historiska data och aktuella trender för att förutsåga uppkomsten och omfattningen av naturkatastrofer och göra det möjligt för myndigheter att införa motägtärder och evakueringsplaner i förväg, eller åttinstone i ett tidigt skede.
- Följa data i realtid från källor som sociala medier, sensorer och setallitbilder för att få en uppfattning om graden av skada, identifiera områden som behöver omedelbar hjälp och optimera resurstilldelning under olika typer av kriser.
Al-kommmissionen anser att de myndigheter, och andra aktörer, som ingår i Sveriges civila beredskapsystem måste stärka sin förmåga att använda sig av Al på dessa och liknande sätt. \({ }^{}\) Men äveni dessa fall in innebär en ökad användning av Al inte bara möjligheter, utan också riser. I detta sammanhang handlar det dock främst om hur ett köt baroende av avanacerad teknik gör varå sbararhet för störningar i de tekniska systemen. Denna fråga kommer vi nu att behandla närmare.
Att bygga motståndskraft i ett högteknologisk samhälle.
Ökad tekniknavändning i samhället, och då inte minst i samhällsviktiga verksamheter, har lett till ennorma vinster. Kvallitet och servicenivåer har ökat, väntetider har kortats, och stora besparingar gjorts. Samtidigt är ett stort antal samhällsfunktioner i dag helt beroende av att tekniken fungerar. Exemljen år många på hur såväl cyberattacker som oavsiktliga fel och olyckor lett till allvarliga störningar i samhället, som ibland varat i flera dagar. Flyglapster världen över har fått goppa trafiken, hela butikskedjor har behövt stänga då de inte kunnat ta betalt och kommunala förvaltningar har lamslagits då de föflorat kontrollen över sina IT-system.
När användningen av Al ökar i samhället kommer även många verksamheters beroende av att de tekniska systemen fungerar att öka. Denna såbararhet blir därmed än mer påtaglig. Det är därför nödvändigt att samhällsviktiga tekniska system byggs på ett sättom bra från grunden gör dem motståndskraftiga mot hela spektrumet av hot. Systemen måste vara dels robusta, det vill såga täla en hög grad av påfrestingar utan att funktionen påverkas, dels resilienta, alltsä kunna anpassa sig vid och återhämtia sig från störningar. Det behövs också ett mått av redundans, eller reservlönsningar som kan ta över när de ordinarie systemen fallerar. Inget av detta år nytt eller kopplat sig speclift till Al-utvecklingen. Tivratt finns dessa behov i allra högsta grad redan i dag, till följd av den digitalisering som redan skett i samhället. Varje satsning på ökad användning av Al måste dock, för att vara ansvarsfall, ta hänsynt till detta perspektiv.
Hur motståndskraftiga de tekniska lönsingarna anför skors kan vi idrligh helt borte från risken att de fallerar. Avbrottskinslag verksamhet behöver självfallet ha beredskaps- och kontinuitetsplaner för sådana scenarier. Exempelvis är akutsjukhus alltid redo att övergåt till manuell journalföring om de digitala systemen slutar fungera. Många verksamheter få också

\title{
Tillgång till utländska Al-resurser
}
Merparten av det värde som Al ger upphov till skapas när man använder Al, genom Al-plattformar och Al-verktyg. Dessa är oftast amerikanska. För att svenska aktörer även förmet stättnings/vis ska kunna använda och utveckle Al är det viktigt att de har fortsatt tillgång till dessa.
\section*{Innovation och riskkapital}
Al kommer att leda till viktiga innovatinerin om alla områden. Innovationsklimatet är därmed viktigt, inte minst tillgången till riskkapital, för att företag ska kunna uppstå och växa. Det berör såväl privat som offentlig finansiering - till exempel från så kallade affärsånglar, banker eller investeringsfonder samt från offentliga aktörer som Vinnova eller Almi.
\section*{Al-kompetens för alla}
Om Al ska få ett brett genomslag isanhällest behövs kunskap. Kunskap om vad Al är och inte är, samt vilka möjligheter och utmaningarna som följer med tekniken. På kort sikt kommer sannolikt en vis absolutens uppstå på arbetsmarknaden innan nya branscher och företag har fångat upp arbetskraft som friställts på grund av olika Al-jänster. Att omställningssystemet fungerar väl är därför avgörande för synen på Al och hur villliga människor är att ta till sig den nya tekniken.
\section*{Ledarskap}
Vid system övergripande förändring behövs ledarskap och styrning. Beslut måste ofta tas under tidpress och med kient beslutsunderlag. Den politiska styrningsmodellen måste därför tillåta ett snabbt och kraftfullt agerande. Det gäller även i internationella sammanhang där många Al-relaterade frågor avgörs, exempelvis gällande reglering och säkerhet.
\section*{Behovet av komplementära åtgärder - ett exempel}
För att visa hur en bred Al-avändning i samhället kräver åtgärder inom olika områden exempilferar vi med ett av uppdragen enligt Al-kommissionens direktiv: ”. att föreslå åtgärder för en ökad Al-avändning i offentlig föravitning genom datadriven innovation och dataförsörling: ”/rkt uvudförlsgår för detta är en så kallad Al-verktskad, då offentliga aktörer tillsammans kan utveckla lösningar inom ramen för en gemensam infrastruktur för Al. Detta kräver möjligheter att dela data. In ulnäget finns det betydande hinder för detta. Det behövs också beräkningskraft, (form av egna datorer för särskilt känslig information, och utnyttjande av molntjänster. Det senare kräver klargörande av vad som är legalt möjligt och inte. Här kommer också säkerhetssaspekten in.
För att hitta de bästa lösningarna måste också privat och offentlig sektor samverka, genom att privata företag får lösa utmaningar inom offentlig sektor. Det förutsätter i sin tur dynamiska forskningsmiljiber, där akademin samverkar med aktörer från både privat och offentlig sektor. Det behövs också utbildningsinsatser, så att människor har förutsättningar att ta till sig och utnyttja de möjligheter som användning av Al ger.
Den kanske viktigaste förutsättningen för att ån Al-verkstad på plats är dock att det finns ett dytligt politiskt ledarskap. In formatlallet gör vår decentraliserade föravtlningsmodell att det är svårt att hitta lösningar på problem som spänner över flera olika sektorer. In ulnäget finns det dock en bred samsyn bland kommuner, regioner och statliga myndigheter om behovet av en gemensam infrastruktur för att utveckla och sprida Al-ösningar. Att regeringen nu tar tillfället i akt och svarar upp mot det behovet kommer att vara avgörande för att uppnå den ökade Al-avändningen i offentlig föravtling som man gett uttryck för i våra direktiv.

I syfte att tydliggöra standarder för datahantering inom olima kstoerter bör regeringen överväga att ge myndigheter i uppdrag att ta fram sådana. \({ }^{}\) Här kan myndigheternas respektive ansvarsområde, och den ansvarsfördelning som framgår av bilagan till förordningen (2001:100) om den officiella statistiken, vara vägledande för vilka myndigheter som bör ges menna typ av uppdrag. Det kan därutöver vara lämpligt att ge en myndighet ett uppdrag som dataförvatlare, så kallad data steward, för att stödja andra aktörer när det gäller att åstadkomma god datahantering och underlätta datadelning inom offentig sektor. Expert- kunskap när det gäller frågor om såväl dataförvatlning som datahantering finns hos Statistiska centralsfyrån
\section*{1. 1. 1. 1.}
Som en del i arbetet för god datahantering bör offentiga aktörer uppträta en datplan. Av en sådan datplan bör det framgå vilka data aktören förfogar över och hur data hanteras, inbegripte hur de gör datadelning möjlig. Av datplanen bör det även framgår hur den offentliga aktören avser att utvärdera bevhet av data för att kunna fullgöra sitt uppdrag på bästa möjliga vis. Genom att upprätta och kontinueringt uppdatera sin datplan kommer offentliga aktörer regelbundet behöva ta ställning til om de hanterar data på ett ändamålsenligt vis samt vilka data som de borde ha till änggll till givet aktörens uppdrag.
Vi anser även att det finns starka skäl som talar för att det behöver bli enklare för privata aktörer att få kunskap om var relevant data finns tillgänglig. Genom att invättta en funktion som Data Steward anser Al-kommmissionen att det bör vara möjligt för en selslld att vända sig till denna funktion för där om var specifika offentliga data finns. Utöver vad som nämns innan kan det vara svårt, särskilt för forskare och mindre företag att få tillgång till offentliga data på grund av höga avgifter.
\section*{1. 1. 1. 1. 1.}
\section*{1. 1. 1. 1. 1.}
- Al-kommmissionen föreslår att ett regeringen överväger att ge någon eller några myndigheter i uppdrag att ta fram sektorsspecifika standarder för modern datahantering.
- Al-kommissionen föreslår att regeringen överväger att ge SCB ett sammanhållande uppdrag att verka för en modern datahantering inom offentlig sektor.
- Al-kommissionen föreslår att regeringen ger SCB 1 upprad tag intäträta en Data Steward-funktion som beskrivs i detta avsnitt. Vi bedömer att SCB bör få ett öktå tarligt anslag på 4 miljoner kronor med anlending av detta uppdrag.
- Al-kommmissionen föreslår att regeringen ger SCB uppdrag att se över avgiftsmodeller för tillgång till offentliga data. Syftet är att göra data mer tillgänglig, särskilt för forskare och mindre företag.
Krav på att tillgång till data och interoperabilitet ska ingå i konsekvensutredningar
Som beskrivs i detta kapitel har tillgången till data blivit en allt viktigare faktor, inte enbart för Al-utvecklingen, utan för digitaliseringen i stort. Al-kommmissionen föreslår därför ett tillgång till förordningen (2024:183) om konsekvensutredningar med innebörden att de stka vara obligatorisk att uttreda tillgång till data (datakonsekvensutredningar). När en statlig utredning tar fram lagförslag eller när en förvaltningsmyndighet beslutar om föreskrifter eller mellannåra vid ska alltså en sådan utredning tas fram som en del av konsekvensutredningen.
I en datakonsekvensutredning ska det bland annet redgörsar för vilka typer av data som finns tillgängliga inom det aktuella området samt vilka data som är nödvändiga för att kunna mäta resultaten av det förslag som presenteras. En datakonsekvensutredning ska även bestå av en interoperabilitetsanalyans som berör de tekniska och juridiska möjlighetna att tillgängliggöra den aktuella dataan. Genom ett sådant tillgång anser Al-kommissionen att frågan om tillgång till data kan lyftas och bli till ett naturligt inslag i processen med att ta fram nya lagar, förordningar och föreskrifter.
\section*{Förslag}
- Al-kommmissionen föreslår att förordningen (2024:183) om konsekvensutredningar ska ändras så att datakonsekvensutredning ska vara obligatorisk när en statlig utredning tar fram lagändring eller när en förvaltningsmyndighet beslutar om föreskrifter eller allmänna r.

\title{
Förslag
}
- \(\quad\) Al-kommissionen anser därför att
E-hälsomyndigheten bör få regeringens uppdrag att verka för jämlika förutsättningar för implementering av Al-tillämpningar inom hälsoh cosjukvården. E-hälsomyndigheten bör även kunna agera som ett samlande och samordnande organ för Al-relaterade frågor, för alla aktörer med uppdrag inom hälso-och sjukvården.
Effektivare användning av offentliga resurser Hanteringen av olika ärenden kan snabbas på med stöd av AI, till exempel genom mer automatiserade tillståndsgivningsproceser. Det kommer matt
underlätta tillvaron för både privatpersoner och företag, och öka Sveriges konkurrenskraft och attraktivitet. Även annan service till enskilda kan utvecklas med hjälp av Al. Offentliga aktörer kommer till exempel att kunna erbjuda en mer effektiv och tillgänglig kundtjänst med hjälp av Al. Den potentiella vinsten är betydande eftersom administration i offentlig verksamhet är ganska likartad. Behoven finns alltså hos alla ofentliga aktörer. Medarbetarnas vardag kommer att underlätta med hjälp av Al-verktyg, Därmed frigörs resurser som kan användas för att lösa fler känuppgifiter och ägna mer tid åt bland annat möten med medborgarna eller kompetensutveckling.
\section*{Översättningar och tokltjänst på Domstolsverket}
\section*{Domstolsverket har utvecklat en Al-tjänst somer översätter juridisk text på andra språkt till svenska. Enligt företrädare för Domstolsverket är den beräknade kostnadsbesparingen 90 procent jämfört med när texterna skacks till en översättningsbyrå. Dessutom gå det mycket snabbare. Al-applikationen översätter text på tre sekunder, vilket kan jämföras med att det tog tre veckor att få tillbaka texterna från en byrå. Domstolsverket arbetar för att lansera tjänsten för samtliga domstolar, vilket skulle kunna imebära besparingar på upp till 20 miljoner kronor en.
Enligt beräknångrar från Domstolsverket skulle myndigheten kunna spara ännu mer med en ny maskeringstjänst för anonymisering av dokument. Tjänsten beräknas kunna spara upp till 220
årsarbetskrafter, vilket motsvarar cirka 300 miljoner kronor årligen. Både översättning och maskering av texter är exempel på tjänster som skulle kunna led till besparingar inom stora delar av den offentliga sekton.
Domstolsverket provar också tillsammans med några andra aktörer en tolkningstjänst för snabb tolkning i olika situationer, exempelvis i väntan på en mänsklig toklt. Offentlig förvaltning lägger omkring 1,5 miljarder kronor årligen på tolkningstjänster. Ettor realistiskt antagande är att det i vissa situationer är möjligt använda tolkning med stöd av Al!
20-30 procent av alla tolkningssituationer. Detta innebär en besparingspottential på 300-450 miljoner kronor årligen.

Foto: Tommy Hvitfeldt (Domstolsverket)

IT-utrustning. Det är visserligen i praktiken omöjligt för ett land av Sveriges storlek att göra sig oberoende av andra på detta område - internationellt samarbete, särskilt inom EU, är på många punkter en direkt nödvändighet. Likväl måste vi beakta detta perspektiv i beredskapsplaneringen.
Tillåggen till beråkningskraft, elförsörjning och elektronisk kommunikation för att upprätthålla Sveriges Al-förmåga år också en aspekt av den bredare frågan om digital suveränitet. Sedan slutet av 2010-talet har det varit en levande fråga i vilken utsträckning det är förenligt med svensk och europeisk lagstiftning att använda sig av utländska, oftast amerikanskam, notjintärsen i vissa varksamheter. Detta gäller särskilt när det röserketssbelagd information eller känsliga personuppgifter. \({ }^{[56]}\) Helt separat från vad som är tillåtet är det också högst relevant att fråga sig vad som är lämpligt. Det är exempelvis nödvändigt att beakta i vilken utsträckning klityiga samhälsfunktionerbör vara forebende av digital infrastruktur belägen i utlandet, och av fungerande förbidenser med dessa länder. \({ }^{}\)
Det är av central betydelse för Sverige att upprätthålla ett adevat skåterhenskyd skijt by inte att hindra obe- höriga från att få tillgångt till strategiskt klityg digit digital infrastruktur. Lika viktig att år det att förhindra att känslig svensk teknik hamnar i fél hander. För närvarande pågår den diskussion i EU kring frågor som rör ekonomisk miskerkhet, där Al är en av de framväxandate teknikar som pekats ut som särskilt angelägen alt beakta. \({ }^{}\) Den forskning som bedrivs i Sverige är många fall efterftraktad, inte minst av främmande makt, som i vissa fallågnar sig åt förknssponiggare eller olvolning teknikanskaffning. \({ }^{}\)
Etta ett satt mottervarka ett utländska aktörer frå tillgång till till teknik som är var central betydelse för Sveriges aktörserhet är den nya lagen om grännskang av utländska direktnivkenterstag som riksdagen ontg203. \({ }^{}\) Fagen, som i stir ut utgår från reglering på EU-nivå, ger regeringen rätt att närmare definerila vilka teknologier lagen ska omfatta. En sådan teknologi är Al-algoritmer, som anvander eller genererar data som innehåller känsliga person- eller lokaliseringsuppgifter. \({ }^{}\) Detta innebårat en investering i exempelvis ett företag som utveckhar Al-algoritmer ska ammålas till Inspektionen för strategiska produkter (ISP). (ISP) har
möjlighet att förbjuda en utändsk direktinvestering, till exempel om det är nödvändigt för att förebygga skadlig inverkan på Sveriges säkerhet. Samtidigt som denna lagstiftning är ett välkomet tillskott iverktygslådan för säkerhetskydd, är det viktigt att granskningarna inte utformas på ett sätt som i ondån avskräcker inversterear som annars kan stå för välkomna kapitaltilskott till svenska Al-företag.
Förslag
- Al-kommissionen anser att den som ansvarar för en samhällsviktig verksamhet måste ta hänsyn till påverkan på Sveriges digitala suveränitet när beslut fattas om Al-avnändning i verksamheten. \({ }^{}\)
De direkta tytrte hoten från Al-avnändning Fiertliga statsaktörer hotar tid gsvenska säkerhetinsresen genom oyherangrepp, sabotageförösk, otillöbrlig påverkan och olvolig underärtelseelnhämtning. \({ }^{}\) Redan nu använder dessa aktörer sig av Al-ågot som förväntas öka i takt med att tekniken utvecklas och förmågan att använda den på både gamla och nya sätt ökter. Art de veste med unppenbara användningsområdena av Al för en aktör som villal ska deler establisibera Sverige är informationspåverkan. Desinformation är inte ett nyftnønen, men numera är det möjligt att mycket skabbare än tidigare publicera stora mångder individualisernd information.
Al gör det möjligt att snabbit, enkelt och med liten arbetisnsats skapta stora mångder text, ljud och rörlig bild som ett led i desinformations- och påverkanskampanjer. Al kan också användas för att effektivt sprida material och budskap i exempelvis sociala medier. Spridning av falsk eller missvisande information kan ska ske yiftet tads kaldita tillen till media och samhållsinstitutioner, vilket i sin tur kan leda till en försvånging av det medokratiska statskicket och viljtan ett förvara desmaterna (försvarsviljarn). \({ }^{}\) Denna fråga år särskilt aktuell i samband med allmänna val. Det finns en oro att Al-verktyg skav användas för att manipulera valutgången genom att sprida desinformation eller missinformation. Det sistämnndna begreppet avver Afsk eller missivsande information som utan oda avsikter sprids vidare på grund av okunskap eller bristande källkritet.
\section*{Bor tom den typ av hot som vi nåmner ovan finns även skriter i framtiden med milträt tillämpning av Al-}

\title{
Kreativ förstörelse
}
Joseph Schumpeters teori om kreativ förstörelse beskriver hur innovatiner och entreprenörskap driver ekonomisk utveckling genom att ersätta föråldrade teknologier och affärsmodeller. Enligt Schumpeter leder entreprenörs innovatiner till att befintliga strukturer och företag byrts ned, vilket skapar utvymme för ny utveckling och kunskap. Ett exempel på kreativ förstörelse är hur e-post revolutionerade kommunikation och görde äldre teknologier som fax och telex överföldiga.
Det finns en risk att den kreativa förstörelseprocessen inårslingivet förämsräts påsparen av Al-tuvecklingen om endast de allra största företagen framgångärkilt kommer kunna använda \(\mathrm{Al}\) i sin verksamhet. En nyligen publicerad studie från Statistiska centralbyrån (SCB) visar att Al- användningen i svenska småböruserdag är betyldigt lägre än i svenska stora företag. Enligt EU:s Digital Economy and Society Index (2023) använde runt 10 procent av svenska företag med fler än into anställda Al - motsvarande andel var nut 15 procent i Dannark och Finland.
Användningen av Al gör det mäjligt för större företag att bli mer produktiva och utkästa sina marknadsandelar. Detta påverkar konkurrenssituationen och ställer konkurrensmyndigheterna inför nya utmaningar. Risken för ökad marknadsmakt hos de ledande företagen i en Al-baserad strukturmovandmling måste bedömsa vara relativt stor. Det beror på att den Al-baserade marknaden är starkt förknippad med nätverkseffketer och starka stordriftsfördelar när det gäller att samla stora mängder data och tränning av modeller.
Den kreativa förstörelseprocessen kan förbättras om Al-modeller med öppen källkod blir allmänt tillgångliga. Ett Al-ekosystem med öppen källkod gör det möjligt för olika typer av aktörer från närnänglivet, offentlig sektor, ideella organisationer, akdemin och individella doreda att få en bred tillgång till utvecklade Al-mobdeller. Detta ger dessa aktörer tillgång till branschiedande produktions teknologier som de aldrig annars skulle få tillgång till. Det stimulerar entreprenörskap och den kreativa förstörelseprocessen. Det är således av vikt att se till att modeller med öppen källkod kan verka utan större konkurrensbegränsningar. Samtidigt är det viktigt att vara medveten om att öppen källkod även innebär att användare kan ta bort säkerhetsspärrar i stora grundläggande språkmodeller.
Teknliksprädningen in den svenska företagseksetformhandar emelterliå inte bara om utveckling av unga Al-företag och utslagning av lägproduktiva företag. som inte tillgodogör sig tekniken. Den stora produktivitetsökningen tenderar i stället att komma från att redan etablerade företag ökar sin produktivitet. Detta sker i hög grad genom att koncerner genom uppköp skapar nya dotterbolag och stöttar dem genom implementering av ny teknik, exempelvis Al. I Figur 7 illustreras detta förhållande genom att stapeln för inom-företagseffekter är betyldligt lägren åt spalarna för in- respektive utträde.
Analysen ovant yder på att företagsförv an bli en allt viktigare del för teknikspridning inom Al-området och produktivitetsutvecklingen, men samtidigt utgöra

\section*{Det uppskattade ekonomiska värdet}
Med det uppskattade ekonomiska värdet, som illustreras av exemplen ovan, kommer ett fullständigt införande av Al isvensk ofentlig förvalntning att ge en avsevård ekonomisk nytta. I en gro uppskattning som Digg och konstuffirman McKinsey gjorde 2020 handlar det om cirka 140 milljarder kronor per år. Det motsvarar omkring fem procent av de totala offentliga utgiferna. \({ }^{[150]}\) Den siffran utgår dock från att man använder de Al-tilänster som fams på plats vidldunpkinter för beräkningen. Efter detta bör potentialen ha ökat ytterligare, eftersom Al-tekniken har gjort stora framsteg sedan dess.
Dessa beräkningar tar inte hänsyn till eventuella kostnader för att implementera Al. Men även om bedömningen är mycket osäker indikerar den en stor potentille! konstmadssparing inom offentlig sektor Redan mycket förästikiga antaganden om hur stor del avenna potentital som faktiskt kommer att förverkligs motiverar dock, på rent ekonomisk grund, betydande satsningar. Utverd ett tällkommer även andra mer vallativa nyttor som i dag är svåra att matä och uppskattta, till exempelök soda snabbet, kvalitet, rättskäkerhet och rädald vi.
\section*{Offentlig sektors Al-resa för att skapa}
Fördergår, av skor, av skor, av skor
Offentliga aktörer utvecklar, experimenterat med och implementerar sälunda redan nu ren ad olika Al-tjänster i sina verksamheter. Vi har bara gett ett axplock ovan. Dessa tjänster har ofta utvecklats i egen regi. Det borer på att det för närvarande finns en rad hinder för aktörerna att arbeta i särka Al-miljör för att jobda stillsammas med att utveckla Al-tjänster för likartade behov. Det är kostsamt att säkerställa en tillräckligt utvecklad och säker I-Tmiljö som också kan inkludera plattformar och verktyg för utveckling av Al.
Från tidigare behovansalyser vet vi att det enbart år ett fåtal aktörer inom offentlig sektor som har resurser och kompetens till detta.
Ska offentlig sektor, med sina stora och små myndigheter, kommuner och regioner, kunna använda Al för att sig an sina uppdrag och hantera utmningarna måste ett antal centrala förutsättningar vara uppfyllda. Delvis handlar det om de generella förutsättningar som diskuteras i andra kapitel i denna Fårdplan, som innovation, säkerhet och medarbetare som är kunniga och vågar att testa och utveckla ny teknik. Men det finns också förutsättningar som är mer specifika för att offentlig sektor ska kunna gära nödvändiga förfyltningar och öka sin Al-avändning och därmed samhålslyntan.
En av de centrala förutsättningarna som behövs är en gemensam, säker och robust Al-infrastruktur med koppling till molntjänster för exempelvis beräkningskraft, Utver det behöver styrningen av Al-avändning inom offentlig sektor utvecklas och möjligheterna att utbyta data mellan myndigheter, kommuner och regioner avsevårt förätbrads. Det innebär att ren ad regelverk behöver moderniseras och anpassas till den digita valår vid I/levr. \({ }^{[155]}\) Detta förutsättet är treging och riksdag har förmåga att fatta långsiktigt kloka beslut i frågor som rör in创新型in ven ad onfentliga sektorns utveckling - en utveckling där användningen av Al måste inte en central position.
Förutsättningarna för ökad användning av Al infotlinger verksamet sammanfattas i Figur 1 och i tresten av kapitel diskuteras och lämnas förslag på de olika områdena i figuren 1) Al-infrastruktur och teknikutveckling, 2) Data, 3) Tillt och förtroende, samt 4) Styrning och möjliggörande regelverk.

För att säkerställa långsiktig tillgång till kompetens som stannar i Sverige är det angeläget att lärosätena erbjuder attraktiva Al-tjänster för nydisputerade forskare. Ett nationellt post doc-program bör därför inärtäts inom Al. Forskarna bör ges årlig finansiering, utan krav på Internationell placering, men med goda möjligheter till internationella utbyten. Syftet med detta är att ge doktorander och forskare, inte minst de med internationell bakgrund, möjlighet att fortsätta sin karriär i Sverige. Det skulle bidra positivt till den nationella forskningen och stärka landets innovationskapacitet.
- Al-kommissionen föreslår att det avsärts meddell till 200 Al-post doc-tjänster till en kostnad om 500 miljoner kronor totalt under tio år. Vetenskapsrådet, i samråd med övriga forskningsfinansiär, bör vara huvudman för en
utlysning och fördela resurserna i konkurens mellan universitet och högskolor.
För att locka hit internationella toppforskare bör också attraktiva paket för utländska vägstrofessorer skapas. Genom att koppla sådana paket till vår unika tillgång till data och de föreslagna excellenscentren och forskarskolorna kan Sveriges attraktionskraft öka.
- Al-kommissionen föreslår att det avsätts medeell mitlmsvarande 50 heltidsanställda vägstrofessorer inom Al till en kostnad om 300 miljoner kronor totalt under tio år. Vetenskapsrådet, i samråd med övriga forskningsfinansiärer, bör vara huvudman för en utlysning och fördela resurserna i konkurens mellan universitet och höigskolor.
\section*{Vision: Förstärk Sveriges profil som forskningsnation inom Al}
Gästprofessorema kan med fördel få i uppgift att lösa de tio viktigaste samhälisproblemen på Al-området. Problemen formuleras av det internationella forskningssamfundet och annonseras i samband med Nobel-festligheterna. Gästprofessorema får bygga upp sina egna forskargrupper inom ramen för de föreslagna excellenscentren och de forskarskolor som föresläs i detta kapitel, för att angripa problemen.
För att öka kopplingen mellan forskning och samhället i övrigt, och samtidigt säkerställa att så många doktorer som möjligt stannar i Sverige, bör flier Kombinationstjänster inrätta, där forskare delar sin anställning mellan akademin och offentlig sektor eller företag. För detta finns exempelvis regelverk som gör det möjligt att adjugnera en professor eller möjlighet att förena anställningar som lektor eller professor med en anställning vid en sjukvårdsenhet. Reglengen om forméone daast-ällningar är dock i dagslagdet begränsad till anställningar som förenas mellan högskol och sjukvårdsenheter. Regeringen har därför remitterat en promemora med förslag om ändringar i högskololelagen i syfte att möjliggöra förenaande anställningar även mellan högskolon och andra sektorer. Dötta våklamorna Al-kommissionen.
Om förslaget går genom vill Al-kommissionen uppumtuna og varshorea att kombinera arbete vid ett lärosäte med arbete i en kommun, myndighet eller företag. Om man<|im_start|> att den myndighet eller det företag då forskaren jobbar delt stärf harflav kostnaden, skulle 500 sådana kombinationstjänster under en tioårspiredel oblasta statsbudgeten med 350 miljoner kronor.
- Al-kommissionen föreslår att det avsätts med delt miltsvorandene 500 kombinationstjänster inom Al till en kostnad om 350 miljoner kronor totalt under 10 år. Vetenskapsrådet, i samråd med övriga forskningfinansiärer, bör vara huvudman för en utlysning och fördela resurserna i konkurens mellan universitet och högskolor.
Den globala konkurrensen om Al-kompetens år hård og much lånader vidtarikte ådtagder för att locka till sig och behälta. Vi att för att för att för att för att Sverige sticker ut på ett. Här kan vi konstans från stäma anledningen är att våtra negativtåndet för<|im_start|>ta anledningen är att våtra en för att för att för att用药ationsregler ofta lelder till att personen från andra länder som har gjort sin doktorsutbildning I Sverige tvings låsman landet, i stället för att stanna och bidra till samhället med sin expertis. Dessa regler uttres nu, \({ }^{197}\) vilket Al-kommissionen vålkomnar. Det är viktigt att en förändring genomförds för att underlätta för fller internationella latgarten stamma varkar efter avslutad utbildning, Utdøver att förenkla processerna och erbjuda bätte vilkvor, i enlighet med våra förslag i detta avsnitt, bör Sverige även fuvosera på att skapa bra villkor som främjar möjligheterna till långsiktig bosättning i landet. Ett uppdrag om detta har gets

\title{
Klinisk bedömning av mammografibilder med Al
}
Bröstcancer är den vanligaste cancerenformen bland kivinnor. 2020 fick 7400 kivinnor diagnosen invasiv bröstcancer i Sverige. På 1980-talet infördes screening där man med hjälp av mammografien genomför röntgenundersökning för att upptäcka bröstcancer. Närmare en miljon kivinnor kallas varje är till en sådan screening, och 60 procent av alla bröstcancerfall upp täcks genom mammografien. Der röntgenbilder som tas granskas av två bröstradologer, som det i dag råder stor brist på.
I en svensk studie från 2023 som omfattade 80000 kivinnor bedömnes häften av kivinnorna av
två radiologer, medan den andra häften bedömdes med Al-stödd screening. Studien visade att granskning med Al resulterade i 20 procent fler identifiabledancerfall, men bara 3 procent fler falska positiva, det will såga där cancermissstanken försvann efter kompletterande utredning, Samtidigt minskade arbetsbördan för radiologen med 44 procent. En radiolog granskar i snitt so mmammografundersökningar på en timme. Det innebär att denna Al-tillämpning spara inte fem månaders jobb på de 40000 sereeningünderökningarna i gruppen som granskades med Al. \({ }^{313}\)
\footnotetext{
[31] Se Kristina Lång, Viktoria Josefsson, Anna-Maria Larsson, Stefan Larsson, Charlotte Hogberg, Hanna Sartor, Solveig Hofvind, Ingvar Andersson, Aldara Rosso, Artificial intelligence-supported screen reading versus standard double reading in the Mammography Screening with Artificial Intelligence trial (MASAI): a clinical safety analysis of a randomised, controlled, non-Hydrolytic, single-blind, screening accuracy study. The Lancet Oncology, Volume 24, No. 802, 2023, 90:649
}
\section*{Lempixel varar hitt gllgling till viktruertruderade data av 6h/kg vallitet är en absolut nödvändighet för att Al ska kunna utvecklas och användas. Men det visar också nödvändigheten i att beakta riskenra för att data den som används återspeglar historiska och befintliga öjelmillkether, så kallad bias, vilka i sin tur speglar mänskliga beslut. Om resultatet från en Al okitriskt accepteras kan det med andra olde tälal att dessa jörliklather, ofte med historisk grund, återskapas. Det är emerlletid osannolikt att antta att data någonsin inte kommer att vara färgåd av blas. Att data är färgåd av blas kan desutom vara en förutsättning för att en Al ska lära sig vad som är bias.
Något förenklat kan man säga att tillgång till data av -information - är nödvändig när det gäller tillämpningen av Al av två olika anledningar. Dels handlar det om så kallad tränngsdats, valiket är den typ av data som beskrevs i början av atttde avsnitt. Dels handlar det om produktionsdata, som används när Al-motiskdellen är i drift. Det är möjligt att utveckla avcervaneade Al-modeller, men utan tillgång till produktionsdata blir modellerna det närmaste värdelösa. Om man inte har tillgång till röntgenbilder att använda den cancertränade algoriment på har man heller ingen nytta av densamma. Det här kan typaska trivlalt, men kan vara en reellt umaning eftersom det ofta finns olika legala begränsningar för delning av data, inte minst inom hälsområdet.

出

\title{
Dataleining, samarbete och problemlösning Innovation inom Al och al Om verktyg för produktoch processinnovatoner har blivit en allt viktigare del av innovationsmarknaden. \({ }^{}\) Här är tillgången till data avgörande. Det gäller privata företags tillgång till data från offentlig sektor, men också datadelning inom offentlig sektor, liksom mellan privata aktörer. Utan tillgång till data blir det ingen Al-innovation. Det är extra tyldigt för unga och mindre företag eftersom tillgångent till eggenonerare dada är begränsad. Större företag kan ha tillgång till egna data, men även här finns det stora behov. framför allt att få tillgång till offentlig data för att träta och utveckla Al-modeller.
Som noteras i kapitlet Data som en förutsättning för Al-utvecklingen, är Sverige onavligt välförsett med data, men möjligheterna att dra full nyttta av den här begränsade. Al-kommissionen föreslär därför en rad åtgärder för att ytterligare underlätta tillgången till data, både till privata och offentliga aktörer.
En åtgär där en omvänd logik i Offentlighets- och sekretesslagen. Grundregeln ska vara att det inte räder sekretess till skydd för den enskilda mellan myndigheter och mellan självständiga verksamhetsgrenar inom en myndighet. Vi föreslär också att en funktion inärttas för att hjälpa och bistå aktörer som vill få tillgång till data från offentlig sektor. Det här skulle påtagligt förbättra förutsättningarna för Al-driven innovation i samhället. \({ }^{}\)
Tillgång till data är dock inte allt som behvös för att utnyttja de innovationsmöjligheter som ligger i Al. Det krävs också samarbete, såväl inom som mellan organisatiserone. Det är en utamning, men framför allt en möjlighet, för ett samarbetsorienterat land som Sverige.
Den typen av samarbeten kan i varierande grad förväntas uppstås constant. Inom respektive organisation ställs det till exempel högre krav på samarbete mellan olika typer av funktioner. Personen med kanskap om affärsproblemen måste exempelvis samarbete med Al-kunniga, dataansvariga och jurister för att Al-öksningar på ett bra sätt ska kunna bidra till att utveckla verksamheten. Det här är något som faller under varje organisations eget ansvar.
Något annorlunda kan det vara med samarbeten då syftet är att utnyttja synergier mellan olika företag. Det finns ett stort antal företag som äger data som skulle kunna vara till stor nyttta för andra organisationer och för samhället i stort. I vissa fall kan företagen säljva hitta affärmsässiga lönsningar som gör att dessa synergier kan utnyttjas till gagn för samtliga parter. AstraZeneca har exempelvis ställt sin data till förfogande till mindre innovationsföretag, se ruta Data som konkurrensmedel - ett exempel från
\section*{Data som konkurrensmedel - ett exempel från AstraZeneca}
\section*{I utvecklandet av nya Al-baserade teknylplattformar}
är tillgången till högvalitativa data nåvdångig och utgörföta en begränsande faktor när Al-baserade kösningar ska utvecklas. Det gäller framför allt mindre innovationsbolgas som inte hara smånger egna
\section*{Som en del av AstraZenecas ambition att stötta och samarbete med mindre innovationsbolg, har företaget etablerat AZ BioVentureHub. Det är en innovationsplattform som erbjuder mindre innovationsblag, en med bågande, en med bågande, en infrastruktur och industriknundande som finns inom AstraZeneca. Miljön har attraherat bolag från bland annat USA, Storbritannien, Nederländerna och Israel. Inom ramen för BioventureHubs verksamhet, och isyfte att kaltselyera utvecklandet av Al-baserade metodologier, erbjuder AstraZeneca tillgång till sin data i de fall ömsesidigt fördelaktiga upplägg kan skapas.
Etbjudandet har resulterat i att israeli innovation Authority (IIA), landets motsvarighertet till Vinnova, nu ger stödt til Israeliska bolag som vill etablera verksamhet i BioventureHub i Göteborg. Ett resultat av detta är att bolaget QurisAl under våren 2024 valde att förläga en del av sin utvecklingsverksamhet till Sverige. QurisAl kombinerar den senaste teknologin inom chip-baserad boligog med Al-genererad predkition. En teknologiplattform som inte bara påskyndar klinisk utveckling utan också reducuerar inte för att.
Exemplet belyser att tillgång till data är ett kraftfullkonkurrensmedel. Genom att erbjuda tillgång till industriella data kan företag på ett kostnadffektivt sätt attrahera räldsledande samarbetspartners. Uplägget bidrar samtidigt till att öka kompetensen inom det aktuella området.

Andra områden där ett nordiskt samarbete kan vara fruktbart gäller samordning av kravställande på utländska etableringar i datacenter. Genom gemensamma, eller likartade, krav på företagen och tillgång till beräkningskraft för inhemska intressenter kan mervårdet för samhället maximeras.
Mot bakgrund av ovanstående välkomnar Al-Hommisisonen planerna på att inrätta ett nordiskt Al-center med delfinansiering av Nordiska ministerrådet.
\section*{Förslag}
- Arbeta för ett mer samordnat nordiskt agerande i internationella förhandlingar om Al-reglering, inte minst i EU.
\section*{Strategiska internationella samarbeten}
Eftersom kompetens om den nya tekniken utvecklas över hela världen är det av stor vikt att vi söker strategiska samarbeten med de bästa forsknings-och innovationsmiljöerna.
Utöver EU-samarbetet och ett tätare nordiskt samarbete bör vi utöka kontakterna med länder som ligger på framkant i sin utveckling av Al. USA, Kanada, Tyskland, Frankrike, Storbritannien och Singapore är exempel på länder som ligger högt på olika Al-rankningar. Det kanadensiska exemplet kan tjana som inspiration på ett lyckat samarbete. Sverige har varit närvarande i det kanadensiska Al-ekosystemet sedan 2022. Relationerna mellan samarbetspartnerna i respektive land är nu mycket väl etablerade. Ett antal konkreta samarbeten är också under uppstart, bland annat inom sjukvården med utbyte mellan fiera kanadensiska och svenska sjukhus.
\section*{Förslag}
- Tekniska attachéer med bred kunskap om det svenska Al-ekosystemet bör stationeras incykelländer, för att bygga strategiska samarbeten mellan svenska och utländska parter. För detta bör totalt 15 miljoner kronor per år anslås till lämpliga myndigheter och organisationer.
- Regeringen bör driva på det transatlantiska samarbetet till exempel i Trade and Technology Council (TTC). Som ett litet och starkv前十eroerende land, är det transatlantiska samarbetet av särskild vikt för Sverige, som skulle gynnas av gemensamma transatlantiska regler.

Andra områden där ett nordiskt samarbete kan vara fruktbart gällt er samordning av kravställande på utländska etableringar I datacenter. Foto: MTZ Graphics/Shutterstock

bör hateras. Det ska även utveckla riktlinjer och besst practices för säker Al-användning och utveckling.
Institutet bör vara en självständig myndighet och ha möjlighet att söka och ta emot externa medel och/eller bidrag för att finanslera sin forskning. Den exterm finansierade forskningen ska vara obereonde från finanslären vad avser metod, resultat och publiceringssätt. Likas bäb/rit institutet inrätta ett expertråd med representanter från relevanta myndigheter, som försvarets radioanstalt, försvarsmarkmen, Myndigheten för samhällsskydd och beredskap, Myndigheten för psykologiskt försvar, Säkerhetspolisen samt från lärosäten och näringslivet. Rådets uppgift ska vara att identifiera nya möjligheter och utmanigar inom Al-säkerhet samt ge strategisk rådgivning till institutet och till regervener. Här ser vi det som viktigt att rådet består av personer med olika kompetenser och erfarenheter.
Vad gäller institutets hemvist föreslär vi att det placeras i nära insalntning till THÖ och Cybercampus Sverige, där forskning om Al och cybersäkerhet predvis (se förslaget ovan). Övriga svenska universitet, högskolor och forskningsinstitut bör också upprunnras att etablera forskningssamarbeten med det nya institutet. Id yetstifter anser vi att en särskilds konvåksningnda på 50 miljoner kronor bör avsättas. Forskningsfonden ska också ha möjlighet att ta emot bidrag från privata aktörder. Al-ksommissionen föreslär också att institutet ska ha kapacitet att vara värd för ett hötgeknologiskt testlaboratorium för Al-säär kehert, inom ramen för våra föreslagna satsningar på berångiskgraft.
Utöver vad som nämnts onva ska institutet aktivt verka för att etablera samarbeten med liknandete institut för Al-säkerhet, såväl inom som utanför EU. Dessa samarbeten ska framst syfta till att stärka den gemensamma kompetensen kring al-Säkerhet. Men vi ser även att de ska likundera bilaterala samarbetssavtal med strategiskt viktiga medlemsländer, initiativat med en med en med en med en med en med en med en med en med praxis inom Al-säkerhet. Därutöver finns det skäl för institutet that a samarbeten med de större internationella företagen som utvecklar Al. Genom dessa åtgärder kan Sverige ta en ledande roll att i säkerställa säker och påtillig Al-utveckling, samtidigt som vi starker vår internationella position som en ledande nation in omidigt in innovation och säkerhet.
En satsning av denna karaktär kan också stärka Sverige internationella ställning. Sverige vore nämligen inte först med ett institut i likhet med det vi föreslär. Tvärtom finns i dag institut för Al-säkerhet i länder- såsom Japan, Singapore, Storbritannien och i USA. \({ }^{}\ I maj 2024 hölls ett globalt möte, Al Seoul Summit 2024, för att diskutera internationellt samarbete mellan nationella institut som arbetar med Al-säkerhet. Vid mötet enades man bland annat om att etablera ett internationellt nätverk av institut som arbetar med Al-säkerhet. Tio länder samt EU komer överens om att samarbete kring standarder för Al-säkerhet, forskning och testning. Sverige har länge varit en föregångare inom digitalisering och teknisk involgation. Med inrättandet av ett institut för Al-säkerhet kan Sverige vara med och ta en ledande roll utvecklingen av säker och påtillig Al, inte minst inom EU. Detta är speciellt angeläget med kanke på upbyggnaden av EU's Al-byrå i Bryssel.
\section*{Förslag}
- Al-kommissionen föreslär att ett svenskt institut för Al-säkerhet ska etableras. Vi bedömer att ett årligt anslag för att drivia institutet på 30 miljoner kronor är nödvändigt. Till institutet ska en forskningsfond på 50 miljoner kronor koppels.
\section*{Al - en existentiell risk?}
Det är möjligt att den snabba teknikutvecklingen så såmningom - enligt visa kanske redain inom de närmaste åren - kommer att leda till att vi än fram till artificiali generell intelligens (AGI). I praktiken innebär det programvara som kan utföra en mängd olika uppgifter och lösa en mängd olika problem, och som är betydligt mer kapabel än de Al-system vi har i dag. Att nå fram till AlG1 skulle innejdära enorma nya möjär att för att för att för att för att för att<|im_start|><|im_start|> det att en sådan utveckling skulle kunna hota mänskighetens existens. Detta genom att Al-syssmen utvecklar en egem vilja som skiljer sig från vår, och betraktar oss som ett hot mot sin eget enfotlevnad eller sina egna mål. Rälen ist ett sådant scenario, ligger i att systemen helt enkelt blir mer kapabla än vi människor och att vår formåga inte rärcker till för att hindra en okontrollerad utveckling.
De som varnat för en sådan existentiell risk har bland annat föreslägt en paus i utvecklingen av Al. \({ }^{}\) Motet har dock invänts att en paus enbart skulle gynna illasinnad aktörer, eftersom dessa sannolikt inte skulle respektera ett sådant beslut. Al-kommissionen anser att onor kring den så kallade existentella risken är något som måste beaktas, men belkagar samtidigt att delar av den offentliga diskussionen om säkerhet har tenderat att fukosera på spekulativa risker med mycket låg sannolikhet. Som vi berör i föregående avsnittär det våt viktigt att samhällets arbete för att han tera Al-relaterade säkerhetstisär er evidensbaserat.

Politisk styrning
Itider av snabb systemövergripande förändring ökar behovet av ledarskap och styrning. Inom området Politisk styrning ligger Sverige på plats 57.
Figur 2 visar jämförelseländernas placering på den vertikala axelin och de poäng som ligger till grund för placeringen på den horisontella axelin. Poängskalan fungerar så att det bästa landet får maximala 100 poäng. Figuren visar att Sverige har en svag position relativt jämförelseländerna; endast Schweiz har en såmre placering. \({ }^{}\) Danmark, Nederlanderna och Norge har nästan dubbelt så många poäng som Sverige, medan Singapore ligger långt före övriga jämförelseländer. Sammantaget indikerar Figur 2 att Sverige har ett omfattande arbete framför sig när det gäller den politiska styrningen av Al-fägor.
Figur 2. Politisk styrning - ranking och poäng

samverkan, \({ }^{1170}\) Aviskten à artt aktörerna ska träffas
deud vis regelbundenhet och utbyta information, men själva fatta de beslut som de bedömer à nödvändiga.
Myndigheten för samhällässlydd och beredskap (MSB) har ien aktuell rapport analyserat hur myndigheter har hanterat tre olika samhällskriser som inträffat under det senaste året. \({ }^{[160]}\) Myndigheten kunde i sin rapport konstatera att det räder brist på proaktiv ågerande, särskilt när det saknas information. MSB föreslår i sin rapport att regeringen bör överväga att införa en handlingsprincip. En sådan princip innebär att berödra aktörer ska agera proaktivt och vidta nödvändiga åtgärder även olsära situationer med brist på information. \({ }^{[181]}\) Al-kommissionen anser att en sådan princip även kan främja användningen av Al i Sverige.
Cybersäkerhet är en fråga som spänner över i princip samtliga kestorer av samhället. I december 2023 beslöt regeringen att ge över 100 myndigheter ett återrapporteringskrav eller updrag i sina regleringsbrev där myndigheten ska redogöra för hur de arbetar med frågor relevaterade till cybersäkerhet. \({ }^{[182]}\) När regeringen i september 2024 beslutade att samla det nationella cybersäkerhetsarbetet i ett nationellt cybersäkerhetcenter med FRA:s ledning, motivaterades det med bett obway en et lyngläre styrning från regeringen. \({ }^{}\) När det gäller säkerhetswyskyd har vissa myndigheter setts begrofenahet att utvotila slylnö yer andra myndighet henter samt att ge ut vägledning kring hur myndigheter och privata aktörser ska tolka lagstiftningen. \({ }^{}\)
\section*{Internationella erfarenheter av central styrning Frågan om ifall det teknologiskfölte \(\mathrm{Al}\) utgör behöver en ny form av central styrning, åttinstone under en viss tid, har i någon mån behandlats i den internationella litteraturen. \({ }^{}\) Frågan om begrofener och möjligheter hos Center of Government (CoG), vad vi i Sverige kallar Statsrådsberredningen och i andra länder premiärnimsterns kalli eller motsvarande, behandlas i ett földe och rambere bute vid OECD. \({ }^{}\) En nylig rapport därfirån keparkar att CoG mahmat under allt större press att kunna navigera i allttemorkplexa miljär med synkorniserade riskre, polarisering och sjunkande förtroende för offentliga institutioner.
För att överbyrgiga gapelet mellan politiken och förlvattningen, förlvatta samhanhällingspolitik, vägleda reformer av den offentliga förlvattningen och samarbeta med medborgare och andra intressenter beskriver OECD att central styrning kräver ett antal förtusättningar. OECD understyrke vikten av en tydlig hundring, en värden som avgörande för resultatet.
Den centrala furling, med med med och av 100 mär. Vi har mandat, mandat, tydligt avgridsade roller, kombinerat med tillimt mellan CoG'er och fackdepartementen. Det är viktigt med ett öppet förtloligt informationutsbyte för att fackdepartementen ska genomföra riktlinjer eller standarder som besulats av CoG'er. CoG'er behöver dessutom rätt personer, med rätt kompetens, på rätt plats. Färdigheter som politiskt förnuft, helhetstänkande, medling och dataanalys har visat sig viktigt. Dessutom behöver CoG rätt stöd (ill exempel budguser eller dataaktöms) för att funnera effektivt.
\section*{Systemövergripande förändring kräver central styrning}
Det är Al-kommissionens uppfattning att vi som samhälle inte enbart kan förlita os på befintliga modeller för styrning för att kunna ta tillvara de stor nytort eller de risker som Al för med sig. Den svenska förvaltningsmodellen tjänar oss väl men är, som ovan visat, inte optimal när det handlar om att lösa en utmanning som spänner över flera olika sektorer. Det behövs, åtminstone temporätt, ett tydligare centralt ledarskap som kan blicko över samtliga samhällssektorer. Nedlan lämmar vi förslag som vi bedömer är nödvändiga för att vi som land ska kunna ta tillvara de möjligheter och hantera de risker som Al för med sig.
\section*{Förslag}
- Regeringen bör fatta beslut om en Al-strategi för Sverige under 2025 där denna Frådplan utgör grunden.
- Behovet av snababa beslut om
systemövergripande åtgärder utgör själva andlenningen till att Al-kommissionen valde att tidigärelägga sin rapport. Finansieringen av regeringens Al-strategi bör därfor grämg a avkrpoppositionen för 2025 eller av en extra ändringsbudget som lämnas till risksdagen under våren 2025.
Al-utevcklingen kräver samordning av politiska beslut i en hastighet som stemet i dag inte
\footnotetext{
[170] See SDI 2005;43-3. 159 för en sammanställning over en del av dessa fora.
[180] See MDI, Ärdenning av den aktorgeneramärn manformen av för att upplikator av ett kribendeskapanspektivf 1920A4/103604.
[181] See MSB, Anvar, samwerkan, Infångatur, ånglärder för stärkt inbåredensväg utfärfn årenfeterna från skogbräden Värtsmanard 2014 (Lü015/1400/5SW).
[182] https://www.reggeringen.se/pressmededlander/2023/12/staatka-krav-v-mydighetens-religginsbiv-kninginformationos-och-schøyserakhetatsbeträkt/.
[183] See MDI, Ärdenning av den aktor generat med för att upplikator av ett kribende skopf, med en med en regeringen 2020. Från dem 1 november 2024 har FRA serverat för centret, dess verksamhet regleras i förordning och dessa theft utes direkt av regeringen.
[184] See MDI, Ärdenning av den aktor generat med för att uppikator av ett kribende skopf.
[185] Se bland samat United Nations System White Paper on Al Governance, 2024, Europeiska Revisionsstränter, Särskild rapport 08/2024, ELS ambitioner 1/4fga om 10/2024, ELS för att upplikator av ett kribende skopf, med en med en 1/4fga om 10/2024, ELS för en 1/4fga om 10/2024.
[186] OECD, Steering form the Centre of Government in Times of Complexity: Compendium of Practices, OECD Publishing, Paris, https://doi.org/10.1787/69b1f1129-en 2024.

har och som behövs för att utveckla avancerade Al-ösningar. Akademin har å andra sida begränsad tillgång till näringslivsbaserade data som krävs för att utveckla nya innovativeren inom vissa områden. Dett koordinationsproblem gör att innovationsmarknaden hämmas. För att motverka problemet har Kanada implementerat ett näringsliv-sakademi samarbete, Mitacs-programmet, där företag definierar affärsmöjligheter och ddrillt hörande Al-tekniska problem och tillhandahåller data, och där forskarstudenter, under handledning av disputerade forskare, arbeter med al-tekniska lösningarna, se faktarutan Exempel från Kanada.
\section*{Den fortsatta utvecklingen av stora språkmodeller för det svenska språket}
De kommersiella stora språkmodeller (eng. Large Language Models) som finns tillgängliga i dag, till exempel ChatGPT, är trånade på textdata från nätet. Efftersom svenska är ett relativt litet språk, och därmed utgör en liten andel av all textdata på nätet, tenderar de har modellerna att bli mindre bra på svenska. Denna begränsning gör att tokltjänster, den med med med med med med med med med med med med med med med med med med med med drikna alportirmer som utvecklas på basis av kommeriellt framtagna språkmodeller riskerar att hålla en förlåg kvalitet för att vara användbara. När en språkmodell tränas på ett annat språk (som amerikansk engelska) och sedan översätter till en svensk version, försvinner dessutom nyanser i språket. Översättningarna tenderar att bli amerikaniserade i sin språkliga logik
Mot den här bakgrunden finns det en rad skäl för att utveckla språkmodeller på svenska, till exempel att bevara den svenska kulturella identiteten och hänsyn till nationell säkerhet. Detta eftersom även samhällskritiska tjänster och verktyg annars måste utvecklas på basis av bristfälliga språkmodeller. Att utveckla en egen språkmodell bidrar också till värdefull uppbyggnad av kompetens. Al-kommissionen anser därför att det i Sverige ska utvecklas språkmodeller på svenska.
Tillgången till stora språkmodeller på svenska bör ses som en kollektiv nyttighet \({ }^{1380}\) som staten bör ha det yttersta ansvaret för. Det motiveras bland annat av att det annars finns en risk att större företag, med resurser att bygga egna stora modeller, skaffar sig ett tollbörligt konkurrensövertag över mindre företag genom att begränsa tillgången till dessa modeller. Modellerna bör dock kunna tas fram i samarbete mellan offentlig och privat sektor så att de kan användas både i det privata näringslivet och inom den offentliga sektorn. Av effektivitvetskäl är det önskvärt med ett samarbete inom Norden eller EU.
\section*{Vad finns i dag?}
En aktör när det gäller att skapa en svensk språkmodell är kungliga biblioteket (KB. Sedan 1661 har myndigheten samlat in allt som tryckts i Sverige. Myndigheten har också digitaliserat enorma mängder dagstidningar, musik, radioprogram, med mera. Det innebär att KB har tillgång till ett i princip världsuntik material. Den tillgång är en stor fördel i arbetet med att skapa en svensk språkmodell med god kvalitet. KB har även använt materialet för att producera flera mindre, skråddarsydda språkmodeller på svenska

- Uppdra át lämplig myndighet attså snart som möjligt tablera en funktion för rådgivning och regulatorisk sandlåda för både privat och offentlig sektor i linje med Al-förordningens krav. Det är centralt att regeringen agerar för att undanörja den osäkerhet som i dag finns kring tillämpningen av den nya förordningen. En utredning tillsattes i september för att lämna förslag kring tillämpningen av Al-förordningen. \({ }^{1133}\)
\section*{Åtgärder för en effektiv finansiering av}
Al-innovation
En väl vannerande finansiell marknad är av stor betydefe för en Al-baserad strukturovmandling. Finansiella marknader har dock vissa specifika utmanigan. Ett grundläggande problem vid företagsfinansiering är asymmetrisk information. Företagts ägare vet normalt per om dess möjligheter än potentiella finansierär. Som en konsekvens av detta riskerar vissa lönsamma investeringar att inte få finansiering till marknadsmässiga villkor eftersom ägarna värderar företaget högre än finansäärerna.
Det kan också vara en utmaning för unga Al-företag att få banklån. Banker är nåmligen sämre lämpade att hantera investeringar i immateriella tillgångar, till exempel algoritter och modeller, som ofta åt det tillgångsslag Al-företag har. Riskkapitalmarknaden och aktiemarknaden är därför oftast bättre lämpad att finansiera investeringar \(\mathrm{I} / \mathrm{Al}\), eftersom de hanterar risken om att ta sen stor del av den potentiella uppidan i företagets avkastning om företaget utvecklas vä.
\(\square\) even om Finansiering via kapital- och aktiemarknaden gerenellt sett är mycket välfungerande kan det finns strukturella orssar som gör det problematisk för vissa Al-företag. Det hänger bland annat ihop med att vissa Al-investeringar sker mycket tidigt i ett företags utveckling. Investeringarna är ofta förknippade med extremt höga initiala kostnader, hög teknikrisk och långsamma intikstströmbran. Det gör att det inte alltid passar riskkapitalbolagen att gå in idessa verksamheter i en tidig fas. Företagen kan ha en sund affarsidé men om tidshorisont, teknikrisk och initiala investeringe int stämmer överens med riskkapitalbolagens naturalg dynamicmark finanseringeren utabl. Det finns därför en risk att det sker för å sådana investeringar uffrån ett samhällsekonomiskt per spektiv. Särskilt viktigt är det att främja utvecklingen av discryptiva teknologier och förmåör som syftar till att ta fram nya möjligheter till innovation. Det handilar
om morgondagens banbrytande innovatiner som har potential att, precis som utvecklingen av stora språkmodeller har gjort, revolutionera användningen av Al. Det kan även vara svårt att hitta finansiering för företag som inte är tillräckligt skalbara för riskkapitalbolagens avkastningskrav. Om affärsidenå mer lokalt betingad, exempelvis i form av att lösa ett problem som är specifikt för svensk sjukvård, kan det också bli problem med finansieringen, även om iden är livskraftig och kan leverera stor samhällsyntta.
Det är också av stor vikt att svenska innovationsföretag dar nyttta av den resars som EU utgör. I kapitlet Beräkningskraft redogrs för EU-initiative Al Factory. Initiativet innebär att bland annat små och edelsterna företag, genom samfinansiering mellan värlandet och EU, får kraftigt subventionerad tillgång till beträkningskraft och resurser för innovation och utveckling av Al-modeller och applikationer. Sverige har också ansök att om att få bilå vår för en Al Factory. Om anskøng bodkänns skulle det bidra ytterligare till att utveckla ett ekosystem för Al bland svenska företag. I kapitlet Internationella positioner föresläs att också åtgärder för att öka möjligheterna att delta i olika U-Program, en med med med med med med med med med med med med med med med med med med med medmed med med med med med med med med med med med med med med med med med med med vid och med och med och med och med och med och med och med och med och med och med ochmed och med och med och med och med och med och med och med och med och med och
\section*{Örslag}
- Utka östö dötvå Vinnova och Almi till projekt och livskraftiga Al-startups som av strukturella skäl, till exempel för hög teknikrisk, exempelvis till föjd av utveckling av disruptiva teknologiler, eller bristande skalbarhet, inte får finansiering i privata kapitalmarknaden. Sammanlagt bör det årliga stödet öka med 100 miljoner kronor årligen, under en femårsperiod.
\section*{Den kreativa förstörelseprocessen och teknikspridningen}
Avgbrande för en effektiv Al-driven kreativ förstörelseprocesser att åt de företag som anvander den nya tekniken mest effektivt växer (organiskt eller genom förväry) medan mindre effektiva företag minskar infkorvär) medan mindre effektiva företag minskar en störlekt. Detta krävet väl fungerande konkurrens. Konkurrenspolitikem har därför en viktig roll genom att den motverkar marknadsmisslyckanden i form av marknadsmakt hos dominerande företag.

\title{
Beräkningskraft
}
\section*{ChatGPT sammanfattar:}
Beräkningskraft utgöryygraden i det digitala samhället och är en avgörande faktor för den pågående Al-revolutionen.
I detta kapitel dyker' vi ner i den komplexa världen av beräkningsresurser, där traditionella datacenter möter den specialiserade kapaciteten hos grafikprocessorer (GPU) som driver Al-innovationer framåt. V uttforskar de strategiska valen mellan polntjänster och egna supderatorer, och hur dessa beslut påverkar allt från forskning till kommersiell användning. Dessutom delsikuter av vid vaderande bighver inom olika skretor och den roll staten kanotsla att skapa en hålltar och pottu furrenskraft in inverstrutt för befakningskraft, enom att kärka dessa appster som för att för att för att för att avnivera i en framtid där Al's møjligheter är gränkösa, men där resurserna är begränsade.
\section*{Vad är beräkningskraft?}
Beräkningskraft utgörs av enskilda eller sammankopplade datorer som utför beräkningarna som ligger till grund för all digitalisering. \({ }^{[17]}\) Vi använder den dagligen, ofta utan att tänka på det. Beräkningskraft är bland annat nödvändigt när vi surfar på mobilen, använder sociala medier eller följer väderprognoser.
Det finns många olika typer av datorer. Det mest centrala i den atör där den enhet som utför beräkningar. dagens datorer används i huvudsak två olika typer av beräkningsenheter, PCUer och GPUr'el', CPUer är designade för att vara generella beräkningsenheter, som ofta gör många beräknningar efter varandra. mycket nabbatt. GPUr (gräfiskprocessorer) var från början bygga förd att skapa bilder på en skärm (dåvarn namnet) vilket baseras på att man utför beräkningar parallelt med olika data. GPUr'er har utvecklats mycket nabbatt och fått en mer generell användning. Deras förmåga att räkna parallelt på många olika data gör att de lämpar sig väldigt väl för dataintensiva beräkningar som till exempel tråning av Al-modeller. En supderator består av ett stort antal enheter, CPUer, GPUr'eler en blandning av dessa, som är sammankopplade med sennabå nätverk och arbetar till SAMSMAN (parallellt).
Den specialiserade beräkningskraften för Al bygger såldes i dag ofta på många parallella grafikprocesser (GPUr'E), Det är till stor del tack vare utvecklingen av nya, mer kraftulla, GPUr's emon de senaste framstegen inom Al har gjorts möjliga. Utmärkande för storskalig infrastruktur för Al är att den på samma
vis som traditionellt kraftfulla datorer är dry, mycket energikrävande och ställer höga krav på kylning.
\section*{Inkött eller egen beräkningskraft}
Tillgång till beräkningskraft är en förutsättning för att privata och offentliga aktörer ska kunna utveckla och använda Al. I dagsläget är det möjligt att få tillgång till beräkningskraft genom två tillvägångssämt. Antingen genom inköpta polnätjänster, vilket innebär att man hyr in sig på ett datacenter som ågs av externa aktörer. Här därer den klar amerikanks dominans. Det andra alternativet är att infösskaffa egen beräkningskraft genom att köpa datorer.
Den största fördelen med att använda molntjänster av att det är enka till att använda och möjligger 0 sennåk uppstart. Man kan därmed ska valp som sin Al-verk- samhet utan kostsamma investeringar. Det här är sisktär våldfult om behovet av beräkningskraft varierar mycket över tid, eftersom en egen dator da skulle sätt outnyttjad under perioder. Om man kan utnyttja egen beräkningskraft väl blir dock kostnaden per GPU-timme betydlig högre för dessa molnjänster än för egen beräkningskraft.
Det här gör att molntjänster snabbt kan bli mycket dyrt om man använder ett stort antal GPU-timmar.
En ytterligare dackel med molntjänster är att kontrollen över känsliga beräkningar och data kan gå förlorad om data skickas utanför landets gränser, in synerhet utanför EU. Detta kan även gälla vid utnyttjande av utländska leverantörers molntjänster
\footnotetext{
[17] Här definirar vi beräkningskraft som att det förutom datorkraft, även inkluderar lagring av data och interna nätverk mellan data och system.
[18] CPU står för Central Processing Unit och GPUr för Graphical Processing Unit.
}

\section*{Innovation för förnyelse}
Sverige har långe varit ett ledande innovationsland och är det på många sätt forfarande. Med AI har emellertid perspektivet på innovation förändrats. AI är nämligen inte bara en innovation i sig själv - det är framför allt ett verktyg för ytterligare innovation. Möjligheterna till innovation och innovationstakten ha därför ökat dramatiskt. För att Sverige ska åhnga med i utvecklingen måste vi se till att ha ett klimat av att till att till att till att till att till att klimat av kreativiteten i företag, akademi och offentig se ktor verka med så få och små hinder som mögligt. Hår hjälper många av våra förslag in om medån medån till, till exempel gällande datadelning. Det finns dock behov av ytterligare åtgärder.
Som vi tidigare namnett betraktas regleringen av Al-relaterade frågor, i Synnerhet GDPR och Al-förordningen som svårtkodal och komplex, inte minet av små och medelstora företag (SMF). Det beskrivs ofta som det kanske största hindret mot innovation. Därför företasvi av tit integritetsskyddsdynamigheten (IMY) ska gesas utökade resurser att kunna erbjuda regulatoriska andslådor riktande till företag. Vi föreslår också att det utökade resursera skna användas för att vägleda företag om hur olika idéer fungerar till SAMLLammande GDPR. \({ }^{}\)
Ytterligare en satsning för att fråtnja innovationsfarkraften Bland SMF är EU-kommissionens nya koncept som kallas för Al Factory. Det innebär att man på olika pläster i unionen bygger upp funktioner som erbjuder beräkningskraft och tjänster för framför allt SMF, i syfte att dessa ska ha möjlighet att experimentera, träman modeler och vidareutbilda sig, Jarje Al Factory samfinanseras av vårdlandet och EU. Sverige har uttryckt önskemål om att få vara värd för en av dessa, något som är möjligt eftersom det tidigare beslutats täggs en nya en upsuterator, kallad Arrheniya. Denna kommer att vara delfinansierad av EU och del av ett EU-nätverk av supardatorer, något som är ett krav för att få ansöka om vårdskap för en Al Factory. Vi föreslär att regeringen anslär medel för att medfinansiera värdskapet för en Al Factory. En sådan etablering skulle beteya mycket, såväl tekinstik som ekonomiskt, för SMF's möjligheter att bedriv avaancerad Al-innovation.
\section*{För att innovationsföretag ska kunna uppstå och växa behövs det riskkapital. Svenska riskkapitalmarknader fungerar på det stora hela. Fru is er därför inte några generella behov av stöd från det offentliga. Det finns dock behov av vässi krätade åtgärder mot företag och projekt med stort potentiellt mverråde som har svårt att få privat finansiering. Det handlar dels om innovationsfion med mycket hög teknikrisk, så kallade disruptiva teknologier. Dels om affärsidéer som inte är tillräckligt skalbara för att attrahera privat kapital, men där samhällsynytan kan vara betydande. Vi föreslär därför utökade medel för Vinnova och Almi i syfte att stötta denna typ av innovation.
Vi föreslär också att Vinnova utreder möjligheterna att främja sektorsöverskridande projekt, där värdet av en lösning kommer fyllgt gån en anbetar de berörda parterna. Staten måste ha beredskap att skjuta till betydande medel när värdefulla projekt av den här typen identifierfals.
\section*{För att kunna utveckla Al-lösningar behövs ofta tillgångt til stora språkmodeller. Vi lämarr förslag på hur vi bör gå framå med att utveckcia slådana modeller på svenska på ett sätt som tar tillvara den erfarenhet som finns i Sverige och vår unik tillgångt till offentliga data, samtidigt som upphovrätten respekteras.
\section*{Nästa steg}
Det viktiga nu är att sabbnt komma till offensiva beslut som stärker Sveriges konkurrenskraft på Al-området. Det är också anledningen till att vi Al-kommissionen har valt att tidligarelaggå var rapport från viljol 2025 till att stora för att vi ska kunna stärka konkurren reskraffen behövs för att skorvill att skorvill att skorvill att skorvill att skorvil att skorvill att skolma dörriden. I Färdplanen föreslär vi konkreta nåtgåder som vil bedörner kan beslutas och genomföras relativt ogåndende, det vill såga rednan under år 2025. Det handlar om satsningar och upprad till myndigheter som Regeringskanslålet har stor erfarenhet av att bereda och hantera i budgetprocessen. Andra förslag måste utredas vidare för att finan si mer precisa form eller för att kunna representeras som färdliga lagförlsga. Det är dock av största vikt att dessa processer sätts i gång skydsamt.
Regeringen bör därför, så samt som möjligt, anta en Al-strategi baserat på denna Färdplan. Finansieringen av våra förslag för framg av våråndringesbudgeten för 2025, eller av en extra ändringsbudget om lämnas till riksdagen under våren 2025. Implementeringsarbetet bör drivas på och övervakas av en task force placerad på Statsrådsberedningen. En viktig uppgift för den föreslagna task föresblor en klir att fortsätta det arbete som Al-kommissionen har påbörjat: att skapa samförstånd om hur vi kan åstadkomma det samhälle vi har beskrivt avon, med Al i medborgarnas tjänst. Samförstånd skapar förutsättningar för samarbete om frågorna, vilket är ndövändigt för att nå de önskade effekterna.
\footnotetext{
\({ }^{}\) Vem som ska ansvara för den regulatoriska sanddälnen om ramen för Al-förordningen för invertingen Al-kommissionen tar ställning till. Den frågaren kommer att behandliga i utredningen Tyg och intrettiling användning av Al som regervenen till根源 i september 2024.
\({ }^{}\) I syfta att följa selvar Kommitt/förordningen som de drätivs som regervengen beslutat för vårt arbete, kommer vi colsat att latta tyckla denna Färdplan i seiner Stantes effentliga utredningar (BCU).
}

\section*{Förslag}
- Al-kommissionen anser att det är viktigt att offentliga aktörer i sin verksamhet använder Al på ett transparent sätt i syfte att bibehålla medborgarnas tillt. Det är inte minst viktigt för att skapa förtroende för de förslag vil lag ovan. Regeringen bör därför överväga att ge myndigheter i uppdrag att ta fram etiska riktlinjer inom sina ansvarsområden om behov skulle uppstå.
Omårda 4: Stimulera användningen av Al genom långsiktig styrning och möjliggörande regelverk som vi tidigare har påpekat är användningen av Al inte ett mål i sig, men det är ett avgörande verktyg i den nödvändiga utvecklingen och förnyelsen av offentlig verksamhet. Det är därför problematiskt att Al-ijänster inte används i tillräckligt hög grad. I detta kapitle har vi lämnat en rad förslag på nödvändiga åtgärder för att öka användningen i offentlig sektor. \({ }^{167 f}\)
Det är möjligt att detta inte räcker. En ytterligare utmanning för att komma framåt är den förskitighet som på flera sätt pråglar den offentliga förvatlningen. I alltför hög grad har riskminimering varit en vägledande princip för utvecklingen, och är det än i dag. Om den offentliga verksamheten ska kunna digitaliseras måste det i högre grad bli möjligt att pröva nya ösningar. Det kräver förändringar både av regelverken och av innehållet i styrningen av offentlig verksamhet.
I dag hindrar eller bromsar gällande rätt i hög grad nödvändig teknikutveckling, både genom generella regler och utformningen av verksamhetspecifika författningar. Regelverken som styr stora delar av offentlig verksamhet behöver därför bli väsentligt mer möjliggörande. Utan en omfattande översyn kommer inte den nödvändiga utvecklingen mot mer automatiserade arbetsprocesser och seltøt infentlig verksamhet komma till stånd.
\section*{Förslag}
- Regeringen bör genomföra en översyn av regelverk som styr offentlig verksamhet i syfte att anpassa dem till en digital transformation med hög grad av Al. Detta kan exempelvis ske genom uppdrag till myndigheter att se över föreskrifter och allmänna råd som de ansvarar för, och att lämna förslag till ändringar av lagar och förordningar inom respektive verksamhetsområde.
Till sist. De förslag som beskrivs ovan ger offentlig sektor bättre förutsättningar att utnyttja potentialen med Al. Det är sedan upp till de offentliga aktörerna att ta villvara dessa möjligheter och utveckla respektive verksamhet med Al som verktyg.

EU-länder. Detta bidrar ytterligare till en tevkan att använda sig av dessa tjänster.
De regulatoriska hindren och osäkerheterna kring Al-regelverken gör också att europeiska aktörer riskerar att få tillgång till de senaste och mest avancerade Al-verkytgen senare än aktörer i andra delar av världen. Exempelvis meddelade företaget Meta, tidlare Facebook, juli 2024 att man inte kommer att lansera sin senaste multimodala modell i Europa, med motiveringen att den regulatoriska miljön var alltför förötutsäbar. \({ }^{104}\) På sammas sätt lanserade Google sin Gemini \(\mathrm{Al}\) i december 2023, med undant geför EU.
I en tid när utvecklingen av Al går svindlande snabbet kan sådana fördröjningar få stora konsekvenser. De aktörer som snabbast kan tillämpa den senaste tekniken får ett betydande försprång, medan de som halkar efter riskerar att förlora i konkurenskraktf. Aven om EU-lanseringen av en ny modell bara dröjer sex måndaker kan det få väldigt negativa effekter på våra
företags förmåga att håvda sig i den globala konkurrensen. Detta gäller inte minst för mindre företag och startups som är beroende av att kunna skala upp sin verksamhet snabbt. Förutom att det försvårar för existerande och blivande företag i Sverige och EU, ökar det risken för att företag väljer att etablera sig utanför EU. För Sverige och EU som helhet innebär den begränsade tillgängligheten av dessa plattformar och verktyg en risk för att v tappar konkurrenskraft och går miste om de potentiella välfärdsvinster som en utkad användning av Al-lösningar utgör i samhället.
- Al-kommissionen understryker hur avgörande det är att de här plattformarna och verktyng tillgängligörs samtidigt i Sverige och EU som i resten av världen. Som föresläs i kapitlet Internationella positörner är det därmed av stor vikt att svenska företrädare verkar för att den Al-relaterade EU-regleringen inte bidrar till försämrad tillgängligheter för dessa verktyg. De tekniska attachser om föresläs i samma kapitel bör också aktivt verkfa för att minska dessa risker.
\section*{Al-förordningen}
I augusti 2024 trädde EU:s förordning om harmoniserade regler för system för artificiell intelligens, Al-förordningen, i kraft. Förordningen är det mest omfattande Al-regelverket i hela världen. Det nya regelverket ställer krev på säkerhet, etik och sminkliga rättigheter utifrån fyra riskkategorieri; Al system med oaceptabel i risk, höyr sk, begränså för att för att för att för att för att för att för att förbituds medan de som enbart medför minära i är accepterade och därmed inte behöver regleras.
De Al-system som omfattas av förordningen behöver bland annat följa regler som kärver transparentsare, nogrann trestning, att allvarliga incidenter rapporteras och att systemen håller en viss nivå av chepsäkerhet. Förordningen ställer även k rav på att för att för att för att för att för att för att för att för såndlåda för Al. Med detta menas en kontrolleradar ram som inrättats av en behörig myndighet och som erbjuder leverantörer eller portetiella leverantörer av Al-system möjllighet att utveckla, träna, validera och testa sina system enligt en specifik sandäldeplan. Detta sker under en begränsad tid och under med att.
Även om förordningen trädde i kraft den 1 augusti 2024 ska majoriteten av dess regler börja tillämpas först den 2 augusti 2026. Förbuden för Al-system med oaceptabel risk tillämpas dock redan efter sex månader och reglerna för Al-modeller för allmänna ändamål efter tolv månader (räknat från den 1 augusti 2024).
Övervakning och implementering
varje medlemsland ska senast den 2 augusti 2025 utse de nationella myndigheter som ska sansvara för implementeringen av Al-förordningen och marknadsövervakning, särskilt vad gäller Al-system med högr risk. På EU-nivå kommer EU-kommissionen att inrättå Al-byrån (Al Office), som ska koodinera implementeringen i medlemsländerana. Al-byrån får även ett särskilt ansvar att utarbeta detaljerade regler för Al-modeller som är utvecklade för allmänna ändamål, samt att övervakta tillämpningen av dessa. Därtuövertillsärts på EU-nivå tre rådgålvande organ:
1. en europeisk styrelse för Al, bestående av representanter på hög nivå från medlemsländerna.
2. ett rådgivande forum med representanter från forskning, industri och civilsamhället, samt från små och medelseltaor företag i syfte att tillhandahålla teknisk expertis.
3. en vetenskaplig panel med oberoende experter som stöd i implementeringen.
Regeringen gav i september 2024 en särskild utredare i uppdrag att bland annat lämna förslag på vilka svenska myndigheter som ska få uppgifter enligt förordningen samt vilka lagändringar som kommer att bli nödvändiga. Utredaren ska lämna sina förslag senast den 30 september 2025.

Färdplan för Sverige | AI FÖR ALLA
Figur 3: Antal ansökta patent per miljon invånare
Audio-visuell teknologi

Källa: EPO. Norbäck och Persson, 2024, Den Al-drivna strukturomvandlingen av det svenska näringslivet, mimeo, IFN, Stockholm

Tyvärk van i konstatera att det svenska utnyttjandet av dessa program är lägre än önskvärt. Sett till Sverigres relativa folkmängd ligger vi i regel efter övriga nordiska EU-änder, men även icke-EU landet Norge. Det finns troligen fera bidragande förklaringar till det läga svenska utnyttjandet. En orsak är att kunskapen bärkta, för att skorning och av den varvåter kängiga svagheter i våra sjana vara läg. En att<|im_start|> av den varvåter i våra gnafsnieringsmodeller för forskning och innovation, och Vinnovas förutsättningar att underlätta deltagande i den här typen av projekt. För att kunna ansäka om deltagande ett tid. Ett digital Europe projekt krävs att svensk medfinansiering redran är särkad. Vinnova, som är kontaktnydnighter för den här typen av EU-program, har dock i nuläget inte möjlighet att ge förhandsbasked om sådan medfinansiering. Orsaken är bland annat att det skulle utsätta Vinnovas budget för stora risker, eftersom man inte vet på förhand hur magma, eller vilka, av de aktuella EU-projekten som beviljas. Problemet beskrivs av magna aktörer som något av ett moment 22 då EU kräver förhandsbesked för om medfinansiering från Sverige. Vinnova kan i sin tur bara godkänna medfinansiering av EU-projekt som har kompletterande finansiering klar. Om Vinnova har för att<|im_start|> av den varvåter i våra sjana av förmåt som av den av den av de attom medfinansiering. Det är storning och av den varvåter i våra sjana av förmåt som av den varvåter i våra sjana av förmåt som av den är storning och av den varvåter i våra sjana av förmår. Det av den varvåter i våra sjana av förmåt som av den/skorning och av den varvåter i våra sjana av förmåt om av den/skorning och av den varvåter i våra sjana av övriga.
- Ge Vinnova möjligheten att finansiera flærliga EU-projekt, genom att ge myndigheten ett beställningsbemyndigande. Al-kommissionen väklomnar regeringens förslag om detta.
- Tilldela Vinnova och Vetenskapsrådet ökade medel för medfinansiering av Digital Europeprojekt på i storleksordningen 160 milioner kronor per år. Det bör också skapas ett bufferutrymme för att i högre grad än i dag kunna ge förhandsgaranti om medfinansiering. Genom en öronnäkt buffertford, med en initial kapacitet på cirka 50 milionjer kronor, bör årliga svängningar i andelen beviljade ansökningar kunna hanteras.
- Utred möjligheten att, enligt norsk modell, ge ex post-anslag till lärosäten i relation till hur magna Eu-projekt man deltar i. Syttet är att kompensera för den ofta läga overhead-kompensationen från EU. Regeringen bör avsätta ca 30 milionjer kronor årligen för sådan kompensation.
- Det svenska deltagandet i EU-finnansierade program är särskill lågt på företagsssidan. Regeringen bör därför se över den nuvarande modellen för att bevaka utlynsningar och sprida information om EU-finansierade program till alla relevanta aktörer. Samrådet med näringsliv och akademi är viktigt också ur de nanna synvinkel. Vinnova och Digg bör också vidta åtgärder för att öka synligheten för EU-program som Horizon Europe, inklusive EIC deep teeth fund, och Digital Europe.
\section*{Mycket att vinna på nordiskt samarbete} Det finns många skål för de nordiska länderna att fördjupa itt samarbete på Al-området. Det gäller inte minst i arbetet med att utforma internationella en av de integas och av de integas och av de integas och utveckling. Med tanke på våra många gemensamma drag, till exempel en stark demokratisk tradition och värderingsburna samhällsvisstem, skulle ett mer samordnat norditsk ägerande öka våra chanser att påverka arbetet i en riktning mot innovativ, säker och etsiv ankvändning av Al. Här borde det finnas utrymme för Norden att skapa sig en nisch inom utveckling och användning av Al. Det faktum att Norden, och även Norden-Baltikum, år en region med en stark position inom IT och IT-relaterad innovation, innebär att ett samordnat agerande i internationella sammanhang skulle kunna få en betydande politisk tyngd. Inte minst inom EU.

<|im_end|>};

<smiles>[CH]

- Svenska synpunkter tidigt förs fram till EU- kommissionen på deras planerade arbete på Al-området (även innan formella förslag läggs på bordet).
- Hitta gemensamma lösningar på frågor kopplade till upphovsrätt och avändande av generativ Al.
- Svenska intressen och befintliga standarder beachtas när nya datastandarder tas fram inom EU.
- Inrätta öppna och transparenta strukturer för samråd med näringsliv, akademi och representanter för kommuner och regioner, för att möjliggöra att svenska förhandlare har tillgång till information om den senaste utvecklingen och svenska intressen på området.
\section*{Sverige måste dra mer nytta av EU:s forskningsoch infrastrukturprojekt}
EU gör för närvarade stora satsningar på olika forsknings- och infrastrukturprojekt med Al-koppling. Det presenteras kontinuerligt projektförslag som forskare och innovatörer från medlemsländerna får ansöka om att leda eller delta! \({ }^{[174]}\) Sädana projekt utgör utmärtka tillfällen för svenska aktörer att bygga kompetens och nätverk, och stärka Sveriges roll som framstående forskningsnation inom Al. Genom att projekten hel- eller delfinansieras via EU-budgeten (i Digital Europe till 50 procent hålls också. kostnaden nere för svenska intressenter; i Faktaruta Horizon Europe och Digital Europe beskrivs dessa program.
\section*{Horizon Europe och Digital Europe}
Horizon Europe omfattar forsknings- och innovationssamarbeten inom många olika sektorer. Budgeten för perioden 2021-2027 år 93,5 miljarder euro. Programmet satsar på Al i olika delar.
1. Satnsningar på grundläggande forskning.
2. Finansiering av lovande innovatörer och SMF:s genom European Innovation Council.
3. Projekt som ligger i den absoluta vetenskapliga framkanten ("scientific excellence") genom European Research Council.
4. Europeiska forskningspartnerskap mellan privata och/eller offentliga aktörer inom ramen för "Al, Data and Robotics".
Digital Europe-programmet är särskilt inriktat på Al-samarbetet. För perioden 2021-2027 år budgeten på 7,6 miljarder euro. Det finns sex delområden.
1. Högpresterande datorsystem (HPC).
2. Artificialt intelligens, data och polnlösningar.
3. Cybersäkerhet.
4. Avancerade digitala färdigheter (Advanced digital skills).
5. Säkerställa bred användning av digitala teknologier (Accelerating the Best Use Of Technology.
6. Halvledare (Chips).

till ett tiolt myndigheter som samordnas av Tillväxtverket. Al-kommissionen ser fram emot resultaten från det uppråget. \({ }^{}\)
\section*{Satsningar för att stärka den långsiktiga}
spetskompetensen
Ett bra sätt att utveckla den långsiktiga spetskompe-tensen i Sverige är genom att inrätta forskskolor, där Al-kunskap integrersa i ämnesspecifika forsatkurbildningar. Forskarskolorna bör organiseras ämnessis, eftersom olika områden kommer att ha olika tillämpningar. De bör inkludera gemensamma kursmoment i grundläggande Al-teknologi men också andra ämnen sömast elt, juridik och säkerhet. Utöver detta bör respektive inriktning ha fördjupningskurseran appassade till respektive ämnesområde.
Forskarskolorna bör vara nationella, där doktorandtjänsterna fördelas i konkurrens mellan universitet och högskolor. Här kan inspiration med fördel hämtas från befintliga koncept inom till exempel WASP, WASP-HS \({ }^{}\) och SciulféLab, som organiserar en forskarskola med 200 doktorander inom datadriven life science med fokus på Al. Al-kommissionen anser att en lämplig målsättning är att utbilda 600 doktorer med Al-kunskap under en tioärsperiod. I de föreslagna forskarskolorna kan det med fördel ingå både doktorander som har sin anställning vid ett lärosäste och doktorander inom det privata näringslivet eller ofentlig sektor.
- Al-kommissionen föreslår att det inrätta forskskolor inom Vetenskapsrådets huvudområden och att det avssätts 2,4 miljarder kronor för detta ändamål under en tioärsperiod, vilket motsvarar 600 doktorer. Vetenskapsrådet, i samråd med övriga forskningsfinansiärer, bör vara huvudman för en utlysning och fördela resurserna i konkurrens mellan universitet och högskolor.
\section*{Al4Science - Al som ett nyttt vetenskapligt instrument}
Användningen av Al inom forskning har gölts namnet Aläscience, och kommer ha potential att helt förändra hur kunskap skapas. Utvecklingen av nya vetenskappliga metoder som baseras på Al och maskininlärning har potential att accelerateera den vetenskapliga utvecklingen mångfalt. Stora framstag her radan gjorts inom biologi, materialvetenskap, meteroologi men även inom humaniora och samhällsvetenskap, Al kan påskydna upptäckter och förbättra forskningsprocesser genom att till exempel integrera resonende al-System, datadriven maskininlärning, stora språkmodeller och datavisualisering i ämnesspecifik förvärs.
Forskarskorlar inte bara den teoretiska och praktiska forskningen inom olika områden, utan förändran samtidigt hela den vetenskapliga processen. Al förändrar hur undervisning bedrivs, uppsatser skriv, litteratur konsumeras och doktorander handleds. Att på bästa sätt utnytja Als potential inom vetenskapen kommer att bli avgörande för hur den nationella forskningsut, excklingen kan både skapa ny, och inom redan starka områden bibehålla, internationell konkurrenskraft. För att till dog@doziga sig de nya möjligheter som tekniken har grävs utbildningsinsatser förskare, doktorander och teknisk personal. Det måste ske på en tvårvetenskaplig, generell nivå som spänner över många, om inte alla vetenskapliga discipliner. Men det
måste också ske på mer specialisarede nivåer med fokus på ämnesspecifika verktyg som är centrala för respektive vetenskapsomårde. \({ }^{}\)
Det kräver också att vi systematiskt bygger orga- nisation och infrastruktur till stöd för utveckling, implementering och integration av Al i vetenskapliga sammanhang. Sådan infrastruktur omfattar både hårdvara för träning och användning (inferens) av Al-modeller, samt kompetenslukster som kan stötta den tvårvetenskapliga utveckling och integration som krävs. I den snabba utvecklingen med kortare kasdammal flanskoning och tillämpning krävs också ett stark samarbete mellan akademä, näringsliv och offentlig sektor. \({ }^{}\)
Ett flertal olika åtgärder kan snabba på utvecklingen av Aläscience och ge svensk forskning en konkurrensfördel. Möljiga frågeställningar är att: utveckla teori och praktik runt Al-systems mell or i forskningen och dess förmåga att resonera, baserat på data at biomination med etablerad vetenskaplig kunskap; förbättra metoder för insamling, hantering kring och syntes av hökvaltativa vetenskapliga data - företråd visivens attentionellt; ta fram kritlinjer runt ett kilit, och ansvarsfrågon inom Al för vetenskap; och utveckla robusta utvärderingsmetoder som på ett ansvarsallt sätt kärgor prestanda, förämpar och begränsningar kring den vetenskapliga användningen av Al.

<|im_end|>};

田

Sveriges goda förutsättningar för att husera stora datacenter har gjort att flera företag valt att placera sina verrar i svenska anläggningar, se faktarutan Etableringar av datacenter. Ur ett globalt utsläppperspektiv är detta positivt, eftersom utsläppen från Al minskar när den elintensiva datoranvändningen i högre grad sker med el som är fossilfri, i stället för fossilbeorende. Samtidigt kan denna utveckling leda till att stora mängder energi avsätts för datacenter i Sverige. Givet det ökande elbehovet i andra samhällssektorär är det därför viktigt att elanvändningen i datacenter skapar samhällsytta. Detta gäller framför
allt vid större etableringar som kräver mycket el. \({ }^{}\) Till exempel ökar datacentrens relativa samhällsynfffat om svenska aktörer får tillgång till datorkraften i centren. \({ }^{}\)
Förslag
- Al-kommissionen föreslår att regeringen bör inleda dialog med aktörer som är i färd med att etablera stora och energikrävande datacenter. Förslaget syfårt till att nå överenskommelser som värnar Sveriges intressen.
\section*{Etablering av datacenter}
\begin{tabular}{|c|c|}
\hline Företag & Beskrivning \\
\hline Meta & \begin{tabular}{l} 
Under 2013 öppnade Meta serverhallar i Luleå, vilket var företagets första etablering utanför USA. Server- \\
hallarna är en del av Metas globala infrastruktur som används för att drivta Metas plattformar, inklusive \\
Facebook, Instagram och WhatsApp.
\end{tabular} \\
\hline Amazon Webb Service & \begin{tabular}{l} 
Företaget har sedan 2016 datacenter i Västerås, Eskilstuna och Katrineholm, som används till att erbjuda \\
verländer och Al.
\end{tabular} \\
\hline Microsoft & \begin{tabular}{l} 
Microsoft tillkännagav under 2024 en investering på 33,7 millilder kronor i datacenter anpassade för \\
molntjänster och Al.
\end{tabular} \\
\hline Coreweave och EcoDatacenter & \begin{tabular}{l} 
På EcoDataCenters anläggning i Falun kommer det amerikanska bolaget Coreweave att investera i ett \\
datacenter för att kunna erbjuda tjänsten "GPU as a service". Investeringen uppgår till totalt åtta millärder \\
kronor.
\end{tabular} \\
\hline EcoDataCenter & \begin{tabular}{l} 
EcoDataCenter har förvärvåt delar av Kvarnsvedens papersprukki Börlänge för att bygga Al-infrastruktur. \\
Företaget har även initierat ett projekt i Ostersut till samhammas med bland andra Jämtkraft och företaget \\
WASM, med en total investering om cirka 18 millrälder kronor. Projektet värtar i skrivande stund på \\
miljötillständ.
\end{tabular} \\
\hline
\end{tabular}
\section*{Räcker elen för Al?}
Al-tekniken har accelerareat samtidigt som kferna andra energikrävande samhällsprojekt ska genomföras. Sveriges elbehov har legat i det närmaste konstant under de senaste omkring 30 åren för att nu börjta öka. Enligt svenska myndigheter förväntas den årliga elanvändningen öka från dagens 140 ThU till 200-340 ThW år 2045. Dessa siffror trar dock inte höjd för den potentiella köning av efterfrågan på el som förväntas ske talt med att användningen av Al ökrat. \({ }^{}\) Sifbrorna inkluderar heller en det pintovis effekt Al kan ha på rentergisystemen.
Den framtida efterfrångtan på el drivs del av övergänen från användningen av fossila bränslent il, ek exempelvis i transporter, dels av ny elförbrukning som uppstår genom ökad förädling av svenska fråvaror. Ett sådant exempel är en ökad stälproduktion från svensk järndmalm. Av framväxten av nya industrier påverkär
elanvändningen, såsom tillverkning av elektrobränslen, batteriproduktion och ny gruvverksamhet. Många av dessa projekt är ännu osåkra, vilket förklarar det stora spannet i uppkastntingarna om framtida elbehov. Erobrende på de antaganden som görs om bland annat industrins elektrifiering uppskattar myndigheterta ett albehovet fram till 2045 kan öka med mellan 40 och 140 procent.
Uvetckling och användning av Al är en del av lerna utveckling. Aven om det inte går att uppkastmtka hur stor nettopåverkan blir på framtida elbehov, står det kart att Al kommer att vara ebroende en av stabil och riklig elförsörjning. Rätt hanterat kan Sveriges elproduktion skapa gynnsamma förutsättningar för Al, till exempel som att säkerställa att datacentren skapar samhällsynfts. Men det innebär samtidigt att perioder av elunderskott kan få långtgående konsekvenser för utvecklingen av Al i Sverige.

\title{
Tillgång till internationella Al-resurser
}
\section*{ChatGPT sammanfattar:}
Ien värld där artificiell intelligens formar framtiden, står Sverige inför både möjligheter och utmaningar när det kommer till tillgången till internationella Al-resurser. Medan nationer som USA och Kina leder utvecklingen, kämpar EU för att hänga med i denna snabba teknologiska omvandling.
Detta kapitel utforskar den kritiska betydelsen av att svenska och europeiska företag har fortsatt tillgång till de senaste Al-verktygen och plattformarna, trots den komplexa globala konkurrensen. Vi granskar den värdekedja som ligger till grund för Al:s värdeskapande och hur det är tillämpningen av dessa verktyg som de verkliga fördelarna kan realisera. Genom att undersöka både möjligheterna och de regulatoriska hindren, belyser vi hur viktig öppenhet och tillgång är för att Sverige ska kunna håvda sig i den globala Al-arena.
\section*{Öppenhet är viktigt}
Sverige är en liten öppen ekonomi, och som sådan i hög grad beroende av vår omvärld. Detta gäller inte minst inom Al-området. Det faktum att produktion och utveckling av Al-relaterade varor och tjänster inte är jämnt fördelad över världen utgör en särskild de utraming. Området domineras i ulnålet av USA och Kina. Aven länder som Israel, Kanada, Singapore, Storbritannien och Taiwan har framskjutna positioner. Detta hängt är att skor tillgång till hängt av 1000 företan. Denna utveckling är en viktig faktor bakom den ökade strävan mot strategisk autonomi, något som inte minst prägral politiken inom eiu. EN central del i den processen är att bygga upp europeiska förmågor på strategiska områden. Det här är en lovwärd ambition och ett viktigt skäl bakam Al-kommissionens uppomming om att öka de svenska insatserna på Al-området. Det är dock viktigt att den här ambitione inten innebär ett fjärmande från resten av världen. Det vore inte bra på lång sikt, men på kort och medellång sikt kan konsekvenserna bil förödande, i synnerhet inom Al-området.
\section*{Värdeskapande i produktion och utveckling av Al -fokus på tillämpning}
Var själva värdeskapandet sker i produktion och utveckling av utveckling av Al kan illustreras i en så kallad värdeledja som sträcker sig från grundforskning till praktisk användning av färdiga Al-verktyg och plattformar (tillhandahållina via molntjänster). Sådana plattformar kan brevisvars som digitala miljörder där privatpersoner, företag och aktörer från offentig sektor mot betalning får tillgång till olika typer av Al-tjänster. Plattformarna tillhandahållas typsikt sett av stora amerikanskaa techbolag. Denna värdekedja kan beskrivas i fem huvudsakliga steg:
Steg 1: Grundforskning och algoritmutveckling Det har håndlar om toeretisk forskning och innovatiner som ligger till grund för framtida tekniska framstig. Exempel på banbrytande forskning är artikeln Attention Is All You Need, som är 2017 introducerade de så kallade Transformer-arkitekturen, vilket var starten för de stora spårkmodellerna. \({ }^{}\)
\section*{Steg 2: utveckling och produktion av infrastruktur och berkänkingsresurser}
Det rör bland annat specialiserad hårdvara och molntjänster som tillhandahåller nödvändig berkänngskraft och lagrigsutymme för att utveckla, träna och implementera Al-modeller. Exempel på detta är design och produktion av särskilda haldvledare lämpade för Al-beräkningar, så kallade Graphic Processing Units (GPU:er) och Tensor Processing Units (TPU:en), samt molntjänster tillhandahållina av stora amerikanska techbolag.
\section*{Steg 3: utveckling av avancerade och förtrånade Al-modeller}
Den grundlångande utvecklingen i steg 1 utgör ofta en utgångspunkt för mer specifika anpassningar som utvecklas för olika uppgifter. Exempel på sådana modeller är de stora språkmodeller som släpps med lämnä mellanrum, exempelvis GPT-serien från Open AI.
\section*{Steg 4: utveckling av Al-platformar och verytkang}
Ur de stora språmkodellerna utvecklas ofta Al-plattformar och verytkg. Desma möjigger 9t för aktörer att integrera AI i sina produkter och tjänster. Utbudet av sådana verktyg är stort och inkluderar all från verktyg för maskinilning och musik- och bildskapande, till att optimera industriella processer.

\title{
Al och arbetsmarknaden - en kort översikt av konsekvenserna av ny teknik \({ }^{}\)
}
\begin{abstract}
Hisatoriskt har teknisk utveckling innepuriti bakad produktivitet oan viastånd, samt att fier och bättre jobb skapats. Men omställning har ofta tagf tid och medfört att fler människor under perioder har blivit arbetslösa då arbetsuppgifter och kryner försuunnit. Det ökade välståndet har heller inte alltid kommit alla till gagn. I daggålsgålet gåäär mycket forskning om Al:s inverkan på arbetsmarknaden och om historiska samband står sär.
\end{abstract}
Vad är nytt med Al jämfört med tidigare tek- nikkskiften?
Ä tvänäs påverka arbetmarknaden på flera olika sätt. I likhet med tomatskärnade och räderer, i mägåtning och fjerbruk. Vi mägåtning och fjerbruk. Det kommer att bli lättare för oss att lösa uppgifter näsbatt och med bättre precision. I många fall kommer det hårt att utveckla arbetsuppgifterna så att till exempel tråkiga eller farliga arbetsommen kan udvikas. I vissa fall kommer det emeltertid att teda allt utrykern och hela branchser försvinner, eftersom uppgifterna bättre om och var. I<|im_start|>jävåp som av. Al, iARE emelterid också "en så kaldad "Innäjävåp och var. I och av. I och av. I och av. I och av. I och av. I och AB, I mägåtning och 14. I mägåtning och 14. I mägåtning. (IM), vilket betyder att Al kommer att påverka hela innovations- och forskningsprocessen. Det innebär att vik kommer att hitta metoder att göra saker som i vid agte nhan förestållas oss. Dehtär kommer i sin tur att teda till framväxten av helt nya yrken och hela av. I och av. I och av. I och av. I och av. I
Det hår är en förklarg till att Al kan förväntas påverka arbetsmarknaden annorlunda jämfört med tidigare tekniksfänt. Eexpelvisväntas fär kyrken påverkas en tidigare - inte minst jäntäsemannaykern. Uppgifterna om hur stor andel av jobben som kommer omfattas varierar. Det finns stakkningar baserade på av, i eretet i av, i eretet i av, i eretet i av, i eret et i av, i eretet i av, i eretet i av, i erat i av, i erat i av, i erat i av, i erat iav, i erat i av, i erat i av, i erat i av, a
la allt äggn. I däggålsgålet gåäär mycket forskning om Al:s inverkan på arbetmarknaden och om historiska samband står sär.
Vad är nytt med Al jämfört med tidigare tek- nikkskiften?
Ä tvänäs påverka arbetmarknaden på flera olika satt. I likhet med tomatskärnade på var, i arbetmarknaden och förverkunder Al stora möjlighet till arbetmarknaden och förverkunder Al stora arbetmarknaden och förverkunder Al stora arbetmarknaden och föremer att bittare arbetsuppgifter. Det kommer att bittare för att föres och 14. I mägåtning och 14. I mär, i arbetmarknaden och 14. I mär, i arbetmarknaden och 14, I mär, i arbetmarknaden och 14, I mär, I arbetmarknaden och 14, I mär, i arbetmarknade och 14. I mär, i arbetmarknade och 14. I mär, I arbetmarknade och 14. I mär, i arbetmarknaden och 14
\section*{och digitalisering. Med Al är det inte särkert att det förhänländel förstänter att gälla.111] en analys av jonskna data tycks rent av personer med kortare utbildningar gynnas något mer av Al-teknologi än personer med längre utbildningar. Al kan nåmligen möjliggöra för arbetstagare att utföra mer avanderag uppgifter som tidigare krävde högre nivåer av specialisering. Här är således sambandet mellan utbildningsnivå och väntad sysesslänts�inutvecklinget merjänkilt, jämfört med hur utvecklingen sett und under de senaste årens strukturomvandling. Skilinderna mellan olika utbildningsnivåer är dock av och forskarna är gänksa förstikliga i sina slutskatser.
\section*{Ustaatån andra för fördåndig på grund av Al. \\ Vytsen som bygger på förmågor som kreativitet, resnenomang och skapande, och där textharingen, bildebarebentning, analyser av data och faktainsamling är vanliga arbetsuppgifter, bedöms oft påverkas mer än andra yrken. Gemensamt för dessa arbetsuppgifter är att du en går att automatisera och förbättra i högre utsträckning än tidigare.
\section*{Omsällning och framtidsprogenor}
Det går natutligvis att skisera andra utvecklingsringar för dade jobb och ekonomi än utvecklingsrivats av ovan. Eempelvis är organisatiser den som besoch OECD bycligila si nasvningar om att Al ochså kan innebära önskade konsekvenser- inte minst avseende inkomster och jämlkhet. Även om de beskriver risken för detta som stört i lander med minre utvecklade ekonomier, betonar de vikten av. att säkerställa att denna nya teknik inte bara gynnar vissa gruppe.
\section*{Ustaatån är den veryker bland annat vikten av furgandera offentliga trygghsetssystem och goda möjligheter till tidbildning för de som behöver ställa om. Utn dessa skykod och stöd ökar risken för att Al skoråkade inkomstskillander och ökad ajömlikhet - tvårt emot vad som år önskvårt. OECD understyker smältidigt viken av väl underbyggdga - evidensbaserade- åtgärder för att förebygga och hantera oÖnskade konsekvenser av den nya tekniken. Organisationen påpekar samtidigt det uppenbara dilemnat att utvecklingen är så nabbåt att det är svårt att stå la fast vad som är väl underbyggdga åtgärder.
\section*{I 111] Fakturatan bygger på SNSa Konjunkturänds rapport 2003, Strukturomvandling på avsens arketsmarknad och polycitygåder.}
\({ }^{}\) Se til avser�pej Brynjöllsson, En ift (2023), Generative Al at Vork, NBEI, Working Paper 31161 eller Choi, J. m ft (2023). Lawgering in the Age of Artificial Intelligence, Minnesota Legal Studies Research Paper no. 29-31.

Figur 1. Sveriges ranking i GAlI 2020-2024 samt Sveriges ranking för de sju områdena under 2024
Figur 1a

Källa: The Global AlI Index
Som framgår av Figur 1a har Sveriges ranking föråsmärts från plats 15 till plats 25 sedan GAlI introducerades år 2020. Fallet mellan åren 2023 och 2024 kan delvis förklaras av att indexets konstruktion har ändrats mellan dessa år. 2024 års placering tar mindre hänsyn till att länder varierar i storlek, vilket gör att minder länder generellt ett starkans sämer än tidigare. Denna nya vinkning motivaters med att bittare spéglar att AlI hög grad förknippas med stordriftsförledar, vilket ger törre länder och ekonomier ett naturligt övertag. Den här förändringen är docke inte hela förklaringen bakom det svenska tappet i ranking, eftersom både Bifelannd (5 placeringar) och Damank (6 placeringar) har fallt mindre än Sverige (8 placeringar). Datta är belymersamt, eftersom sveriges, som är större än Damank och Finland, rent för att föräld och föräld och klasma mindre negativt av förändringar in indexets konstruktion.
Sveriges ranking inMG AlIl3 ollka områden i 2024 års mätning framgår av Figur 1b. De största utamningar för sverige ligger inom områdena Politisk styrning (plats 57) och Utveckling (plats 30). Politisk styrning avser ländernas strategiska arbete runt Al-ffågor, medan Utveckling speglar Sveriges innovationskraft inom Al. Sveriges starkaste om rärde, Operativ miljö, speglar bland annat befolkningens atttill Al och tillitt till tekniken.
Nedan går vi igenom de sju områdena i Figur 1b var för sig och beskriver Sveriges position i förhållande till
Figur 1b

andra länder. Vi tittar också på vilka indikatorer som används inom respektive område, för att ge en bättre bild av vad som faktiskt påverkar vår ranking. Det gör det lättare att bedöma i vilken mån de åtgärder vi föreslår kan förväntas leda till en förbättring av Sveriges position. Efter den diskussionen föreslår vi ett mål för hur vår ranking bör kunna förbättras fram till år 2030 inom de sju områdena. Det valda itsperspektivet speglar att vi tror att de åtgärder vi föreslår kan förväntas ha fått full effekt då. Avslutningsvis, utifrån de målisatta nivåerna för de olika delområdena, diskuterar vi ett ambitiöst men rimligt mål för det sammanvågda GAlI för år 2030.
För att landjämförelser ska vara meningsfulla är det viktigt att de sker utifrån relevanta jämförelseländer, sett till exempelvis storlek, öppenhet eller nuvarande ranking i GAlI. Vi har därför valt att jämföra Sverige med våra nordiska grannländer Danmark, Finland och med med med med med med med med med med med Nederänderna och Schweiz, medbiska ekonomier, sättningar som Sverige. Vi jämförlö också med de tre främtas länderna i det sammanvågda GAlI sett till de storlekjusterade indikatorerna, mänligen Singapore, USA och räal - dessa tre är allstå de som presterar bäst vägt sin relativa storlek. I genomgången av de sjängångå en förbar vid estudom de tre bästa änderna inom respektive märde. Alt detta för att få en känsla för vad som är anmäties och rimlignåkättning för Sverige år 2020, givet de satsningar som görs i denna Färdplan.

fertel Al-ti|änster har fördröjts eller utelvilit, med hänvisning till EU:s regligerang par Al- och dataområdet. De här två aspekterna måste hela tiden balanseras mot varandra.
I den vision vi nyss beskrev har Sverige lyckats dra nytta av de stora potentiella vinsterna som ligger i en ökad Al-användning. Men de här möjligheterna till ökad konkurrenskraft och välfärd har också en baksdida: om vi inte lyckas blir konsekvenersna allvarliga. Problemet är att konsekvenersna på kort sikt inte kommer att vara särskilt markbara och tydliga för var och en. I stället kommer effekten att krypa på oss, i form av att våra företag får svårare att konkurrera. Det skulle både täll påfrestningar på arbetsmarknaden och en såmre inkomstutveckling. Likaså skulle den offentliga sek tornor få allt svårare att leva upp till sitt samhällsåtagande, med negativa effekter på människors tillt och vilja att betala skatt. Vi skulle också vara såmre rustade att hantera problem kopplade till illivillig användning av Al.
Passivtter så silan någon bra strategi, i synnerhet inte nåt tillvaron förändras. Det är Al-kommissionenns ambition att vår Fårdplan ska utgöra en väckarklocka och inspirera till att med tillförsikt anta den gemensamma utmaning och möjlighet som Al-utvecklingen innebär. Till sammans kan vi se till att utnyttja Al till vår gemensamma nytta och till gagn för samhället, precis som vi har gjort vid tidigare teknikskiften. Det finns ingenting som säger att vi inte skulle kunna göra det även denna gång.

Einrides självkörande, eldrivna lastbli, T-pod på svenska vägar. Bilden tagen i Jönköping i maj 2019. Foto: Wiktor_swe/Shutterstock

\title{
Al för alla
}
\section*{Kompetenslyft för alla}
\section*{ChatGPT sammanfattar:}
\begin{abstract}
Al kommer att paverka hela samhället, och för att Sverige ska kunna dra maximal nytta av denna Utveckling kräve att omhättande kompetenslyft, I aktorca att Al-technolögibr allmer ingegrard vär vardag, från åbetsplatser till skolor och hem, måste vi-såkerstälta att alla har möjlighet att förståt och använda dessa verkling. Det handlar inte bara om att utbilda tekniker och forskare, utn också om att gle hebalfokningen, oavsett ålder eller bakgrund, grundläggande kunskaper om AI. Utan dessa insatser riskerar samhället att skapa nya klyftor, där vissa grupper står utanför den digitala omvandlingen.
Detta kapitel kommer att diskutera hur folklidning, utbildningsvåsendet och arbetsmarknadsen aktörer kan samarbeta för att säkerställa en bred och inkluderande kunskapsspridning. Genom att investera i utbildning och livslångt lärande kan vi inte bara stärka Sveriges konkurrenskraft, utan också säkerställa en hållbar och rättvis Al-utveckling. Här presenteras AI-kommissionens förslag för att nå dessa mål.
\end{abstract}
\section*{Folkbildning}
Erfarenheter från tidigare stora strukturovmandvlingar visar att ny teknik ofta kommer olika samhållsgruppert till del i olika hög grad. Det här kan bli problematiskt, eftersom ett land med en stor del andel av belfolkningen som an vandärde AI kommer att uppnåt stora konkurrensfördelar. Ett land där människor förstärt de grundläggande möjligheterna och riskerna med AI, oräcksåt till end som på ett säkert och etsikt sätt kan använda Al för samhällets bästa. Ett land som erbjuder tryggethunder omställning är ett land som kan genomföra storate teknikskiften med bred legitimitet. Det är sålunda avgörande för samhällets av utveckling och sammanhållning att alla vil, vägar och kan ta del av ny teknik.
För att Sverige fullt ut ska dr yntta av den nya tekniken behövs en hög läggstaniav vad gäller grundläggande Al-kompetens hos Sveriges belkning. Tyvärr finns det grupper i dag som befinneren sig ett digtjältat utanförskap. Engl itetsnettillests rapport \({ }^{}\) med Svenskarna och internet från 2024 bedöms förvisso bara frya procent av svenskaarna vara digitalt utanför, men sinffran gömmer en snedfördelning i befolkningen - bland äldre är motsvarande siffra 22 procent. Den snabba teknikutvecklingen riserkar desusdom att skapa nya kunskapskyltor. \({ }^{}\) Samma rapport visar exempelvis att 30 procent av belfolkningen har använt ett \(\mathrm{Al}\)-verktyg under 2023, framför allt för privat bruk. Bland yngre svenska är det betydligt vanlrigare. I äldern 18-34 vår svarar 60 procent att de har använt Al-verktyg. Det kan jämföras med svenskar läldern 65-84 år denadst fem procent svarar att der ha gjort det. Framför allt är det Al-verktyget ChatGPT som anvandårs. För att fullt ut kunna skörda frukterna av den nya tekniken, men också udwika avarter såsom bedrägerier, faktåförvanskning och intrånade fördomar i automatiska beslut, kärss ut vi minskar sjängens kunskapskyltor och motverkar framtidå känka. Här kan Sveriges långa tradition av folkbildingspela en viktig roll.
Flera organisationer och myndigheter abretar i dag med dessa frågor, delvis inom ramen för satsningen Digitalkad som drivs av Post-och telestyrelsen (PTS). Andra exempel är Internetstiftelsens sajt

en tids I-avbrott. I dessa fall behöver ver anavcereand
reservetrivinet inter förberedes. \({ }^{}\)
Om vi framtiden har gjort oss beroende av I 6 r att
kunna upprätthål vissa samhälisktjung kintuoner,
behöver i hø en plan för scenarler där den nödvån-
digta tekniken slås ut under oaceptabelt lång tid,
avosett skål. Problemet är att det i många fall sannø-
likt inte långre kommer vara ett realistiskt alternativ
att övergåt till manuell - eller densit, man El-olös
- hantering, eftersom arbetsbelastningen då skulle
bil ohanterligt stor. I stället kommer kversamheter
behöva bedrivas på ett i grunden annat sätt. Fokus
måsde då liga på att efter bästa förmåga till godose
de grundläggande behoven och intressena, såsom
tilgång till el, kommunikation, livsmedel, försörjning
av prundatta eller ekonomisk säkerhet för befolk-
ningen. \({ }^{}\) dessa fall är det viktigt att det lagar och
föreskrifter som styr verksamheten har tagit höjd för
sådana scenarier, och medger att ordinarie regler och
rutiner tillifällig åsidosatts.
\section*{förslag}
- Al-kommissionen anser att regeringen
behöver ta initiativ för att säkerställa en god
beredskapsplanering på de områden där bortfall
av digital förmåga annars riskerar att så ut viktiga
samhällsfunktioner.
- Al-kommissionen anser också att det finns ett
behov av att utveckla och uprätthålla robusta
beredskapsplaner och responsstrategier
för att sabbnt kuntha antera och minksa de
negativa konsekvensna av Al-relaterade
säkerhetsincidenter. Al-specific skanerier
bör tas fram och inkluderas i nationella
säkerhetsövningar, och samarbetet stärks
lemmal nyndigheter, företag och lärosäten för att
effektivt svrata på h0. \({ }^{}\) Det kan även innebära
att befintliga rutiner och planer stärks av sätt även
de i detta avsnitt behandlade riskerna beaktas i
tillräcklig omfattning.
\section*{avtsultningsvis är det vätt att återvända till frågan}
om digital suveränitet som berördes tidigare i detta
kapitel. Det finns goda skål för Sverige att säkerstälta
full rådighet över vissa digital tillgångar. Trots detta
kan vi inte bortre från att samhällets motsndskårt
vissa lägen kan öka, snarare än minksma, om viktig
data och Al-tjänster läggs i kommersiella molljtänster
med bas i utlandet i stället för - eller utöver - i egna
datacenter inom landets gränser. Detsamma gäller
om valet står mellan att få tillgång till en förmåga utan
full rådighet över den, och att vara helt utan. Detta är
något som regeringen och ansvarig aktör kommer att
behöva analysera noggrant och från fall till fall.
\section*{Etsik och säker Al-avnändning}
Som beskrivits i denna Fårdplan används Al i dag
breftt i vårt samhälle. Användningen av Al inom en
viss sektor kan medföra etiska överväganden. Det
kan därför vara relevant för varje samhällssektor att
ta fram etiska kritlinjer för hur Al bör användas inom
olika yrkesgrupper med beaktande av etablerade
standarder och principer, såsom god avdokatsed, god
revisorsed eller god läkaretik. Försvarsberedningen
har exempelvis lynlige efterfrågat atste sklar tijnler nä
det gäller Al på försvasormrådet. Dessa är åtknä atta
adressera både de särskilda juridiska och etiska pro-
blem som följer med Al inom militära tillämpningar. \({ }^{}\) Vidic med med med med med
med med<|im_start|>etetetta kritlinjer är det naturligt att
ellerätas samarbet personarligas personarligationer
eller motsvarande som i regelning av betvändande
kunnande. Det kan även vara relevant att betvändande
FN-organet Unesco: riktlinjer om ett okch Al. \({ }^{}\) Därut-
över anser vi att det är viktligt att det etiska kritlinjenna
även beaktar hur användningen av Al kan påverka
barns rättigheter. \({ }^{}\)
\section*{Ett nytt institut för Al-säkerhet}
Det finns ett stort behov av kunskap ring der risker
som utvecklingen och användningen av Al kan föra
med sig för vårt samhälle. För att stärka samhällets
dosttnskraft, men även för att hantera den oro
som finns vad gäller Al, anser vi att ett institut för
Al-säkerhet ska detalferas. Instituts teudyduoppgift
ska vara att bedriva och främja forskning ring skär-
hetslärster förknippade med Al. På så vi kan institutet
brida till att faktiska säkerhetsrisker förknippade med
Al besleys och blir adresserade. I de fall institutets
forskning visar att en viss säkerhetsrisk ir almajän
eller överdriven bör den ha ett tydligt uppdrag att
pätala det. På så vi kan vi iste allt dusslossinen
kring säkerhet och Al grundsas i evidens och fakta,
samt att samhällets resurser läggs på att hantera
faktiska risker. Instituttet ska därför årligen publicera
en rapport om tvigelsest fårgroma vad gäller Al-sä-
kerhet samt ge rekommendationer kring hur dessa

\section*{Förslag}
- Förr att öka kunskagen om Al bland alla människor som bor isverige föreslår Al-kommissionen en förstärkning av medel för folkbildningsinsatser till folkhögskolorna, Medilemyndigheten, \({ }^{1081}\) PTS och studieförbunden under åren 2025-2029 med 100 miljoner kronor per år. Syftet med satsningen är att minsko och förebygga de klyftor som kan öka i samhället vid stora teknisktfärem samt att öka kunskapen om de möjligheter och risker Al kan medföra. Därmed kan legimititeten och acceptansen för den omställning som sker öka.
- Al-kommissionen föreslår också en satsning på folkbiblioteken om 100 miljoner kronor per år under åren 2025-2029. Det stärker deras uppdrag att främja användningen av informationsteknik för kunksapisnähmtning och lälandera. Satsningen syffert till att allmänheten ska kunna få hjälp med att vropa på och använda Al-verktyg kostnadsfritt.
- Likt hem-pc-reformen på 1990-talet bör staten erbjuda alla medborgare kostnadsfrit tillgång till utvert alval v kvalitetssikrade Al-tjänster: "Al för alla":-ferommen. \({ }^{108}\) Genom att demokratisera aktomsten till dessa verktyg kan Sverige positionera sig som en föregångare inom Ananvändning. En sådan bred satsning skulle inte bara stimulera innovation och produktivitet på individinåvä, utan också ge Sverige en unik profil informationletit - ett land som prioriterar med en förläut, satspetsteknologit tillgänglig för alla medborgar medför att för att för att för att för att 100 miljoner. I den Al-hubb som Al-kommissionen föreslår på sidta 70. Vikla Al-tjänster som ska inkluderas i Al-hubben måste snabbutredas. Den last force vi föreslär i kapitlet Ledarskap och styrning för att genomföra Färdplanen ska ha till uppgift att snabbutreda och förhandla med leverantörer om förutsättningarna för att erbjuda gratis Al-tjänster, sätta ett taf för hur mycket satsningen får kosta och under hur lång tid erbjudandet ska fortgå.
- Staten bör också överväga att ege lever, studenter och lärare kostnadsfrit tillgång till Al-tjänster under en längre period. Vid bibliotek och vid Statens servicecenter ska det finnas Al-tjänster utan kostnad att använda, samt personal som kan vara behjälpliga i hur de kan användas. Även detta måste snabbutredas av den task force vi föreslär i kapitlet Ledarskap och styrning för att genomföra Färdplanen. I vörigt är det arbetsgivarnas ansvar att de ansställda har de hjälpmedel som behövs för att lösa sina uppgifter och förbättra verksamheten.
\section*{Det reguljära utbildningsväsendet}
\section*{Al i skolan}
Syftet med Al-kommissionens uppradg är att säkerställa att Sverige som delande forsknings-, industrioch vållårdnasten blättre ska villvarata möjligheterna och hantera riskerna med Al. Ska Sverige kunna behälla och stärka denna position i kommande generator är det avgörande att v ochäså rstar barn och unga för framtiden.
\section*{Al-kommissionen lämnar dock inte några förslag på skolans område i denna rapport, eftersom vi enligt direktiven är förhindrade till det. Samtidigt ska kommissionen beakta möjligheter och risker som Al kan medföra med avseende på skolväsendet. Al kommer, och bör, ha en stor inverkan på skolan. Det gäller såväl innehållsmässigt som pedagogiskt. Det handlar om att ta vara på teknikens möjligheter men också att tycliggöra dess problem.
Som i samhället i övrigt erbjuder Al stora möjligheter för skolpersonal att förbättra och rationalisera administrativa uppgifter. Lärare kan exempelvis använda Al för att ta fram lektionsplaner, scheman, rapporter och genomföra dokumentation. Härigenom frigors tid som kan läggas på pedagogisk utveckling och på elevernas individuella behov, för att nämna några exempel.

\title{
Figur 5: Nheservingar från riskkapital (VC) i Al- och dataföretag
}
VC-investeringar i Al per land

Källa: OECD,Al.
Den sammantagna bilden vittnar om att det generella innovationsklimatet är gott i Sverige. Det är något vi måste slå vakt om. Den generella innovationspolitiken bör därför fortsätta att fokusera på stabila spelregler, att begränsa regelbördor och säkerställa en väl fungerande kapitalmarknad. Det vore dock ett stort misstag att tolka nuläget som att vi kan slå oss till ro. För det första finns det indikationer på att situationen inte är lika gynnsam på Al-området som i övrigt. Att vi ligger så lågt som på 18:e plats i Kommersialisering och på 30:e plats i Utveckling i Global Al Index är en tydlig signal om att allt inte står rätt till. För det andra visar erfarenheten att när det handiar om stora teknikkiften, som dessutom utvecklas med en väldig fart, är det viktigt att inte bara titta på historiska data utan också på utvecklingens riktning och hastlighet.
Här är Al-kommissionens uppfattning att införandet av Al som innovationsverktyg generellt sett går för långsamt. Det är något som vi har erfartiv tid otliga kontakter med aktörer från både privat och offentlig sektor. Det finns också flera undersökningar som visar att svenska företag använder Al i mindre utsträckning än företag i jämförbara länder. \({ }^{\text {123 }}\) Figur 6 visar till exempel hur svenska företag ligger efter många EU-länder i användningen av Al.

\title{
Brottsbekämpning på Skatteverket
}
\author{
Varje år biträder Skatteverkets brottsutredare, på \\ direktiv av åklagare, i cirka 1 600 förundersökningar \\ som gäller ekonomisk brottslighet. Samtidigt finns \\ det ständigt ett omfattande antal ärenden som väntar \\ på att utredas. Skatteverket satsar på att med ett \\ antal Al-drivna tjänster skynda på utredningarna och \\ samtidigt snabbare kunma minska ärendebalansen \\ Esempel på Al-drivna tjänster som kombineras \\ år strankribering av förhör i realtid, analystjänster
}
baserat på en stor finjusterad spräkmodell som hjälp till att hitta mönster i såväl förhör som övrigt förundersökningsmaterial, och maskningstjänster som ananomysimeru oppgifter vid behov. Denna kedja av Al-drivna tjänster beräknas effektivsiera arbetet för brottsutredarna med 20-45 procent vilket kortar lefditerna. Andra effekter som är tvölliga, men svåra att kvantifiera, är att Al-tilänsterna bitrar till högre rättssäkerhet och ökad kvalitet.
\section*{En friskare befolkning}
Inom hälso- och sjukvården ökar användningen av Al snabbt i hela världen. Kvalliteten i vårdinsatser förbrättras, liksom effektiviteten i sjukvårdens organisation och verksamhet. Al-teknik kommer att i mycket högre grad kunna användas för att identifiera risker för olika sjukdomar och för att förebygga dem. Det gör
det möjligt att snabbare diagnostisera sjukdomar och utveckla behandlingsmetoder. Sensorteknik för att exempelvis förebygga falksador hos äldre är ett annat sätt att utnyttja ny teknik för förbättrad hälsa. Det kan således bili tätare att förebygga och förhindra ohälsa och sjukdomar, vilket framför allt minskar mänskligt lindande men också kostnader.
\section*{Fallsensorer med Al-teknik för att minska fallolyckor i Sundsvall}
\section*{Perioden januari-september 2023 testeda Sundsvalls} kommun fallensorer utrustade med Al-teknik på
Norra Kajens äldreboende i Sundsvall. Syftet med projektet var att minska risken för att boende ramlade. Fallensorn, som nu är fullt operativ, är installerad i den boendes rum. Med hjälp av infrarött ljus skannar den av rörelser i rumet och reagerar på olika typer av rörelser. I samråd med den boende ställer personalen på äldreboendet in vilka rörelser sensorn ska reagera på
\section*{Fallsensom kan också ställas in så att den reagerar under olika tidpurkter av dygnet. Registrerar fallensorn en rörelse som den är inställd att reagera}
på utlöjer den ett lærn. När lärmet går tillkallas personalen på tre sätt: via en applikation i mobilen, via sms och via befintligt kallesesystem som larmar med ljud iden boendes rum. Fallensorn testades av 20 personer och en utvårdering visade bland annat att fallyologkorna hade minskat markant (77 procent). Resultet blev att tillgjare förelej<|im_start|>am har ersatts fullut . \({ }^{[144]}\)
Berkängnkor som Sundsvalls kommun gjort visar att en användning av fallsensorer i hela landet skulle leda till besparingar på omkring åtta miljarder kronor årligen, bland annati i minskade vårdokstnader för fallskador. \({ }^{[147]}\)
\section*{Trots att hälso- och sjukvården är ett område där Al kan göra stor nyttta. \({ }^{[148]}\) saknas i dag nationella strateger, mål och handlingslpner för AI i hälsooch sjukvården samt tillräckliga investeringar för att skapa förutsättningar för införande. Detta kan vara en bidragande orsat kall att Syreige halkar efter jämförbara länder i Al-mogland inom häso- och trodjer.}
\footnotetext{
[145] En bekringvår av testet med fallaspersoner finns på https://utveckling.sundsvall.se/snitivat/keepeml-fran-vardagen/2023-08-16-fallseorser-med-ai-lekink-minskartik/colykor på-nora-kajens-iradbøende.
[146] Inom hälso- och sjukvården är ett förebygga för att förebygga för att förebygga för att förebygga injrapston för för 2020 till 16.8 miljarder kronor, varna 11.3 miljarder kronor varner direkta kostnader för region och kommun. https://www.socialstyrelsen.se/globalassets-hirasport-clunkrinner/kalketaliskatogr/09/2022-5-7923.pdf.
[147] En best land annat-Haug, C. J. et al. 2023. Artificial Intelligence and Machine Learning in Climate Medicine, 2023-10-08. | Med 2023. 398.
[148] Inom hälso- och sjukvården. Inom hälso- och jukvården. Inom hälso- och jukvården. | Inom hälso- och jukvården. | Inom hälso 2020. | Studien, publicar för defrikor 2024, gjorde en för omvort. | Inom hälso. | Inom hälso- och sjukvården. | Finalerativat var årsut Versig fölager efter de andra landerna i magnadsgrad.
}

\section*{Förslag}
- AI-kommissionen anser att det är mycket angeläget att staten stödjer arbetet med att få etableren an AI Factory i Sverige, något som bland amnat skulle öka tillgången på beräkningskraft till ett subventionerat pris för små och medelstora företag. För detta behövs motfinansiering från Sverige på minst 50 procent av kostnaderna. Det skulle innebära 30 miljoner eure 1072"som en engångsinvestering och sedan 10 miljoner eure för är i fyrå år för drift. Ansvaret bör delas mellan Vetenskapsrådet och vinnova då målgruppen är små och medelstora företag.
\section*{Offets}
Offentlig sektor behöver också beräkningskraft för att träna Al-modeller, men kontviken här ligger framför allt på att implementera Al-tilnster för att erbjuda samhällsservice. Att erbjuda offentliga Al-tilnster ställer höga krav på säkerhet och stabil drift. Tjänstserna måste både vara tillgängliga dygnet runt och för många användare samtidigt. Allt som oftast innebär dessa krav att det är svårt för offentlig verksamhet att köpa beräkningskraft via molntjänster, i synnerhet då det i dagslägdet föreligger juridiska hinder för offentlig sektor att upphandla dem. \({ }^{}\)
\section*{I dag finns ingen sammalå beräkningskraft för offentlig verksamhet, utan Al-tilnsterna levereras främst genom lokal beräkningskraft inom olika myndigheter. Situationen för offentlig sektor är inte helt olik den för företag vågåller olika förutsättningar mellan stora och små aktörer. Större myndigheter, som Stakterveet och Försäkningskassan, ligger i framkant \(\mathrm{I}\) Al-avnändning tack cave deras omfattande resurser och kompetens. Mindre myndigheter, små regineren och kommuner har ofta inte ens börjat använda Al, då de saknar både beräkningskraft och kompetens. Likom för de små och medelstora företagen finns här samordningsvinterat att göra, men dessa förhindras av samma koonderringsproblem som företagen ofta möter. Staten behöver därför skapa förutsättningar för offentlig sektor att utnyttjä gemensern abräkningskraft anpassad för Al, både i egen regi och i form- uvpphandlade molntjänster där det är lämpligt. Ett förjasag som möter offentlig sektors behov av beräkningskraft diskuteras i detæli i kapitetl Al'fören offentlig sektor / framkant.
\section*{Bavaka behoven av framtida beräkningskraft}
Universitet och högskolor, näringsliv och offentlig sektor kommer i allt högre utsträckning att vara berende av al-relaterad beräkningskraft. De satsningar som Al-kommissionen listar i detta kapitel täckerh behov som vi kan överskåda i dagslåget. För att erbjuda långsiktigt konkurrenskraftiga resurser, givet den snabba tekniska utvecklingen, måste planer för uppdgraderingar och förbättringar vara integrereda från början. Det är därför viktigt att regeringen noga följer utvecklingen för att säkerstältil tillräcklig beräkningskraft för tråning av Al-modeller och användning av Al-länster.
\section*{Förslag}
- Al-kommissionen anser att Vetenskapsrådet, heist i samverkan med övriga nordiska vetenskapsråd, årligen bör redovaa till regeringen huruvida beräkningskraften för tråning och användning av Al-modeller i rättkläcklig.
Regeringen bör ha beredskap att öka anslaget för detta ändamlå om Vetenskapsrådets analys visar på ett sådant behov. Det är viktigt att behovsanalysen beaktar att utbuddet av beräkningskraft kan påverka efterfrågan och innovationer. Et tvist överutbud av beräkningskraft är därför att föredra i detta läge.
- Beräkningskraft för tränng av Al-modeller och användning av Al-länster blir en allt viktigare del av det moderna samhället. Al-kommissionen anser därför att den bör ha samma status som annan samhällskritisk infrastruktur, såsom järnvågar och enlåt.
Det vi talat om hittills rör behovet av beräkningskraft för dagens tekniker, men runt nörhten kat vi randen skönja morgandagens teknikkifke. Vktandatorer är ett exempel på ett område där utvecklingen gör stora framsteg och särder och av de av de av de av de av för att förändra hur beräkningskrafts rättarna. Trets och ett bestydande framsteg som glorts krivs dock fortsatt teknisk utveckling och forskning för att nåkpriskta tillämpningar av kwantdatafore. Kwantsystemens potenfati är dock så skor att stiver Figuser björitorera ett aktivt deltagande ideras utveckling. Detta sker redan till vils del genom satsningar från exempelvis Knut och Alcoilleulbenergs Stiftelse. Al-kommissionen ser positivr på att regeringen har gerett vetenskapsrådet utppdrag att ta fram underfag till en nationell kvantstrategi. \({ }^{}\)

Figur 1: Förutsättningar för ökad användning av AI i offentlig sektor
För att lyckas behöver offentlig
verksamhet
Arbeta till SAMMANS med
infrastruktur och utveckling
Säkerställa tillgång till data
Värna enskildas tillit och förtroende
För att uppna det behöver
regering och riksdag
Utvockla styrningen av
offentlig verksamhet för
att bygga gemensam infra-
struktur och främja samarbete
Skapa regelverk som
möjliggör utveckling och
anändning av Al
Källa: Försåkringskassan
Omräde 1: Arbeta tillsammans med Al-infrastruktur och teknikutveckling
För att de offentliga aktörerna ska kunna svara upp mot de allt högre var som ställs på verksamheten kommer det att vara nödvändigt med ett fördjupat omtag av befintliga arbetsätt och verksamhetsmodel, men också att göra ett ökat samarbete möjligt. Detta kan ske med stöd av digitalisering och Al, vilket har varit ett satsmännigt budskap från sektorsföredräare till Al-kommissionen.
Samarbete mellan aktörerna försvåras i dag kraftigt av avasnaknad av en gemensam Al-infrastruktur som uppfyller, del varo som offentlig verksamhet har attta hänsyn till när det gäller sådant som sekretess och offentlighet, personlig ingjetrter, cybersäkerhet, informationsseparation, dataskydd och säkerthetskydd och- som dödtarill åt tillräckligt enkelt för aktörerna att använda. \({ }^{1591}\) stället finns det ofta ett stort antal tekniska system och infrastrukturer som inte kan kommunicera med varandra.
Avsaknaden av en gemensam Al-infrastruktur gör att det inte går att experimentera med, utveckla och driftsätta Al-Iösningar tillsammans. Olika aktörer kan inte heller dela lönsningar och kompetens eller att få till stånd stark ett efferrågade gemensamma tjänster i den omfattning och hastighet som krävs. Privata levefrentörer kan i dag bidra till att lösa en del av offentlig förvaltnings problem, men ofta har de lönsingarna
\footnotetext{
\({ }^{}\) Det räder stor effatsthet i digital infrastruktur: en majoriet av aktörerna saknar en modern grund att stå på. Vidare saknas rådighet i merar v kontroll över undiergandte system samt utvecklingskapacitet. Se Klorovisionens rapport förplärade it-system - hinder för en effektiv digitalisering (RJR 2019:28).
\({ }^{}\) Se kapitel \(A\) och samhärles säkerhet angårmende digital varändertet.
}
inte tillräcklig nivå av säkerhet och erbjuder inte heller den fulla rådighet över lönsinger, infrastruktur och data som behövs.
Den offentliga sektorn måste därför snabbt etablera en sammanhållen teknisk Al-infrastruktur som geråsål statliga myndigheter som kommuner och regioner, oavsett storlek och utgångsläge, förutsättningar det 44 rta hyntta av de möjligheter som Al-utvecklingen skapar - och att kunna göra det i samarbete med varandra. Det behöver också finnas gemensamma stödfunktioner som kortskitigt kan stötta offentliga aktörer med den kompetens de saknar och långsiktigt hjälpa dem att bygga upp sin egen kapacitet. \({ }^{153}\)
Skulle alla offentliga aktörer behöva utveckla sin egen Al-förmåga skulle det ta oacepetbalt lång tid och leda till ytterligare fragmentering med olika lönsningar som har varierande säkerhet och kvalitet. Det skulle också vara en ineffektiv användning av begränsade resurser - i den mån det överhuvutaget skulle vara möjligt. Särskit stor skulle utmaningen vara för de mindre aktörerna, därribland en stor andel av kommunerna.

Figur 6: Användning av Al-teknologier bland små och stora företag (i procent) för olika EU-länder

- Stora företag - Små företag
Notera: "Stora företag" avser företag med över 250 anställda medan "Små företag" avser företag med 10 till 249 anställda. Den vertikala axein visar den precent av företagen som under 2023 någon gång användne en av följande Al-teknologier: "text mining", tailgenkänning, naturligt språkgenering, bildigenkänning och bilbehandning, maskininlärning (tex. udipinlärning) för dataanalys, Al-baserad mjukvaruborotautomation samt autonorna robotar, självkörande fördon och autonorna drönare.
Källa: Eurostat
Al ställer nya krav på innovation, vilka åtgärder behavior då vidtats? För att förstå det behöver viin esat Alt al ställer nya, eller skärpta, krav på innovationsprocessen. Det första att notera är snabbhetenen itu utvecklingen. Det andra år Al:s behov av data, som gör att perspektivet på data och dess värde förändras dramatiskt. Det aktualiserar många komplicerade frågor, inte minst legala. Vi måste se till att tid en passa regleverken till den nya verklighet vi befinner oss . Det pretde jär att vi inte kommer att kunna förverkligla Al:s fula potential om inte forskning, närgälvsi och oftellig sektor hittar nya såt att samarbeta. Det fjärde är att nätkerseffekterna som karaktäriserar många Al-verktyg bidrar till betydande stordriftsfördelar. Det innebär att den som kommer först lätt får en dominerande position.
För att hantera dessa utmaningar och möjligheter på ett efektivt sätt krävs en välgrundad policyansats i form av ett ramverk. Med hjälp av det kan samhällsekonomisk effektiva policyreformer eller policyåtgärder identifieras. En viktig utgångspunkt är att undvika statliga ingipranden, till exempel stödåtgärder, om marknadskrafterna leder till gynsamma utfall utan åtgärder. Ibland förekommer dock så kallade marknadsmisslyckanden. Det innebär att marknadskrafterna på egen hand inte kan förväntas leda till ett samhällsekonomiskt optimalt utfall. Exempelvis kan företag tendera att investera för lite i vidareutbildning av sin personal om det finns en risk att förbara medarbetare till konkurrerande företag. Därför kan det vara motiverat med statliga utbildningsinsatser riktade mot de som är anställda när stora tekniskkiften sker. \({ }^{}\) Att hitta rätt åtgärder kan emellertiv dara förenat med svårigheter eftersom myndigheterna inte har full kunskap, till exempel om hur Al-tekniken kommer att utvecklas. Inom nationalekonomin kallas detta för regleringsmisslyckande. Stödåtgärder kan därmed bil felritkad, kostsamma och i vårsta fall, kontraproduktiva. Det är därför viktigt att noga överväga vad som hindrar respektive gynnar innovation.
I resten av kapitlet kommer vi att titta på några viktiga aspekter av Al-innovation:
- Datadelning, samarbete och problemlösning.
- Vikten av ändamålsenlig och begriplig reglering.
- Åtgärder för effektiv finansiering av Al-innovation.
- Den kreativa förstörelseprocessen och teknikspridning.
- Maximering av synergier i kluster och ekosystem för Al.

I inledningen till denna Fårdplan beskriver AI-kommissionen de utmaningar som vårt land står inför. Mot den bakrunden finns ett stort behov av att den offentliga sektorn i sin helhet omfamnar och realiserar den utvecklingspotential som AI erbjuder.
Utmaningarna för offentlig verksamhet En av de största utmaningarna för offentlig sektor är den demografiska utvecklingen, som speglar att vi lever längre och föder allt färre barn. Under 2020-talet beräknas exempelvis andelen personer över 80 år öka med 49 procent medan andelen personer i yrkesför ålder endast kommer att öka med fyra procent. Den här utvecklingen innebär en stor påfrestning för den offentliga verksamheten, såväl I termer av arbetsuppgifter som för finansiering via skattemedel. I takt med att samhället i övrigt utvecklas ökar också människors förväntningar på vad offentlig sektor ska leverera. För att svara upp mot det måste offentlig sektor klara fler uppgifter, eller samma uppgifter, till högre kvalitet.
Kostnaderna för den offentlig verksamheten kan dock inte öka i motsvarande grad, bland annat för att den sysselsatta delen av befolkningen minskar över tid. Arbetsmarknaden kommer inte heller att kunna tillgodose den offentliga sektorns kompetensbehov i samma utsträckning som tidigare. Forskning visar att Sverige inom ett par år behöver klara av 125 procent av välfärden i relation till dagens mått, med bara 75 procent av bemanningen. \({ }^{}\) Verksamheten måste därför kunna utföras mer kostnadseffektivt och mindre personalintensivt.
Offentlig förvatning förväntas kunna leverera snabba resultat, inte minst när tempot i samhället i övrigt ökar. Därttil innebär serviceåtagandet krav på tillgånglighet och alltmer individanpassade svar. Offentlig sektor behöver därmed utveckla sin förmåga att snabbare möta både medborgare och företag.

Foto: Jeppe Gustafsson/Shutterstock
\section*{Dygnet runt alla dagar i veckan på Skatteverket}
\(\mathrm{Pa}\) Skatteverket finns närmare trettiolatel Ai-jänster i produktion dygnet runt varje dag. Enbart under mars månad 2024, inför deklarationsperioden, svarade Skatteverkets chatbot Skatti på 225 000 konversationer, jämfört med mer normala 50 000 konversationer per månad. Varje konversationen innehåller flera olika frågor varav 47 procent besvarades när skatteupplysningen var stängd. Med stöd av Skatti när Skatteverket ut till fler kunder än tidigare, med service som är tillgänglig digynet runt hela året. Samtidigt har myndigheten kunnat avlasta sin manuela skatteupplysning och gjort kosstandsbesparringar.
Varje år kommer det in ungeffär 300000 ärenden om företagsregistrering till Skatteverket. För att hantera dessa behöves det tidigare 200 heltidstjänster. Tack vare en Al-ijänst har Skatteverket minskat handläggningstiden med ungeffär tre och en halv dag. Detta har frigjort tid för över 100 medarbetare och 40 chefer som nu kan ägna sin tid åt annat mer valficerat arbete som till exempel kvalitetssäkring och utveckling.

Tabell 1. Sívar nágra av de indikatorer som utgör grunden för området Politisk styrning. Indikatorerna itabellen visar om landet har centrala strukturer och resurser på plats för Al-utveckling, såsom öronnärkta medel, en Al-strategi med mätbara mål och effektiva uppföljningsmekanismer.
Tabell 1. Grön färg innebär att villkoret är uppfyllt.
\begin{tabular}{|c|c|c|c|c|c|c|c|}
\hline & \begin{tabular}{l} 
Departement \\
med ansvar \\
för Al-frågor
\end{tabular} & \begin{tabular}{l} 
Oronnärkta \\
pengar till Al
\end{tabular} & \begin{tabular}{l} 
Regeringen \\
har en Al- \\
strategi
\end{tabular} & \begin{tabular}{l} 
Regeringen \\
har mätbara \\
Al-mål eller \\
KPIer
\end{tabular} & \begin{tabular}{l} 
Mekanismer \\
för att följa \\
ups patningar \\
påAl
\end{tabular} & \begin{tabular}{l} 
Akademi, \\
näringaliv etc. \\
har bidragt \\
tillAl-strategi
\end{tabular} & \begin{tabular}{l} 
Stats- \\
chefnar har- \\
underteknatt \\
Al-strategi
\end{tabular} \\
\hline Sverige & & & & & & & \\
\hline Finland & & & & & & & \\
\hline Norge & & & & & & & \\
\hline Danmark & & & & & & & \\
\hline Nederlanderna & & & & & & & \\
\hline Schweiz & & & & & & & \\
\hline Singapore & & & & & & & \\
\hline Israel & & & & & & & \\
\hline Saudiarabien & & & & & & & \\
\hline USA & & & & & & & \\
\hline Kanada & & & & & & & \\
\hline
\end{tabular}
Källa: The Global Al Index, 2023 års upplaga.
Sveriges situation skulle förbättras avsevät om regeringen följer Al-kommissionens förslag i kapitlet Ledarskap och styrning för att genomföra Fårdplanen. Med en beslutad Al-strategi och tydligare styrning skulle många av rutorna i Tabell 1 bli gröna.
En annan viktig indikator inom detta område - som inte ingår i tabellen - är den offentliga sekörtns investeringar i Al-utveckling, som enligt indikatorn är låga i Sverige. Detta kan delvis bero på vår förvaltningsmold, är tuendveilla al Raterade investeringar inte finns enkelt sammanställda för myndigheter, kommuner och regionen. I kapitlet Ledarskap och styrning för att för att för att för att för att förärrportetingskrav och särskilda upprag till inmydigheter för att komma till rätta med detta. I kapitlet Al för en offentlig sektor i framkant finns också en rad förslag på Al-relaterade investeringar exempelvis en nationell infrastruktur för Al (en Al-verkstad) som bland annat möjligsgära implementering och delning av Al-önningar mellan myndigheter, regioner och kommunen.
Vad kan då vara en ambitiös och realistisk målsättningen för området Politisk styrning? Denna fårdplan innehåller flera förslag som kan förbättra Sverige ranking och det finns redan många goda exempel från stvensk offentlig sektor men som idag är svåra att måta. Det gör att Sverige borde kunna rankas bland de 10 främsta länderna år 2030.
- Sverige ska senast år 2030 ha gått från plats 57 till att tillhöra topp 10 i området Politisk styrning
\section*{Utveckling}
Al öppnar upp nya möjligheter till utveckling. Det handlar om omväldande innovativeren som kan skapa stort värde i form av nya kreativa lösningar och tillämpningar. Inom området Utveckling ligger Sverige på plats 30.
Figur 3 visar Sveriges och jämförelseländernas pletering och poång inom området Utveckling. Området spegräl i hög grad det som skulle kunna beskrivas som innovationskraft. Sveriges poång är låga: endast 5 av 100 möjliga. Även om många inlaktorer inom detta område gynnar stora länder, ligger relativt små länder som Finland, Schweiz och Nederlanderna före Sverige poångmässigt. De ledade jämförelseländerna, Singäpore och Israel, får nästan fyra gånger högre poång. Detta är ammärkningsvärt i och med att Sverige ofta ses som ett innovationsland och vanligtvis tranks högt i internationella jämförelser av innovationskraft.

\title{
Ekosystemet
}
Om vi ska kunna förbättra den svenska konkurrenskraften genom utveckling och användning av Al, samtidigt som vi minimerar dess risker, kan vi inte förlita oss på enskilda insatser. Isolerade satsningar faller ofta platt, om de inte utgör en del av en bred satsning. Det krävs i stället en palett av åtgärder som kompletterar varandra och skapar ett ekosystem för Al i samhället. Enligt Al-kommissionen är följande beståndsdelar nödvändiga i ett väl fungerande Al-ekosystem.
\section*{Elektricitet}
För att utveckla och använda Al-modeller krävs att tillgången till elektricitet är riklig och pålitlig. Med stigande efterfrågan på el från andra håll får det inte uppstå oro om att elproduktionen inter räcker till. Det skulle snabb tunkna leda till omolkalisering av Al-företag.
\section*{Beräkningskraft}
Beräkningskraft är en förutsättning för att privat och offentliga aktörer ska kunna utveckla och använda Al. I dag är det möjligt att få tillgång till beräkningskraft på två sätt. Antingen genom inköpta montjänster, vilket innebär att man hyr in sig på ett datacenter som ägs av externa aktörer. Det andra tillvägagångssättet är att införskaffa egen beräkningskraft genom att köpa datorer designade för just Al-användning.
\section*{Säkerhet}
Etilk- och säkerhet är viktigt vid Al-användning. Det rör säkerhet i relation till livlilig användning av Al och användningen av molntjänster. Men det rör också säkerhetsfrågor relaterade till den fortsatta utvecklingen av Al ochoron för existentiella risker och Al-verkty som en viktig komponent i säkerhetsarbetet.
\section*{Telekom}
Telekommäten möjliggör snabb dataöverföring och reattidskommunikation, vilket är avgörande för Al-tjänster som kräver stora mängder data och snabba svarstider. Nästa generation Al-tjänster kommer att ytterligare höjä kraven på snabbhet och täckningsgrad i telekomnäten.
\section*{Data}
I Sverige finns både stora mängder och långa tidsserier av data. För att utnyttja dessa måste data vara tillgängliga och av hög kvalitet, väkstruturerade och standardiserade samt möjliga att hitta för användaren. Regelverken måste också medge att data kan delas effektivt, med beaktande av skydd för den personliga integriteten och upphovsrätten.
\section*{Spetsforskning}
I Al-utvecklingen är avståndet kort mellan grundläggande forskning, tillämpning, innovation och produkt. Spetsforskningsmiljör i samverkan mellan akademi, privat och offentlig sektor behövs därför. Som ett litet land måste vi dra till oss kompetens och idéer utfrån, samtidigt som forskare kan stanna i Sverige.

\title{
Arbetet i olika internationella forum
}
- Inom OECD diskuterade man tidigt vilka principier som borde gälla för tillförlitlig Al. Principier antogs redan 2019 och uppdaterades i maj 2024. Bland annat EU's Al-fördorning, Europarådet, FN och USA använder sig av OECD's definition av ett Al-system.
- Hiroshima Al Process lanserades under Japan GS7-ordförandeskap och ledde till att G7 enades om International Guiding Principles for all Al Actors i december 2023.
- FN antog i mars 2024 en resolution med stöd av mer än 120 länder inklusive USA och Kina om respekt, skydd och främjande av mänskliga rättigheter vid design, utveckling, tillämpning och användning av Al. Generalförsamlingen underströk också Al-systemens potential att acceleraera och möjliggöra framsteg i att nå de 17 målen för hållbar utveckling. Vid toppnötet Summit of the future i september 2024 enades man om en Pact for the Future som bland annat innehåller en Global digltal compact som är det första omfattande globala ramverket för digitalt samarbete och styrning av Al.
- I maj 2024 antog Europarådet en Framework Convention on Artificial Intelligente and Human Rights, Democracy and the Rule of Law som kommer att vara rättsligt bindande för de som underteknar konventionen. Konventionen öppnades för underteknande den 5 september 2024.
- Inovember 2023 hölls ett globalt Al-säkerhetsmöte i Storbritannien, vilket leded till The Bletchley Declaration on Al Safety. Man kom överens om vikten av säkerhetstester av nya Al-system och att utarbeta en state of the science rapport för att bygga internationell konsensus om förmågor och risker med frontier Al.
- Al är ett prioriterat samarbetsområde inom ramen för Trade and Technology Council (TTC) mellan EU och USA. Inom TTC söker man gemensamma ansatser för Al-området, bland annat vad gäller riskhantering, interoperabilitet och transparens
\section*{Vissa av dessa samarbeten är mer globala till sin} karaktär, medan andra är regionala. Båda perspektiven är viktiga och det är centralt att regeringen och myndigheter avsätter tillräckliga resurser för att kunna delta. Givet våra begränsade resurser är det dock viktigt att svenska myndigheter fokuseran insatserna där de gör störst nytta. För att göra det är det viktigt att ensom av milka aspekter som är mest centrala ur ett sensstpektpektiv, och låta detta avspegaslas i det svenska deltagandet och agerandet i respektive organisation.
Antalet enskilda länder som antagigt egna nationella Al-regelverk växer också snabbit i takt med att medvetendehnt om behovet av att reglera ökar. Enligt Stanford Al Index som graskat Al-lagfistfning 128 länder från 2016 till 2023 har 32 länder åtminstone en Al-lagstiftning. \({ }^{}\) Men det handlar inte bara om regelverk. Många länder har också antagit en Al-strategi. Kanada var det första landet med en nationell Al-strategi 2017. I dagsläget finns 75 nationella Al-strategier och flær under utveckling. \({ }^{}\)
EU-samarbetet är centralt för både reglering och utvecklingen av Al i Sverige
EU-samarbetet är det enskilt viktigaste internationella engagemanget för Sverige. Ramarna för hur vi kan utforma svenska lagar och regler sätts ofta på EU-nivå, genom gemensamma beslut i EU:s ministreråd och i Europaparamentet. Aktiviteten i arbetet med tarrel Ala in om EU har en surender a杠ar av hög. EU var till exempel först i värden med att anta ett rättsligt regelverk för specifika användningsområden av Al. Al-fördorningen trådde i kräft den 1 augusti 2017. Vi att<|im_start|>ning som syrtfall att selt till At alInom EU utvecklas på ett<|im_start|>iskäkert sätt som säkerställer.
In av medborgans grundsläggande månskliga rättigheter. medbergoransga en med-ordrängering, och implementeringen, är en gemensam EU-reglering, och implementeringen, är också viktigt för att undvika att EUs inre marknad fragsmetiseras genom olika nationella regelverk. Det kan vara avgörande för ett forätets möjligheter att växka i Europa-i stället för att direkt etablera sig i USA.

\section*{Ändamälsenlig och begriplig reglering}
Ett av de tydligaste budskapen Al-kommissionen har fått vid sina träffar med företrådare från näringslivet, är behovet av ändamälsenlig och begriplig reglering. Det står klart att utformningen och implementeringen av ny reglering ofta upffattas som svåbergplig och varierande mellan EU:smedlemsländer. Mario Draghi framhäljer i sin rapport Den europeiska konkurrenskräftens framtid att europeis skreling måste hitta en förälden företråd en företråd en företråd en skor till en företskiftighet och innovation med mer sammämmighet mellan EUs medlemsländer.
Enligt rapporten utupper mer än 60 procent av EU-företagen att medligeringar utgört either finder/investeringar. 55 proceret<|im_start|> så och medhela storførget med, eller på regelverk och ministrativa böröord som deras största utmaningar.
På Al-området är det bland annat den kommande Al-förordningen. \({ }^{177}\) Datskdydsfördörningen (GDPR) och den nya datafördörningden som är relevanta.
I dessa fall är det uppenbart att det finns många olkharerå other att problemen att tolka reglerna är stora. Detta aktualiserar åtgärder som minskar inforsamoms- och koordinationsproblemen mellan olika tillsynsmyndigheter inom Al-området. Företagens kostnader för att följa reglerna kan också minskas genom att harmonisera regleringar där myndigheter förlenna och varierpanda ansvarsområden, eller genom att förenlka och integrera relaterade regleringar. Detta gäller inom Sverige såväl som inom EU. Al-kommisionsen väklommer därför regleringens uppradgt fill elva myndigheter att minksfa företagens regelbörda, liksom tillsättning av förekningsrådet. \({ }^{110,111,112,113 \mathrm{~K} \mathrm{~m}\) emissionen understryker vikten av att arbetetet läggeret stor vikt vid Al-relaterade regelumningar.
Ett sätt att klarkjora och utveckla regelverken för Al-användning in näringslivet är genom regulatoriska sand- lador. Här får företag möjlighet att utveckla och träna Al-verktyg i en avgråsend och säker miljö. Ett regulorlatorsikt sandåde-program ger en institutionell ram som gör det möjligt för tillsynsmyndigheter att aukkarisera och övervaka företag som testar en innovativ produkt eller affärsmodell, ofta med wisss regulatorisk support eller lättnad för deltagande företag. Genom att använda sandlädör ökar företagens incitament att upfftnina och kommersialisera inom Al. \({ }^{1139}\) För att innovation ska gynnas maximalt är det viktigt att myndigheten ger aktivt stöd till företagen kring hur de kan anpassa sina affärsmodeller för att telva upp till beffintliga regelverk. Implementeringen av regulatoriska sandlädor innebär inte bara att företagen får hjälp att navigera inom beffinitliga regelverk. Det ger också tillsynsynmyndigheter möjlighet att lära sig om ny teknik på ett bättre och snabbare 3tch od därför snabbare tunna utveckla bättre reglering av ny teknik. \({ }^{114,115}\)
I Sverige finns sedan 2022 en regulatorisk sandålda under integritetskyddsdysmighetens (IMY) ansvar. Den är öppen för både privat och offentligt verksamhet och är inrättad utöver den regulatoriska sandläda som ska inätta i senlighet med Al-förordningen. I kapitet Al för en offentlig sektor i framkant föresläs att IMY:s sandläda utökas med ett särskilt spår för offentligt sektor, inklusive en rädglivingsfunsttion. Privata företag har samma behov av att kunna testa affärlsäder i en kontrollerad miljö, och att få vägledning närket gäller Al och dataksvidsdöregelverket. \(^{116}\)
\section*{Förslag}
- Utvecka och expandera IMY:s regulatoriska sandläda för privata företag. För att möta näringslivets behov av klarhet kring. dataskyddsregelverket bör IMY ges i uppdrag att inom maximalt fiva kracker og kensilda privata aktörer cike-bindande besked om olika Allönsningar är i linje med dataskydskingsregelverket. Myndigheten bör också kunna ge information till små och medelstora företag om var de bör vänsa sig i frågor som ligger utanför IMY:s kompetensområde. Det berkåndake behovet av anslag för dessa uppdrag år 8 miljoner kronor årligen. Tjänsten ska vara tillgånglig genom den samlade ingång som Al-verkstaden föresläs utgöra.
\footnotetext{
Al-förordningen beskrivs i kapitet Tjängång till internationella Al-ressurer.
 See Updrag att föreknäke regelverk si yhet att minskar regelbörder av förängslavitå från ett föreknäksom med med med med med med med med med med med med med med med med med med med med metsom med med med med med med med med metsom med med med med med med metsom med med med med med med metsom med metsom med med med med med med med med med metsom med med med med med med med metsom med med med med med med med med med med med med med med med med med med mitsom med med med med med med med med med med med med med med med med med med
}

\title{
Ledarskap och styrning
}
\section*{Internationella positioner}
\section*{ChatGPT sammanfattar:}
I en tid då I utvecklas snabbt och påverkar alja sektorer globalt, blir internationellis samarbete avgörande för Sveriges framtida konkurrenskraft, För ett exportberørende land som Sverige, där den Inhemska marknaden är begränsad, är det nödvändigt att ha en aktiv roll i det globala Al-samarbetet.
Detta kapitel underöker hur Sverige kan delta I internationella policyprocesser och regleringar för samarbetet har vi en unik möjlighet att forma framtidens Al-regler, samtidigt som vi måste navigera i en värdå där olika länder inför egna Al-strategier. Sveriges styrkor, såsom en avancerad tuch eskotor och en offentlig sektor med stor datalligång, ger oss en plattform för att bli en stark internationell Al-aktör, För ett särkestalla framgång kravs dock en tydlig strategi för att attrahera internationell kompetens och samarbeta med de främsta forskningsmiljerna. Detta kapitel utforskar också hur Sverige kan ta en ledande roll i nordiskt och globalt Al-samarbete.
\section*{Det internationella policylandskapet}
Det internationella policysamarbetet tar sig många former och sker i många olika forum. FN, OECD, G7 och G20 har alla börjar utarbeta riktlinjer i takt med att
Al-utvecklingen går allt snabbare, se faktaruta Arbetet i olika internationella forum för en kort beskrivning av arbetet. Syftet är att hitta gemensamma principer när det gäller etik, transparens, ansvar och rättvisa.

Summit of the future Foto: UN Photo/Mark Garten

Det gäller särskilt möjligheten att använda Al för att utveckla autonoma vapen, som kan agera offensivt utan mänsklig kontroll. Riskerna med system med kapacitet att döda utan mänskligt beslutsfattande är uppenbara, och det är problematiskt att en tydlig folkärtslig reglering av sådana vapen än sålänge saknas. Säväl FN som Internationella Rödakorskommittén har efterlyst en snar hantering av de många och svåra frågeställningar som omger autonoma vapen. \({ }^{[13,}\)
Även leke-militär användning av Al exponerar oss för nya hot från frålliga aktörer. Al-system kan till exempel vara känsliga för manipulation eller angrepp som innebär att omäkrbara förändringar i data kan få al-12 hatta fela best. Dulta kan vara särskilt allarligt vid skäkerhetskritiska system, som självikörande fordon eller militära tillämmignur. Det finns också risker med att Al-drivina säkerhetssystem kan reagera för snabbart eller felaktigt på upplevda hot, vilket kan leda till att kontlifteret eskalerar utan mänsklig inblandning eller förståelse.
\section*{Inre hot - brotslighet och extremism utmanter det öppna samhället}
Även innanför landets gränser har säkerhetslåget förämsraturd en senaste åren. Till följå av organisserad brotslighet har antalet spärndgåd och dödligt skutvapenvåld ökat kraftigt. Det har också blivit betydeltlyt planigare att även oskyldiga utomstående bradbas. Lätsjöbelauserat extremis har för med sig att vi lever med en föröhjd risk för torrerddå, och hot den ötpöpa samhällets institutioner. \({ }^{}\) Vindäre är den kriminella ekonomin om affuntade - Polislymngtil, för att för att<|im_start|> till cirka 100-150 miljärder arlängen. \({ }^{}\) Det orsakaar med en till cirka 100-150 miljärder arlängon för enskilda, genom både vålfärdbörtsor tillännäna och bedrägerier. Kriminella använder al-Verktryg i yfte att stärka sin förmåga. Bottspläggå, sasöm bidraggroot which bedrägerier, underlättas till exempel när Al gör det lättare att villeeda myndigheter och enskilda. Al kon kacss användans som stöd för att göra många former av ekonomisk brotslighet mer svårtpäckta.
När det gäller terrorism bedöver Säkerhetsopilisen att attentatshot mot servev i första hand kommer från ensamagerande extremister, snarare ån remer starka organisationer eller statskotterer. \({ }^{}\) Här finns att answers<|im_start|> med att exempel på hot kopplat tillde tillnativveckliginen: med Al-öståd kon en sådan ensamagerande
aktör få betydande hjälp att planera och förbedreda ett attentat.
Al stärker Sveriges säkerhet och försvar
Al används i dag för att stärka Sveriges säkerhet och av
förvarsförmånga. Genom Al-losningar för dataanalys, besultsstöd och underärtelsetjänst går det att analysera stora mångder data på mycket kort tid.
Detta kan användas för bland annat logistikplanering, särbarhetsanalys eller strid. Danda tillämpningar för
försvar och säkerhet är händelse- och bildigenkän- ning, exempelvis för att upptäcka fientliga rörelser eller skilja fiedner från de egna styrkorna. Al kan också användas för att på ett realistiskt och dynamiskt vilsimulera situationer och fieder i spel och övningar.
Tillgången till, och förmågan att tillämpa, Al-teknik är därmed av betydelse för militära maktförhållanden. \({ }^{}\)
Al erbjuder även effektiva möjligheter att bekämpa för och förebygga brott. Inte minst skapas helt nya möjligheter till kunskaputveckning där data från rättssväsendet kan kombineras med data från exempelvis föråskirngskassan, skatterveket och socialtjänsten, Sådana data kan bland annat rättsvärdande myndigheter använda för analys med hjälp av Al på ett sätt förs med tidjäte in vart*v i mjdligt. \({ }^{}\) Det kan leda till nya inskinte intere bar kring hur brott kan upptäckas och bevirus, utan också förebyggs och förhindras. Möjligheterna till biometrisk identifierfång \({ }^{}\) kan också förväntas öka i takt med att Al-tekniken utvecklas, med fökade möjligheter att identifiera sjävål gärningsmän som brottsoffer som följåd. \({ }^{} \mathrm{Al}\) kan även bidra positivt till brottstudningar genom digital förelans som, så ett r instructions om vådande omidra som bilir alt viktigöare för effektiv brottsbekämpknins. Stultilgen kan den tänmåns att Al även kan användas för att snabbt kunta indientföre och hantera disinformation.
\section*{Al:s roll i försvarevt mot cyberangrepp}
Fördermaren al Al in om cybersäkerhet är betydande. Med hjälp av Al-teknik kan skyddet stärks mat sökval antagonistiska cyberangrepp som mot oavtskilliga IT-Incidenter. Genom att analysera somneter i stora datadmänger kan al identifiera potenfål�la hot som är svåra att upptäcka för människor, och varna cybersäkerhetspersonal. Al används också för att autometaristura utnupggifter inom säkerhet och därmed förigra förd i för personalen att någa sin på mer komplex avpgifter som de tekniska systemen inte klarar lika
\footnotetext{
\({ }^{}\) See Fakten nationenans generatfalsning medikation AT\&E/S/27/211 om dödliga autonoma pavejsentiven.
\({ }^{}\) Sti ett exempel Gemensamt utlandene du 5 köböror 230 Trif 154 generalsikterärera och internationella Rödakorskommittlits erforderande om behovet av att eletbara nya förbud och begärmningar avserende autonoma pavesystem.
\({ }^{}\) See Fakten nationenans generatfalsning mediktions av de 1000 polysolids controllers den hattetning som rättensmeditsen andbetrittelse 2023/2024: 9, hff, samt Mottandiskraft och handlingskraft - en nationel strategi mot originaler minister. See 2023/2024: 9, hff, samt Muttandiskraft och handlingskraft - en nationel strategi mot originaler minister.
\({ }^{}\) Seeåktenatspolisions årsärtetelse 2023/2024: 9, hff.
\({ }^{}\) Seeåktenatspolisions årsärtetelse 2023/2024: 9, hff.
\({ }\) See värny pelleåt Data som en för utsträttning för Al-utvecklingen angande möjligheterna att dela data mellan olika aktörer.
\({ }^{}\) Med biometriks identifiering menats att en person identifiera gennen hystesjika ter livolysogiska skärssad in genom mänskias anskinte, flingeravtryck eller med med med med med med med med med med med med med med med med med med med med.
\({ }^{}\) på temperantenspromemorian förBiftadtrøde möjligheter för polsien att användas kamerabavekning (DS 2024-11) föreslats att Polismyndigheten och Säkerhetsoplisen viska fiskå fikt tillånd att användas system för biometriks fikt identifiering i realtid på all minåts för brottskämpninggåndmäkn. \({ }^{48}\)
}

\title{
Figur 4. Infrastruktur - ranking och poäng
}
\begin{tabular}{|c|c|c|c|c|c|c|}
\hline USA - plats 1 & 100 & & & & & \\
\hline Kina - plats 2 & 66 & & & & & \\
\hline Singapore - plats 3 & 50 & & & & & \\
\hline Nederänderna - plats 7 & 40 & & & & & \\
\hline Schweiz - plats 11 & 34 & & & & & \\
\hline Finland - plats 12 & 33 & & & & & \\
\hline Sverige - plats 21 & 26 & & 35 & Poång för plats 10 & & \\
\hline Norge - plats 22 & 25 & & & & & \\
\hline Danmark - plats 25 & 25 & & & & & \\
\hline Israel - plats 26 & 25 & & & & & \\
\hline 0 & 20 & 40 & 60 & 80 & 100 & \\
\hline Poång \(\quad \mathrm{KPI}\) & & & Poång & & & \\
\hline
\end{tabular}
Notera: Placeringen för varje land inom Infrastruktur visas efter landets namn. Den horisontella axeln visar poången för varje land, beräknad utifrån indikatorer relaterade till området. Den högsta poåing som ett land kan få år 100. Den språ stepeln representerar den poåing som krävs för att placiera sig på plats 10 infrastruktur 12024 års upplaga.
Källa: The Global AI Index, 2024 års upplaga.
Inom området Infrastruktur fokuserar de underliggande indikatorerna främst på beräkningskraft, som bedöms utifrån två huvudsakliga typer av indikatorer. Den första utgår från Topp 500-Istan över världens kraftfullaste datorer. Den andra handlar primårt om import och export av material till hvalledare, som är en viktig förutsättning för AI.
För att bedöma infrastrukturen inom telekom utgår man från genomsmittlig nedladdningshastighet, antibolambonenm大桥, samt delanen av befolkningen med tillgång till internet.
Denna Fårdplan försåldr ett antal åtgärder för att stärka Sveriges infrastruktur för AI. Framför allt ligger tonvikten på beräkningskraft för tränning och användning av Al-modeller. I kapitlet Beräkningskraft finns flera förslag riktad mot både akademin och privat sektor. I kapitlet Al för en offentlig sektor i framkant föreslår vi ytterligare beräkningsresurser för offentlig sektor.
Vad kan då vara en ambitiös och realistisk målsättningen för området Infrastruktur? Med de förslag som läggs i denna rapport och med de möjligheter ett närmare EU-samarbete kan innebära så borde Sverige kunna klättra i ranking. Det gör att Sverige borde kunna rankas bland de 10 främsta länderna år 2030.
- Sverige ska senast år 2030 ha gått från plats 21 till att tillhöra topp 10 i området Infrastruktur
\section*{Forskning}
Utvecklingen inom Al pråglas av ett kort steg från forskning till tillämpning och produkt. Företag i framkant måste därför bedriva egen forskning eller samverke med delande akademiska institenter. Spetsforskning inom Al är avgörande för att Sverige ska behålla och stärka sin konkurrenskraft. Inom området Forskning ligger Sverige på plats 19.
Figur 5 visar att Sverige ligger på samma nivå poångmåssigt som många av jämförelseländerna, därlband Finland, Norge och Danmark. Dock är det långt till toppen. Singapore, Schweiz och Israel ligger långt före övriga jämförelseländer, både vad gäller poång och placering.

<|im_end|>};

att driva på digitaliseringen av väljårdstjänster. Det innebå er förlågningen att internet blir nödvändigt för att få tillgång till samhällsservice.
En snabb och stabil uppkoppling i samtliga delar av landet kan med andra ord verka för att utjämna ekonomiska förutsättningar och främja demokratiskt deltagande. Dårför anser vi att det är särskilt positivt att regeringen har för avsikt att ge Post- och telestyrelsen (PTS) updragg att utreda hur stöd kan utformas för egografiska områden där det saknas förutsättningar för kommersiell utbyggnad av mobil täckning och kapacitet. \({ }^{[14]}\)
\section*{Investeringar i mobilnätet}
Betydelsen av en stabil och snabb internetuppkoppling, som diskuteras i föregående avnsitt, kan inte nog belysas. Man vedår det för teknik vi syftar på när vi diskuterat detta område?
Internetuppkoppling är huvudsak beroende av tvåd centrala komponenter: fibernät och mobilnät. Oftak- använder vi fibernätet för att koplla upp oss på internet, till exempel via WiFi på jobbet eller i hemmet. När vi är röleske kan vi koppla upp oss på internet via mobilnätet ( \(t\) dag representat genom olika generation av mobilnät: 3G, 4G och 5G). I framtiden är det dock inte sättet arkit tv i kommer vara beroende av lokal WiFi, eftersom moderna mobilnät erbjuder högre säkerhet och tilltförlitighlet. \({ }^{[19]}\) på blir det möjligt att het rigetera mobilnät och molnintäster för att gefretoag och privatpersoner konstant och säker uppkoppling, avosett plats. En sådan utveckling skulle leda till att den traditionella IT-infrastrukturen på arbetsplatser och i hemmetomverråders och förenklas. Detta är särskilt vårfedult med kante på att arbetsvitel har�ilt mindre platsbundet och flerarbetar helt eller delvis på distätt.
\section*{Happensning av medeligat i framkant när det} välier utbyggnaden av telekom, både avseende en susteena teknologi och täckning. En illustration-av hatta det är att Sverige har ett relativt väl vähutgffgiftenär, särskilt i städer och tätorer, vilket ger en bragrund för att erbjuda Al-ölnssningar. I kontrast står sig Sveriges mobilnät sämre vid en internationell jämförelses.
Utygbngaden av 5G, den senaste generationens mobilån, har skett tidlige och snabbare i USA och Asian en i Sverige. Enligt den senaste upplavan av GMSA:s 5G-konnektivitetsindex hamnar Sverige på plats tjugoett. \({ }^{[18]}\) Våra nordiska garnar Norge, Finland och Danmark tillhörla del av súa bäta i världen. Likas påclarer sig så the rota Al- nationerna Kina och USA USA före Sverige. Vidare visar indexet att av de tjugo
högst placerade länderna på lista不过 tillhör end antist sex EU. Det är av särskilt stor relevans för Sverige eftersom vår telekommarknad i hög grad är pråglad av EU-lagstiftningen.
När perspektivet bredads till att omfatta hela mobilnätet, och inte enbart 5G, framstår situationen som något mer positiv. I en mätning av mobilnätens tillförtilighet, utförd av Opensignal under 2024, placeras sig Sverige på en sjätpletals globalt, med liknande poäng som Norge. Länder som Danmark, Sydkorea och Japan placerar sig alla före Sverige. Opensignal måter även den genomsnittliga nedladningshastigheten i mobilnätet, där Sveriges hastighet är lägre än i våra grannländer Finland, Norge och Danmark.
Hastigheten i mobilnätet kan delvis förklaras av Sve riges tora tyo och glese beholgningstruktur, som har gjort att operatörerna valt att prioritera låga frekvensband. Låga frekvensband kan täcka störreytor, och det behövs förre basestationor och antenner. Nackelen är att låga frekvensband är mindre lämpade för att transportera storaa datamängder.
Globalt används 3.5 GHz-bundet som huvudsaklig bräare för den nya 5G-tekniken, vilket ger möjlighet att överföra stora mängder data. I Sverige tillämpas detta frekvensband endast i störrestäder, vilket kan utgöra en växutade unmanning i takt med att ekonomig digitaliseras. För närvarande abretar Sveriges mobilepoloreret med att bygga ut 3,5 GHz-bandet till fier torer, mente delt kräver att efterfrågan finns från både pública och privåtata aktörer. Det kommert att flera år innan små och medelstora otherar 3,5 GHz, täckning. Under ditten levereras enklare 5G-tjänstervia deljare frekvensbanden.
Att få bättre täckning med högre frekvensband är viktig för utvecklingen av Al, men ännu viktigare å att mobilnätens kämnt (sälja 5G-intelligensen) blir full utbyggda. Kärnnäten behöver upgraderas till full 5G-funkfunctionalitet, så kallad 5G SA, se fakturatan 5G-teknikgör Al möjlig. Först då blir det möjligt att införa ny funktionalitet och nya tjänster, något som kommer att behövas för Al i framtiden.
Exakt vad som förklarar Sveriges låga investeringar i mobilnäten är inte lätt att fastställa. En förklaring som ofta lyfns fram är att låg lönsamhet för operatörerna påverkar deras vilja att investera negativtt. Et tecken på operatörernas låga lönsamhet är att den generellt sett ligger under kostnaden för kapital. Lönsamheten påverkas av att den svenska operatörsmarknaden är fragmentiserad, med många olika aktörer och hög

\title{
4 Ledarskap
}
\section*{och styrning}
När det sker stora förändringar som ger nya förutsättningar för samhällsutvecklingen ställs det annorlunda krav på ledarskap och styrning. Utvecklingen av Al är en sådan systemövergripande förändring. I kapitlet Internationella positioner diskuterar vi först vad det betyder för vårt internationella engagemang. Vill vi vara med och styra utvecklingen måste vi ta en aktiv del i det internationella samtalet om Al.
I kapitlet Ledarskap och styrning diskuterar vi slutligen vilka krav förändringen ställer på vårt politiska ledarskap och styrningsprocesser samt vilka anpassningar som behövs för att relevanta åtgärder ska kunna vidtas och implementeras.
Denna del innehåller:
Internationella positioner ..... 108
Ledarskap och styrning för att genomföra Färdplanen ..... 116

\text{

\section*{Utredningen konstaterade också att det bästa vore att vända på utgångspunkten i OSL, och som huvudregel tillåta att uppgifter om enskilda får utbytas mellan myndigheter. Utredningen hade dock inte möjlighet att lämna sådana förslag inom ramen för sitt updrag. \({ }^{}\)
\section*{Förslag}
- Al-kommissionen föreslår därför att regeringen bör utreda möjligheten att ändra logiken för OLS: bestämmelser, så hutuvredegeln är att det inte räder sekretess till skydd för den enskilda mellan myndigheter och mellan självståndiga verksamhetsgrenar inom en myndighet. Regeringen bör också klargöra vilken slags sekretess till skydd för enskilda som ska kvarstå.
- Al-kommissionen föreslår att regeringen bör överväga om den generella sekretessbryttande bestämmelsen som föreslås i SOU 2024:63 bör utsträckas till att även omfatta sådana uppgifter som skyddas av hålos och sjukvärdssekretess. Det bör exempelvis vara möjligt att bryta sådan sekretess om det behövs för att förbättra håsloch sjukvärdens möjligheter att stålla diagnoser.
\section*{Skyddet för den personliga integriteten och möjligheter att använda data}
Skyddet för den personliga integriteten är grundlägande och framgår av såväl Exporakvonventionen, EU-stadan om de grundläggande dättigheterna, som av regervingsformen. Även EU:s allmänna datskyddsförfördenning, GDPR, är central när det gäller skyddet för den personliga integriteten. \({ }^{}\) Förordningen, som är tillämplig i såväl offentig som i privat veryksamhet, innehåller ett national principiella bestämmelser för hur data som innehåller personuppgifter för behandlas och delas. \({ }^{}\) När det gäller behandling av personuppgifter som utförs av myndigheter i syfte att förebygga, förthårla eller upptäcka brottsligt verksamhet, utreda eller lagföra brott eller verkställa straffättsliga påpöljeder, gäller siltålet UEs atskydadsyddreaktiv som isvensk rätt genomförgs genom brottsdatelagan (2018:1177).
\section*{GDPR kompletteras i fera olika offentliga verksamheter av registerförfattningar. Det finns en stor mängdåsädana författningar. Dessa registerförfattningar gäller framför allt myndigheters behandling av personuppgifter och utgört ett komplement till den allmänna regleringen och förokmerbade båle form av
lag och förordning. Syftet med registerförfattningarna är att anpassa regleringen till de särskilda behov som myndigheterna har i sina respektive verksamheter, samt att göra avvägningar mellan behovet av effektivitet i verksamheten och behovet av skydd för den enskildes integritet. \({ }^{}\)
Senare års samhällsutveckling har emellertid lett till att många registerförfattningar i dag är föräldrade. Al-kommissionen anser därför att registerförfattningarna ska smokerminerasis yfette att göra det enklare för myndigheter att behandla personuppgifter i deras verksamhet.
dragståget är det inte fullt möjligt för myndigheterna att på ett effektivt sättfulgårsa sina författningsenliga uppradg. Det beror på att specifcerade andämälsbæ. stämmelser samt detaljerade uppräkningar av vilka personuppgifter som för behandlas utgört ett storten hinder. Givet den nuvarande upbyggnagden av reglerförfattningar, där en stor del av författningarnas bestämmelser finns i lag, krävs biland riskadsgbeshandling när nya uppgifter behöver behandlas i myndigheterens verksamhert. Dämder är det onständligt att förfärdra regelverket. I vitsusträckning krävs att bestämmelser anges i lag, men regleringen behöver bli mer tätörlig, eftersom nya behov kan upppkomma snabbt, och fler bestämmelser bör kunna lämpa sig för att regleras genom förordningar. \({ }^{}\) Genom att högre grad använda förordningar för registerförfattningarna blir del tättare att genomföra ändringar vid behov.
För flera myndigheter och verksamheter har registerförfattningarna börjat ses över och det finns förslag, eller redan genomförda förändringar, som innebär en mer eller mindre modern och ändämalsenlig reglering. Som exempel kan nämna förslaget på nya registerförfattningar för S4tkeverket, Tullverket och Kronfogedmyndigheten (SOU 2023:100). Vi anser dock att motsvarande förändringar, där de lada förslagen kan utgöra utgångspunkt, behövs för samtliga venas smykndigheter. Det skulle varne att första steg för att megyndigheterna bättre möjligheter att utnytja potentialen med ökad Al-asavändning. Det finns också fördelar med att liktarade formuleringar användas i olika registerförfattningar.
Detta första steg kan genomfäs relativt snabbet, men det är inte tillräckligt för att underlätta användningen av Al in de ontfiliga verksamheten. I ett längre perspektiv är det involter att förändra den svenska

\title{
Kommissionens ätgärdsförslag
}
Utöver denna inledning består rapporten av tre övergripande delar: En stabil grund att bygga på, Al för alla och Lederskap och styrning. Under respektive del finns ett antal kapitel som rör specifika sakfrågor av vikt för användningen av Al i Sverige. Dessa har bruttsar i rind imade avsentil. Ivar jearc kapitel finns också ett antal förslag på satsningar som Al-kommissionen anser ndövändiga för att utvecklingen ska gå tätt håll.
Nedan sammanfattar och motiverar vi de huvudsakliga förslagen. Vi har alt att gruppera förslagen i fem kluster: Politiskt ledarskap behövs, ett kunskapslyft förla, framltssäkra vålfärden, Forskning i vårlds klass samt Innovation för utveckling.
Som vi betonar i beskrivningen av ekosystemet ska våra förslag ses som en helhet där samtliga förslag kompletterar varandra. Att exempelvis genomföra hållten av förslagen skulle inte ge halva effekten - det är helheten som ger resultat.
\section*{Politiskt ledarskap behövs}
Al-kommissionen har i sina konatter med företdrörare för olika grupper i samhället fått förmedlat en bild av betydande frustration. Många år otäliga att få ta delar av de effektivitets- och vålfärdsvinter som Al kan erbjuda, och bekymrade över att det går för långsamt. De anser att dettas många initiativ men att bristen på samordning och strategisk planering gör att hindren är för sota och framstegen ofta utebli.
Grundproblemet verkar vara kopplat till svagt centralt ledarskap och en bristande förmåga att hantera en
systemövergripande teknik som Al på ett ändamålsenligt sätt. Det räder en närmast total samstämmighet om att den svenska förvaltningsmodellen, med en långt driven delegering, har sina begränsningar när man ska hantera den typ av vårsketoriella utmaninger som Al representerar. Under mer normal som ätandigheter fungerar modellenål. Men itider av snabb, omvålvande och systemövergripande förändring kan det behövas mer central styrning, på samma sätt som man på ett säljukhus temporätt går upt tillifälligt stabslåge i kritiska situationer.
Vi föreslår därför att en särslåd task force tillsatts vid Statsådsbærdendenign i Regeringskansliet, i syfte att bevaka att nödvändiga åtgärder gällande Al styrning. Vi har all och styrning, en för en byrgga mellan politiken och de medarbetare på Regeringskansliet som jobbar med de enskilda sakfrågorna samt eregbulendet samråda med representanter för samhället i övrigt (nåringsliv, arbetsmankelnars antare, kommernur de groginer). Gruppen bör leds av en statussekreterare med erfarenhet från arbette i Regeringskansliet och innehålla både parallelist- och specialistkompetenser. efter fem år bör det utvårderas om denna kast force ska fortsätta sitt alberte, eller om man kan återgå till ett mer normal läge vad gäller hanteringen av Al-relaterade frågor inom förvaltningen.
En viktig uppift för gruppen bör vara att bedreå ett regeringsbeslut om Al-strategi, baserad på denna Fårdplan. Ett sådant beslut bör tas under våren 2025. Regeringen, genom denna task force, bör årligen följa upp de åtgärder som vidtas i syfte att uppfylla målen

<|im_end|>};

\title{
Bilaga A Förslagsista
}
Tabellen nedan innehåller sammanfattningar av samtliga förslag som presenteras i Färdplanen tillsammans med en sammanräkning av deras kostnader. Kostnadsatta förslag är ofärgade, och förslag som förväntas kunna genomföras inom ram/utan extra kostnad är [Bi]. Förslag på utredningar som kommer ha behov av lämplig finansiering är orange].Dessa utredningar har getts ett schablonbelopp på 5 mnkr, men den exakta finansieringen kommer behöva fastställas fall för fall. Förslag vars kostnader täcks inom ett annat förslag är röda.

att skapa en ny Al-myndighet, utan att myndigheter med olika ansvarsområden för Al-frågor länkar sina respektive funktioner till Al-verkstaden. Det betyder till exempel att Digg's vågledningar och IMY's repulatoriska sandlädor för att träna Al-modeller, för både privat och offentlig sektor, enkelt går att finna via Al-verkstaden.
Den Al-hubb som föreslås i kapitlet Kompetenslyft föralla kan också ha sin hemistiv har. I hubben ska det finnas information om vilken kompetens som kommer att efterfrågas framvor och vilka utbildningsar som finns för att fylla dessa behov.
\section*{Förslag}
- Al-kommissionen anster att en portfallunktion som beskrivs avon är en viktig del att färmjaa användningen av, och kunskapen om, Al i sambållet. Hur en sådan funktion ska kunna åstadkommas behöver dock utreda vidare, förägsvis isamband med den utredning som görs för etableringen av Al-verkstaden.
Område 2: Särkeställa tillgång till data.
I offentlig verksamhet finns strona mängder relevanta data som kan användas för att utnyttja potentialen med Al, men det finns en rad hinder som behöver.
åtgärdans. Offentliga aktörer måste först och fråt avverkatt till att data som genereraseris värkade och avverkatt.
och håller hög kovillar för att föräksade och bålla tillåtet att dela dameläma, delar av det offentliga och att ansvarnida data tillträckigt utsträckning. Det har handlar om att förändra der tätsliga förutsättningarna. Dess omt förbehöver det vara tekniskt möjligt och enkling ett offentlig verksamhet att komma åt, använda och dela dosta som respektive aktor behöver. Säkenhet och vasernitt behöver vårensis i sammanhänget.
Inomo datamrådet behövs en tydlig förflytning för att offentlig sektor fulut ut ska kunna nyttja möjligheterna med Al och minimera dess risker. Konkreta åtgärder diskuteras i kapitlet Data som en förutsättning för Al-utvecklingen.
Område 3: Värna enskildas tillt och förtroende 1 få andra länder har medborgarna en så hög tilltro till staten och sina medmänniknor som i Sverige
Det gör det mesta mindre tungrott och mer effektivt när det kommer till beslutsfattande och mänskligt umgänge. \({ }^{}\) Som exempel kan nämnas det stora förtroende vi har för vår skattemyndighet, vilket till exempel har möjliggjort smidiga sätt för oss att delkarera våra inkomster. \({ }^{}\) Den här tilliten och förtroendet för det allmänna får dock inte tasa för givet, den behövas vånas. \({ }^{}\) Vid etableringen av en Al-verkstad, och vid en ökad användning av Al-tjänster inom offentig sektor måste således tilliten till det offentliga beaktas.
En samhällsutveckling som gör offentliga aktörers beslutsfattande mer svårbegripligt kan snabbt leda till en urcholkning av förtroendet. I takt med att Al används i allt större utsträckning för beslutsunderlag, och på sikt även direkt myndighetsutvöning, är det därför nödvändigt att vidta åtgärder för att bevara tillten och förtroendet. Det handlar också om att vårna om att svensk myndighetsutvöning även framvöder ska vara transparent. In inte minst förvatningslagen ställer krav på att det offentliga ska se till att enskilda ska kunna förstå på vilka grunder ett vist beslut har fattats. Det finns twärr exempel i nörråld från Sverige och andra till grund för myndighetsutvöning som varta i avgat. felaktig och även oglång. \({ }^{}\) In som visas ist av gär.
Myndighetsutvöning där den denskilde inte ges möjligt att förstå hur ett beslut har fattats kan leda till en bristande tillit, i förhålland tilde tilkeniken som sådan och till det offentliga. Det är därför viktigt att den enskilde kan förstå hur ett beslut har fattats, såskilt om digitala hjälpmed enlåvands som beslutsvåd. Mot den bakgrunden är det betydelsefullt att utvecklare, i den mån så är möjligt, använder sig av öppen källdod. Bd blir det enklare att förstå varför ett Al-verktyd producerat ett vist resultat. Det blir också enklare att granska hur en offentligt aktör fattat ett vis bestul set. UEs Al-förordning kommer att spela en roll det attare betge enorm att ställa krav på transparens, spårbarhet och information. Aven etiska riktlinjer är viktiga för att bygga en sold两边 (11.162)

internetkunskap.se som hjälper till att utbilda säkra och medvetna internetanvändare genom att samla kunskap om nätet och digitala tjänster som vi har användning för i vardagen. Bland annat har man en sektion om Al där man beskriver vad det är och hun det används, samt vilka möjligheter och risker som finns med tekniken. Myndigheten för psykologiskt försvar (MPF) har ett särskilt ansvar att se till att Svejrgies befolkning har en god beredskap när det gäller avgåttet att för att skor. Vi att skor och att avgått ett avgått att samderna det nationella arbetet med medle-och informationskunnighet. Som ett led i detta har Medisemyndigheten fått i uppradg att genomföra en nationelt satsning för stärkt medie-och informa ochsCHAunnighet under 2024 och 2025. Satsningen ska hjälla amhånetens kunskap om hur Al kan användas i Informationsfödet och där medt och rälda till att stärkka samhällets öntäkskratt mot bland annat ansdeinformation och otillbörlig informationspåverkan.
Folkbildning utövas i olika former men viär på samna grundidå om att frä�nja livslångt ärlandare, aktiv medborgarskap och demokratiskt deltagande. Mest kända är kanske de olika studieförbunden, men föreningslivet och olika privata och offentliga initiativ är också omfattande; se faktarutan nedan för exempel på folkbildningsinitiativ runt digitalisering och Al Samantaget deltar över 80000 personer bara i studietförbundens och folkhögskolornas verksamheter varje år. \({ }^{107}\) Det svenska folkbildningssystemet erbjuder därmed utmärtka möjligheter att sprida kunskap om Al till amhänheten - kunskap om möjligheterna, men också om riskerna. Särskilt värdefull är möjligheten att nå ut till grupper med ett högre digitalt utanförskap, såsom funktionsnedsatta och äldre.
\section*{Folkbildningsinitiativ inom digitaliserings- och Al-området}
SeniorNet (seniornet.se) är en ideoll organisation som hjälper senieror använda digital teknik och digitala tjänster.
Studiefrämjandet och biblioteken har inom ramen för Digidelnätverket, skapat Medborgarveckan (digidel) se) som arbetar för samverkan och kunskapsdelning. Under en vecka varje år sker en rad aktiviteter med fokus på allmånhetens möte med digitala tät.
Samverkansplattformen Digitalidag (PTS.se), som årligen och nationell kraftsamlar för att inspirera alla människor till att vilja och kunna vara en del av den digitala utvecklingen. Under Digitalidag 2023 medverkade till exempel 375 aktörer (privatoch offentlig sektor, universitet och högskolor, arbetsmarknadens parter och civilsamhället) och tillsammans genomförde man över 1000 aktiviteter i 216 kommuner. Genom att bygga på befintliga strukturer kan utbildningsaktiviteter arrangeras på platser där människor känner sig trygga och genom aktörer som är relevanta just för målgruppen.
Digitalhäljpen (PTS.se) ger vägledning och tips för den som är digital nybörjare.
Svenska Science Centre, 2020, 2025, 2026.
Intrittet med ett före och förrestråder, främjar och utvecklar branschen med 20 science centers över hela andet- en samlad nationali resurs för livslängt lärande. De arbetar tillsammans för att ge främst barn och unga, oavsett bakgrund, möjlighet att öka sitt vetenskapliga kapital genom kunskap inom Al, naturvetenskap, teknik och matematik samt stärkta förmågor inom entreprenörskap och innovationsutveckling.

Foto: Anna Gerdén/Tekniska Museet

<|im_end|>};

för Sveriges Al-strategi. För att underlätta uppföljning av implementeringen, och för att få en indikation på om åtgärderna har avsedd effekt, föreslår vi uppföljningsindikatorer (Key Performance Indicators), baserade på Tortoise Global Al index. Inget index är perfekt, men vår bedömning är att detta år bäst för dänjärbølser som ska spegla utvecklingen av vår konkurrensförmåga på Al-området.
Vi föreslår också att regeringen uppmumtrar myndigheter att använda Al i sin verksamhet. Till exempel genom att införa åtrrapporteringskrav på myndigeter om hur de arbetar för ansvarfull implementering av Al eller genom specifika uppdrag om att öka Al-användningen.
För en liten öppen ekonomi som den svenska har de globala markonderna och det internationella samarbetet varit mycket viktigt för att nå den välfård vi har id,ag EU-samarbetet inter i detta fall en särställning, eftersom vi som medlemmar förhandlar fram gemensamma lagar och regler som ih görg bald blir styranden för vad vi kan göra nationellt. Ett engagerat och proaktiv tager andenime onEU är därför viktigt. Det gäller inte minst Al-el的生活ade frågor som användning och delning av data eller gemensamsa matsangir på beråkningskraft och starka forskningsmiljöer. Vå bild är att mer kan göras i detta avseende. Vi föreslår därför satsningar på att öka den svenska representationen i EU:s institutioner, inte minst för att ta plats i den nystartade Al-byrån i Bryssel. Vi föreslår också åtgärder för att öka det svenska utnyttjandet av gemensamma EU-sartsningar inom forskning och innovation, vilket ofta kräverviss svensk medfinansiering.

Som framgår i Mario Draghis rapport Den europeiskakonkurrenskraftens framtid tenderar EU-regleringen att i många fall utgöra ett hinder för företagsamhet och innovation inom unionen. Vi menar att regeringen måste verka för att EU-regleringen ska göra det möjligt för Al-öksningar att komma till. En viktig del i det arbetet är att göra vad vi kan för att implementeringen av EU-regler, som exempelvis GDPR och Al-fördningen, blir mer enhetlig mellan medlemsländerna. I har det att för att det att det att det att det i medlementerning och tolking av EU:s regelverk utgör ett handelshinder. Det senare illustreras bland annat av att flera amerikanska teknikföretag har dröjt med, eller helt avstätt från, att lansera sina mest avancarede Al-modeller för användning i EU. Med tanke på hur snabb den tekniska utvecklingen är utgör detta en stor risk för svenska företag. Det här är en bidragande orsat kill att vi, utvör arbetet inom EU, måste satsa på bilateral samarbete med de bästa i värden inom Al. Vi föreslår därför en ökad satsning på tekniska attacher med mjup knapskom det versenska Al-ekosystemet.
\section*{Ett kunskapslyft för alla}
Al är ett kraftfullt verkty, men tekniken i sig räcker inte för att skapa nyttå för människor och samhälle. För det krävs användning, det vill säga interaktion mellan människor och Al-verkty. Det här handlar inte heller bara om ingenjörer och tekniknörard- a lla kan på något sätt dra nytta av att använda Al-tjänster.
För att användningen av Al ska ta fart i hela samhället behövs docken grundläggande nivå av Al-kunskap i hela befolkningen. Alla ska kunna delta i ett samtal om Al, runt köksbordet, under jobblunchen eller i styr-

\[
\mathcal{L} = \left\{ \begin{array}{ll}
\displaystyle \frac{1}{2} \sum_{i=1}^{n} \frac{1}{2} \sum_{j=1}^{n} \frac{1}{2} \frac{1}{2} \frac{1}{2} \frac{1}{{\rm Tr} \left( \frac{1}{2} \frac{1}{2} \frac{1}{4} \right)} \frac{1}{2} \frac{1}{2} \frac{1}{n} \frac{1}{2} \frac{1}{2} \frac{2}{2} \frac{1}{2} \frac{1}{2} \frac{n}{2} \frac{1}{2} \frac{1}{2} \frac{\partial}{\partial n} \nonumber \\
\displaystyle \frac{1}{2} \sum_{i=1}^{n}{\rm Tr} \left( \frac{1}{2} \frac{1}{n} \frac{1}{n} \frac{1}{2} \frac{1}{{\rm Tr} \, \left( \frac{1}{2} \frac{1}{2} \frac{\partial}{n} \frac{1}{2} \frac{1}{2} \frac{\partial}{{\rm Tr} \, \left( \frac{1}{2} \left( \frac{1}{2} \frac{1}{2} \right) \frac{1}{2} \frac{1}{2} \frac{1}{3} \frac{1}{2} \frac{1}{2} \frac{12}{{\rm Tr} \, \left( \frac{1}{2} \right) \frac{1}{2} \frac{\partial}{\partial n}} \right)}{2} \frac{1}{2} \frac{1}{2} \frac{2}{{\rm Tr} \, \left( \frac{1}{2} {\rm Tr} \, \left( \frac{1}{2} \frac{1}{{\rm Tr} \, {\rm Tr} \, \left( \frac{1}{2} \frac{2}{2} \frac{1}{4} \right) \frac{1}{2} \frac{1}{2} \left( \frac{1}{2} \frac{\partial}{\partial n} \frac{1}{2} \frac{1}{2} \frac{11}{{\rm Tr} \, \left( \frac{1}{2} {\bf \partial} \right) \frac{1}{2} \frac{1}{2} \right)} \right)} \nonumber
\end{array} \right\}
\]

mäste alla också förstå thar Al kan användas i den egna ssvselsättningen. All utbildning måste därför Al-säkras genom att Al-kunskap integreras över hela linjen - oavsett om det handlar om samhällsvetenskap, juridik, teknik eller biologi. Det här gäller även arbetsgivans norma validaureutbildning av sin personal.
För att åstadkomma detta föreslär vi en omfattande satsning på utbildning i hela samhället - ett kompetenslyft för alla. Här ingår satsningar på folkbildning, inklusive folkbilitonekoten, och en möjlighet för alla att gratis ta del av kvalitetstestade Al-verktyg. Med andra ord en satsning i likhet med vad som gjordes i samd and med hem-PC-reformen 1998. Satsningen är tänkett att göras inom ramen för en så kallad Al-hubb, där man kan få information om hur man som individ och ran tränyta av. Li förjeslår vidare sin satsning på vidareutbildning av lärare vid universitet och högelskor, så att Al kan bil en del av all högre utbildning.
Ökad kunskap om Al är en viktig faktor för att ötka skyddet för individer och samhälle mot exempelvis krilllig användning av Al. Det är också viktigt för att människor ska bil medvetna om andra typer av sikrer och utmanring kopplade till Al-användning. Al är emellertid också ett mycket effektivt verktyg för att motverka riser för samhället. För att ytterligare öka säkerheten i samhället föreslår vål dörför en satsning på ökad forskning om Al och cybersäkerhet, inklusive tekniker för att förbätra den personliga integriteten. Dessutom föreslär vi att ett institut för Al-säkerhet skapas, med uppgift att bedriva och främja forskning skagneshärterskiker förknippade med Al. På svis av kän institutet bidra till att faktiska säkerhetrsikser bylesses och blir adresserade. Vi belyser även behovet av att istiska riktlinjer för användningen av Al.
\section*{Framtidssåkra välfärden}
Den demografiska utvecklingen utgören allvanlig utmanring för den offentliga sekontv- i bli roll att färre som ska tha an mod allt fler, 1 takt med att samhället iörvigt utvecklaska ökar också kraven på offentliga tjänster. Inom ett par är bedöms sekorn behöva leverare 125 procent av dagens väpfårlård, men med 75 procent av dagens bemanning. Offentlig service ska också finnas tillgånglig hela tiden och overallt. Det gör det mycket svakt ärft många aktörer, till exempel små kommuner, vars uppdrag inte skiljer sig från de storas. Al-kommissionen anser att en ökad användning av Al-tjänster är helt nödvändig för att den offentliga sekontem ska kunna klara av att tleva upp till sitt åtagande, något som är en viktig del av hela vårt samhällskontrakt. Det finns dock ett antal hinder i vägen för en sådan utveckling.
Ett viktigt hinder är dagens begränsade tillgång till data, och svårigheterna att dela data mellan och inom myndigheter. Sverige har en värdefull tillgång i form avtana. I dagslåget har vi dock väldigt svårt att använda denna, såväl i offentlig som i privat sektor. Resultatet blir att många potentella lösningar inom områden som värd och omsorg, brottsbekämpning och kontakten av håld och av håld och myndigheter förblir outnyttjade. Orsaken ligger till<|im_start|> av håld förändningå som ofta utformades då värdet av att kunna rättångå en varbetyldigt mindre än id. ågg. Sett till hur datadelning in dag kan ska pavår, måste regelverket i vissa delar kallbreras om. Det finns också en bydetyande osäkerhet om hur det existerande regelverket ska tolkas när det gäller möglicher att dela data. Detta gäller inte minst EUs: stadskyddsförordning, GDPR. Vår bild är att osäkerheten leder till att beslutsfattare i både offentlig och privat sektor tenderar att ta det såkra före det osäkra. Det innebär det att helle avstråf från att pröva en möjlig Al-lösning än att ta risken att byrta mot reglerna. Den här osäkerheten måste minnas.
För att gödra data mer tillgängliga förl A-avändning och minksna de legala osäkerheten lägger vi fram en rad förslag. Det handflar bland annat om att ändra logiken för offentlighets- och sekretsslagen, OSL, så ut huvudregeln är att det inte räder sekretss till skydd för den enskilda mellan myndigheter. Det vill säga att utgångspunken ist tället bör vara att myndigeter ska kunna utbyta information med varandra. Fört mottervka fragmentering och lice henteg. tillämpning av registerforfattningar bör det utredas hur en ramlagför för att förbattning.
För att fört tillämpningen av GDPR i Sverige utformas. Vid att göra det tätkare för enskilda att bitta för att företaslär vi också att en så kalldata Deta offentlig.
För att glauftningen för vägledning ska inättas vid Sta listiska Centralbryan (SCB). Med dessa åtgärder skulle möjligheterna öka betydlig för aktörer från både offentlig och privat sektor att använda sig av data för Al-tjänster. Vi föreslår även att det offentliga aktörernas möjligheter att använda sig av moltjänster som erbjuds av företag utanför EU förtydligas.
Utverb ättre tellega och praktiska möjligheter att dela data krävs det också nya förutsättningar för samarbete om den offentliga sektorn ska kunna svara upp mot sina utmannger. Det handflar om samarbete mellan offentliga aktörer, men också mellan offentlig och privat sektor. Anledningen till att samverkan i dag är svår att få till berior framför all på att vi saknar en gemensam Al-infrastruktur som uppfyller den offentligliga verksamhetens krav. Systemen i offentlig sektor är fragmenterade med stora problem att kommunicera med varandra, något som kraftigt försvårår utvecklingen av gemensamma Al-lösningar.

Vision: En svensk hällsomodell
\section*{Sammanfattning}
Denna vision syftar till att bestkriva hur man skulle kunna inritta ett nationellt projekt för utveckling av en storskallg Al-modell baserad på svenska hälso-
data. Projektet, kallat Svensk Hälsomodell (SHM), skulle bygga vidare på den existerande infrastrukturen health bank vid Stockholms universitet. Syftet är att skapa en världsunik resures för forskning och innovation inom hälso- och sjulvården.
\section*{Svensk hälsomodell}
Sverige står infor en historisk möjlighet att bli världedande inom Al-drin vählo- och sjukvård. Vårt land ar hen unik position med omfattande digitaliserade hälsodata och ett personommurysystem som gör en sömlös koppling mellan olika datakällor möjlig. Den infrastrukturen, i kombination med vårt starkar ykte av att skapa en världsunikation. Vårt land av att skapa en världsunkin avtångspunkt för att revolutionera global hälsovård. health bank-projektet vid Stockholms universitet har redan visat den enorma potentialen i att använda sådana data för utveckling av Al-baserade verktyg för hälso- och sjukvården (Dalianis et al., 2015). Nu är det dags att ta nästa eget och lansera ett nationellt signalprojekt i klass med USA: Apolloprogram - ett projekt som inte bara syfart till att skapa en världsunkin resurs för hälsoforskning, utan också positionera Sverige som det globala navet för hälsoinnovation.
Syftet med detta ambitöisa initiativ skulle vara att skapa en nationell Al-drin hälsomodell av internationell topplkass - Svensk Hälsomodell
(SHM) - som kan revolutionera allt från diagnostik till behandlingslparenring.
SHM skulle utgöra en stor spkråkomdel med omfattvande förståelse för hälso- och sjukvårdsproblem. Genom att samla och bearbeta data från fäkarllor som elektroniska patientjournaler, kallitetresgister och genetiska databaser, kan modellen formera som ett krafttult verktyg inom värden, och ge besultsstödt till vårdpersonal, identifiera riskfaktorer och förutsåga sjukdomsförlop. \({ }^{}\)
SHM har potentialen att förändra vården i grunden genom mer precis, individanpassad och proaktiv behandling. Det skulle inte bara förbätta forkløhålsan utan också leda till stora samhälsekomoniska besvargner. Antitiivet kan även dra stil sig betydande internationella investeringar i den svenska hälsosektorn, ska upistenats well/favilerade jobb och attrahera vårtdslade enterer inom Al, medicin och datvatenstaps
till Sverige. Genom att stärka vår position som en global kunskapsnation skulle SHM driva en kraftfull tillväxt inom hälsotekniskt entreprenörskap och etablera Sverige som ett internationellt centrum för innovation inholtsetoknolgi. Dessutom kan SHM bana väg för en ny exportindustri inom Al-drinvårtdektnologi och bidra till förbåtrad hälsa världen över.
Genom att satsa stort på projektet visar Sverige ledarskapi i en av vår tids mest avgörande frågor - hur vi kan använda den senaste teknologi för att förbättra hälsa och livskvalitet för alla. Detta är inte bara ett forskningsprojekt, utan en nationell vision med potential att omdefiniera Sveriges roll i den globala ekonomin och samtidigt bidra till en friskare värld.
\section*{Projektets struktur}
Projektet är uppfövgggt kring tre huvudsakliga områden: styrning, som syftar till att samordna och förakra initiativet inom de samhällssektorer som behöver engageras, en regulatorisk kommitté, med ansvar för att hante räfagor kring dataskyd, det eik och patientsäkerhet samt ett forskningscenter, vars främsta uppgift år årt utveckla SHM.
\section*{Styön}
Syrön ska styra SHM behöver sarna el man ångblaf av perspektiv och samtidigt förankra initiativet både praktistok och elitiskt. En möjliglönning kan vara att inträta en oberoende stiftelse som har det övergripande ansvaret för initiativets styrning och uppgiften att förvajtala data. Med tanke på SHM:s värvitetenskapliga natur, bör en sådan stiftelse inkludera framstående representanter från många olika samhällsekrotter.
För att tillföra viktig vetenskaplig expertis behövs. ledande forskare inom Al, medicin och etik, både från svenska och internationella toppuntiversitet. Närings livet bör vara representera att ekobphal med expertis inom Al och storsklig datahantering, globala läkemeksföretag intresserings, en med med med medet och medet och kraftetag med praktiskt skankspikav av implementering. Offentiget sektor spelar en central roll i detta initiativ, med representanter från myndigheter som Socialstylresen och Datainsptektionen, samt grejtoner och kommuner som kan säkerställa kopplingen till praktisk sjukvård. Styrelsen bör även inkluderis invereraste, såsom venture capital-firmor och stora pensionsforder, som kan bidra med nödvändiga resurser för långsiktigt utveckling. Internationella organisationer som WHO OH eU:s hälsokommission bör inkluderas för att se till tprojektet följer (12015)
\footnotetext{
[120] Attnetets inpatientsfotskälla är internationale projekt som Med FallM 2, e hälsomodell med att pluksvårm om medionelka frågor. Se Singsal, Kran, Tto, Värt, Goffslivos, Föry Søvnes, Bjel Weyckin, Y. Lhas Kien, Värt, Graft et al., Tavarese expert-kan del socialisation avserning avtårliga nagrojmeda-avtör-varprän xiv/2:2030.06/17 (2023), och för exempel på tillmillningar: Galin, Järn, Cheznylig, Jan, Chanan Zhang, Guoqing Cal, and Bechang Lu, A. L. Värt Cancer QuostorAnwæng. System Based on Next-Generation Intelligence and the Large Model Med-PalM 2. International Journal of Computer Science and Information Technology 2, no. 1 (2020): 28-30.
}

omsorgsdokumentation samt genom att ge E-hälomsomydigheten ansvar för att ta fram interoperabilitetslösningar för hälso- och sjukvården.
Sammanfattningsvis är det tylligt att det finns en polityskil vija att undertätta möjligheten att dela och använda data. Trots de vidtagna initiativen finns det emellertid fortfarande betyndande svårigheter, såväl legala som mer tekniska, för hela samhället att dra full nytta av den strategiska resurs som våra data utgör. Det här spegral bland annat att datt gesregelvert inte alla delar är anpassade till dagens utmaningar och möjligheter. All regelutformning kräver att beslutsfattare väger samman olika intressen, för- och nackdelar, för att hitta en rillig jämvikt. Vårdet av er剃ering mäste alltid vägas mot värdet av att inte ha regleringen, eller ha en mindre omfattande reglering. Här finns det anledning att finna en delvis ny balsom.
Innan användningen av Al tog fart togs det av naturliga skäl inte hänsynt till de betydande fördelarna som i dag vore möjliga genom en mer delningsvänlig reglering. Kostnaden av att ha ett strikt regelverk, i form av stoppade möjligheter, var helt enkelt mycket lägre längret tillbakka i tiden. Det är Al-kommissionens uppfattning att det här har lett till att den reglering som berör möjlighetera till datdelningen generelt sett skulle behöva balanseras om i syfette att göta data mer tillgånglig.
Nedan går vi ingen omolka ohrmaden med bärng på datadelning där vil bedömer att åtgärder är nödvånagda. Utvöder förslag som lämnas: itta kapitel geret Al-kommissionen sitt stödt til de förslag som lämnas i tidrigare nämnda Utredningen om interoperabilitet vid datdelning samt Utredningen om infrastruktur för hälsodata som nationellt intresse.
\section*{Offentlighet och sekrestees}
Inmydigheters - och i vissa privata - veryksamther gäller offentlighets- och sekretesslagen (OSL). Lagen anger under vilka omständigheter en viss uppfigt omfattas av sekrestees eller är offentlig. Sekrestes kan finnas för att skydda olika intressen, såväl enskilda som allämna. Det kan exempelvis handla om slydkat bodyd för den personliga integriteten hos en patient i sjulvården eller sekretstes för att skydda Sverige särgesäkerhet. \({ }^{}\)
Huvudregeln i OSL är att en uppfgift för vilken sekretseg skapler inte får föjas förens schilda eller för andra mydigheter. Denuta åtgapspunkt gäller inte enbart mellan myndigheter, utan även mellan olika verksamhetsgrenar inom en myndighet när de är att betrakta som självständiga förhållande till varandra.
Informationsutbyte av uppgifter där sekretess gäller, får endast förekomma om det särskilt angles i OSL eller annan författning.
De nuvarande bestämmelsernia: OSL, om sekretess mellan och inom myndigheter, försvårar utbytet av data. Det har också varit avsikten, då syftet med reglerna är att våma enskildas personliga integritet och förebygga missbruk av tillgången till information.
Samtidighet har lagstifferan konstaterat att det finns ett stort antal fall där det är motiverat för myndigheter att kunna dela information med varandra. Det finns därför ett antal undantagsregler från principen om sekretess, genom olika uppgifftskyddligheter och sekretessbyrande bestämmelser. Dessa har blivit så ränga och invexkulade att de har skapat en komplex och svårtoklad zagfittning, dem ett lapftpäcke av regler. Desutom tar de sekretessbyrande bestämmelserna ofta fasta på att vissa uppgifter får delas i ett enskilt fall, det vill säga i ettårende. Många av de möjligheter som Al erbjuder bygger dock på tillgång till stora datamängder, för bland annat analys. Det till säga uttrinnässigt utbyte av större datamängder. Vi inte en ettet berät sker för ensmikla ärenden utan på en vorter ogvergåne livå och med fokus på föreleiselser.
För att en sådan delning av data som många Al-killlämpningar kräver ska vara möjlig, anser vi att ett paradigsmikfte för offentlighets- och sekretesslagstiftningen är nödvändigt. Utgångspunkten bör varada ontfittel man myndigheter i stället för sekretessen är delt gäller skydd för enksildas personliga och ekomniskora hållanden. Sekretess är fortsatt lämpligt i vissa fall, men det bör vara undantaget, inte regeln.
I september 2024 presenterade Utredningen om förbättrade möjligheter til informationusbutte formal minydigheter sitt förslag till en ny gerenell sekretessbyrande bestämmelser. \({ }^{}\) Utredningens förslag innebät att myndigheter ska ges möjlighet att, för att för att för att för att för att för att information med varandra. Detta förutsatt att det är nödvändigt för att en annan myndighet ska kunna förebygga, förhindra, uppttacka eller treda fusk eller regeløverträdelser, eller för att kunna handlägga ärenden i författningsreglerad verksamhet. Den föreslagna regeln har emellertid ett antal undantag. Bland annat oftarden inte uppgifter som skydsad av hälos- och sjukvårdsekretess. I sitt betänkande uttalar utredningen att den föreslagna bestämmelsen kan gröra det möjligt för myndigheter att utveckla och använda Al-modeller i sin verksamhet i större utsträckning än vad som är möjligt i dag. \({ }^{}\)

Vi f®eslår dårför att det inrätts at gemensam käninfrastruktur för utveckling och leverans av Al-drina tjänster inom offentlig sektor - en så kallad Al-verkstad. I verkstaden ska offentliga aktörer - statliga myndigheter, regioner och kommuner - kunna utforska, utveckla och driftsätta nya Al-tjänster och funktioner. De ska också kunna dela och använda kvalitetssäkrade data, modeller och komponenter. Hår har privat sektor en viktig roll att tvara med och utveckla lösningar.
I kraft av sin erfarenhet och kompetens på området föreslår vi att försäkringskassan och Skatteverket blir leverantörsmyndigheter för Al-verkstaden. Alla aktörer ska dock kunna bidra till, och använda verkstaden. Det görs möjligt med hjälp av en differentierat avgiftsmodell för användningen. I anslutning till Al-verkstaden ska det också finans egensamma stödfunktioner för offentliga aktörer som saknar egen kompetens. Exempelvis kan en liten kommun begå harjälp av verkstaden. En "Insatsstykra" bestående av experter och generilarister bistår då med hjälp att identifiera vilke behav som finns och kommer med förlasg på lösningar. Al-verkstaden skulle också kunna fungera som perional en våse: injåd man kan vända sig för att få information om olika Al-relaterade frågor.
\section*{Forskning i världsklass}
Fört var Sverige ska kunna stärka sin konkurrensfakt formå hjälp av Al är det helt centralt att vi har forskningsmiljär för världsklass. Den högå utvecklingstakten, och det krympande avståndet mellan grundforskning och färdliga kommersialiserade produkter, innebårt att dessa miljöer måste vara mycket dynamiska och präglas av ett nära samarbete mellan akademi, näringsliv och offentlig sektor. I detta arbete är det centralt ut Sverige både lyckas locka hit och behålla spetskompetens. Vi föreslår dårför en bred satsning på att stärka våra forskningsmiljöer och vår internationella attraktionskraft. Det omfattar att skapa ett antal excellcenteren inom Al-forskning och särskilda satsning på nationella post doc-tjänster och internationella sägtprofessorer. Vidare föreslår vi en satsning på kombinationstjänster, där forskare kombinerar sin tjänst vid ett lärosåte med arbete inom privat eller offentlig sektor.
Vi föreslår också injärtandtet av nationella forskarkstor inom Al. Dessa ska erbjuda ämnesspecifika forskarutbildningar där Al-kompetensen integreras i det bitlningen, som bör täcka beda samhällsfrågor. En lämplig målsättning är att buttida 600 dokterer med Al-kunskap under en tioårsopider.
För att Sverige ska kunna bedriva en Al-forskning i världsklass krävs också beräkningskraft i form av tillgång till supderatorter anpassade för Al. Genom Knut och Alice Wallenbergas Stiftelse (KAW) har svenska Al-forskare tillgång till superdatorn Berzelius vid Linköpgins universitet. It takt med den snabba tekniska utvecklingen ökar emelertid behoven snabbt på ytterligare beräkningskraft för forskning. Det handlar både om tränning av Al-modeller, vilket kan pågå i flera månader, och användning (inferens) av redan tränade modeller. Det senare ställer andra krav på datorkraft eftersom dattom må måste kunna reagera inom bråkdelen av sen kund på frågor och uppgifter från ett stort antal användare samtidigt. Vi föreslår dårför två separata satsningar på ytterligare beräkningskraft för träning respektive användning av Al-modeller för forskningssyfte.

Jan-Ingvar Jönsson inviger superdatorn Berzelius. Foto: Thor Balkhed/Linköpings universitet

Tittar vi på totala privata investeringar i AI, satt i relation till BNP, visar det sig att endast Israel har en större andel än Sverige, se Figur 2.
Figur 2: Privata investeringar i AI uttryckt som andelar av BNP, 2023
Totala investeringar (\% av BNP), 2023

Notera: Figuren visar de 15 länder som hade högsta privata investeringar i AI.
Källa: The Artificial Intelligence Index, Stanford, 2024, figur 4.3.8. Norbåck och Persson, 2024, Den AI-drivna strukturomvandlingen av det svenska närlingal/vert, mimeo, IFN, Stockholm.
Samtidigt har Sverige tappat placeringar i Global AI Index. I det delindex som mäter Kommersialisering, som ger en relativt bred bild av AI-företags situation i ett land, har Sverige gått från 16:e till 18:e plats jämfört med förra årets mätning. I delindex över Utveckling är motsvarande förändring från plats 17 till plats 30. I Bilaga 2 KP1: för uppföljning diskuteras Global AI Index i mer detalj. Ett annat viktigt mått på innovationsresultat är antal sökta patent. Som framgår av Figur 3 ligger Sverige mycket högt i flera teknikklasser, såsom digital kommunikation och datorteknologi.

\title{
Till
}
\section*{statsrådet}
\section*{Erik Slottner}
\section*{Regeringen beslutade den 7 december 2023 att till:} sätta en kommitté med uppdrag att identifiera behov av och lämna förslag på åtgärder som kan bidra till att stärka utvecklingen och användningen av artificiell intelligens (A) i Sverige på ett hållbart och säkert sätt. Uppdraget skulle slutredovisas senast den 1 jull 2025, men kommittén har valt att tidigarelägga redovisningen till november 2024. Det speglar kommitténs övertygelse att det brådskar att få viktiga politiska beslut på plats.
Som ordförande förordnades från och med den 7 december 2023 styrelseordförande Carl-Henric Svanberg. Som ledamöter förordnades samma dag informationssäkerhetspescialist Anne-Marie Eklund Löwinder, professor Fredrik Hettiz, digitaliseringschef Olof Hernell, förbundsordförande Ulrika Lindstrand, samhällspolitisk chef Nicklas Berild Lundblad, IT-direktör Marcus Matteby, verkställande ledamot Sara Mazur, professor Sylvia Schwaag Serger, chefredaktör Mathias Sundin samt direktör Martin Svensson.
Som ohngänväxeperter utsåg kommittén själv den 26 januari 2024 tidigare chefen för EU-kommissionen i Sverige Katarina Areskoug, verkställande direktör Börje Ekholm, styrelseordförande Mats Granryd, verkställande direktör Martin Lundstedt, generaldirektör Katrin Westling Palm, styrelseordförande och tidigare Europachef Magnus Tyreman samt generaldirektör Nils Öberg.
Som experter förordnades den 30 april 2024 ämnersråd Sara Bringle, stabschef Jörgen Eklund,
Fredrik Heintz
Olof Hernell
Ulrika Lindstrand
Nicklas Berild Lundblad
Anne-Marie Eklund Löwinder
Marcus Matteby
Sara Mazur
Sylvia Schwaag Serger
Mathias Sundin
Martin Svensson
katarina Areskoug
Börje Ekholm
Mats Granryd
Martin Lundstedt
Katrin Westling Palm
Magnus Tyreman
Nils Öberg
Susanne Ackum
Anton Eklöf
Mattias Hector
Jon Olofsson
Aron Verständig

\title{
En stabil grund att bygga på
}
\author{
Energi
}
\section*{ChatGPT sammanfattar:}
Als utveckling står och faller med en stabil och riklig tillgång på el, samtidigt som vi befinner oss ien tid där efterfrågan på fossilfri energi ökar snabbt. Hur ska Al:s energ\©behov mötas, när också industrins och transportsektorns elektrifiering kraver alliter?
I det här kapitlett utforskar vi hur Al kan vara både en utmaning och en lösning för framtidens energsystem. Dessutom ser vi på varför Sveriges fossifria erproduktioner og ress en unik fördeli en deng jobols Al-kapplöpningen. Prägan om elens tillräcklighet blir avgörande - både för Al-teknikens framsteg och för samhällets grona omställning.
\section*{Al och elkonsumtionen}
Al är en energjuntensiv teknologi. Orsaken ligger i den stora mängd data och beräkningskraft i form av tillgång till trakffulla datorer, som krävs för att tråna och använda olika Al-verktyg. Detta gäller särskilt för system med breda användningsområden - exempelvis stora språkmodeller och generativ Al - som kan analysera både text, bild och andra medietyper. Mindre generella allopritmer, med smalare användningsområden, använder mindre beräkningskraft och därmed också mindre energi. Ett exempel på detta är de allopritmer som Skatteverket använder för att upptäcka avvikleser i deklarationerna.
Storleken på Al:s framtadia elbehov är di aghög stisöker och går inte att för utfätsåme med någon precision. Osäkerheten berör på hur tekniken kommer att vettckas och användas. I takt med att modellerna blir mer avancerade och får bredare tillämpning ökar elbehovet. Faktum är att den beräkningskraft som krävs för Al har mångdubblats varje är sedan generativ Al introducerades, vilket har lett till en motsvarande ökning av energianvändningen. Om den nuvarande
trenden för energianvändningen skulle hålla i sig skulle tekniken komma att kräva väldligt mycket el film tradition. \({ }^{}\) Erfarenheten visar dock att digital teknik tenderar att utvecklas snabbt, och det utvecklas hela tiden energieffektivare dataprocessorer och algoritner. \({ }^{}\) Mot den bakgrunden finns det anledning att förvänta sig en betydligt mindre ökning av elförbrukning kopplat till utveckling och användning av Al. Sammantaget är det rimligt att anta att Al kommer att står för någon eller några procent av den totala elkonsumtionen i vården.
Även om Al kommer att öka efterfrågan på elektricitet betyder det inte att den totala energianvändningen i samhället kommer att påverkas på samma sätt. Al är nämligen ett centralt verktyg för att effektivisera energianvändningen i samhället, vilket gör all tilen viktig nyckel för att klara av den gröna omställningen. I faktarutan Tre exempel på hur Al kan effektivisera energianvändningen ges exempel på detta.

**kern	**kern
*clefF4	*clefG2
*k[b-e-a-d-g-c-f-]	*k[b-e-a-d-g-c-f-]
*M3/4	*M3/4
=-	=-
8BBB- 8BB-L	4b- 4bb-
8EE-	.
8BB-	4r
8EE-	.
8BB-	4r
8EE-J	.
=	=
8BBB- 8BB-L	4b- 4bb-
8EE- 8E-	.
8BB- 8B-	4r
8EE- 8E-	.
8BB- 8B-	4e-[ 4ee-[
8EE- 8E-J	.
=	=
8BBB- 8BB-L	4e-] 4ee-]
8EE- 8E-	.
8BBB- 8BB-	4B- 4b-
8EE- 8E-	.
8BBB- 8BB-	4G- 4g-
8EE- 8E-J	.
=	=
8BBB- 8BB-	2.f- 2.ff-
8E- 8A- 8c-	.
8BB- 8B-	.
8E- 8A- 8c-	.
8BB- 8B-L	.
8E- 8A- 8c-	.
8BB-  8B-	8e- 8ee-
8E- 8A- 8c-J	8r
=	=
*-	*-

Al har också stor potential att kunna användas som pedagogiskt verktyg i skolundervisningen, förutsatt att den vetenskapliga forskningen visar på gynnsamma inlärningseffekter. Här kan Al-tekniken innebära nya möjligheter att bland annat individualpassa undervisningen och utjämna elevarens skiftande förutsättningar, till földiv att vill exempel föräldrarnas olika utbildningsbakgrund. Om skolan inte introducerar eleverna till Al finns därmed risken att elevernas olika socioekonomiska bakgurder leder till växande klytfor inom detta område. Hemmet blir då den plats där kunskapen om, och användningen av, Al äger rum.
\section*{Vision: Al-verktyg i undervisningen}
Al-tjänster har potentialen att förändra perspektiven på vad undervisning kan åstadkomma. Dels kan undervisning bli mycket mer individanpassad, dels kan läraze ges mer itd att ägna sig åt dem som behöver mer stöd. Det mest släende är kanske att med en "privatlärare" som anpassar sin pedagogik efter dennes förutsättningar - en läraer som dessutom har obergasnt med och är tillgänglig dynget runt. Anpassningen kan bland annat innebära med en "privatlärare" som anpassar sin med en "sprivatlärare" och avskar digeråd svårighetsgrad för att utmana elever som behöver det. Undervisningen skulle också kunna interferers, vilket innebär att undervisningen anpassas baserat på elevens tidigare prestationer och reaktioner. Al skulle därmed kunna erbjuda en flexibel och stegvis inlärning/processes där varje steg bygger på det<|im_start|>. Al inlärning/probesning, en avskar, avskar.
Al undervisningen skulle också kunna ge en djupare insikt i varje elevs iörprocess. Lärare kan se vad elever studerar, har länge de gör det, vad som fastnar, vad som är svårt, och vad som är intressant eller intressant. Det här får vivestis inte aventyra elevernas integritet. Men använt med omdöme ger
det utrymme för stöd där det verkligen behövs och möjlighet att följa lärandet i realtid på ett sätt som inte är möjligt i dag. En viktig effekt av allt detta år att det skulle kunna öka skolans möjligheter att utjämna livschanser för elever med olika förutsättningar, för att det avkort och avskar, och avskar, och avskar, exempelvis att föräldrars utbildningsbakgrund har stor bydelse yer barrens framgång i skolan, där barn med högskoleutbildade föräldrar har en klar fördel jämfört med andra - en öjämlikhet som i hög grad skulle kunna begränsas med tillgången till en Allärare. Lektationen med Al-läraren behöver eleven läraer lever avskar, om den inte förstår och kan fråga har mänga gångens.
Al i undervisningen har också potentialen att förbättra undervisningen på ett nationellt plan. Eftersons Al möjliggrör storskalig analys och mönsterigenkänning skulle man, genom att samia in och analysera data från stambals alver, kunna identifiera mönster från stambals, i kovacser som annars skulle vara omöjliga att upptäkera. Detta kan leda till insikter som förbättrar undervisningen är den åtta, samt bårtet systematiskt sätt, samtidigt som det kan brytas ner på enskilda skolor och klasser.

Foto: Ground Picture/Shutterstock

\title{
Innovation, entreprenörskap och riskkapital
}
\section*{ChatGPT sammanfattar:}
Al har redan nu börjat omforma vårt samhälle på djupet, och dess potential att främja innovation är oändlig. Men frågan vi måste ställa oss är: hur står Sverige sig i denna omvandling? Vi har länge varit framstående inom innovation, men den snabba utvecklingen av Al ställer oss inför helt nya utmaningar. Samtidigt som vil ligger i framkant på vissa områden, halkar vi efter när det gäller Al's kommersialserling och utveckling. Denna obalans måste åtgärdas om vi ska kunna behälta vår konkurrenskraft.
I det här kapitlet utforskar vi Sveriges nuvarande position i en global innovationskontext och belyser de åtgärder som krävs för att utnyttjta Al's fulla potential. Hur kan vi, genom rätt policy och riktade insatser, säkerställa att vi inte bara hänger med utan leder vägen framät?
Sveriges nuvarande position inom innovation, entreprenörskap och riskkapital
Sverige har låne varit ett av världens mest innoivativa länder. Det är något som bekråftas i många internationella jämförelser. The European Innovation Scoreboard (EIS) måter olika besapketer av innovationsprestation, organiserade i frya kategorier och tomsdimensioner. Sverige och Danmark har under de senaste åren toppat detta index. Bilden stärks också av det breda innovationsindexet The Global Innovationsen Index, som årligen ges ut av World Intellectual Property Organization. Där rankades Sverige 2023 som nummert två efter Schweiz. Indexet Ease of Doing
Business från Världsbanken visar hur bra eller ädanaämåsenligt företagsregleringar fungerar i olika länder över tid. Under de senaste måtperioderna har Sverige rankats som fjärde bästa land. \({ }^{}\) An
Generellt ligger Sverige relativt väl framme i forskning och utveckling (FoU) och investeringar. Bild infirån superdatom Berzelius. Foto: Thor Balkhed/Linkpings universitet

konkurrens. Denna fragmentering har delvis sin grund i EU:s konkurrensregler, som i vissa fall inte medgett företagsförvärv eller ställt krav på åtaganden. Den har problematiken är inte specifikt svensk, utan i hög grad gemensam inom EU. I de så kallade Lettaoch Draghi-rapporterna lyfts också behovet av att reformera den europeiska telekommarknaden. Den senare pekar på behovet av konsolidering och att lätta på konkurrensreglerna liksom åtgärder kopplade till frekvenslicenser och spektrumtildelning.
\section*{Förslag}
- Sverige har i över ett sekel varit ett föregångsland när det gäller utbyggnad och tillämpning av telekommunikationer. Tyvärr har vi nu halkat efter, vilket riskerar att minska vår innovations- och konkurrenskraft. Det är därför av yttersta vikt att åtgärda orsakerna till detta.
- Frågan är dock komplex, och Al-kommissionen ser regeringens beslut att tillsätta en utredning för att påskynda utbyggnaden av 5G och fiber I Sverige som ett viktigt steg i rätt riktning.
\section*{Utredningen ska föreslå komplettering} och anpassningar för att möta kraven i EU:s fördning om gigabitinfrastruktur. Denna EUfördonding syftar till att minska kostnaderna för utbyggnaden av höghastighetsnät för elektriskomunkation.
- Al-kommissionen anser att utredningen även bör ges i updrag att analysera de relevanta förslagen i den linjigen publicerade Draghi-rapporten, som berör telekommarknaden. Här föreslår vi till exempel att utredningen särskilt ska analysera de förslag som rör hur konkurrensrätten påverkar företagskonsolideringar och lägga förslag på vilka positioner som Sverige bör drivia i EU på Al-mördet. Utredningen bör även överväga hur Sverige kan drivia på för ökade investeringar i telekom på EU-nivå, såsom investeringar i bandbredd och 5G SA, Kostnaderna för dessa åtgärder bör vägas mot det värde de kan skapa i termer av en konkurrenskraftig Al-sektor.
\section*{5G-teknik gör Al möjlig}
5G-tekniken gör Al möjlig genom snabbare och mer robust dataöverföring. Den ökade hastigheten och kapaciteten i 5G-nätverk skapar nya förutsättningar för Al-baserade system som bygger på att bearbeta stora mängder data i realtid. 5G SA (stand alone) ger operatörer möjlighet att avsätta specifik näruserskapacitet baserat på individuella kunders behov. Denna flexibilitet gör att kunder med höga krav på stabilitet och påtiltighet kan få en anpassad uppkoppling, vilket inte var möjligt med tidigare nättet kenerationer där resurserna delades lika mellan alla användare.
Rattikera, för att skor, för att skor.
redan börjat göra skillnad. I Kankbergsgruvan i Västerbotten använder Boliden ett lokalt 5G-nät för att fjärstryra fordon på 400 meters diguj. Detta förbättrar både säkerheten och effektiviteten i gruvdriften, eftersom maskinerna kan styras från för att för att företagning och 5G. 5G. 5G. 5G. 5G. 5G.
att "hjärnan" som styr maskinerna kan placeras på en annan plats än själva maskinen. Tekniken öppnar upp för framtida implementering av smarta Al-baserade system, utan att man för den sakens skull behöver installera specifik hårdvara i 5G.
Denna utveckling visar på potentialen för framtida samverkan mellan 5G-teknik, Al och andra avancerade system. Genom att kombinera dessa teknologier kan vi förvänta oss nya innovationer som förändrar hur vi använder digjtala nätverk. 5G SA, I kombination med Al, har potential att spela en nyckelroll i den kommande digjtala transformationen, även om tekniken ännu inte är tillgångligi va senska operatörer.

Foto: Boliden

\title{
WASP
}
WASP, som står för Wallenberg Al, Autonomous Systems and Software Program, initierades 2015. Det är Sveriges största enskilda forskningsprogram någonsin, och en viktig katalysator för samarbetet mellan lärosäten och företag inom områdena Al, autonoma system och mjkukvara.
WASP bygger på den kombinerade befintliga spetskompetensen vid Sveriges fem större Informationsoch kommunikationsteknik (IKT) universitet: Chalmers tekniska högskola, Kungliga Tekniska högskolan, Linköpings universitet, Lunds universitet och Umeå universitet samt ledande forskargrupper vid Örebro universitet, Uppsala universitet och Luleå tekniska universitet.
\section*{En ögonblicksbild över WASP}
- Budget: 6,5 milljarder kronor fram till 2031.
- Mål: examinera 600 doktorer och rekrytera 80 värdsledande forskare.
- Utfall hittills:
- Mer än 580 doktorander har antagits och mer än 140 har examinerats vid universiteten och högskolorna som bildar nätverket enligt ovan.
- 67 internationella toppforskare har rekryterats till berörda lärosäten.
- 80 företag och myndigheter har engagerat sig i satsningen.
Satsningar som snabbt kan öka spetskompenensen ländert
För att Sverige ska kunna konkurera i en värld där vägen från innovation till produkt blir kortare hela tiden, krävs det att grundforskning kopplas direkt till företag på nya ohnivativa sätt. Grundforskning och tillämpad forskning måste vävas samman genom att vi skigar for forskningsmiljörer med tillräcklig kritisk, massa för att uppnå både excellens för forskningen och synerigefIKter ut i samhället. Det kanadensiska, exempelt är intressant att studera. Man har bland annat satsat på en modell med excellencenter, som har visat sig framgångärsik. \({ }^{}\) Härutöver har man den så kallade Mitcas-satsningen, där företag och lärosäten förs samman, vilket hjälper bägge parter att bättre förståvarandras komparativa fördelar. \({ }^{}\) Det har fört med sig att fler professor, doktorander och studenter fått anställining inom det privata näringslivet och därmed kommit närmare industriell innovation och utveckling.
Flera svenska lärosäten har också bildat centra som fokuserar både på grundforskning och tillämpad Al, Dessa centra fungerar som nav där forskare, studenter och industripartners samarbetar och tar sig an viktiga utmanigar och möjligheter inom en rad olika områden. Genom att skapa deidkerade Al-initiativ och träjma tvåvnetenskapligt samarbete positionarher-
sig dessa universitet och högskolori i det nationella och internationella Al-landskapet. Det är viktigt att de initiativ som Al-kommissionen föreslår beaktar de pågående aktiviteterna inom området, eftersom det måste finnas god mottagarkapacitet för nya resurser och initiativ. Vi anser också att lärosätena bör engageras tidigt i planeringen av dessa föreslagna initiativ, framför allt för att säkerstälia önskad effekt och att initiativen ingår i ett större sammanhang.
\section*{Förslag}
- Al-kommissionen föreslår att regeringen inättär ett begränsat antal excellenscenter inom Al, med med med med med med med med med med med med med med med med med med med med
100 miljoner kronor per är under 10 år, till ett totalbelopp om 300 miljoner kronor årligen. Centren ska samverka med forskare och projekt vid olika svenska och internationella lärosäten, samt samarbeta med både näringslivet och offentlig sektor. Dessa center kan vara virtuella och omfatta flera lärosäten, men ska ha ett lärosäte som huvudman. Vetenskapsrådet, i samråd med övriga forskningsfinansiår, bör vara huvudman för en utlysning och fördela resurserna i konkurrens mellan universitet och högskolorl.

\title{
Data som en förutsättning för Al-utvecklingen
}
\section*{ChatGPT sammanfattar:}
Historiskt har data spelat en central roll i beslutsfattande, men i en tid präglad av snabb teknologisk utveckling har dess betydelse blivit mer tuttad än någonsin.
I detta kapitel utforskar vi hur tillgången till relevant och högkvaltativa data är en avgörande förutsättning för utvecklingen och tillämoningen av artificialn intelligens (AI). Med konkreta exempel, såsom algormitter för cancerdiagnostik, visar vi hur kvalitet och variation i data inte bara påverkar teknikens effektivitet utan också dess etsika implikationer. För att möjligöra en effektiv användning av dessa teknologier är det avgörande att lagfettlinnengen utformas på ett digitaliseringsvänligt stätt, vilket innebär att man tidigt bär beaktla hur ny teknojogi kan användas för att skapa samhåljenytta. Följ med oss när vid yderk jüpare in i den komplex relationen mellan data, Al och samhällets framtid.
\section*{Behovet av relevant data}
Framväxten av Al har förändrat betydelsen av data och statistiki i grunden. Tidigare var data främst ett medel för att särkeerställa att beslut av olika slag togs på rätt grunder - ett nog så viktigt syfte. Med Al har data blivit ett verktyg som har potential att förändra samhället.
Ett exempel kan tjäna som illustration av den ökade betydelsen av data: En utvecklare har fått i uppdrag att skapa en algormit för att diagnostisera cancer utifrån röntgenbilder, avsedd att fungera som ett stödt för läkare. För att algoritmen ska fungera effekttivt krävs tillgång till ett stort och varierat datamaterial - det vill säga en stor mängd röntgenbilder som har analyserats av ferrarna radiologer, både bilder som visar cancettmorøer och sådana som inte gör det. Ju fler röntgenbilder utvecklaren har till sitt förfogande, desto bättre bli alloritmen på att ställa rätt daglinos. I det här förenklade exemplet är det röntgenbilder som har å de data som är nödvändiga för att kunna trina algo, itrtmen. För andra typer av Al-modeller kan det handla om exempelvis text, bilder eller ljudinspeningar.
\(\checkmark\) Aven om tillgång till data år helt avgörande för användningen av Al räcker det inte enbart med stora datamängder, även kvaliteten är avgörande. Om radiolegerna som skapade träningsdata i vårt exempel hade gjort ett slavrigt jobb och missat att märka ut vissa tumörer, skulle modellen ha presterat såmre. Det är även nödvändigt att data struktureras noggraft
i enlighet med de specifikationer som gäller inom det aktuella området. \({ }^{\text {s }}\)
Modelien i exemplet hade också fungerat såmre om röntgenbilderna endast representerade en viss grupp av människor, eftersom olika kroppsteyer, åldrar eller sjukdomsmönster kan påverka hur cancerutmörner framträder på bilderna. Detta skulle göra algoritmen mer effektiv för de kroppstyper som finns i dataunderaktet, men mindre tillförlitlig för andra grupper. Om vi exempelvis antar att vi enbart har tillgång till röntgenbilder för man skulle dessa inte vara särskilt relevanta för pat prognosticera cancer hos kvinnor.
Utvecklaren hade kunnat lösa detta genom att komplettera med en mer varierad databas som inkluderklare röntgenbilder från människor med olika bakgrund. I ett sådant fall kan dock andra problem uppstå. De nya bilderna kan till exempel vara i ett annat format eller inkludera en bredare definition av tumörer, till exempel bilder på gordattade tümörer eller förstädar till cancer. Med andra ord behöver tränigandsata noggrant spegla det specifika problem som Al ska lösa, vilket underlätta genom att använda standarder och dokumentation vid datainsamlingen.

\title{
Figur 7. Talang - ranking och poäng
}
\begin{tabular}{|c|c|c|c|c|c|c|c|c|c|}
\hline \multicolumn{10}{|c|}{ USA - plats 1} & 100 \\
\hline \multicolumn{10}{|c|}{ Indian - plats 2} & 42 & & & & \\
\hline \multicolumn{10}{|c|}{ Tyskand - plats 3} & 35 & & & & & \\
\hline \multicolumn{10}{|c|}{} & 30 & & & & & \\
\hline \multicolumn{10}{|c}{ Schweiz - plats 5} & 30 & & & & & \\
\hline \multicolumn{11}{|c|}{ Israel - plats 7} & 27 & & & & & \\
\hline \multicolumn{11}{|c}{} & 23 & & & & & & \\
\hline \multicolumn{11}{|c}{ Nenleränderna -... } & 23 & & & & & \\
\hline \multicolumn{11}{|c||}{ Finland - plats 14} & 17 & & & & & \\
\hline \multicolumn{11}{|l|}{ Sverige - plats 15} & 17 & & 25 & & Poäng för plats 10 & \\
\hline \multicolumn{11}{|c|}{ Danmark - plats 16} & 17 & & & & & \\
\hline \multicolumn{2}{|c|}{ Norge - plats 24} & 14 & & & & & & & \\
\hline \multicolumn{2}{|c|}{} & 0 & 20 & & 40 & & 60 & & 80 & 100 \\
\hline
\end{tabular}
Notera: Placeringen för varie land inom Talang visas efter landets namn. Den horisontella axein visar poängen för varie land, beräknad utifrån indikatorer relaterade till området. Den högsta poäng som ett land kan få år 100. Den grå stapeln representerar den poäng som krävs för att placera sig på plats 1017 alang12024 års upplaga.
Källa: The global AI index, 2024 års upplaga.
\section*{De underliggande indikatorerna för detta område \\ inkluderar bland annat antalet nytuxaminera de inom \\ STEM och IT samt data från Linkedin. \({ }^{}\) Indikatorerna \\ rör även mått på här utkiva aktörer från olika länder är \\ på populära diskussionsforum för AI-utveckling. \({ }^{}\) \\ Antagandet är att länder med ett stort digital \\ avtryck in om delar av internet där AI-utvecklare är \\ verksamas också har bättre tillgång till kvalificerad \\ arbetskraft inom Al. Denna färdplan innehåller flera \\ konkreta åtgärder för att säkerställa Al-kompetens i \\ Sverige. I kapitlet Kompetenslyft för alla presenteras \\ exempelvis förslag om ett lärkjärfyt inom Al vid aka \\ medenin. Även kapitlet Spetsforskning i samverkan \\ innehåller initiativ som kan stärka Sveriges tillgång till \\ kvalificerad arbetskraft, band annat förslaget om att \\ utbida 600 doktorander under en tioårsperiod. \\ \\ \\ \\ \\ \\ \\ \\ \\ \\ \\ \\ \\ \\ \\ \\ \\ \\ \\ \\ \begin{tabular}{l} 
Vad kan då vara en ambitiöis och realistisk målsätt- \\
ningen för området Talang? Genom de strategiska \\
satsningar som görsi denna Färdplan och av andra \\
aktörer borde Sverige kunna farkas bland de 10 \\
främsta länderna senast år 2030.
\end{tabular} \\
\hline
\end{tabular}
- Sverige ska senast år 2030 ha gått från plats 15 till att tillhöra topp 10 iområdet Talang
\section*{Operativ miljö}
En stödjande och förutsägbar operativ miljö är viktig för att skapa förutsättningar för utveckling och implementering av Al. Det handlar om aspekter som tillit, säkerhet samt regelverk kring data och personlig integritet. Inom arndödet Operativ miljö ligger Sverige på plats 5.
Figur 8 visar Sveriges ranking i relation till jämfö-relser. Sverige och andra nordiska länder medlareländerna. Sverige och andra nordiska länder planäder planäder planäder planäder. Sverige har en kompartivå fördel inom detta område, särskilit jämfört med länder som placerar sig högt in andma ordnäden, såsom Singapore och Israel. Dessa hamnar här betyligt långræn, mer på plats 48 respektive 65. Noterbart är Italins starka position. \({ }^{}\)
\footnotetext{
[5] Notera att det här handlar om nytuxaminera de inom STEM-ndrätet. I området Forskning används antalet forskare in somma område.
[6] För att dokumentera förkomststen av Al-experter i olika länder samlas data in om antalet personer som beskrivser sig som "tigenjö". "Forskansar" eller "forskansar" eller "forskansar" eller "forskansars" eller "forskansar" eller "forskansar" eller "fforskansar" eller "fforskansar" eller "fforskansars" eller "fforskansar" eller "fforskansar" eller "forskansar" eller "fforskansars" eller "fforskansars" eller "fforskansar" eller "forskansar" eller "forskansare" eller "fforskansare" eller "fforskansare" eller "fforskansar" eller "fforskansare" eller "fforskansare" eller "forskansare" eller "fforskansare" eller "forskansar" eller "fforskansare" eller "fforskansar" eller "forskansare" eller "fforskansar" eller "fforskansar" eller "främsta länderna senast år 2030.
}

<|im_end|>};

\title{
The future - of European competitiveness
}

Mario Draghis rapport Den europeiska konkurrenskraftens framtid om europeisk konkurrenskraft.
Foto: Alexandros Michalidis/Shutterstock
bild som har bekräftats vid våra många möten med representanter för olika samhällsgrupper: vi måste bli bättre på att utnyttja Al för att adressera våra samhållsutmaningar, och det måste gå fort.
Det finns många sått att illustrera Sveriges efterslägning på Al-området. Ett är att titta på internationella jämförelser. The Global Al Index från Tortoise Media är ett väletablerat index, lämpat för att jämföra Al-utvecklingen i samhället som helhet mellan länder. \({ }^{}\) Avsikten är att bedöma ett lands kapacitet att nyttja Alå. Avrets upplaa av indexet visar tyvärr att Sveriges relativa position är svag och att den har församhärs. I det samminavgåa indexet faller Sverige från plats 17 (år 2023) till plats 25 (år 2024) av 83 jämföra länder. Blind E-Ulånderna placerar vi oss för att för att för att för att för att för att för att för att för att för att för till stått för all<|im_start|>. Särskilt svag är vår ranking avseende den politiska styrmingsdimensionen (Government strategy) där Sverige rankas så lågt som på plats 57. Vi maner att det här är en avgörande svaghet, eftersom politiskt ledarskap beshövs för att få behövliga åtgärder på plats. Det har också varit ett av de starkaste budskapen vi har fått i våra kontakter med representanter för olika grupper i samhället.
Även om den här typen av ranking mellan länder aldrig är perfekt i alla delar, så vår bedömning att den generella bilden är tvördig- Sverige halar efter. Att Sverige inte bara rankas lågt globalit, utan även inom EU, år särskit bekymmersamt eftersom EU också tenderar att halka efter på Al-området, något som betonas i Mario Draghis rapport Den europeiska konkurrenskraftens framtid. \({ }^{}\) Som exempel är den amerikan ska och kinesiska dominansen på marknaden för molntjänster närmast total, genom att de tio största Al-plattformarna i EU ägs av företag från juln just dessa två länder. Den här situationen är problemastik ur flera synvinklar. För det första år den ett underbetyg åt europeisk innovations-, tilväxt-och regleringspolitik, som måste blir men innovations- och tillväxtvärlig. För det andra illustrerar den tydligt hur särbara europeiska aktörer är.
Frågan om särbarhet är dock komplex och rymmer två viktiga aspekter. I dagens säkerhetspolitsilka läge bör man noga ge akt på olika bereondeställningar, framför allt inom samhällskritiska tjänster och sektorer. Det här talar bland annat för att en engar kapacitet bör byggas upp in omolne krycelområden. I Fårdplanen lägger vi ett antal förslag i den här riktningen. Men det är samtidigt helt centralt att säkerkittå allt svenska och europeiska aktörer har fortsatt tillgång till den senaste Al-teknologin, som i dag offta erbjuds genom amerikanska Al-plattformar. Det år i detta perspektiv problematiskt att lanseringen av ett
\footnotetext{
[1] Se Bliga B för en fördjud på presentation av The Global Al Index.
Se faktaruta i kapitalet Internationella positioner för mer information.
}

\title{
Förord från ordförande: Vi har gjort det förut
}
\author{
Al är en omtumlande, framforsande teknik som \\ i människans händer skapar förutsättningar för \\ grundläggande förbättringar av våra samhållen, \\ precis som järnvägen, elen och telefonen gjorde.
}
Takten och förändringarna är sådana, att det inte går att stå vid sidan och invänta den totala och fullständiga kunksågen, utan det gäller att orientera under färd, etablera principer och färdriktning och sedan agera. Det är anleddningen till att Al-kommissionen bett om att få tidigarelägga denna rapport.
Al är redan en del av vår verklighet i Sverige inom företagen, organisationerna och myndigheterna. Det sker, som ofta är fallet med y teknik, lite oordnat, prestigt - inte utifrån någon stor, övergripande plan, utan beroende på var viljan, entreprenörskraften och resurserna finns i samhället. Vid tidigare teknikkiften har Sverige ofta tidigt kunna visa på ett teknikledarskap genom att nya, snabbt våsande och senare varidelsdande företag etablerat. Vis är eärnu inte något sådant när det gäller Al och Sverige.
Al förstås bäst om det anknyts till redan befintliga verksamheter och funktioner i ett samhälle: utbildning, vård och omsorg, forskning, produktutveckling, dataanalys, arbete mot penningtvätt, smitskyddsarbete, trafiksäkerhet, skatteuppbrd, offentliga transfereringssystem, kundvård eller vad det nu kan vara.
Därmed blir det också uppenbart att en regering, eller en Al-kommission, inte kan formulera någon stor plan om "hur" en ny teknik ska användas. Men något som det finns ett stort behov av - och som också denna rapport fokuserar på - de principier, standarder och kraftsamlingar som krävs för att spridningen och användningen av Al ska bli så snabb, säker, omfattande och positiv som möjligt för det svenska samhället. Det krävs helt enkelt övergripande trafikregler, bränsleförsjönning och målbilder för Al. Al kräver särskilk vaksamhet och ett särskilt tokus på de risker som exempelvis manipulation, splitting och fake news som Al kan göra möjlig.
Al-kommissionens ordförande, Carl-Henrik Svanberg. Foto: Volvo
Även om Sverige annu inte har något ledarskap kring Al, utan snarare läggerer offens, finns det andra aspekter som tarkar till Sveriges fördel: Sverige är ett tillitsbanserat samhälle, där samarbete och kompromisser mellan olika samhälitsnressen skapat en närmast unik förmåga till omställning av samhället, ofta utifrån teknikskiften. Sveriges målsättning att alla ska med signalerar också behovet av trygghet under förändring. Företag kommer att grundas och våka sig stora både som ett resultat av första, andra och tredje vägen av Al - och så vidare.
Med detta sagt måste naturlitygista tentiltis att de grundläggande förutsättningarna finns på plats. Ett tydligt politiskt ledarskap måste till it der av stor och snabb förändring, men det krävs också samordnade resurser för vår offentliga sektor, säker entliförsel, tillgänglig beräkningskraft, snabb digital frinfrastruktur, god kompetensförsjönning och stabila entreprenörsvillkor.
Al blir ett hot om vi står vid sidan om och passivt åser en teknik- och samhällsförändring ut att angera och koppla den till våra övergripande strävanden i Sverige om en bättre tillvaro för alla. Al blir en möjlighet först när vi med självförtroende, kloka kraftsamlingar och tycliga spelregler utnyttjar Al välfärdens, tillväxtens och upbypagnaen av det goda Sveriges tjänst.
Carl-Henric Svanberg
Stockholm i november 2024

\title{
Figur 7: Olika faktorers bidrag till förändringen i produktivitet i det svenska näringslivet från 1999 till 2021
}
\author{
Antal tusen kronor per anställd
}

\title{
Tre exempel på hur Al kan effektivisera energianvändningen \({ }^{\text { }}\)
}
\author{
Al kan användas för att förbättra matchningen mellan utbud \\ och efterträgan för värderberoende energikällor \\ För att minska risken för felprognostiseringar mellan utbud \\ och efterträgan för sol- och vindkraft har Vattenfall utvecklat \\ självlärande algoritmer som kombinerar historiska väderdata \\ med realitidsinformation om molnrörelser. Dessa algoritmer \\ möjliggör mycket precisa närtidprognoser. Med dessa \\ prognoser minskar Vattenfall sin risk och kostnader, samtidigt \\ som de bidrar till ett mer stabilt elsystem.
}
\section*{Datadriven minimering av driftstörningar och strömavbrott med hjälp av Al}
E.ON har utvecklat en algoritm som förutspår när ett medelspänningsnät behöver bytas ut, vilket har minskat antalet strömavbrott med upp till 30 procent. Italienska Enel installerade 2019 sensorer på kraftledningar för att läsa av vibrationsnivärer. Utifrån maskininlärningsalgoritmer kunde Enel identifiera och åtgärda potentiella problem. Denna insats resulterade i att antalet strömavbrott minskade med 15 procent.
\section*{Smartsystem optimerar energianvändningen i byggnader genom att förutspå elpriset}
ABB har utvecklat en Al-modell för prognoser över energianvändning i kommersiella byggnader, vilket hjälp er fastighetsägare att undvika höga elpriser och dra nytta av rörliga elavtal. Om liknande mekanismer implementeras brett i samhället kan det bidra till en bättre matchning mellan utbud och efterträgan på el, vilket i sin tur leder till ett stabilare elsystem. Al har även potential att uppnå energibesparingar genom att optimera luftkonditioneringsoch belysningssystem i byggnader. Enligt Schneider Electric kan Al minska energianvändningen i byggnader med 15-25 procent de kommande fyra åren.
Se Why Al and energy are the new power couple, International Energy Agency, 2023: Energy-Guzzling Al Is Also the Future of Energy Savings, The Wall Street Journal, 2024.